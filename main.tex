\documentclass[mat2, anglescina]{fmfdelo-mod}
% \documentclass[fin2, anglescina, tisk]{fmfdelo}
% \documentclass[isrm2, anglescina, tisk]{fmfdelo}
% \documentclass[ped, anglescina, tisk]{fmfdelo}
% Če pobrišete možnost tisk, bodo povezave obarvane,
% na začetku pa ne bo praznih strani po naslovu, …

%%%%%%%%%%%%%%%%%%%%%%%%%%%%%%%%%%%%%%%%%%%%%%%%%%%%%%%%%%%%%%%%%%%%%%%%%%%%%%%
% METAPODATKI
%%%%%%%%%%%%%%%%%%%%%%%%%%%%%%%%%%%%%%%%%%%%%%%%%%%%%%%%%%%%%%%%%%%%%%%%%%%%%%%

% - vaše ime
\avtor{Izak Jenko}

% - naslov dela v slovenščini
\naslov{K3 ploskve z vidika izpeljanih kategorij}

% - naslov dela v angleščini
\title{K3 surfaces from a derived categorical viewpoint}

% - ime mentorja/mentorice s polnim nazivom:
%   - doc.~dr.~Ime Priimek
%   - izr.~prof.~dr.~Ime Priimek
%   - prof.~dr.~Ime Priimek
%   za druge variante uporabite ustrezne ukaze
\mentor{.}
% \somentor{...}
% \mentorica{...}
% \somentorica{...}
% \mentorja{...}{...}
% \somentorja{...}{...}
% \mentorici{...}{...}
% \somentorici{...}{...}
% mentorja morate navesti še z angleškim nazivom:
% - doc.~dr. = Assist.~Prof.~Dr.
% - izr.~prof.~dr. = Assoc.~Prof.~Dr.
% - prof.~dr. = Prof.~Dr.
\supervisor{.}
% \supervisor{...}
% \cosupervisor{...}
% \supervisors{...}{...}
% \cosupervisors{...}{...}

% - leto magisterija
\letnica{2025}

% - povzetek v slovenščini
%   V povzetku na kratko opišite vsebinske rezultate dela. Sem ne sodi razlaga
%   organizacije dela, torej v katerem razdelku je kaj, pač pa le opis vsebine.
\povzetek{.}

% - povzetek v angleščini
\abstract{.}

% - klasifikacijske oznake, ločene z vejicami
%   Oznake, ki opisujejo področje dela, so dostopne na strani https://www.ams.org/msc/
\klasifikacija{.}

% - ključne besede, ki nastopajo v delu, ločene s \sep
\kljucnebesede{.}

% - angleški prevod ključnih besed
\keywords{.}

% - neobvezna zahvala
\zahvala{.}

% - ime datoteke z viri (vključno s končnico .bib), če uporabljate BibTeX
\literatura{references.bib}

% - ime datoteke z razširjenim povzetkom v slovenskem jeziku
\razsirjenipovzetek{razsirjenipovzetek.tex}

%%%%%%%%%%%%%%%%%%%%%%%%%%%%%%%%%%%%%%%%%%%%%%%%%%%%%%%%%%%%%%%%%%%%%%%%%%%%%%%
% DODATNE DEFINICIJE
%%%%%%%%%%%%%%%%%%%%%%%%%%%%%%%%%%%%%%%%%%%%%%%%%%%%%%%%%%%%%%%%%%%%%%%%%%%%%%%

% Packages

% Fonts
% \usepackage{newpxtext}

% \usepackage{concmath}
\usepackage[T1]{fontenc}

% \usepackage{fourier}
% \usepackage[T1]{fontenc}

% Additional packages
% \usepackage[parfill]{parskip}

\usepackage{enumitem}
% \usepackage[shortlabels]{enumitem}

\usepackage{xparse}

% Colors
\usepackage{xcolor}

\usepackage{tikz}
\usetikzlibrary{cd, babel}
\usepackage{quiver}

% For comments in pdf
\usepackage{todonotes}
% \usepackage[pdftex,dvipsnames]{xcolor}
\newcommand{\info}[1]{
    \todo[linecolor=olive,backgroundcolor=olive!25,bordercolor=olive, size=\tiny]{#1}
}

% Krajšave v slovenščini
\newcommand{\ti}{t.~i.\ }
\newcommand{\tj}{tj.\ }
\newcommand{\oz}{oz.\ }
\newcommand{\npr}{npr.\ }

% Acronyms in english
\newcommand{\ie}{i.e.\ }
\newcommand{\eg}{e.g.\ }
\newcommand{\resp}{resp.\ }
\newcommand{\cf}{cf.\ }

% Shorthands
\newcommand{\R}{\mathbf R}
\newcommand{\N}{\mathbf Z_{\geq 0}}
\newcommand{\Z}{\mathbf Z}
\newcommand{\C}{\mathbf C}
\newcommand{\OO}{\mathcal O}
\newcommand{\F}{\mathcal F}
\newcommand{\G}{\mathcal G}
\newcommand{\A}{\mathcal A}
\newcommand{\B}{\mathcal B}
\newcommand{\I}{\mathcal I}
\newcommand{\J}{\mathcal J}
\newcommand{\D}{\mathcal D}
% \newcommand{\k}{\texttt k}


% Isos, equivalences, ...
\newcommand{\iso}{\simeq}
\newcommand{\htpy}{\simeq}
\newcommand{\natiso}{\simeq}

% Categorical constructions
\newcommand{\im}{\operatorname{im}}
\newcommand{\coim}{\operatorname{coim}}
\newcommand{\coker}{\operatorname{coker}}
\newcommand{\colim}{\operatorname{colim}}
\newcommand{\Ob}{\operatorname{Ob}}
\renewcommand{\hom}{\operatorname{Hom}}
\newcommand{\Hom}{\operatorname{Hom}}
\newcommand{\id}[1]{\operatorname{id}_{#1}}
\newcommand{\op}{\text{op}}

% Specific categories
\newcommand{\Set}{\mathsf{Set}}
\newcommand{\Ab}{\mathsf{Ab}}
\renewcommand{\mod}[1]{\mathsf{Mod}_{#1}}
\newcommand{\coherent}[1]{\mathsf{coh}(#1)}
\newcommand{\quasicoh}[1]{\mathsf{qcoh}(#1)}
\newcommand{\com}{\mathsf{Ch}}

% Homotopy category of complexes
\newcommand{\K}{\mathsf{K}}
\newcommand{\kcat}[1]{\K(#1)}
\newcommand{\kplus}[1]{\K^+(#1)}
\newcommand{\kminus}[1]{\K^-(#1)}
\newcommand{\kb}[1]{\K^{b}(#1)}

% Derived categories
\newcommand{\derd}{\mathsf{D}}
\newcommand{\dercat}[1]{\derd(#1)}
\newcommand{\derplus}[1]{\derd^+(#1)}
\newcommand{\derminus}[1]{\derd^-(#1)}
\newcommand{\derb}[1]{\derd^b(#1)}

% Derived functors
\newcommand{\rderived}[2]{\mathbf{R}^{#1}#2}
\newcommand{\lderived}[2]{\mathbf{L}^{#1}#2}

% \NewDocumentCommand{\rderived}{o m}{
%     \IfValueTF{#1}{%
%         R^{#1} #2%
%     }{%
%         R #2%
%     }%
% }

\newcommand{\derivedtensor}[3]{#1 \otimes_{#3}^{\operatorname{L}} #2}
\newcommand{\localtor}[4]{\mathcal{T}or^{#4}_{#1}(#2, #3)}
\newcommand{\localext}[4]{\mathcal{E}xt^{#1}_{#4}(#2, #3)}

% Cohomology
\newcommand{\td}{\operatorname{td}}
\newcommand{\tdclass}[1]{\td_{#1}}
\newcommand{\firstchern}{\operatorname{c}_1}
\newcommand{\ch}[1]{\operatorname{ch}(#1)}

% Lattices
\newcommand{\hyperbolic}{\mathcal U}


\newcommand{\inv}{^{-1}}
\newcommand{\abs}[1]{\left\lvert #1 \right\rvert}


% Quotient
% \newcommand{\bigslant}[2]{{\raisebox{.2em}{#1}\left/\raisebox{-.2em}{#2}\right.}}

%%%%%%%%%%%%%%%%%%%%%%%%%%%%%%%%%%%%%%%%%%%%%%%%%%%%%%%%%%%%%%%%%%%%%%%%%%%%%%%
% ZAČETEK VSEBINE
%%%%%%%%%%%%%%%%%%%%%%%%%%%%%%%%%%%%%%%%%%%%%%%%%%%%%%%%%%%%%%%%%%%%%%%%%%%%%%%

\begin{document}

\section{Triangulated categories}

\subsection{Additive, $k$-linear and abelian categories}

We mainly follow \cite{kashiwara2006categories}.

A categorical \emph{biproduct} of objects $X$ and $Y$ of a category $\mathcal C$ is an object $X \oplus Y$ together with morphisms
\begin{center}
    \begin{tabular}{c c}
        $p_X\colon X \oplus Y \to X$ & $p_Y\colon X \oplus Y \to Y$ \\
        $i_X\colon X \to X \oplus Y$ & $i_Y\colon Y \to X \oplus Y$ \\
    \end{tabular}
\end{center}
for which the pair $p_X, p_Y$ is the categorical product of $X$ and $Y$ and the pair $i_X, i_Y$ is the categorical coproduct.

% $X \xleftarrow{p_X} X \oplus Y \xrightarrow{p_Y} Y$ is the categorical product of $X$ and $Y$ and $X \xrightarrow{i_X} X \oplus Y \xleftarrow{i_Y} Y$.

\begin{definition}
    A category $\A$ is additive (resp.\ $k$-linear) if all the hom-sets carry the structure of abelian groups (resp.\ $k$-modules) and the following axioms are satisfied
    \begin{enumerate}
        \item[A1] For all all objects $X$, $Y$ and $Z$ of $\A$ the composition
        \[
            \circ \colon \hom_{\A}(Y, Z) \times \hom_{\A}(X, Y) \to \hom_{\A}(X, Z)
        \] 
        is bilinear.
        \item[A2] There exists a \emph{zero object} $0$, for which $\Hom_\A(0,0) = 0$.
        \item[A3] For any two objects $X$ and $Y$ there exists a categorical biproduct of $X$ and $Y$. 
    \end{enumerate} 
\end{definition}

% Whenever it will be clear from context, we will sometimes omit the subindex $\A$ from $\Hom_\A(X, Y)$ for clarity and write just $\Hom(X, Y)$ instead.

\begin{definition}
    A functor $F \colon \A \to \A'$ between two additive (resp.\ $k$-linear) categories $\A$ and $\A'$ is \emph{additive} (\resp \emph{$k$-linear}), if its action on morphisms
    \[
        \hom_\A(X, Y) \to \hom_{\A'}(F(X), F(Y))
    \] 
    is a group homomorphism (resp.\ $k$-linear map).
\end{definition}

For a morphism $f\colon X \to Y$ in an additive category $\A$ recall, that the \emph{kernel} of $f$ is the equalizer of $f$ and $0$ in $\A$, if it exists, and, dually, the \emph{cokernel} of $f$ is the coequalizer of $f$ and $0$. It is well-known and easy to verify, that the structure maps $\ker f \hookrightarrow X$ and $Y \twoheadrightarrow \coker f$ are mono and epi respectively. We also define the \emph{image} and \emph{coimage} of $f$ to be
\begin{align*}
    (\im f \to Y) &:= \ker(Y \to \coker f) \\
    (X \to \coim f) &:= \coker(\ker f \to X).
\end{align*}

\begin{definition}
    An additive category $\A$ is \emph{abelian}, if it contains all kernels and cokernels and satisfies axiom A4.
    \begin{enumerate}
        \item[A4] For any morphism $f\colon X \to Y$ in $\A$ the canonical morphism $\coim f \xrightarrow{\ \sim \ } \im f$ is an isomorphism.
        \[\begin{tikzcd}[column sep = small]
            % https://q.uiver.app/#q=WzAsNixbMSwwLCJYIl0sWzIsMCwiWSJdLFswLDAsIlxca2VyIGYiXSxbMywwLCJjb2tlciBmIl0sWzEsMSwiY29pbWYiXSxbMiwxLCJpbWYiXSxbMiwwXSxbMCwxLCJmIl0sWzEsM10sWzAsNF0sWzEsNV0sWzQsNSwiaCJdXQ==
            {\ker f} & X & Y & {\coker f} \\
            & \coim f & \im f
            \arrow[hook', from=1-1, to=1-2]
            \arrow["f", from=1-2, to=1-3]
            \arrow[two heads, from=1-2, to=2-2]
            \arrow[two heads, from=1-3, to=1-4]
            \arrow[hook', from=2-3, to=1-3]
            \arrow["\sim", from=2-2, to=2-3]
        \end{tikzcd}\] 
    \end{enumerate}
\end{definition}

\begin{remark}
    We obtain the morphism mentioned in axiom A4 in the following way. Since $(\im f \to Y) = \ker(Y \to \coker f)$ and the composition $X \to Y \to \coker f$ is equal to $0$, there is a unique morphism $X \to \im f$ by the universal property of kernels. The composit $\ker f \to X \to \im f$ then equals $0$, by the fact that $\im f \hookrightarrow Y$ is mono and $\ker f \to X \to Y$ equals $0$. From the universal property of cokernels we obtain a unique morphism $\coim f \to \im f$, since $(X \to \coim f) = \coker(\ker f \to X)$.
\end{remark}

\begin{definition}
    Let
    \[
        X \xrightarrow{\ f \ } Y \xrightarrow{\ g\ } Z
    \]
    be a sequence of composable morphisms in an abelian category $\A$ satisfying $g \circ f = 0$ . We say it is \emph{exact}, if the canonical morphism $\im g \xrightarrow{\ \sim \ } \ker f$ is an isomorphism.
\end{definition}

\subsection{Triangulated categories}

In order to formulate the definition of a triangulated category more concisely, we introduce some preliminary notions. A \emph{category with translation} is a category $\D$ together with an auto-equivalence $T\colon \D \to \D$ called the \emph{translation functor}. If $\D$ is additive or $k$-linear, $T$ is moreover assumed to be additive or $k$-linear. We usually denote its action on objects $X$ with $X[1]$ and its action on morphisms $f$ with $f[1]$.

A \emph{triangle} in a category $\D$ with translation $T$ is a triplet of composable morphisms $(u, v, w)$ of category $\D$ having the form 
\[
    X\xrightarrow{\ u \ } Y \xrightarrow{\ v \ } Z \xrightarrow{\ w \ } X[1].
\]
A \emph{morphism} of triangles $X\xrightarrow{u} Y \xrightarrow{v} Z \xrightarrow{w} X[1]$ and $X'\xrightarrow{u'} Y' \xrightarrow{v'} Z' \xrightarrow{w'} X'[1]$ is given by a triple of morphisms $(\alpha, \beta, \gamma)$ for which the diagram below commutes.
\[\begin{tikzcd}
    % https://q.uiver.app/#q=WzAsOCxbMCwwLCJYIl0sWzEsMCwiWSJdLFsyLDAsIloiXSxbMywwLCJYWzFdIl0sWzAsMSwiWCciXSxbMSwxLCJZJyJdLFsyLDEsIlonIl0sWzMsMSwiWCdbMV0iXSxbMCwxXSxbMSwyXSxbMiwzXSxbNCw1XSxbNSw2XSxbNiw3XSxbMCw0LCJcXGFscGhhIiwyXSxbMSw1LCJcXGJldGEiLDJdLFsyLDYsIlxcZ2FtbWEiLDJdLFszLDcsIlxcYWxwaGFbMV0iLDJdXQ==
	X & Y & Z & {X[1]} \\
	{X'} & {Y'} & {Z'} & {X'[1]}
	\arrow[from=1-1, to=1-2]
	\arrow["\alpha"', from=1-1, to=2-1]
	\arrow[from=1-2, to=1-3]
	\arrow["\beta"', from=1-2, to=2-2]
	\arrow[from=1-3, to=1-4]
	\arrow["\gamma"', from=1-3, to=2-3]
	\arrow["{\alpha[1]}"', from=1-4, to=2-4]
	\arrow[from=2-1, to=2-2]
	\arrow[from=2-2, to=2-3]
	\arrow[from=2-3, to=2-4]
\end{tikzcd}
\]
One can compose morphisms of triangles in the obvious way and the notion of an isomorphism of triangles is defined as usual.

\begin{definition}
    A \emph{triangulated category} is an additive or $k$-linear category $\D$ with translation $T$ equipped with a class of \emph{distinguished triangles}, which is subject to the following axioms.
    \begin{enumerate}
        \item[TR1] \begin{enumerate}[(i)]
            \item Any triangle isomorphic to a distinguished triangle is also itself destinguished.
            \item For any $X$ the triangle
            \[
                X \xrightarrow{\id_X} X \longrightarrow 0 \longrightarrow X[1]
            \] 
            is distinguished.
            \item For any morphism $f \colon X \to Y$ there is a distinguished triangle of the form
            \[
                X \xrightarrow{\ f \ } Y \longrightarrow Z \longrightarrow X[1]
            \]
            \end{enumerate}
        \item[TR2]
        \item[TR3]
        \item[TR4]    
    \end{enumerate} 
\end{definition}

% \begin{definition}
%     A \emph{triangulated category} is an additive or $k$-linear category $\D$ together with two additional pieces of structure. 
%     \begin{enumerate}    
%         \item An additive or $k$-linear auto-equivalence
%         \[
%         T\colon \D \to \D,
%         \]
%         called a \emph{translation functor}. Usually we denote its action on an object $X$ of $\D$ with $X[1]$ and its action on a morphism $f$ with $f[1]$ instead of $T(X)$ and $T(f)$.
        
%         \item A collection of \emph{distinguished triangles}, which are triplets of composable morphisms $(u, v, w)$ of category $\D$ having the form 
%         \[
%             X\xrightarrow{u} Y \xrightarrow{v} Z \xrightarrow{w} X[1],
%             \]
%     \end{enumerate}
%     The above is constrained by the following axioms.

%     \begin{enumerate}
%         \item[\text{TR1}] 
%         \item[TR2] 
%     \end{enumerate}
% \end{definition}

\begin{remark}
    For a distinguished triangle
    \[
        X\xrightarrow{\ u \ } Y \xrightarrow{\ v \ } Z \xrightarrow{\ w \ } X[1]
    \]
    morphisms $u$ and $v$ are said to be of degree $0$ and morphism $w$ is said to be of degree $+1$. They are also sometimes diagramatically depicted as triangles, with the markings on morphisms describing their respective degrees.
    \[\begin{tikzcd}[column sep = small, row sep = 1.5em]
        % https://q.uiver.app/#q=WzAsMyxbMSwwLCJaIl0sWzAsMiwiWCJdLFsyLDIsIlkiXSxbMCwxLCIrMSIsMl0sWzEsMiwiMCJdLFsyLDAsIjAiLDJdXQ==
        & Z \\
        \\
        X && Y
        \arrow["{+1}"', from=1-2, to=3-1]
        \arrow["0", from=3-1, to=3-3]
        \arrow["0"', from=3-3, to=1-2]
    \end{tikzcd}\]
\end{remark}

\begin{definition}
    Let $\D$ and $\D'$ be triangulated categories with translation functors $T$ and $T'$ respectively. An additive functor $F \colon \D \to \D'$ is defined to be \emph{triangulated} or \emph{exact}, if the following two conditions are satisfied.
    \begin{enumerate}[(i)]
        \item There exists a natural isomorphism of functors 
        \[
            \eta \colon F \circ T \iso T' \circ F.
        \]
        \item For every distinguished triangle
        \[
            X\xrightarrow{u} Y \xrightarrow{v} Z \xrightarrow{w} X[1]
        \]
        in $\D$, the triangle
        \[
            F(X) \to F(Y) \to F(Z) \to F(X)[1] 
        \]
        is distinguished in $\D'$, where the last morphism (of degree $1$) is obtained as the composition $F(Z) \xrightarrow{Fw} F(X[1]) \xrightarrow{\eta_X} F(X)[1]$.
    \end{enumerate}
\end{definition}

\begin{definition}
    Let $H\colon \D \to \A$ be an additive functor between a triangulated category $\D$ and an abelian category $\A$. We say $H$ is a \emph{cohomological functor} if for every distinguished triangle $X \to Y \to Z \to X[1]$ in $\D$, the induced long sequence in $\A$
    \[
        \cdots \to H(X) \to H(Y) \to H(Z) \to H(X[1]) \to H(Y[1]) \to H(Z[1]) \to \cdots
    \]
    is exact.
\end{definition}

\begin{example}
    For any object $W$ in a triangulated category $\D$ the functors 
    \[
    \Hom_\D(W, -) \colon \D \to \Ab \text{ and } \Hom_\D( - , W) \colon \D^\op \to \Ab
    \]
    are cohomological.
\end{example}

\newpage
% \chapter{Derived categories}
\section{Derived categories}

\subsection{Categories of complexes}
In order to define derived categories of an additive category $\A$ we first introduce the category of complexes and the homotopic category of complexes of $\A$. We will equip these with a triangulated structure and . Throughout this section $\A$ will be a fixed additeve or $k$-linear category and we also mention that we will be using the cohomological indexing convention.

As a preliminary we introduce graded objects in ??

\subsubsection*{{Category of complexes}}

By a \emph{chain complex} in $\A$ we mean a collection of objects and morphisms 
\[
    A^\bullet = \left((A^i)_{i \in \Z}, (d^i_A\colon A^i \to A^{i+1})_{i \in \Z}\right),
\] 
where $A^i$ are objects and $d^i$ are morphisms of $\A$, called \emph{differenitials}, subject to equations $d^{i+1}\circ d^i = 0$, for all $i \in \Z$. A complex is \emph{bounded from below} (\resp \emph{bounded from above}), if there exists $i_0 \in \Z$ for which $A^i = 0$ for all $i \leq i_0$ (\resp $i \geq i_0$) and is \emph{bounded}, if it is both bounded from below and bounded from above. A \emph{chain map} between two chain complexes $A^\bullet$ and $B^\bullet$ in $\A$ is a collection of morphisms in $\A$
\[
    f^\bullet = (f^i\colon A^i \to B^i)_{i \in \Z},
\]
for which $f^{i+1}\circ d^i_A = d^i_B \circ f^i$ holds for all $i \in \Z$. This may diagramatically be described by the following commutative ladder.

\[
\begin{tikzcd}
    % https://q.uiver.app/#q=WzAsMTAsWzEsMCwiQV57aS0xfSJdLFszLDAsIkFeaSJdLFs1LDAsIkFee2krMX0iXSxbMSwyLCJCXntpLTF9Il0sWzMsMiwiQl5pIl0sWzUsMiwiQl57aSsxfSJdLFs2LDIsIlxcY2RvdHMiXSxbNiwwLCJcXGNkb3RzIl0sWzAsMCwiXFxjZG90cyJdLFswLDIsIlxcY2RvdHMiXSxbMCwzLCJmXntpLTF9Il0sWzEsNCwiZl5pIl0sWzIsNSwiZl57aSsxfSJdLFswLDEsImRee2ktMX1fQSJdLFsxLDIsImReaV9BIl0sWzQsNSwiZF57aX1fQiJdLFszLDQsImRee2ktMX1fQiJdLFs1LDZdLFsyLDddLFs4LDBdLFs5LDNdXQ==
	\cdots & {A^{i-1}} && {A^i} && {A^{i+1}} & \cdots \\
	\\
	\cdots & {B^{i-1}} && {B^i} && {B^{i+1}} & \cdots
	\arrow[from=1-1, to=1-2]
	\arrow["{d^{i-1}_A}", from=1-2, to=1-4]
	\arrow["{f^{i-1}}", from=1-2, to=3-2]
	\arrow["{d^i_A}", from=1-4, to=1-6]
	\arrow["{f^i}", from=1-4, to=3-4]
	\arrow[from=1-6, to=1-7]
	\arrow["{f^{i+1}}", from=1-6, to=3-6]
	\arrow[from=3-1, to=3-2]
	\arrow["{d^{i-1}_B}", from=3-2, to=3-4]
	\arrow["{d^{i}_B}", from=3-4, to=3-6]
	\arrow[from=3-6, to=3-7]
\end{tikzcd}
\]

We then define the \emph{category of chain complexes} in $\A$, denoted by $\com(\A)$, to be the following additive category. 
\begin{center}
    \begin{tabular}{r l}
        \textsl{Objects:} & chain complexes in $\A$. \\
        \textsl{Morphims:} & $\Hom_{\com(\A)}(A, B)$ is the set of chain maps $A \to B$, \\ & equipped with a group structure inherited from $\A$ \\ & by applying operations componentwise.
    \end{tabular}
\end{center}
The composition law is defined componentwise and is clearly associative and bilinear, and the identity morphisms $1_{A^\bullet}$ are defined to be $(1_{A^i})_{i \in \Z}$. The complex $\cdots \to 0 \to 0 \to \cdots$ plays the role of the zero object in $\com(\A)$ and the biproduct of complexes $A$ and $B$ exists and is witnessed by the chain complex
\[
    A \oplus B = \left( (A^i \oplus B^i)_{i \in \Z}, (d^i_A \oplus d^i_B)_{i \in \Z}\right),
    % \quad \quad \cdots \to A^i \oplus B^i \xrightarrow{d^i_A \oplus d^i_B} A^{i+1} \oplus B^{i+1} \to \cdots,
\]
together with the cannonical projection and injection morphisms arising from biproducts componentwise.

Additionally, we also define the following full additive subcategories of $\com(\A)$.
\begin{center}
    \begin{tabular}{c l}
        $\com^+(\A)$ & \emph{Category of complexes bounded below}, spanned on \\ &complexes in $\A$ bounded below. \\
        $\com^-(\A)$ & \emph{Category of complexes bounded above}, spanned on \\ &complexes in $\A$ bounded above. \\
        $\com^b(\A)$ & \emph{Category of bounded complexes}, spanned on \\ &bounded complexes in $\A$. \\    
    \end{tabular}
\end{center}
\begin{remark}
    Whenever $\A$ is $k$-linear, all the categories of complexes $\com^*(\A)$ become $k$-linear as well in the obvious way.
\end{remark}

On all the categories of complexes mentioned above, we can now define the translation functor
\[
    T\colon \com^*(\A) \to \com^*(\A)
\]  
given by its action on objects and morphisms as follows.

\begin{center}
    \begin{tabular}{r l}
        \textsl{Objects:} & $T(A^\bullet) = A^\bullet[1]$ is the chain complex with $(A^\bullet[1])^i := A^{i+1}$ \\ & and differentials $d^i_{A[1]} = - d^{i+1}_{A}$. \\
        \textsl{Morphims:} & For a chain map $f^\bullet\colon A^\bullet \to B^\bullet$ we define $f^\bullet[1]$ to have \\ & component maps $(f^\bullet[1])^i = f^{i+1}$.
    \end{tabular}
\end{center}
The translation functor $T$ thus acts on a complex $A^\bullet$ by twisting its differential by a sign and shifting it one step to the \emph{left}, which is graphically pictured below.

\[\begin{tikzcd}[row sep = 1.2em]
    % https://q.uiver.app/#q=WzAsMTcsWzAsMSwiQV5cXGJ1bGxldFxcY29sb24iXSxbMCwyLCJBXlxcYnVsbGV0WzFdXFxjb2xvbiJdLFsxLDAsIlxcY2RvdHMiXSxbMywxLCJBXjAiXSxbNCwxLCJBXjEiXSxbMiwxLCJBXnstMX0iXSxbMiwyLCJBXjAiXSxbMywyLCJBXjEiXSxbNCwyLCJBXjIiXSxbNSwxLCJcXGNkb3RzIl0sWzUsMiwiXFxjZG90cyJdLFsxLDEsIlxcY2RvdHMiXSxbMSwyLCJcXGNkb3RzIl0sWzMsMCwiXFx0ZXh0dHR7MH0iXSxbNCwwLCJcXHRleHR0dHsxfSJdLFsyLDAsIlxcdGV4dHR0ey0xfSJdLFs1LDAsIlxcY2RvdHMiXSxbNiw3XSxbNyw4XSxbNSwzXSxbMyw0XSxbNCw5XSxbMTEsNV0sWzEyLDZdXQ==
	& \color{linkcolor}\cdots & {\color{linkcolor}{\mathtt{-1}}} & {\color{linkcolor}\small{\mathtt{0}}} & {\color{linkcolor}\small{\mathtt{1}}} & {\color{linkcolor}\small{\mathtt{2}}} & \color{linkcolor}\cdots \\
	{A^\bullet} & \cdots & {A^{-1}} & {A^0} & {A^1} & {A^2} & \cdots \\
	{A^\bullet[1]} & \cdots & {A^0} & {A^1} & {A^2} & {A^3} & \cdots
	\arrow[from=2-2, to=2-3]
	\arrow[from=2-3, to=2-4]
	\arrow[from=2-4, to=2-5]
	\arrow[from=2-5, to=2-6]
	\arrow[from=2-6, to=2-7]
	\arrow[from=3-2, to=3-3]
	\arrow[from=3-3, to=3-4]
	\arrow[from=3-4, to=3-5]
    \arrow[from=3-5, to=3-6]
    \arrow[from=3-6, to=3-7]
\end{tikzcd}\]

\begin{remark}
    We remark that the translation functor $T$ is clearly also additive or $k$-linear, whenever $\A$ is additive or $k$-linear.
\end{remark}

Since $T$ is an auto-equivalence there exists a quasi-inverse $T\inv$ to $T$, which is defined and unique up to a natural isomorphism. We may then speak of $T^{k}$ for any $k \in \Z$, whose action on a complex $A^\bullet$ is described by $(A^\bullet[k])^i = A^{i + k}$ with differential $d^{i}_{A[k]} = (-1)^k d^{i+k}_A$.

\subsubsection*{Homotopy category of complexes}

In this subsection we construct the homotopy category of chain complexes associated to a given additive category $\A$ and equip it with a triangulated structure. The main motivation for its introduction in this work is the fact that we will later on use it to construct the derived category of $\A$. In particular the homotopy category of $\A$, as opposed to the category of complexes\footnote{It is still possible to construct the derived category of $\A$ without passing through the homotopy category of complexes, however equipping it with a triangulated structure in this case becomes less elegant.} $\com(\A)$, can be enhanced with a triangulated structure which will afterwards descend to the level of derived categories.

\begin{definition}
    Let $f^\bullet$ and $g^\bullet$ be two chain maps in $\Hom_{\com(\A)}(A^\bullet, B^\bullet)$. We define $f^\bullet$ and $g^\bullet$ to be \emph{homotopic}, if there exists a collection of morphisms $\left( h^i\colon A^i \to B^{i-1} \right)_{i \in \Z}$, satisfying
    \[
        f^i - g^i = h^{i+1} \circ d^i_A + d^{i-1}_B \circ h^i
    \] 
    for all $i \in \Z$ and denote it by $f^\bullet \htpy g^\bullet$. We say $f^\bullet$ is \emph{nullhomotopic}, if $f^\bullet \htpy 0$.
    \[\begin{tikzcd}
        % https://q.uiver.app/#q=WzAsMTAsWzEsMCwiQV57aS0xfSJdLFszLDAsIkFeaSJdLFs1LDAsIkFee2krMX0iXSxbMSwyLCJCXntpLTF9Il0sWzMsMiwiQl5pIl0sWzUsMiwiQl57aSsxfSJdLFs2LDIsIlxcY2RvdHMiXSxbNiwwLCJcXGNkb3RzIl0sWzAsMCwiXFxjZG90cyJdLFswLDIsIlxcY2RvdHMiXSxbMCwzLCJmXntpLTF9Il0sWzEsNCwiZl5pIl0sWzIsNSwiZl57aSsxfSJdLFswLDEsImRee2ktMX1fQSJdLFsxLDIsImReaV9BIl0sWzQsNSwiZF57aX1fQiJdLFszLDQsImRee2ktMX1fQiJdLFs1LDZdLFsyLDddLFs4LDBdLFs5LDNdLFsxLDMsImheaSJdLFsyLDQsImhee2krMX0iXV0=
        \cdots & {A^{i-1}} && {A^i} && {A^{i+1}} & \cdots \\
        \\
        \cdots & {B^{i-1}} && {B^i} && {B^{i+1}} & \cdots
        \arrow[from=1-1, to=1-2]
        \arrow["{d^{i-1}_A}", from=1-2, to=1-4]
        \arrow["{f^{i-1}}"', from=1-2, to=3-2]
        \arrow["{d^i_A}", from=1-4, to=1-6]
        \arrow["{h^i}"', from=1-4, to=3-2]
        \arrow["{f^i}"', from=1-4, to=3-4]
        \arrow[from=1-6, to=1-7]
        \arrow["{h^{i+1}}"', from=1-6, to=3-4]
        \arrow["{f^{i+1}}"', from=1-6, to=3-6]
        \arrow[from=3-1, to=3-2]
        \arrow["{d^{i-1}_B}", from=3-2, to=3-4]
        \arrow["{d^{i}_B}", from=3-4, to=3-6]
        \arrow[from=3-6, to=3-7]
        \end{tikzcd}\] 
\end{definition}

\begin{lemma}
    \label{nullhomotopics form a subgroup}
    Let $A^\bullet$, $B^\bullet$ and $C^\bullet$ be complexes in $\com(\A)$, and let $f, f' \in \Hom(A^\bullet, B^\bullet)$ and $g, g' \in \Hom(B^\bullet, C^\bullet)$ be chain maps.
    \begin{enumerate}[(i)]
        \item The set of all nullhomotopic chain maps $A^\bullet \to B^\bullet$ forms a subgroup of $\Hom(A^\bullet, B^\bullet)$.
        \item If $f \htpy f'$ and $g \htpy g'$, then $g \circ f \htpy g' \circ f'$.
    \end{enumerate}
\end{lemma}

The homotopy category of complexes in $\A$, denoted by $K(\A)$, is defined to be the additive category consisting of
\begin{center}
    \begin{tabular}{r l}
        \textsl{Objects:} & chain complexes in $\A$. \\
        \textsl{Morphims:} & $\Hom_{K(\A)}(X, Y) := \Hom_{\com(\A)}(X, Y)/_\htpy$
    \end{tabular}
\end{center}
The composition law descends to the quotient by lemma \ref{nullhomotopics form a subgroup} (ii), \ie $[g]\circ[f] := [g \circ f]$, for composable $[f]$ and $[g]$, and for any $A^\bullet$ the identity morphism is defined to be $[1_A^\bullet]$. All the hom-sets $\Hom_{K(\A)}(X, Y)$ are abelian groups by lemma \ref{nullhomotopics form a subgroup} (i) and compositions are bilinear maps. 

As in the case of categories of complexes in $\A$, we can also define the following full additive subcategories of $K(\A)$.
\begin{center}
    \begin{tabular}{c l}
        $K^+(\A)$ & \emph{Homotopy category of complexes bounded below}, spanned on \\ &complexes in $\A$ bounded below. \\
        $K^-(\A)$ & \emph{Homotopy category of complexes bounded above}, spanned on \\ &complexes in $\A$ bounded above. \\
        $K^b(\A)$ & \emph{Homotopy category of bounded complexes}, spanned on \\ &bounded complexes in $\A$. \\    
    \end{tabular}
\end{center}

We now shift our focus to the construction of a triangulated structure on $K^*(\A)$. The translation functor $T\colon K^*(\A) \to K^*(\A)$ is defined on objects and morphisms in the following way.
\begin{center}
    \begin{tabular}{r l}
        \textsl{Objects:} & $A^\bullet \longmapsto A^\bullet[1]$. \\
        \textsl{Morphims:} & $[f^\bullet] \longmapsto [f^\bullet[1]]$.
    \end{tabular}
\end{center}


We define any triangle in $K(\A)$ isomorphic to a triangle of the form
\[
    A \xrightarrow{ \ f \ }  B \xrightarrow{\ \pi \ } C(f) \xrightarrow{\ \tau \ } A[1]
\]
to be distinguished.

\begin{proposition}
    The homotopy category $K(\A)$ together with the translation functor $T\colon K(\A) \to K(\A)$ and distinguished triangles defined above is a triangulated category.
\end{proposition}

\begin{remark}
    The proposition still holds true, if we replace $K(\A)$ with any of the categories $K^+(\A)$, $K^-(\A)$, $K^b(\A)$.
\end{remark}


\newpage
\begin{lemma}
    Let $f^\bullet \colon A^\bullet \to B^\bullet$ be a quasi-isomorphism and let $g^\bullet\colon C^\bullet \to B^\bullet$ be a morphism. Then there exist a quasi-isomorphism $u^\bullet\colon C_0^\bullet \to C^\bullet$ and a morphism $v^\bullet\colon C_0^\bullet \to A^\bullet$, such that $f \circ u = g \circ v$ \ie the diagram below commutes.
    \[
    \begin{tikzcd}
        {C_0^\bullet} && C^\bullet \\
        \\
        A^\bullet && B^\bullet
        \arrow["u^\bullet", dashed, from=1-1, to=1-3]
        \arrow["\sim"', draw=none, from=1-1, to=1-3]
        \arrow["v^\bullet"', dashed, from=1-1, to=3-1]
        \arrow["g^\bullet", from=1-3, to=3-3]
        \arrow["f^\bullet", from=3-1, to=3-3]
        \arrow["\sim"', draw=none, from=3-1, to=3-3]
    \end{tikzcd}
    \]
\end{lemma}

The above also holds in categories $K^+(\A)$, $K^-(\A)$, $K^b(\A)$.

\begin{proposition}
    \label{injective resolution}
    Suppose $\A$ contains enough injectives. Then every $A^\bullet$ in $K^+(\A)$ has an injective resolution.
\end{proposition}

\begin{proof}
    
\end{proof}

\begin{remark}
    As is evident from the proof by close inspection we have only used two facts about the class of all injective objects of $\A$ -- that every object of $\A$ embeds into some injective object and that the class of injectives is closed under finite direct sums. This will later on be used in subsection \ref{Subsection on F-adapted classes} when constructing a right derived functor of a given functor $F$ in the presence of an $F$-adapted class.
\end{remark}

\begin{lemma}
    Let $A^\bullet$ be any acyclic complex in $K^+(\A)$ and $I^\bullet$ a complex of injectives from $K^+(J)$. Then
    \[
        \hom_{K^+(\A)}(A^\bullet, I^\bullet) = 0.
    \]
    In other words every morphism from an acyclic complex to an injective one is nulhomotopic.
\end{lemma}

\begin{proof}
    
\end{proof}

\begin{theorem}
    Assume $\A$ contains enough injectives and let $\J \subseteq \A$ denote the full subcategory on injecitve objects of $\A$. Then the inclusion $K^+(\J) \hookrightarrow K^+(\A)$ induces an equivalence of categories
    \[
        K^+(\J) \iso D^+(\A).
    \]
\end{theorem}

\subsection{Derived functors}

\subsubsection{$F$-adapted classes}
\label{Subsection on F-adapted classes}

Unfortunately our categories at hand will sometimes not contain enough injectives, as can already be seen with the category of coherent sheaves $\coherent{X}$ on a scheme $X$, which is not a point. In this case our previously defined method of constructing right derived functors will not work. Luckly however, there exists a method of obtaining a right derived functor $\rderived{}{F} : D^+(\A) \to D^+(\B)$, given a left exact functor $F: \A \to \B$, even when $\A$ does not contain enough injectives. To this end we first introduce $F$-adapted classes.

\begin{definition}
    \label{F-adapted}
    A class of objects $\I \subset \A$ is \emph{adapted} to a left exact functor $F : \A \to \B$, if the following three conditions are satisfied.
    \begin{enumerate}[(i)]
        \item $\I$ is stable under finite sums. 
        \item Every object $A$ of $\A$ embeds into some object of $\I$, \ie there exists an object $I \in \I$ and a monomorphism $A \hookrightarrow I$.
        \item For every acyclic complex $I^\bullet$ in $K^+(\A)$, with $I^i \in \I$ for all $i \in \Z$, its image $F(I^\bullet)$ under $F$ is also acyclic.
    \end{enumerate}
\end{definition}

\begin{remark}
    We imediately observe that whenever $\A$ contains enough injectives, the class of all injective objects forms an $F$-adapted class for \emph{every} functor $F\colon \A \to \B$.
\end{remark}

In the presence of an $F$-adapted class $\I$ we will now construct the right derived functor of $F$. Firstly one can upgrade the class of objects $\I$ to a full additive subcategory $\J$ of $\A$ having its class of objects be preciesly $\I$. This is done by declaring $\hom_\J(X, Y) := \hom_\A(X, Y)$ for all objects $X$, $Y$ of the class $\I$. Then we can define a functor $K^+(F) \colon K^+(\J) \to K^+(\B)$ between triangulated categories, which acts on objects of $K^+(\J)$ as 
\[
    \left(\cdots \to A^i \xrightarrow{d^i} A^{i+1} \to \cdots\right) \longmapsto \left(\cdots \to F(A^i) \xrightarrow{F(d^i)} F(A^{i+1}) \to \cdots\right)
\]
and on morphisms as
\[
    f^\bullet = (f^i \colon A^i \to B^i)_{i \in \Z} \longmapsto (F(f^i))_{i \in \Z}.
\]
It is then clear that $K^+(F)$ is exact because of condition (ii) of definition \ref{F-adapted} and it descends to a well defined functor on the level of derived categories 
\[
    D^+(\J) \to D^+(\B).
\]
Since we want to define the derived functor $\rderived{}{F}$, whose domain is $D^+(\A)$, it remains to construct a functor $D^+(\A) \to D^+(\J)$, which we can then post-compose with $K^+(F)$ to obtain $\rderived{}{F}$. What we will show instead is that the inclusion of categories $\J \hookrightarrow \A$ induces an exact equivalence of triangulated categories $D^+(\J)$ and $D^+(\A)$.

It remains to be shown that the inclusion of categories $\J \hookrightarrow \A$ induces an the equivalence of categories $D^+(\J)$ and $D^+(\A)$. Firstly we remark that this inclusion clearly descends to a well-defined functor on the level of derived categories

This hinges on the following lemma.

\begin{lemma}
    \label{F-acyclic resolutions}
    For every complex $A^\bullet$ in $K^+(\A)$ there is a quasi-isomorphism $A^\bullet \to I^\bullet$ where $I^\bullet$ is a complex in $K^+(\J)$.
\end{lemma}

\begin{proof}
    By inspecting the proof of proposition \ref{injective resolution} we see that only conditions (i) and (ii) of definition \ref{F-adapted} were actually used, so the proof of this lemma follows mutatis mutandis from said proposition.
\end{proof}



% it is clear that the restriction of the functor $F$ to $\J$ descends to the right derived functor

% localization of K^+(\I) by qis equivalent to D^+(\A)

\begin{proposition}
    \label{F-acyclic is F-adapted}
    Let $F : \A \to \B$ be a left exact functor between two abelin categories. Let $\I \subset \A$ be an $F$-adapted class of objects in $\A$. Then the class of all $F$-acyclic objects of $\A$, denoted by $\I_F$, is also $F$-adapted. 
\end{proposition}

\begin{lemma}
    \label{F-adapted is F-acyclic}
    Every object of an $F$-adapted class is also $F$-acyclic.
\end{lemma}

\begin{proof}
    We only have to recall the definition 
\end{proof}

\begin{proof}[Proof of proposition \ref{F-acyclic is F-adapted}]
    We verify conditions (i)--(iii) of definition \ref{F-adapted}. (i) As the higher derived functors $R^iF$ are clearly additive, $\I_F$ is stable under finite sums. (ii) Holds by lemma \ref{F-adapted is F-acyclic}. (iii) Suppose $A^\bullet$ is an acyclic complex in $K^+(\A)$ with $A^i \in \I_F$ for all $i$. Then using acyclicity of $A^\bullet$, we may break this complex up into a series of short exact sequences as follows.
    \begin{align*}
        0 \to A^0 \to & A^1 \to \ker d^2 \to 0 \\
        0 \to \ker d^2 \to & A^2 \to \ker d^3 \to 0 \\
        & \vdots
    \end{align*} 
    The first short exact sequence yields the following long exact sequence associated to $F$.
    \begin{multline*}
        0 \to F(A^0) \to F(A^1) \to F(\ker d^2) \to \rderived{1}{F}(A^0) \to \cdots \\ \cdots \to \rderived{i}{F}(A^0) \to \rderived{i}{F}(A^1) \to \rderived{i}{F}(\ker d^2) \to \rderived{i+1}{F}(A^0) \to \cdots.
    \end{multline*}
    Exacness of the sequence together with $F$-acyclicity of $A^0$ and $A^1$ shows that $\ker d^2$ is also $F$-acyclic and by only considering the begining few terms of the sequence we obtain the  short exact sequence 
    % From the fact that $A^0$ and $A^1$ are $F$-acyclic we see that $\ker d^2$ is $F$-acyclic and we obtain a short exact sequence 
    \[
        0 \to F(A^0) \to F(A^1) \to F(\ker d^2) \to 0.
    \]
    After inductively applying the same reasoning for each of the original short exact sequences of \eqref{}, we are left with the following set of short exact sequences\footnote{To be more precise we have used that $F$ is left exact in order to swap the order of $F$ and $\ker$ to reach \[\ker Fd^i \simeq F(\ker d^i)\].}.
    \begin{align*}
        0 \to F(A^0) \to & F(A^1) \to \ker Fd^2 \to 0 \\
        0 \to \ker Fd^2 \to & F(A^2) \to \ker Fd^3 \to 0 \\
        & \vdots
    \end{align*}   
    Collecting them all together we can conclude that the complex $F(A^\bullet)$ is acyclic.\qedhere
\end{proof}
\newpage
\newpage

\section{Derived categories}
\label{Chapter: Derived categories}

The first goal of this section is to construct the derived category of an abelian category $\A$ over $k$ and equip it with a triangulated structure. Our arguments, although specialized to the homotopy category $\kcat{\A}$ and the class of quasi-isomorphisms, do not differ tremendously from the general theory of localization of categories. 
% Our arguments will try to be minimal and will not rely on the already established theory of localization of categories. Our approach might at times seem less elegant, but we include the exposition for completness nonetheless.
A more comprehensive and formal treatment of this topic is laid out in Chapters 7, 10 and 13 of \cite{kashiwara2006categories} or the meticulous \cite{milicic-dercat}.
The second part covers the construction of derived functors. We see how the established framework of derived categories nicely lends itself for the definition of derived functors and we also relate them back to the classical higher derived functors. To close the chapter, we prove a correspondence between the Hom-sets of $\derplus{\A}$ and certain $\Ext$-modules.
% more general framework realtes back to the classical higher derived functors. 

Throughout this chapter $\A$ will denote an abelian category over $k$ and $k$ will be assumed to be either a field or the ring of integers $\Z$.

\subsection{Derived categories of abelian categories}

In algebraic geometry cohomology of a geometric object, like a scheme or a variety, $X$ with respect to some coherent sheaf $\F$ plays a very important role. One way of computing $H^i(X, \F)$, which we shall also feature in Section \ref{Subsection: Derived functors in geometry}, involves the following. Instead of directly 
% \info{intrinsically might not be the best word as resolutions in certain cases do represent a sheaf intrinsically in the sense of generators and relations... (Hilbert syzygy thm)}
studying the sheaf $\F$, we represent it with a so-called \emph{resolution}, which consists of a complex of sheaves $\E^\bullet$, built up from sheaves $\E^i$, for $i \in \Z$, belonging to some class of sheaves, which is well behaved under cohomology, and a quasi-isomorphism of the form $\E^\bullet \to \F$ or $\F \to \E^\bullet$. 
% After observing that using two different resolutions of $\F$ 
After noting that the sheaf $\F$ can be seen as a complex concentrated in degree $0$ and observing that replacing a resolution of $\F$ with another one
results in computing isomorphic cohomology groups, we are motivated  
not to distinguish the sheaf $\F$ from its resolutions in the ambient (homotopy) category of complexes any longer. 
% identify the sheaf $\F$ with the collection of all its resolutions. 
Taking a step back, we would like to modify the homotopy category $\kcat{\coherent{X}}$ in such a way that $\F$ is identified with all its resolutions, or in other words, we want all the quasi-isomorphisms of $\kcat{\coherent{X}}$ to turn into isomorphisms in a ``universal way''. We will take the latter to be our inspiration for the definition of the derived category $\dercat{\A}$ of a general abelian category $\A$. This is stated more formally in the form of the ensuing universal property.

% assigning the derived category $\dercat{\A}$ to a general abelian category $\A$. 

% is to consider a certain complex built from objects all originating from some special class, in a way representing $\F$, called its \emph{resolution}. In a way we would like to identify $\F$ with its resolution and the correct relation, which acheves this, is via quasi-isomorphisms. Inherently, computing the cohomology from said complex losses some information 

% One of the most important invariants of a geometric object $X$ in algebraic geometry is its cohomology $H^i(X, \F)$ with respect to some (coherent) sheaf $\F$. 


% The central idea in this part of algebraic geometry is 

% Our goal in general is then fairly simple, modify the homotopy category $\kcat{\A}$ in such a way that all the quasi-isomorphisms become isomorphisms in a ``universal way''. This is stated more formally in the form of the ensuing universal property.

% \begin{definition}
%     \label{universal property of D(A)}
%     Let $\A$ be an abelian category and $\kcat{\A}$ its homotopy category. A triangulated category $\dercat{\A}$ together with a triangulated functor $Q\colon \kcat{\A} \to \dercat{\A}$ is the \emph{derived category of $\A$}, if it satisfies:
%     \info{maybe I replace triangulated in this def. w/ $k$-linear and then add to proposion 2.8, that when $Q$ and $F$ are assumed to be triangulated $F_0$ is as well. This is so that I can use it directly to construct homology functors on $\dercat{\A}$}
%     \begin{enumerate}[label = (\roman*)]
%         \item For every quasi-isomorphism $s$ in $\kcat{\A}$, $Q(s)$ is an isomorphism in $\dercat{\A}$.
%         \item For any triangulated category $\D$ and any functor $F\colon \kcat{\A} \to \D$, sending quasi-isomorphisms $s$ in $\kcat{\A}$ to isomorphisms $F(s)$ in $\D$,
%         % for which $F(s)$ is an isomorphism in $\D$ for all quasi-isomorphisms $s$ in $\kcat{\A}$, 
%         there exists a triangulated functor $F_0 \colon \dercat{\A} \to \D$, which is unique up to a unique natural isomorphism, such that $F \natiso F_0 \circ Q$.
%         % , and moreover $F_0$ is required to be unique up to a unique natural isomorphism.
%         In other words, the diagram below commutes up to natural isomorphism. 
%         \[\begin{tikzcd}
%             % https://q.uiver.app/#q=WzAsMyxbMCwwLCJLKFxcQSkiXSxbMCwxLCJEKFxcQSkiXSxbMSwwLCJcXEQiXSxbMCwyLCJGIl0sWzAsMSwiUSIsMl0sWzEsMiwiRl9cXEEiLDIseyJzdHlsZSI6eyJib2R5Ijp7Im5hbWUiOiJkYXNoZWQifX19XV0=
%             {\kcat{\A}} & \D \\
%             {\dercat{\A}}
%             \arrow["F", from=1-1, to=1-2]
%             \arrow["Q"', from=1-1, to=2-1]
%             \arrow["{F_0}"', dashed, from=2-1, to=1-2]
%         \end{tikzcd}\]
%     \end{enumerate}
% \end{definition}

\begin{definition}
    \label{universal property of D(A)}
    Let $\A$ be an abelian category and $\kcat{\A}$ its homotopy category. A category $\dercat{\A}$ together with a functor $Q\colon \kcat{\A} \to \dercat{\A}$ is called the \emph{derived category of $\A$} if it satisfies:
    \begin{enumerate}[label = (\roman*)]
        \item For every quasi-isomorphism $s$ in $\kcat{\A}$, $Q(s)$ is an isomorphism in $\dercat{\A}$.
        \item For any category $\D$ and any functor $F\colon \kcat{\A} \to \D$, sending quasi-isomorphisms $s$ in $\kcat{\A}$ to isomorphisms $F(s)$ in $\D$,
        % for which $F(s)$ is an isomorphism in $\D$ for all quasi-isomorphisms $s$ in $\kcat{\A}$, 
        there exists a functor $F_0 \colon \dercat{\A} \to \D$, which is unique up to a unique natural isomorphism, such that $F \natiso F_0 \circ Q$.
        % , and moreover $F_0$ is required to be unique up to a unique natural isomorphism.
        In other words, the diagram below commutes up to natural isomorphism. 
        \[\begin{tikzcd}
            % https://q.uiver.app/#q=WzAsMyxbMCwwLCJLKFxcQSkiXSxbMCwxLCJEKFxcQSkiXSxbMSwwLCJcXEQiXSxbMCwyLCJGIl0sWzAsMSwiUSIsMl0sWzEsMiwiRl9cXEEiLDIseyJzdHlsZSI6eyJib2R5Ijp7Im5hbWUiOiJkYXNoZWQifX19XV0=
            {\kcat{\A}} & \D \\
            {\dercat{\A}}
            \arrow["F", from=1-1, to=1-2]
            \arrow["Q"', from=1-1, to=2-1]
            \arrow["{F_0}"', dashed, from=2-1, to=1-2]
        \end{tikzcd}\]
    \end{enumerate}
\end{definition}

\begin{remark}
    We recognize this definition as a special case of localization of categories \cite[\S 7, Definition 7.1.1.]{kashiwara2006categories} or \cite[\S III.2, Definition 1]{gelfand2002methods}. In particular it defines $\dercat{\A}$ to be the \emph{localization of a triangulated category} $\kcat{\A}$ by the family of all quasi-isomorphisms in $\kcat{\A}$.
\end{remark}

A naive way of constructing $\dercat{\A}$ out of $\kcat{\A}$ would be to artificially add the inverses to all the quasi-isomorphisms in $\kcat{\A}$ and then impose the correct collection of relations on the newly constructed class of morphisms. As this can quickly lead us to some set theoretic problems, we will construct a specific model, which achieves this, instead. Our construction is a priori not going to result in a locally small\footnote{A category $\mathcal C$ is called \emph{locally small} if for all objects $X$ and $Y$ of $\mathcal C$ the Hom-sets $\Hom_\mathcal{C}(X, Y)$ are actual sets.} category, but as we shall soon see in practice all the categories we will encounter are going to be locally small.
% be concerned with will be locally small. 
% still going to reach outside the scope of sets, but as we will se in practise 
% this can quickly become very tricky, we will construct a specific model instead.
% This will be achieved by constructing a specific model. 
% Instead we will construct a specific model, which achieves this. 

\subsubsection{Construction}

To start, we first need a technical lemma resembling the Ore condition from non-commuta\-tive algebra.

\begin{lemma}
    \label{Ore condition}
    Let $f \colon A^\bullet \to B^\bullet$ and $s\colon C^\bullet \to B^\bullet$ belong to the homotopy category $\kcat{\A}$, with $s$ being a quasi-isomorphism. Then there exists a quasi-isomorphism $u\colon C_0^\bullet \to A^\bullet$ and a morphism $g\colon C_0^\bullet \to C^\bullet$, such that the diagram below commutes in $\kcat{\A}$.
    \[\begin{tikzcd}[column sep = 2em]
        % https://q.uiver.app/#q=WzAsNCxbMCwyLCJBXlxcYnVsbGV0Il0sWzIsMiwiQl5cXGJ1bGxldCJdLFsyLDAsIkNeXFxidWxsZXQiXSxbMCwwLCJDXlxcYnVsbGV0XzAiXSxbMCwxLCJmIl0sWzIsMSwicyJdLFszLDIsImciLDAseyJzdHlsZSI6eyJib2R5Ijp7Im5hbWUiOiJkYXNoZWQifX19XSxbMywwLCJ1IiwyLHsic3R5bGUiOnsiYm9keSI6eyJuYW1lIjoiZGFzaGVkIn19fV0sWzMsMCwiXFxzaW0iLDAseyJzdHlsZSI6eyJib2R5Ijp7Im5hbWUiOiJub25lIn0sImhlYWQiOnsibmFtZSI6Im5vbmUifX19XSxbMiwxLCJcXHNpbSIsMix7InN0eWxlIjp7ImJvZHkiOnsibmFtZSI6Im5vbmUifSwiaGVhZCI6eyJuYW1lIjoibm9uZSJ9fX1dXQ==
        {C^\bullet_0} && {C^\bullet} \\
        \\
        {A^\bullet} && {B^\bullet}
        \arrow["g", dashed, from=1-1, to=1-3]
        \arrow["u", dashed, from=1-1, to=3-1]
        \arrow["\sim"', draw=none, from=1-1, to=3-1]
        \arrow["s", from=1-3, to=3-3]
        \arrow["\sim"', draw=none, from=1-3, to=3-3]
        \arrow["f", from=3-1, to=3-3]
    \end{tikzcd}\]
\end{lemma}

\begin{proof}
    We are now in a \emph{triangulated} category $\kcat{\A}$ and first extend $s \colon C^\bullet \to B^\bullet$ into a distinguished triangle
    \[
        C^\bullet \xrightarrow{\ s \ } B^\bullet \xrightarrow{\ \tau \ } C(s)^\bullet \xrightarrow{\ \pi \ } C[1]^\bullet.
    \]
    The composition $\tau f \colon A^\bullet \to C(s)^\bullet$ then also extends to a distinguished triangle, for which we have the following commutative diagram with solid arrows
    \[\begin{tikzcd}[column sep = normal, row sep = 2.5em]
        % https://q.uiver.app/#q=WzAsOCxbMSwwLCJBXlxcYnVsbGV0Il0sWzIsMCwiQyhzKV5cXGJ1bGxldCJdLFsxLDEsIkJeXFxidWxsZXQiXSxbMiwxLCJDKHMpXlxcYnVsbGV0Il0sWzAsMSwiQ15cXGJ1bGxldCJdLFszLDEsIkNbMV1eXFxidWxsZXQiXSxbMywwLCJDKFxcdGF1IGYpXlxcYnVsbGV0Il0sWzAsMCwiQyhcXHRhdSBmKVstMV1eXFxidWxsZXQiXSxbMCwxLCJcXHRhdSBmIl0sWzIsMywiXFx0YXUiXSxbNCwyLCJmIl0sWzMsNSwiXFxwaSJdLFsxLDMsIiIsMCx7ImxldmVsIjoyLCJzdHlsZSI6eyJoZWFkIjp7Im5hbWUiOiJub25lIn19fV0sWzAsMiwicyJdLFsxLDZdLFs2LDUsIiIsMSx7InN0eWxlIjp7ImJvZHkiOnsibmFtZSI6ImRhc2hlZCJ9fX1dLFs3LDQsIiIsMCx7InN0eWxlIjp7ImJvZHkiOnsibmFtZSI6ImRhc2hlZCJ9fX1dLFs3LDBdXQ==
        {C(\tau f)[-1]^\bullet} & {A^\bullet} & {C(s)^\bullet} & {C(\tau f)^\bullet} \\
        {C^\bullet} & {B^\bullet} & {C(s)^\bullet} & {C[1]^\bullet}
        \arrow["u", from=1-1, to=1-2]
        \arrow[dashed, from=1-1, to=2-1]
        \arrow["{\tau f}", from=1-2, to=1-3]
        \arrow["f", from=1-2, to=2-2]
        \arrow[from=1-3, to=1-4]
        \arrow[equals, from=1-3, to=2-3]
        \arrow[dashed, from=1-4, to=2-4]
        \arrow["s", from=2-1, to=2-2]
        \arrow["\tau", from=2-2, to=2-3]
        \arrow["\pi", from=2-3, to=2-4]
    \end{tikzcd}\]
    By Axiom \ref{TR2}, there exists a morphism $C(\tau f)^\bullet \to C[1]^\bullet$ making the whole diagram commutative. As the bottom triangle is distinguished an application of the long exact sequence in cohomology, guaranteed by Proposition \ref{Cohomology is cohomological functor on K(A)}, shows that $H^i(C(s)^\bullet) \iso 0$ for all $i \in \Z$, because $s$ is a quasi-isomorphism. Applying the long exact sequence now to the top triangle shows that $u \colon C(\tau f)[-1]^\bullet \to A^\bullet$ is a quasi-isomorphism. Taking $C^\bullet_0$ to be $C(\tau f)[-1]^\bullet$ proves the lemma. 
\end{proof}

Equipped with the preceding lemma, we are now in a position to construct the derived category $\dercat{\A}$ of an abelian category $\A$. The derived category $\dercat{\A}$ will consist of
\[
    \textsl{Objects of $\dercat{\A}$:} \quad \text{chain complexes in $\A$},
\]
\ie the class of objects of $\kcat{\A}$ or $\com(\A)$, and a class of morphisms, which is quite intricate to define. 
% Its class of objects is formed by chain complexes in $\A$ \ie equals the class of objects of $\kcat{\A}$ and $\com(\A)$, but the definition of morphisms in this case is a bit more intricate. 
For fixed complexes $A^\bullet$ and $B^\bullet$ we define the Hom-set\footnote{What will be defined here is a priori not necessarily a set, but a class, so $\dercat{\A}$, defined in this section, is not necessarily a locally small category.
% category in the usual sense. 
% We will however prove that in specific cases some variants of the derived category, especially concrete ones used later on, will from categories in the usual sense.
} $\Hom_{\dercat{\A}}(A^\bullet, B^\bullet)$ in the following way.

% \vspace{0.3 cm}

\noindent
\textsc{Hom-sets.}
A \emph{left roof spanned by $A^\bullet$ and $B^\bullet$} is a pair of morphisms $s \colon C^\bullet \to A^\bullet$ and $f\colon C^\bullet \to B^\bullet$ in the homotopy category $\kcat{\A}$, where $s$ is a quasi-isomorphism. This roof is depicted in the following diagram
\begin{equation}
    \label{eq: left roof}
    \begin{tikzcd}[row sep = small, column sep = small]
    % https://q.uiver.app/#q=WzAsMyxbMCwxLCJBIl0sWzIsMCwiQyJdLFs0LDEsIkIiXSxbMSwyLCJmIiwyXSxbMSwwLCJzIl0sWzEsMCwiXFxzaW0iLDJdXQ==
	&& C^\bullet \\
	A^\bullet &&&& B^\bullet
	\arrow["s", from=1-3, to=2-1]
	\arrow["\sim"', from=1-3, to=2-1]
	\arrow["f"', from=1-3, to=2-5]
    \end{tikzcd}
\end{equation}
and denoted by $(s,f)$.
Dually, one also obtains the notion of a \emph{right roof spanned by $A^\bullet$ and $B^\bullet$}, which is a pair of morphisms $g\colon A^\bullet \to C^\bullet$ and $u\colon B^\bullet \to C^\bullet$, where $u$ is a quasi-isomorphism, and is depicted below.
\[\begin{tikzcd}[row sep = small, column sep = small]
    % https://q.uiver.app/#q=WzAsMyxbMCwxLCJBIl0sWzIsMCwiQyJdLFs0LDEsIkIiXSxbMSwyLCJmIiwyXSxbMSwwLCJzIl0sWzEsMCwiXFxzaW0iLDJdXQ==
	&& C^\bullet \\
	A^\bullet &&&& B^\bullet
	\arrow["g"', to=1-3, from=2-1]
	\arrow["\sim"', to=1-3, from=2-5]
	\arrow["u", to=1-3, from=2-5]
\end{tikzcd}\] 
Our construction of $\Hom_{\dercat{\A}}(A^\bullet, B^\bullet)$ will be based on left roofs, for nothing is gained or lost by picking either one of the two. Both work just as well and are in fact equivalent (see \cite[\S 7, Remark 7.1.18]{kashiwara2006categories}). Despite our arbitrary choice, it is still beneficial to consider both, as we will sometimes switch between the two whenever convenient. 

\begin{definition}
    \label{Definition of equivalence of roofs}
    Two left roofs $A^\bullet \xleftarrow{\ s_0 \ } C_0^\bullet \xrightarrow{\ f_0 \ } B^\bullet$ and $A^\bullet \xleftarrow{\ s_1 \ } C_1^\bullet \xrightarrow{\ f_1 \ } B^\bullet$ are defined to be \emph{equivalent} if there exists a quasi-isomorphism $u\colon C^\bullet \to C_0^\bullet$ and a morphism ${g\colon C^\bullet \to C_1^\bullet}$ in $\kcat{\A}$, for which the diagram below commutes (in $\kcat{\A}$). 
    \begin{equation}
        \label{eq: equivalence of left roofs}
    \begin{tikzcd}[row sep = small, column sep = small]
        % [row sep = 1.5em, column sep = 2em]
        % https://q.uiver.app/#q=WzAsNSxbMCwyLCJBIl0sWzEsMSwiQ18wIl0sWzQsMiwiQiJdLFszLDEsIkNfMSJdLFsyLDAsIkMiXSxbMSwyXSxbMywwLCJcXHNpbSIsMCx7ImxhYmVsX3Bvc2l0aW9uIjo2MH1dLFszLDJdLFs0LDEsIlxcc2ltIiwyXSxbNCwzLCJnIiwyXSxbMSwwLCJcXHNpbSIsMl0sWzQsMSwidSJdXQ==
        && C^\bullet \\
        & {C_0^\bullet} && {C_1^\bullet} \\
        A^\bullet &&&& B^\bullet
        \arrow["\sim"', from=1-3, to=2-2]
        \arrow["u", from=1-3, to=2-2]
        \arrow["g"', from=1-3, to=2-4]
        \arrow["\sim"', from=2-2, to=3-1]
        \arrow["\sim"{pos=0.6}, from=2-4, to=3-1]
        \arrow[from=2-4, to=3-5]
        \arrow[from=2-2, to=3-5, crossing over]
    \end{tikzcd}
    \end{equation}
We denote this relation by $\equiv$.
\end{definition}

Note that since $C^\bullet \to C_0^\bullet \to A^\bullet$ is a quasi-isomorphism, the same is true for the composition ${C^\bullet \to C_1^\bullet \to A^\bullet}$, concluding that $g\colon C^\bullet \to C^\bullet_1$ is a quasi-isomorphism. Also observe that in the diagram \eqref{eq: equivalence of left roofs} we may find a new left roof, namely 
\begin{equation*}
    \begin{tikzcd}[row sep = small, column sep = small]
    % https://q.uiver.app/#q=WzAsMyxbMCwxLCJBIl0sWzIsMCwiQyJdLFs0LDEsIkIiXSxbMSwyLCJmIiwyXSxbMSwwLCJzIl0sWzEsMCwiXFxzaW0iLDJdXQ==
	&& C^\bullet \\
	A^\bullet &&&& B^\bullet,
	\arrow["s_0 \circ u"', from=1-3, to=2-1]
	\arrow["\sim", from=1-3, to=2-1]
	\arrow["f_1 \circ g", from=1-3, to=2-5]
    \end{tikzcd}
\end{equation*}
which is also equivalent to the two roofs we started with, $(s_0, f_0)$ and $(s_1, f_1)$. 

\begin{lemma}
    The equivalence of left roofs on $A^\bullet$ and $B^\bullet$ is an equivalence relation.
\end{lemma}

\begin{proof}
    The relation is clearly reflexive -- we take both $u$ and $g$ to be $\id{C^\bullet}$. By the note above it is also symmetric, since $g$ is a quasi-isomorphism. It remains to show transitivity. Suppose left roofs $A^\bullet \longleftarrow C_0^\bullet \longrightarrow B^\bullet$ and $A^\bullet \longleftarrow C_1^\bullet \longrightarrow B^\bullet$ are equivalent and left roofs $A^\bullet \longleftarrow C_1^\bullet \longrightarrow B^\bullet$ and $A^\bullet \longleftarrow C_2^\bullet \longrightarrow B^\bullet$ are equivalent. This is witnessed by the diagrams
    \begin{center}
        \begin{tabular}{l c r}
            \begin{tikzcd}[column sep = small, row sep = small]
                % https://q.uiver.app/#q=WzAsNSxbMCwyLCJBXlxcYnVsbGV0Il0sWzEsMSwiQ18wXlxcYnVsbGV0Il0sWzQsMiwiQl5cXGJ1bGxldCJdLFszLDEsIkNfMV5cXGJ1bGxldCJdLFsyLDAsIkReXFxidWxsZXQiXSxbMSwyXSxbMywwLCJcXHNpbSIsMCx7ImxhYmVsX3Bvc2l0aW9uIjo2MH1dLFszLDJdLFsxLDAsIlxcc2ltIiwyXSxbNCwxLCJcXHNpbSIsMl0sWzQsMywiZyIsMl0sWzQsMSwidSJdXQ==
                && {D_0^\bullet} \\
                & {C_0^\bullet} && {C_1^\bullet} \\
                {A^\bullet} &&&& {B^\bullet}
                \arrow["\sim"', from=1-3, to=2-2]
                \arrow["", from=1-3, to=2-2]
                \arrow[""', from=1-3, to=2-4]
                \arrow["\sim"', from=2-2, to=3-1]
                \arrow["\sim"{pos=0.7}, from=2-4, to=3-1]
                \arrow[from=2-4, to=3-5]
                \arrow[from=2-2, to=3-5, crossing over]
            \end{tikzcd} & 
            and &
            \begin{tikzcd}[column sep = small, row sep = small]
                % https://q.uiver.app/#q=WzAsNSxbMCwyLCJBXlxcYnVsbGV0Il0sWzEsMSwiQ18wXlxcYnVsbGV0Il0sWzQsMiwiQl5cXGJ1bGxldCJdLFszLDEsIkNfMV5cXGJ1bGxldCJdLFsyLDAsIkReXFxidWxsZXQiXSxbMSwyXSxbMywwLCJcXHNpbSIsMCx7ImxhYmVsX3Bvc2l0aW9uIjo2MH1dLFszLDJdLFsxLDAsIlxcc2ltIiwyXSxbNCwxLCJcXHNpbSIsMl0sWzQsMywiZyIsMl0sWzQsMSwidSJdXQ==
                && {D_1^\bullet} \\
                & {C_1^\bullet} && {C_2^\bullet} \\
                {A^\bullet} &&&& {B^\bullet.}
                \arrow["\sim"', from=1-3, to=2-2]
                \arrow["", from=1-3, to=2-2]
                \arrow[""', from=1-3, to=2-4]
                \arrow["\sim"', from=2-2, to=3-1]
                \arrow["\sim"{pos=0.7}, from=2-4, to=3-1]
                \arrow[from=2-4, to=3-5]
                \arrow[from=2-2, to=3-5, crossing over]
            \end{tikzcd}
        \end{tabular}
    \end{center}
    % \[\begin{tikzcd}
    %     % https://q.uiver.app/#q=WzAsOCxbMCwzLCJBIl0sWzEsMiwiQ18wIl0sWzYsMywiQiJdLFs1LDIsIkNfMiJdLFszLDIsIkNfMSJdLFsyLDEsIkRfMCJdLFs0LDEsIkRfMSJdLFszLDAsIkMiXSxbMSwyXSxbMywwLCJcXHNpbSIsMCx7ImxhYmVsX3Bvc2l0aW9uIjo2MH1dLFszLDJdLFsxLDAsIlxcc2ltIiwyXSxbNSwxLCJcXHNpbSIsMix7ImNvbG91ciI6WzI0MCw2MCw2MF19LFsyNDAsNjAsNjAsMV1dLFs1LDQsIiIsMCx7ImNvbG91ciI6WzI0MCw2MCw2MF19XSxbNCwwLCJcXHNpbSIsMix7ImxhYmVsX3Bvc2l0aW9uIjo3MH1dLFs0LDJdLFs2LDMsIiIsMCx7ImNvbG91ciI6WzMwMCw2MCw2MF19XSxbNiw0LCJcXHNpbSIsMix7ImNvbG91ciI6WzMwMCw2MCw2MF19LFszMDAsNjAsNjAsMV1dLFs3LDUsIlxcc2ltIiwyLHsic3R5bGUiOnsiYm9keSI6eyJuYW1lIjoiZGFzaGVkIn19fV0sWzcsNiwiIiwwLHsic3R5bGUiOnsiYm9keSI6eyJuYW1lIjoiZGFzaGVkIn19fV1d
    %     &&& C^\bullet \\
    %     && {D_0^\bullet} && {D_1^\bullet} \\
    %     & {C_0^\bullet} && {C_1^\bullet} && {C_2^\bullet} \\
    %     A^\bullet &&&&&& B^\bullet
    %     \arrow["\sim"', dashed, from=1-4, to=2-3]
    %     \arrow[dashed, from=1-4, to=2-5]
    %     \arrow["\sim"', color={rgb,255:red,92;green,92;blue,214}, from=2-3, to=3-2]
    %     \arrow[color={rgb,255:red,92;green,92;blue,214}, from=2-3, to=3-4]
    %     \arrow["\sim"', color={rgb,255:red,214;green,92;blue,214}, from=2-5, to=3-4]
    %     \arrow[color={rgb,255:red,214;green,92;blue,214}, from=2-5, to=3-6]
    %     \arrow["\sim"', from=3-2, to=4-1]
    %     \arrow[curve={height=12pt}, from=3-2, to=4-7]
    %     \arrow["\sim"'{pos=0.2}, from=3-4, to=4-1]
    %     \arrow[from=3-4, to=4-7]
    %     \arrow["\sim"{pos=0.6}, curve={height=-12pt}, from=3-6, to=4-1]
    %     \arrow[from=3-6, to=4-7]
    % \end{tikzcd}\]
    By Lemma \ref{Ore condition} the right roof $D_0^\bullet \to C_1^\bullet \leftarrow D_1^\bullet$ may be completed to form a commutative square
    \[\begin{tikzcd}[column sep = 2em]
        % https://q.uiver.app/#q=WzAsNCxbMCwyLCJEXzBeXFxidWxsZXQiXSxbMiwyLCJDXzFcXGJ1bGxldCJdLFsyLDAsIkReXFxidWxsZXRfMSJdLFswLDAsIkNeXFxidWxsZXQiXSxbMCwxXSxbMiwxLCJcXHNpbSJdLFszLDAsIlxcc2ltIiwyLHsic3R5bGUiOnsiYm9keSI6eyJuYW1lIjoiZGFzaGVkIn19fV0sWzMsMiwiIiwwLHsic3R5bGUiOnsiYm9keSI6eyJuYW1lIjoiZGFzaGVkIn19fV1d
        {C^\bullet} && {D^\bullet_1} \\
        \\
        {D_0^\bullet} && {C_1^\bullet,}
        \arrow[dashed, from=1-1, to=1-3]
        \arrow["\sim"', dashed, from=1-1, to=3-1]
        \arrow["\sim", from=1-3, to=3-3]
        \arrow[from=3-1, to=3-3]
    \end{tikzcd}\]
    proving that $A^\bullet \leftarrow C_0^\bullet \rightarrow B^\bullet$ and $A^\bullet \leftarrow C_2^\bullet \rightarrow B^\bullet$ are equivalent.
\end{proof}

For a left roof \eqref{eq: left roof} we let 
% $\left[s \setminus f\right]$ or 
$\left[A^\bullet \xleftarrow{\ s\ } C^\bullet \xrightarrow{\ f \ } B^\bullet\right]$ denote its equivalence class under $\equiv$. 

We then define $\Hom_{\dercat{\A}}(A^\bullet, B^\bullet)$ to be the class of left roofs spanned by $A^\bullet$ and $B^\bullet$, quotiented by the relation $\equiv$. That is
% \begin{equation*}
%     \Hom_{\dercat{\A}}(A^\bullet, B^\bullet) := 
%     \left\{ (s, f) \in \coprod_{C^\bullet \in \Ob(\kcat{\A})} \Hom_{\kcat{\A}}(C^\bullet, A^\bullet) \times \Hom_{\kcat{\A}}(C^\bullet, B^\bullet) \ \middle\vert \ \text{$s$ quasi-isomorphism} \right\}
% \end{equation*}
% \[
% \Hom_{\dercat{\A}}(A^\bullet, B^\bullet) := \\ 
% \left\{
% \left[
% \begin{array}{l c r}
% & C^\bullet & \\
% \quad \swarrow & & \searrow \quad \ \\
% A^\bullet & & B^\bullet
% \end{array}
% \right]_\equiv
% \, \middle| \,
% \begin{array}{l}
% C^\bullet \in \Ob \kcat{\A}, \\
% f \in \Hom_{\kcat{\A}}(C^\bullet, B^\bullet), \\
% s \in \Hom_{\kcat{\A}}(C^\bullet, A^\bullet) \text{ quasi-iso.}
% \end{array}
% \right\}.
% \]
\[
\Hom_{\dercat{\A}}(A^\bullet, B^\bullet) := 
\left\{
\left[
\begin{tikzcd}[row sep = small, column sep = 0.8em]
    % https://q.uiver.app/#q=WzAsMyxbMCwxLCJBIl0sWzIsMCwiQyJdLFs0LDEsIkIiXSxbMSwyLCJmIiwyXSxbMSwwLCJzIl0sWzEsMCwiXFxzaW0iLDJdXQ==
	&& C^\bullet \\
	A^\bullet &&&& B^\bullet
	\arrow["s", from=1-3, to=2-1]
	\arrow["\sim"', from=1-3, to=2-1]
	\arrow["f"', from=1-3, to=2-5]
\end{tikzcd}
\right]
\, \middle| \,
\begin{array}{l}
C^\bullet \in \Ob \kcat{\A}, \\
f \in \Hom_{\kcat{\A}}(C^\bullet, B^\bullet), \\
s \in \Hom_{\kcat{\A}}(C^\bullet, A^\bullet) \text{ quasi-iso.}
\end{array}
\right\}.
\]

% \vspace{0,3 cm}

\noindent
\textsc{Composition.} Next we define the composition operations
\[
    \circ \colon \Hom_{\dercat{\A}}(A_0^\bullet, A_1^\bullet) \times \Hom_{\dercat{\A}}(A_1^\bullet, A_2^\bullet) \to \Hom_{\dercat{\A}}(A_0^\bullet, A_2^\bullet),
\]
for all objects $A_0^\bullet$, $A_1^\bullet$ and $A_2^\bullet$ of $\dercat{\A}$. Let $\phi_0\colon A_0^\bullet \to A_1^\bullet$ and $\phi_1\colon A_1^\bullet \to A_2^\bullet$ be a pair of composable morphisms in $\dercat{\A}$. Next, pick their respective left roof representatives, $A_0^\bullet \xleftarrow{\ s_0\ } C_0^\bullet \xrightarrow{\ f_0 \ } A_1^\bullet$ and $A_1^\bullet \xleftarrow{\ s_1\ } C_1^\bullet \xrightarrow{\ f_1 \ } A_2^\bullet$, and concatenate them according to the solid zig-zag diagram below. 
\[\begin{tikzcd}[row sep = small, column sep = small]
    % https://q.uiver.app/#q=WzAsNixbMCwyLCJBXzAiXSxbMiwxLCJDXzAiXSxbNCwyLCJBXzEiXSxbOCwyLCJBXzIiXSxbNiwxLCJDXzEiXSxbNCwwLCJDIl0sWzEsMiwiZl8wIiwyXSxbMSwwLCJzXzAiXSxbMSwwLCJcXHNpbSIsMix7Im9mZnNldCI6LTEsInN0eWxlIjp7ImJvZHkiOnsibmFtZSI6Im5vbmUifSwiaGVhZCI6eyJuYW1lIjoibm9uZSJ9fX1dLFs0LDIsInNfMSJdLFs0LDMsImZfMSIsMl0sWzUsMSwidSIsMCx7InN0eWxlIjp7ImJvZHkiOnsibmFtZSI6ImRhc2hlZCJ9fX1dLFs1LDQsImciLDIseyJzdHlsZSI6eyJib2R5Ijp7Im5hbWUiOiJkYXNoZWQifX19XSxbNCwyLCJcXHNpbSIsMix7Im9mZnNldCI6LTEsInN0eWxlIjp7ImJvZHkiOnsibmFtZSI6Im5vbmUifSwiaGVhZCI6eyJuYW1lIjoibm9uZSJ9fX1dLFs1LDEsIlxcc2ltIiwyLHsib2Zmc2V0IjotMSwic3R5bGUiOnsiYm9keSI6eyJuYW1lIjoibm9uZSJ9LCJoZWFkIjp7Im5hbWUiOiJub25lIn19fV1d
	&&&& C^\bullet \\
	&& {C_0^\bullet} &&&& {C_1^\bullet} \\
	{A_0^\bullet} &&&& {A_1^\bullet} &&&& {A_2^\bullet}
	\arrow["u", dashed, from=1-5, to=2-3]
	\arrow["\sim"', shift left, draw=none, from=1-5, to=2-3]
	\arrow["g"', dashed, from=1-5, to=2-7]
	\arrow["{s_0}", from=2-3, to=3-1]
	\arrow["\sim"', shift left, draw=none, from=2-3, to=3-1]
	\arrow["{f_0}"', from=2-3, to=3-5]
	\arrow["{s_1}", from=2-7, to=3-5]
	\arrow["\sim"', shift left, draw=none, from=2-7, to=3-5]
	\arrow["{f_1}"', from=2-7, to=3-9]
\end{tikzcd}\]
By Lemma \ref{Ore condition} there are morphisms $u\colon C^\bullet \to C_0^\bullet$ and $g\colon C^\bullet \to C_1^\bullet$, depicted with dashed arrows, completing the diagram in $\kcat{\A}$. In this way we obtain a left roof, formed by a quasi-isomorphism $s_0 \circ u$ and a morphism $f_1 \circ g$, the equivalence class of which we define to be the composition 
\[
    \phi_1 \circ \phi_0 :=
    \left[A_0^\bullet \xleftarrow{s_0 \circ u} C^\bullet \xrightarrow{f_1 \circ g} A_2^\bullet\right].
\]
It can be shown that this is a well defined composition, independent of the choice of left roof representatives of $\phi_0$ and $\phi_1$ and independent of the choice of the peak $C^\bullet$ and morphisms $u\colon C^\bullet \to C_0^\bullet$ and $g\colon C^\bullet \to C_1^\bullet$, for which the square at the top of the diagram commutes. It is also true that the operation is associative. We leave out the proof, because it is routine, but refer the reader to a very detailed account by Miličić \cite[Ch.~1.3]{milicic-dercat}. 

% \vspace{0,3 cm}

\noindent
\textsc{Identities.}
The identity morphism on an object $A^\bullet$ of $\dercat{\A}$ is defined to be the equivalence class of $A^\bullet \xleftarrow{\id{A^\bullet}} A^\bullet \xrightarrow{\id{A^\bullet}} A^\bullet$. It is not difficult to check that these classes play the role of identity morphisms in $\dercat{\A}$. 

% \vspace{0,3 cm}

\noindent
\textsc{Functor $Q\colon \kcat{\A} \to \dercat{\A}$.} The \emph{localization} functor $Q\colon \kcat{\A} \to \dercat{\A}$ is an identity on objects functor with the action on morphisms defined by the assignment 
\[\begin{tikzcd}[row sep = 0.3em, column sep = 0.75em]
    % https://q.uiver.app/#q=WzAsNCxbMSwwLCJcXEhvbV97XFxkZXJjYXR7XFxBfX0oQV5cXGJ1bGxldCwgSV5cXGJ1bGxldCkiXSxbMCwwLCJcXEhvbV97fShBXlxcYnVsbGV0LCBJXlxcYnVsbGV0KSJdLFswLDEsIlxcbGVmdCggZlxcY29sb24gQV5cXGJ1bGxldCBcXHRvIEleXFxidWxsZXRcXHJpZ2h0KSJdLFsxLDEsIlxcbGVmdFtBXlxcYnVsbGV0IFxceGxlZnRhcnJvd3sgXFwgXFxpZHtBXlxcYnVsbGV0fSB9IEFeXFxidWxsZXQgXFx4cmlnaHRhcnJvd3sgXFwgZiBcXCB9IEleXFxidWxsZXRcXHJpZ2h0XSJdLFsyLDMsIiIsMix7InN0eWxlIjp7InRhaWwiOnsibmFtZSI6Im1hcHMgdG8ifX19XSxbMSwwXV0=
	{\Hom_{\kcat{\A}}(A^\bullet, B^\bullet)} & {\Hom_{\dercat{\A}}(A^\bullet, B^\bullet)} \\
	{\left( f\colon A^\bullet \to B^\bullet\right)} & {\left[A^\bullet \xleftarrow{ \ \id{A^\bullet} } A^\bullet \xrightarrow{ \ f \ } B^\bullet\right].}
	\arrow["{Q}",from=1-1, to=1-2]
	\arrow[maps to, from=2-1, to=2-2]
\end{tikzcd}\]

% \vspace{0,3 cm}

\noindent
\textsc{$k$-linear structure.} To equip the Hom-sets of $\dercat{\A}$ with a $k$-module structure, we consult the following lemma resembling finding a common denominator in the context of left roofs.

\begin{lemma}
    \label{common denominator for left roofs}
    Let $\phi_0 \colon A^\bullet \to B^\bullet$ and $\phi_1 \colon A^\bullet \to B^\bullet$ be two morphisms in $\dercat{\A}$ represented by left roofs
    \begin{center}
    \begin{tabular}{l c r}
        \begin{tikzcd}[row sep = small, column sep = small]
            % https://q.uiver.app/#q=WzAsMyxbMCwxLCJBIl0sWzIsMCwiQ18wIl0sWzQsMSwiQiJdLFsxLDIsImZfMCJdLFsxLDAsInNfMCIsMl0sWzEsMCwiXFxzaW0iLDAseyJzdHlsZSI6eyJib2R5Ijp7Im5hbWUiOiJub25lIn0sImhlYWQiOnsibmFtZSI6Im5vbmUifX19XV0=
            && {C_0^\bullet} \\
            A^\bullet &&&& B^\bullet
            \arrow["{s_0}"', from=1-3, to=2-1]
            \arrow["\sim", draw=none, from=1-3, to=2-1]
            \arrow["{f_0}", from=1-3, to=2-5]
        \end{tikzcd}
        & \quad and \quad &
        \begin{tikzcd}[row sep = small, column sep = small]
            % https://q.uiver.app/#q=WzAsMyxbMCwxLCJBIl0sWzIsMCwiQ18wIl0sWzQsMSwiQiJdLFsxLDIsImZfMCJdLFsxLDAsInNfMCIsMl0sWzEsMCwiXFxzaW0iLDAseyJzdHlsZSI6eyJib2R5Ijp7Im5hbWUiOiJub25lIn0sImhlYWQiOnsibmFtZSI6Im5vbmUifX19XV0=
            && {C_1^\bullet} \\
            A^\bullet &&&& B^\bullet
            \arrow["{s_1}"', from=1-3, to=2-1]
            \arrow["\sim", draw=none, from=1-3, to=2-1]
            \arrow["{f_1}", from=1-3, to=2-5]
        \end{tikzcd}
    \end{tabular}
    \end{center}
    Then there is a quasi-isomorphism $s \colon C^\bullet \to A^\bullet$ and morphisms $g_0$, $g_1 \colon C^\bullet \to B^\bullet$, such that 
    \begin{center}
    \begin{tabular}{l c r}
        \begin{tikzcd}[row sep = small, column sep = small]
            % https://q.uiver.app/#q=WzAsMyxbMCwxLCJBIl0sWzIsMCwiQ18wIl0sWzQsMSwiQiJdLFsxLDIsImZfMCJdLFsxLDAsInNfMCIsMl0sWzEsMCwiXFxzaW0iLDAseyJzdHlsZSI6eyJib2R5Ijp7Im5hbWUiOiJub25lIn0sImhlYWQiOnsibmFtZSI6Im5vbmUifX19XV0=
            && {C^\bullet} \\
            A^\bullet &&&& B^\bullet
            \arrow["{s}"', from=1-3, to=2-1]
            \arrow["\sim", draw=none, from=1-3, to=2-1]
            \arrow["{g_0}", from=1-3, to=2-5]
        \end{tikzcd}
        & \quad and \quad &
        \begin{tikzcd}[row sep = small, column sep = small]
            % https://q.uiver.app/#q=WzAsMyxbMCwxLCJBIl0sWzIsMCwiQ18wIl0sWzQsMSwiQiJdLFsxLDIsImZfMCJdLFsxLDAsInNfMCIsMl0sWzEsMCwiXFxzaW0iLDAseyJzdHlsZSI6eyJib2R5Ijp7Im5hbWUiOiJub25lIn0sImhlYWQiOnsibmFtZSI6Im5vbmUifX19XV0=
            && {C^\bullet} \\
            A^\bullet &&&& B^\bullet
            \arrow["{s}"', from=1-3, to=2-1]
            \arrow["\sim", draw=none, from=1-3, to=2-1]
            \arrow["{g_1}", from=1-3, to=2-5]
        \end{tikzcd}
    \end{tabular}
    \end{center}
    represent $\phi_0$ and $\phi_1$, respectively.
\end{lemma}

\begin{proof}
    The proof of this lemma relies on Lemma \ref{Ore condition} and is otherwise not complicated, therefore we omit it. A proof may be found in \cite[\S 1, Lemma 1.3.5]{milicic-dercat}.
\end{proof}

Using notation of Lemma \ref{common denominator for left roofs}, we define the addition operation on $\Hom_{\dercat{\A}}(A^\bullet, B^\bullet)$ by the rule
\[
    \phi_0 + \phi_1 := \left[ A^\bullet \xleftarrow{\ s\ } C^\bullet \xrightarrow{ g_0 + g_1 } B^\bullet \right].
\]
It is well defined, \ie independent of the choice of $C^\bullet$, $s$, $g_0$ or $g_1$  and depends only on $\phi_0$ and $\phi_1$. It is also associative, commutative and has a neutral element $0 = Q(0)$. This is all very thoroughly expanded on in \cite[\S 1.2]{milicic-dercat}. The action of scalars of the ring $k$ is defined as  
\[
    \lambda \phi_0 := \left[ A^\bullet \xleftarrow{\ s_0\ } C^\bullet \xrightarrow{ \lambda f_0 } B^\bullet \right].
\]
Moreover, the composition law $\circ$ is $k$-bilinear and the localization functor $Q \colon \kcat{\A} \to \dercat{\A}$ is a $k$-linear functor. 

The zero complex $0$ plays the role of the zero object of $\dercat{\A}$ and the direct sum of two complexes is seen to exist and is induced from the direct sum in $\kcat{\A}$ by applying the localization functor $Q$.


% \vspace{0,3 cm}

\noindent
\textsc{Triangulated structure.} The translation functor $T \colon \dercat{\A} \to \dercat{\A}$ is defined to be the usual translation functor on objects and a morphism represented by some roof is sent to the equivalence class of that roof on which we have acted with the translation functor of $\kcat{\A}$. This action respects the equivalence relations, which define the Hom-sets of $\dercat{\A}$, and thus induces a well defined functor, which is also a $k$-linear auto-equivalence, the quasi-inverse being given by translation in the other direction.  

The class of distinguished triangles in $\dercat{\A}$ is defined to consist of all triangles $A^\bullet \to B^\bullet \to C^\bullet \to A[1]^\bullet$, for which there exists a distinguished triangle $A_0^\bullet \to B_0^\bullet \to C_0^\bullet \to A_0[1]^\bullet$ in $\kcat{\A}$, such that $Q(A_0^\bullet) \to Q(B_0^\bullet) \to Q(C_0^\bullet) \to Q(A_0[1]^\bullet)$ is isomorphic to $A^\bullet \to B^\bullet \to C^\bullet \to A[1]^\bullet$ in $\dercat{\A}$. Spelling this out in the case of left roofs, we see that this condition implies the existence of another distinguished triangle $A_1^\bullet \to B_1^\bullet \to C_1^\bullet \to A_1[1]^\bullet$ in $\kcat{\A}$, such that the following commutative diagram, where vertical arrows are all quasi-isomorphisms, exists in $\kcat{\A}$.
\[\begin{tikzcd}[]
    % https://q.uiver.app/#q=WzAsMTIsWzAsMSwiQSJdLFsxLDEsIkIiXSxbMiwxLCJDIl0sWzMsMSwiQVsxXSJdLFswLDIsIlgnIl0sWzEsMiwiWSciXSxbMiwyLCJaJyJdLFszLDIsIlgnWzFdIl0sWzAsMCwiWCciXSxbMSwwLCJZJyJdLFsyLDAsIlonIl0sWzMsMCwiWCdbMV0iXSxbMCwxXSxbMSwyXSxbMiwzXSxbNCw1XSxbNSw2XSxbNiw3XSxbMCw0XSxbMSw1XSxbMyw3XSxbOCw5XSxbOSwxMF0sWzEwLDExXSxbMCw4XSxbMSw5XSxbMiwxMF0sWzMsMTFdLFsyLDZdXQ==
	{A^\bullet} & {B^\bullet} & {C^\bullet} & {A[1]^\bullet} \\
	{A_1^\bullet} & {B_1^\bullet} & {C_1^\bullet} & {A_1[1]^\bullet} \\
    {A_0^\bullet} & {B_0^\bullet} & {C_0^\bullet} & {A_0[1]^\bullet}
	\arrow[from=1-1, to=1-2]
	\arrow[from=1-2, to=1-3]
	\arrow[from=1-3, to=1-4]
	\arrow[from=2-1, to=1-1]
	\arrow[from=2-1, to=2-2]
	\arrow[from=2-1, to=3-1]
	\arrow[from=2-2, to=1-2]
	\arrow[from=2-2, to=2-3]
	\arrow[from=2-2, to=3-2]
	\arrow[from=2-3, to=1-3]
	\arrow[from=2-3, to=2-4]
	\arrow[from=2-3, to=3-3]
	\arrow[from=2-4, to=1-4]
	\arrow[from=2-4, to=3-4]
	\arrow[from=3-1, to=3-2]
	\arrow[from=3-2, to=3-3]
	\arrow[from=3-3, to=3-4]
\end{tikzcd}\]

Verification, that the above defines the structure of a triangulated category on $\dercat{\A}$, is left out, but we remark, that it follows naturally from the triangulated structure on $\kcat{\A}$ and refer the reader to \cite[\S 2, Theorem 1.6.1]{milicic-dercat} or \cite[\S 10, Theorem 10.2.3]{kashiwara2006categories} for more details.

\begin{remark}
    All that has been defined and established in this section with left roofs can analogously also be done with right roofs.
\end{remark}

\begin{proposition}
    \label{universal property for D(A) proposition}
    The constructed derived category $\dercat{\A}$ of an abelian category $\A$ together with the localization functor $Q \colon \kcat{\A} \to \dercat{\A}$ satisfies the universal property of Definition \ref{universal property of D(A)}. Additionally, in notation of Definition \ref{universal property of D(A)}, if category $\D$ and the functor $F \colon \kcat{\A} \to \D$ are $k$-linear or triangulated, the induced functor $F_0$ is as well.
\end{proposition}

\begin{proof}
    For (i) suppose $s\colon A^\bullet \to B^\bullet$ is a quasi-isomorphism. Then the inverse of $Q(s)$ is a morphism $\phi\colon B^\bullet \to A^\bullet$ represented by the left roof $B^\bullet \xleftarrow{\ s \ } A^\bullet \xrightarrow{\id{A^\bullet}} A^\bullet$. Clearly $\phi \circ Q(s) = \id{A^\bullet}$, whereas $Q(s) \circ \phi$, represented by $B^\bullet \xleftarrow{s} A^\bullet \xrightarrow{s} B^\bullet$, can be seen to equal $\id{B^\bullet}$ by the following diagram.
    \[\begin{tikzcd}[column sep = small, row sep = small]
        % https://q.uiver.app/#q=WzAsNSxbMCwyLCJCXlxcYnVsbGV0Il0sWzEsMSwiQV5cXGJ1bGxldCJdLFs0LDIsIkJeXFxidWxsZXQiXSxbMywxLCJCXlxcYnVsbGV0Il0sWzIsMCwiQV5cXGJ1bGxldCJdLFsxLDIsInMiLDJdLFszLDAsIiIsMCx7ImxhYmVsX3Bvc2l0aW9uIjo2MCwibGV2ZWwiOjIsInN0eWxlIjp7ImhlYWQiOnsibmFtZSI6Im5vbmUifX19XSxbMywyLCIiLDAseyJsZXZlbCI6Miwic3R5bGUiOnsiaGVhZCI6eyJuYW1lIjoibm9uZSJ9fX1dLFsxLDAsInMiLDJdLFs0LDEsIiIsMix7ImxldmVsIjoyLCJzdHlsZSI6eyJoZWFkIjp7Im5hbWUiOiJub25lIn19fV0sWzQsMywicyIsMl1d
        && {A^\bullet} \\
        & {A^\bullet} && {B^\bullet} \\
        {B^\bullet} &&&& {B^\bullet}
        \arrow[equals, from=1-3, to=2-2]
        \arrow["s"', from=1-3, to=2-4]
        \arrow["s"', from=2-2, to=3-1]
        \arrow[equals, from=2-4, to=3-1]
        \arrow[equals, from=2-4, to=3-5]
        \arrow["s"'{pos=0.7}, from=2-2, to=3-5, crossing over]
    \end{tikzcd}\]% , where all the unmarked arrows are identities.
    For (ii) the functor $F_0 \colon \dercat{\A} \to \kcat{\A}$ is defined by the assignment on
    \begin{center}
    \begin{tabular}{r r c l}
        \textsl{Objects:} & $A^\bullet$ & $\longmapsto$ & $F(A^\bullet)$. \\
        \textsl{Morphisms:} & $[A^\bullet \xleftarrow{s} C^\bullet \xrightarrow{f} B^\bullet]$ & $\longmapsto$ & $F(f) \circ F(s)\inv$.
    \end{tabular}
    \end{center}
    This is well defined and functorial since the functor $F\colon \kcat{\A} \to \D$ sends quasi-isomorph\-isms of $\kcat{\A}$ to isomorphisms of $\D$. Moreover, $F_0$ is a $k$-linear functor, as can quickly be computed by an application of Lemma \ref{common denominator for left roofs}. Using the same notation as in the lemma, we have
    \begin{gather*}
        F_0(\phi_0 + \phi_1) = F(g_0 + g_1) \circ F(s)\inv = F(g_0) \circ F(s)\inv + F(g_1) \circ F(s)\inv = F_0(\phi_0) + F_0(\phi_1) \\
        \text{and} \quad F_0(\lambda \phi_0) = F(\lambda f_0) \circ F(s_0)\inv = \lambda (F(f_0) \circ F(s_0)\inv) = \lambda F_0(\phi_0).
    \end{gather*}
    It is clear from the definition, that $F_0 \circ Q = F$ and that $F_0$ is unique up to a natural isomorphism for which the diagram \eqref{universal property of D(A)} commutes. 

    Lastly, when $\D$ and $F$ are triangulated, the functor $F_0$ also commutes with the translation functors of $\dercat{\A}$ and $\D$, because $Q$ as the identity on objects and $F$ commutes with translation functors of $\kcat{\A}$ and $\D$. From the fact that distinguished triangles in $\dercat{\A}$ are by definition exactly those triangles, which are isomorphic to $Q$-images of distinguished triangles of $\kcat{\A}$, and the fact that $F$ maps distinguished triangles of $\dercat{\A}$ to distinguished triangles of $\D$, we can conclude that $F_0$ is triangulated as well. 
\end{proof}


% ============================================================== %


\subsubsection{Cohomology}

Since cohomology already appeared as an indispensable tool on the level of the homotopy category $\kcat{\A}$, it would be nice to also have it be accessible at the level of derived categories. 
% also have access to it at the level of derived categories. 
We can in fact define cohomology functors on $\dercat{\A}$ by inducing them from the functors $H^i \colon \kcat{\A} \to \A$. By definition they send quasi-isomorphisms of $\kcat{\A}$ to isomorphisms of $\A$, thus using the universal property of Definition \ref{universal property of D(A)} we obtain $k$-linear functors 
\[
    H^i \colon \dercat{\A} \rightarrow \A, \qquad i \in \Z,
\]
% for $i \in \Z$, 
which as expected return the $i$-th cohomology of a complex $A^\bullet$ and return the morphism $H^i(f) \circ H^i(s)\inv$ when applied to a morphism of $\dercat{\A}$, represented by a left roof \eqref{eq: left roof}.


% ============================================================== %


\subsubsection{Subcategories of derived categories}

% We have already seen $\derplus{\A}$, $\derminus{\A}$ and $\derb{\A}$ be defined as full triangulated subcategories of $\dercat{\A}$ on the class of certain bounded complexes taking values in $\A$.  

As with the homotopy category of complexes $\kcat{\A}$, we also have the bounded versions of the derived category $\dercat{\A}$. They are constructed in the exact same way as $\dercat{\A}$ above, with the adjustment that we replace every instance of the category $\kcat{\A}$ with either $\kplus{\A}$, $\kminus{\A}$ or $\kb{\A}$, to obtain $\derplus{\A}$, $\derminus{\A}$ or $\derb{\A}$, respectively. They are clearly all triangulated as well. All three bounded variants naturally come with the inclusion functors
\[
    \derplus{\A} \to \dercat{\A}, \qquad \derminus{\A} \to \dercat{\A}, \qquad \derb{\A} \to \dercat{\A},
\]
% \[\begin{tikzcd}[row sep = small]
%     % https://q.uiver.app/#q=WzAsNCxbMSwwLCJCIl0sWzAsMSwiQSJdLFsyLDEsIkMiXSxbMSwyLCJEIl0sWzEsMF0sWzAsMl0sWzEsM10sWzMsMl1d
% 	& \derplus{\A} \\
% 	\derb{\A} && \dercat{\A}, \\
% 	& \derminus{\A}
% 	\arrow[from=1-2, to=2-3]
% 	\arrow[from=2-1, to=1-2]
% 	\arrow[from=2-1, to=3-2]
% 	\arrow[from=3-2, to=2-3]
% \end{tikzcd}\]
defined to be the identity on objects and send a morphism, \ie equivalence class of some roof representative with respect to the ``bounded variant'' of the equivalence relation \ref{Definition of equivalence of roofs}, where, again, every instance of $\kcat{\A}$ is replaced by either $\kplus{\A}$, $\kminus{\A}$ or $\kb{\A}$, to the equivalence class of that same roof representative, but now under the original equivalence relation. All three functors are clearly also triangulated.  

% appearing as full triangulated subcategories of the unbounded variant $\dercat{\A}$ by proposition \ref{triangulated subcategory proposition}.
% As with the homotopy category of complexes $\kcat{\A}$, we also have the following bounded versions of the derived category $\dercat{\A}$, appearing as full triangulated subcategories of the unbounded variant $\dercat{\A}$ by proposition \ref{triangulated subcategory proposition}.

% \begin{center}
%     % \begin{tabular}{c l}
%     %     $\derplus{\A}$ & spanned on complexes in $\A$ bounded below. \\
%     %     $\derminus{\A}$ & spanned on complexes in $\A$ bounded above. \\
%     %     $\derb{\A}$ & spanned on bounded complexes in $\A$. \\    
%     % \end{tabular}
%     \begin{tabular}{c l}
%         $\derplus{\A}$ & spanned on objects of $\kplus{\A}$. \\
%         $\derminus{\A}$ & spanned on objects of $\kminus{\A}$. \\
%         $\derb{\A}$ & spanned on objects of $\kb{\A}$. \\    
%     \end{tabular}
% \end{center}

There is also another (equivalent) way of obtaining the categories mentioned above, which will sometimes be more convenient to work with. We introduce them in the form of the following proposition.

\begin{proposition}
    \label{Subcategories of D(A) but bigger}
    Let $\A$ be an abelian category. 
    \begin{enumerate}[label = (\roman*)]
        % \item{The functor $\derplus{\A} \to \dercat{\A}$ (\resp $\derminus{\A} \hookrightarrow \dercat{\A}$) is fully faithful and induces an equivalence between $\derplus{\A}$ and the full triangulated subcategory of $\dercat{\A}$ spanned on (possibly unbounded) complexes $A^\bullet$, for which there is an integer $n \in \Z$, such that $H^i(A^\bullet) \iso 0$ for all $i < n$ (\resp $i > n$).}
        \item{The functor $\derplus{\A} \to \dercat{\A}$ is fully faithful and induces an equivalence between $\derplus{\A}$ and the full triangulated subcategory of $\dercat{\A}$ consisting of (possibly unbounded) complexes $A^\bullet$, for which there is an integer $n \in \Z$, such that $H^i(A^\bullet) \iso 0$ for all $i < n$.}
        \item{The functor $\derminus{\A} \to \dercat{\A}$ is fully faithful and induces an equivalence between $\derminus{\A}$ and the full triangulated subcategory of $\dercat{\A}$ consisting of (possibly unbounded) complexes $A^\bullet$, for which there is an integer $n \in \Z$, such that $H^i(A^\bullet) \iso 0$ for all $i > n$.} 
        \item{The functor $\derb{\A} \to \dercat{\A}$ is fully faithful and induces an equivalence between $\derb{\A}$ and the full triangulated subcategory of $\dercat{\A}$ consisting of (possibly un\-bound\-ed) complexes $A^\bullet$, for which there is an integer $n \in \Z$, such that $H^i(A^\bullet) \iso 0$ for all $\abs{i} > n$.} 
    \end{enumerate}
\end{proposition}

To make the proof of this proposition conceptually clearer, we introduce two complexes associated to a complex $A^\bullet$, called its \emph{right and left truncations}
\begin{align*}
    \rtruncation{n}(A^\bullet) &= (\cdots \to A^{n-2} \to A^{n-1} \to \ker d^{n} \to 0 \to \cdots) \text{ \ and } \\
    \ltruncation{n}(A^\bullet) &= (\cdots \to 0 \to \coker d^{n-1} \to A^{n+1} \to A^{n+2} \to \cdots).
\end{align*}
They come equipped with the natural inclusion map $j \colon \rtruncation{n}(A^\bullet) \to A^\bullet$ and the natural quotient map $q \colon A^\bullet \to \ltruncation{n}(A^\bullet)$, satisfying the following claims
\begin{align}
    H^i(j) \colon H^i(\rtruncation{n}(A^\bullet)) \to H^i(A^\bullet) \text{ is an isomorphism for all } i \leq n. \label{right truncation isos on cohomology} \\
    H^i(q) \colon H^i(A^\bullet) \to H^i(\ltruncation{n}(A^\bullet)) \text{ is an isomorphism for all } i \geq n. \label{left trunctation isos on cohomology}
\end{align}
Indeed, claim \eqref{right truncation isos on cohomology} is clearly true for $i < n$, with $i = n$ being the only interesting case. Here the right commutative square of the diagram below
% \[\begin{tikzcd}[column sep = small, row sep = 1.5em]
%     % https://q.uiver.app/#q=WzAsNCxbMCwwLCJcXGtlciBkXm4iXSxbMiwwLCIwIl0sWzAsMiwiQV57bn0iXSxbMiwyLCJBXntuKzF9Il0sWzIsMywiZF5uIl0sWzAsMSwiMCJdLFsxLDMsImpee24rMX0iXSxbMCwyLCJqXm4iLDJdXQ==
% 	{\ker d^n} && 0 \\
% 	\\
% 	{A^{n}} && {A^{n+1}}
% 	\arrow["0", from=1-1, to=1-3]
% 	\arrow["{j^n}"', from=1-1, to=3-1]
% 	\arrow["{j^{n+1}}", from=1-3, to=3-3]
% 	\arrow["{d^n}", from=3-1, to=3-3]
% \end{tikzcd}\]
\[\begin{tikzcd}[column sep = small, row sep = 1.5em]
    % https://q.uiver.app/#q=WzAsNixbMiwwLCJcXGtlciBkXm4iXSxbNCwwLCIwIl0sWzIsMiwiQV57bn0iXSxbNCwyLCJBXntuKzF9Il0sWzAsMCwiQV57bi0xfSJdLFswLDIsIkFee24tMX0iXSxbMiwzLCJkXm4iXSxbMCwxLCIwIl0sWzEsMywial57bisxfSJdLFswLDIsImpebiIsMl0sWzQsMCwiXFxkZWx0YSJdLFs0LDUsImpee24tMX0iLDJdLFs1LDIsImRee24tMX0iXV0=
	{A^{n-1}} && {\ker d^n} && 0 \\
	\\
	{A^{n-1}} && {A^{n}} && {A^{n+1}}
	\arrow["\delta", from=1-1, to=1-3]
	\arrow["{j^{n-1}}"', from=1-1, to=3-1]
	\arrow["0", from=1-3, to=1-5]
	\arrow["{j^n}"', from=1-3, to=3-3]
	\arrow["{j^{n+1}}", from=1-5, to=3-5]
	\arrow["{d^{n-1}}", from=3-1, to=3-3]
	\arrow["{d^n}", from=3-3, to=3-5]
\end{tikzcd}\]
induces the identity morphism between the kernels of the horizontal differentials and the left commutative square induces the identity morphism between the images of the horizontal differential. Consequently $j$ induces the identity morphism on the $n$-th cohomology $H^n(j) \colon H^n(\rtruncation{n}(A^\bullet)) \to H^n(A^\bullet)$.
Dually, the claim \eqref{left trunctation isos on cohomology} also holds. 
% In order to make the proof of this proposition conceptually clearer, we introduce truncation functors $\tau_{\geq n} \colon \kcat{\A} \to \kplus{\A}$, defined by
% \begin{center}
%     \begin{tabular}{r r c l}
%         \textsl{Objects:} & $A^\bullet$ & $\longmapsto$ & $(\cdots \to 0 \to \coker d^{n-1} \to A^{n+1} \to A^{n+2} \to \cdots)$. \\
%         \textsl{Morphisms:} & $(f\colon A^\bullet \to B^\bullet)$ & $\longmapsto$ & $ 
%         (\tau_{\geq n}(f)^i)_{i \in \Z}$, 
%         % \text{, where } \tau_{\geq n}(f)^i = 
%         % \begin{cases}
%         %     0, \ i < n \\
%         %     \coker d^{n-1}_A \to \coker d^{n-1}_B, \ i = n \\ 
%         %     f^i, \ i > n.
%         % \end{cases}$.
%     \end{tabular}
% \end{center}
% where $\tau_{\geq n}(f)^i$ is the zero morphism for $i < n$, the induced morphism $\coker d^{n-1}_A \to \coker d^{n-1}_B$ for $i = n$ and morphisms $f^i\colon A^i \to B^i$ for $i > n$. 
% % Naturally for every $A^\bullet$ there is a quotient map $q \colon A^\bullet \to \ltruncation{n}(A^\bullet)$.

% We also have $\tau_{\leq n} \colon \kcat{\A} \to \kminus{\A}$, defined by
% \begin{center}
%     \begin{tabular}{r r c l}
%         \textsl{Objects:} & $A^\bullet$ & $\longmapsto$ & $(\cdots \to A^{n-2} \to A^{n-1} \to \ker d^{n} \to 0 \to \cdots)$. \\
%         \textsl{Morphisms:} & $(f\colon A^\bullet \to B^\bullet)$ & $\longmapsto$ & $ 
%         (\tau_{\leq n}(f)^i)_{i \in \Z}$, 
%     \end{tabular}
% \end{center}
% where $\tau_{\leq n}(f)^i$ is $f^i \colon A^i \to B^i$ for $i < n$, the induced morphism $\ker d^n_A \to \ker d^n_B$ for $i = n$ and the zero morphism for $i > 0$.
% % In this case there are natural inclusions $j \colon \rtruncation{n}(A^\bullet) \to A^\bullet$ for every $A^\bullet$.

% \begin{lemma}
%     For every complex $A^\bullet$ the natural quotient map $q \colon A^\bullet \to \ltruncation{n}(A^\bullet)$ induces isomorphisms $H^i(q)$ for $i \geq n$ and the natural inclusion map $j \colon \rtruncation{n}(A^\bullet) \to A^\bullet$ induces isomorphisms $H^i(j)$ for $i \leq n$.
% \end{lemma}

\begin{proof}[Proof of Proposition \ref{Subcategories of D(A) but bigger}]
    For (i) the functor $\derplus{\A} \rightarrow \dercat{\A}$ is readily seen to be fully faithful by considering morphisms to be represented by \emph{right} roofs. For example in the right roof $A^\bullet \to C^\bullet \leftarrow B^\bullet$ spanned by complexes $A^\bullet$ and $B^\bullet$ of $\derplus{\A}$, we first observe that $C^\bullet$ has bounded cohomology from below, thus $q \colon C^\bullet \to \ltruncation{n}(C^\bullet)$ is a quasi-isomorphism for some $n \in \Z$. Extending the aforementioned right roof with quasi-isomorphism $q$, to obtain $A^\bullet \to \ltruncation{n}(C^\bullet) \leftarrow B^\bullet$, then yields an equivalent right roof, whose peak now belongs to $\derplus{\A}$ and the equivalence class of which is therefore sent to the morphism represented by $A^\bullet \to C^\bullet \leftarrow B^\bullet$.
    % taking the viewpoint of morphisms being represented by \emph{right} roofs and extending their peaks using the quotient map into their left truncations. For example in right roof $A^\bullet \to C^\bullet \leftarrow B^\bullet$ spanned on complexes $A^\bullet$ and $B^\bullet$ of $\derplus{\A}$, we first observe that $C^\bullet$ has bounded   
    % truncating their peaks by extening them with the quotient map $q$. 
    % so we verify only that it is essentially surjective onto the subcategory of complexes with cohomology bounded from below. 

    To verify this functor is essentially surjective onto the subcategory of complexes with cohomology bounded from below,
    suppose $A^\bullet$ satisfies $H^i(A^\bullet) \iso 0$, for all $i < n$. Then by \eqref{left trunctation isos on cohomology} the quotient map $q \colon A^\bullet \to \ltruncation{n}(A^\bullet)$ is a quasi-isomorphism, inducing an isomorphism $A^\bullet \iso \ltruncation{n}(A^\bullet)$ in $\dercat{\A}$. 
    
    The case (ii) is proven analogously, now considering left roofs and right truncations. For (iii), the functor $\derb{\A} \to \dercat{\A}$ is viewed as the composition $\derb{\A} \to \derplus{\A} \to \dercat{\A}$. Here $\derb{\A} \to \derplus{\A}$ is seen to be fully faithful via a minor modification of the argument for (ii) and $\derplus{\A} \to \dercat{\A}$ is fully faithful by (i). Essential surjectivity onto the required subcategory then follows as the roof $A^\bullet \to \ltruncation{n}(A^\bullet) \leftarrow \rtruncation{n}(\ltruncation{n}(A^\bullet))$, for some $n \in \Z$, witnesses an isomorphism in $\dercat{\A}$ between a complex $A^\bullet$ with bounded cohomology and a bounded complex $\rtruncation{n}(\ltruncation{n}(A^\bullet))$. \qedhere
    % we view the inclusion $\derb{\A} \hookrightarrow \dercat{\A}$ as the composition $\derb{\A} \hookrightarrow \derplus{\A} \hookrightarrow \dercat{\A}$
    
    % claim that the quotient map $q \colon A^\bullet \to \ltruncation{n}(A^\bullet)$ is actually a quasi-isomorphism.
    
    % This shows that there is an isomorphism in $\dercat{\A}$ between $A^\bullet$ and a complex $\ltruncation{n}(A^\bullet)$, which belongs to $\derplus{\A}$. 
\end{proof}

% \info{Move footnote about $\A \to \kcat{\A}$ being fully faithful over here.}

% More precisely, there is always a fully faithful functor $\A \to \kcat{\A}$ sending an object $A$ to a complex $\cdots \to 0 \to A \to 0 \to \cdots$ concentrated in degree $0$ and a morphism $f\colon A \to B$ to the homotopy class of the chain map defined by $f^0 = f$ and $f^i = 0$, for $i \neq 0$ \cite[Chapter 3, Lemma 1.3.5]{milicic-dercat}.


Recall that in Section \ref{Subsection: Categories of complexes} we introduced a functor $\A \to \kcat{\A}$, including an object $A$ of $\A$ to a complex concentrated in degree $0$ and having a similar action on morphisms.  
Postcomposing this functor with the localozation functor $Q \colon \kcat{\A} \to \dercat{\A}$ results in a functor $\A \to \dercat{\A}$. Due to the next proposition, we may view $\A$ as a full subcategory of the derived category $\dercat{\A}$. 

\begin{proposition}
    \label{A -> D(A) fully faithful}
    The natural functor $\A \to \dercat{\A}$ is fully faithful and identifies $\A$ with the full subcategory of $\dercat{\A}$, consisting of complexes $A^\bullet$, satisfying $H^i(A^\bullet) \iso 0$, for all $i \neq 0$.
\end{proposition}

\begin{proof}
    For fully faithfulness see \cite[\S 13, Proposition 13.1.10]{kashiwara2006categories}. Essential surjectivity onto the required subcategory is \cite[\S 13, Proposition 13.1.12]{kashiwara2006categories} and follows from a similar argument as in the proof of Proposition \ref{Subcategories of D(A) but bigger} (iii) by taking $n$ to be $0$, thus showing that in $\dercat{\A}$ the complex $A^\bullet$ with trivial cohomology in non-zero degrees is isomorphic to $H^0(A^\bullet)$ concentrated in degree $0$.
\end{proof}
% \begin{proof}
%     We first show the morphism action
%     \begin{equation}
%         \label{eq: morphism action of A -> D(A)}
%         \Hom_\A(A, B) \to \Hom_{\dercat{\A}}(A, B)
%     \end{equation}
%     is injective. Suppose morphisms $f_0$ and $f_1 \colon A \to B$ are sent to equivalent roofs, described by the next diagram.
%     \[\begin{tikzcd}[row sep = small, column sep = small]
%         % [row sep = 1.5em, column sep = 2em]
%         % https://q.uiver.app/#q=WzAsNSxbMCwyLCJBIl0sWzEsMSwiQ18wIl0sWzQsMiwiQiJdLFszLDEsIkNfMSJdLFsyLDAsIkMiXSxbMSwyXSxbMywwLCJcXHNpbSIsMCx7ImxhYmVsX3Bvc2l0aW9uIjo2MH1dLFszLDJdLFs0LDEsIlxcc2ltIiwyXSxbNCwzLCJnIiwyXSxbMSwwLCJcXHNpbSIsMl0sWzQsMSwidSJdXQ==
%         && C^\bullet \\
%         & {A} && {A} \\
%         A &&&& B
%         \arrow["\sim", from=1-3, to=2-2]
%         \arrow["u"', from=1-3, to=2-2]
%         \arrow["g", from=1-3, to=2-4]
%         \arrow[equals, from=2-2, to=3-1]
%         \arrow[equals, from=2-4, to=3-1]
%         \arrow["f_0", from=2-4, to=3-5]
%         \arrow["f_1"'{pos=0.7}, from=2-2, to=3-5, crossing over]
%     \end{tikzcd}\]
%     Then firstly we see that $u = g$ and taking the $0$-th cohomology, shows $f_0 = f_1$. For surjectivity we assign to a morphism $\phi \in \Hom_{\dercat{\A}}(A, B)$, represented by a roof $A \to C^\bullet \leftarrow B$, 


%     To see $\A \to \dercat{\A}$ is fully faithful one uses the $0$-th cohomology functor $H^0$ on roof representatives
%     % The functor $\A \to \dercat{\A}$ can be written as the composition $\A \to \derplus{\A} \to \dercat{\A}$, so we only need to verify fully faithfulness of $\A \to \derplus{\A}$. 
%     % This is seen through the fact that any quasi-isomorphism between two complexes concentrated in degree $0$ is an isomorphism. 
%     % That the functor is fully faithful is a consequence of the fact that it can be written as the composition $\A \to \derplus{\A} \to \dercat{\A}$ and any quasi-isomorphism between two complexes concentrated in degree $0$ is an isomorphism. 
%     Essential surjectivity follows from a similar argument as in the proof of Proposition \ref{Subcategories of D(A) but bigger} (iii) by taking $n$ to be $0$, thus showing that in $\dercat{\A}$ the complex $A^\bullet$ with trivial cohomology in non-zero degrees is isomorphic to $H^0(A^\bullet)$ concentrated in degree $0$. \info{this proof is wack.} 
% \end{proof}

% ============================================================== %

\subsubsection[Derived categories of abelian categories with enough injectives]{
    \texorpdfstring{Derived categories \\ of abelian categories with enough injectives}{Derived categories of abelian categories with enough injectives}
    }

After possibly working with proper classes, when constructing the derived category, this section will once again place us back on familiar grounds of set theory. Aside from these concerns the establishing result of this section will have a very important practical application -- construction of derived functors. Under an assumption on the abelian category $\A$, we will establish an equivalence of $\derplus{\A}$ with the homotopy category of a full $k$-linear subcategory of $\A$, consisting of \emph{injective objects}, which we define presently. 

\begin{definition}
    An object $I$ of an abelian category $\A$ is \emph{injective}\footnote{Dually, an object $P$ of $\A$ is \emph{projective} if $P$ is injective in $\A^\op$, or more explicitly, the functor $\Hom_\A(P, -) \colon \A \to \mod{k}$ is exact. Category $\A$ is said to \emph{have enough projectives} if every object $A$ of $\A$ is a quotient of some projective object \ie there is an epimorphism $P \twoheadrightarrow A$ for some projective object $P$.
} if the functor 
\[
    \Hom_\A(-, I)\colon \A^\op \to \mod{k}
\]
is exact. Equivalently, $I$ is injective, whenever for any monomorphism ${A \hookrightarrow B}$ and morphism $A \to I$ there is a morphism $B \to I$, for which the following diagram commutes.
    \[\begin{tikzcd}[column sep = 2em]
        % https://q.uiver.app/#q=WzAsNCxbMSwxLCJBIl0sWzIsMSwiQiJdLFsxLDAsIkkiXSxbMCwxLCIwIl0sWzMsMF0sWzAsMSwiIiwwLHsic3R5bGUiOnsidGFpbCI6eyJuYW1lIjoiaG9vayIsInNpZGUiOiJ0b3AifX19XSxbMCwyXSxbMSwyLCIiLDEseyJzdHlsZSI6eyJib2R5Ijp7Im5hbWUiOiJkYXNoZWQifX19XV0=
        & I \\
        0 & A & B
        \arrow[from=2-1, to=2-2]
        \arrow[from=2-2, to=1-2]
        \arrow[hook, from=2-2, to=2-3]
        \arrow[dashed, from=2-3, to=1-2]
    \end{tikzcd}\]
\end{definition}
A category $\A$ is said to \emph{have enough injectives} if every object $A$ of $\A$ embeds into an injective object \ie there is a monomorphism $A \hookrightarrow I$ for some injective object $I$. 

\begin{remark}
    A simple verification shows, that the full subcategory of an abelian or $k$-linear category $\A$, consisting of all injective objects of $\A$ is a $k$-linear category, as the zero object $0$ is injective and the direct sum of two injective objects is again injective. We denote this category by $\J$.
\end{remark}

\begin{example}
    \label{example of cats w/ and w/o injectives}
    % In the category of abelian groups $\Ab$, injective objects are precisely the divisible groups.     
    The category of abelian groups $\Ab$ contains enough injectives. As the latter is equivalent to $\quasicoh{\spec{\Z}}$, it also contains enough injectives. However, since injective abelian groups are divisible, and non-trivial divisible groups are never finitely generated, the full abelian subcategory of finitely generated abelian groups does not contain enough injectives. The latter category is equivalent to $\coherent{\spec{\Z}}$, so it also does not contain enough injectives.
    
    % This is because every abelian group $A$ may be embedded into $I(A) = \prod_{f \in \Hom(A, \Q/\Z)} \Q/\Z$, where $I(A)$ is injective as a product of injective groups $\Q/\Z$.
\end{example}

Injective objects will be utilized as building blocks for representing complexes. They are the preferred class of objects to work with in this context because they enable a favourable interplay between cohomology and homotopy. 
For example, as a consequence of Lemma \ref{acyclic to injective is nulhomotopic}, every acyclic bounded below complex of injectives $I^\bullet$ is actually null-homotopic \ie isomorphic to the object $0$ in $\kplus{\A}$. 
In light of this, consider a complex $A^\bullet$ of $\kplus{\A}$. A complex of injectives $I^\bullet$ of $\kplus{\J}$ together with a quasi-isomorphism $f \colon A^\bullet \to I^\bullet$ is called an \emph{injective resolution} of $A^\bullet$.
Later we will also see that this injective resolution is unique up to homotopy equivalence.
% of their nice behaviour with respect to cohomology and homotopy

% We call a quasi-isomorphism $f \colon A^\bullet \to I^\bullet$ an \emph{injective resolution} of the complex $A^\bullet$. Later we will also see that this injective resolution is unique up to homotopy \ie any two injective resolutions of $A^\bullet$ are homotopically equivalent. 
% Throughout this section differentials of $I^\bullet$ will be denoted with $\delta^i\colon I^i \to I^{i+1}$.

\begin{proposition}
    \label{injective resolution}
    \emph{\cite[\S 2, Proposition 3.35]{huybrechts2006fouriermukai}}
    % Suppose $\A$ contains enough injectives. Then every $A^\bullet$ in $\kplus{\A}$ has an injective resolution.
    Suppose $\A$ contains enough injectives. Then for every $A^\bullet$ in $\kplus{\A}$ there is a quasi-isomorphism $f \colon A^\bullet \to I^\bullet$, where $I^\bullet \in \Ob \kplus{\A}$ is a complex of injectives.
\end{proposition}

\begin{proof}
    We will inductively construct a complex $I^\bullet$ in $\kplus{\A}$, built up from injectives, and a quasi-isomorphism $f \colon A^\bullet \to I^\bullet$. For simplicity assume $A^i = 0$ for $i < 0$. 
    
    The base case is trivial. Define $I^i = 0$ and $f^i = 0$ for $i < 0$ and take $f^0\colon A^0 \to I^0$ to be the morphism obtained from the assumption about $\A$ having enough injectives.
    \[\begin{tikzcd}[column sep = 3em, row sep = 2.7em]
        % https://q.uiver.app/#q=WzAsOCxbMSwwLCIwIl0sWzIsMCwiQV4wIl0sWzMsMCwiQV4xIl0sWzQsMCwiXFxjZG90cyJdLFswLDAsIlxcY2RvdHMiXSxbMiwxLCJJXjAiXSxbMSwxLCIwIl0sWzAsMSwiXFxjZG90cyJdLFs0LDBdLFswLDFdLFsxLDJdLFsyLDNdLFsxLDVdLFs3LDZdLFs2LDVdXQ==
        \cdots & 0 & {A^0} & {A^1} & \cdots \\
        \cdots & 0 & {I^0}
        \arrow[from=1-1, to=1-2]
        \arrow[from=1-2, to=1-3]
        \arrow[from=1-3, to=1-4]
        \arrow[from=1-3, to=2-3]
        \arrow[from=1-4, to=1-5]
        \arrow[from=2-1, to=2-2]
        \arrow[from=2-2, to=2-3]
        \arrow[from=1-2, to=2-2]
    \end{tikzcd}\]

    For the induction step suppose we have constructed injective objects $I^i$, with differentials ${\delta^i \colon I^i \to I^{i+1}}$, and morphisms $f^i$ for all $i$ up to $n$, such that the diagram below commutes. 
    \[\begin{tikzcd}[column sep = normal, row sep = 3em]
        % https://q.uiver.app/#q=WzAsOCxbMSwwLCJBXntuLTF9Il0sWzIsMCwiQV5uIl0sWzMsMCwiQV57bisxfSJdLFs0LDAsIlxcY2RvdHMiXSxbMCwwLCJcXGNkb3RzIl0sWzIsMSwiSV5uIl0sWzEsMSwiSV57bi0xfSJdLFswLDEsIlxcY2RvdHMiXSxbNCwwXSxbMCwxLCJkXntuLTF9Il0sWzEsMiwiZF5uIl0sWzIsM10sWzEsNSwiZl57bn0iLDJdLFs3LDZdLFs2LDUsIlxcZGVsdGFee24tMX0iXSxbMCw2LCJmXntuLTF9IiwyXV0=
        \cdots & {A^{n-1}} & {A^n} & {A^{n+1}} & \cdots \\
        \cdots & {I^{n-1}} & {I^n}
        \arrow[from=1-1, to=1-2]
        \arrow["{d^{n-1}}", from=1-2, to=1-3]
        \arrow["{f^{n-1}}"', from=1-2, to=2-2]
        \arrow["{d^n}", from=1-3, to=1-4]
        \arrow["{f^{n}}"', from=1-3, to=2-3]
        \arrow[from=1-4, to=1-5]
        \arrow[from=2-1, to=2-2]
        \arrow["{\delta^{n-1}}", from=2-2, to=2-3]
    \end{tikzcd}\]
    Consider the cokernel of the differential\footnote{Strictly speaking this is not a differential of a complex yet, because we have not specified the chain map $f$ entirely. But this is easily fixed by setting $I^i = 0$ and $f^i$ to be the zero morphism $A^i \to 0$, for $i > n$.} $d^{n-1}_{C(f)}$
    \[
        A^n \oplus I^{n-1} \xrightarrow{\ d^{n-1}_{C(f)} \ } A^{n+1} \oplus I^n \xrightarrow{\ e \ } \coker d^{n-1}_{C(f)} 
    \]
    and denote with $j \colon \coker d^{n-1}_{C(f)} \hookrightarrow I^{n+1}$ an embedding to an injective object $I^{n+1}$.
    %  we get from $\A$ having enough injectives. 
    After precomposing the morphism
    \[
        A^{n+1} \oplus I^n \xrightarrow{\ e \ } \coker d^{n-1}_{C(f)} \xrightarrow{ \ j \ } I^{n+1}
    \]
    with canonical embeddings of $A^{n + 1}$ and $I^{n}$ into their direct sum $A^{n+1} \oplus I^n$, we obtain morphisms $\delta^{n}\colon I^n \to I^{n+1}$ and $f^{n+1}\colon A^{n+1} \to I^{n+1}$. As $j \circ e$ is a morphism fitting into the following coproduct diagram,
    % \[\begin{tikzcd}[row sep = 2em]
    %     % https://q.uiver.app/#q=WzAsNCxbMSwxLCJBXntuKzF9IFxcb3BsdXMgSV5uIl0sWzAsMCwiQV57bisxfSJdLFsyLDAsIklee259Il0sWzEsMywiSV57bisxfSJdLFswLDMsImogXFxjaXJjIGUiLDJdLFsxLDMsImZee24rMX0iLDIseyJjdXJ2ZSI6Mn1dLFsyLDMsIlxcZGVsdGFebiIsMCx7ImN1cnZlIjotMn1dLFsxLDBdLFsyLDBdXQ==
    %     {A^{n+1}} && {I^{n}} \\
    %     & {A^{n+1} \oplus I^n} \\
    %     \\
    %     & {I^{n+1}}
    %     \arrow[from=1-1, to=2-2]
    %     \arrow["{f^{n+1}}"', curve={height=12pt}, from=1-1, to=4-2]
    %     \arrow[from=1-3, to=2-2]
    %     \arrow["{\delta^n}", curve={height=-12pt}, from=1-3, to=4-2]
    %     \arrow["{j \circ e}"', from=2-2, to=4-2]
    % \end{tikzcd}\]
    \[\begin{tikzcd}[column sep = small, row sep = 2em]
        % https://q.uiver.app/#q=WzAsNCxbMSwxLCJBXntuKzF9IFxcb3BsdXMgSV5uIl0sWzAsMSwiQV57bisxfSJdLFsxLDAsIklee259Il0sWzIsMiwiSV57bisxfSJdLFswLDMsImogXFxjaXJjIGUiLDIseyJsYWJlbF9wb3NpdGlvbiI6NDAsInN0eWxlIjp7ImJvZHkiOnsibmFtZSI6ImRhc2hlZCJ9fX1dLFsxLDMsImZee24rMX0iLDIseyJjdXJ2ZSI6Mn1dLFsyLDMsIlxcZGVsdGFebiIsMCx7ImN1cnZlIjotMn1dLFsxLDBdLFsyLDBdXQ==
        & {I^{n}} \\
        {A^{n+1}} & {A^{n+1} \oplus I^n} \\
        && {I^{n+1}}
        \arrow[from=1-2, to=2-2]
        \arrow["{\delta^n}", curve={height=-18pt}, from=1-2, to=3-3]
        \arrow[from=2-1, to=2-2]
        \arrow["{f^{n+1}}"', curve={height=18pt}, from=2-1, to=3-3]
        \arrow["{j \circ e}"'{pos=0.4}, dashed, from=2-2, to=3-3]
    \end{tikzcd}\]
    we see that $j \circ e = \langle f^{n+1} , \delta^n \rangle$. Next, we see that $\delta^n\delta^{n-1} = 0$ and $\delta^n f^n = f^{n+1}d^n$ from the computation
    \[
        0 = j \circ e \circ d^{n-1}_{C(f)} = \langle f^{n+1},\ \delta^n \rangle \begin{pmatrix}
            -d^n & 0 \\ f^n & \delta^{n-1}
        \end{pmatrix} = \langle \delta^n f^n - f^{n+1}d^n , \ \delta^n\delta^{n-1}\rangle.
    \]
    Thus far we have constructed a chain complex $I^\bullet$ of injective objects and a chain map ${f \colon A^\bullet \to I^\bullet}$. It remains to be shown that $f$ is a quasi-isomorphism. Recall that $f$ is a quasi-isomorphism if and only if its cone $C(f)^\bullet$ is acyclic. By Remark \ref{versions of cohomology} we known
    \[
        H^n(C(f)^\bullet) = \ker(\coker d^{n-1}_{C(f)} \xrightarrow{\ d \ } A^{n+2} \oplus I^{n+1}),
    \]
    where $d$ is obtained from the universal property of $\coker d^{n-1}_{C(f)}$, induced by $d^n_{C(f)}$. Thus it suffices to show that $d$ is a monomorphism. Consider the following diagram, which we will show to commute. 
    \[\begin{tikzcd}[column sep = 2em]
        % https://q.uiver.app/#q=WzAsNSxbMCwwLCJBXm5cXG9wbHVzIElee24tMX0iXSxbMiwwLCJBXntuKzF9XFxvcGx1cyBJXntufSJdLFs0LDAsIkFee24rMn1cXG9wbHVzIElee24rMX0iXSxbMywxLCJjb2tlciBkXntuLTF9X3tDKGYpfSJdLFs1LDEsIklee24rMX0iXSxbMCwxLCJkXntuLTF9X3tDKGYpfSJdLFsxLDIsImRee259X3tDKGYpfSJdLFsxLDMsImUiXSxbMywyLCJkIiwwLHsic3R5bGUiOnsiYm9keSI6eyJuYW1lIjoiZGFzaGVkIn19fV0sWzIsNCwicCJdLFszLDQsImoiXV0=
        {A^n\oplus I^{n-1}} && {A^{n+1}\oplus I^{n}} && {A^{n+2}\oplus I^{n+1}} \\
        &&& {\coker d^{n-1}_{C(f)}} && {I^{n+1}}
        \arrow["{d^{n-1}_{C(f)}}", from=1-1, to=1-3]
        \arrow["{d^{n}_{C(f)}}", from=1-3, to=1-5]
        \arrow["e", two heads, from=1-3, to=2-4]
        \arrow["p", from=1-5, to=2-6]
        \arrow["d", dashed, from=2-4, to=1-5]
        \arrow["j", hook, from=2-4, to=2-6]
    \end{tikzcd}\]
    Once we show that its rightmost triangle is commutative, we see that $d$ is monomorphic, because $j$ is monomorphic. To see that the triangle in question commutes, we compute
    \[
        p \circ d \circ e = p \circ d^n_{C(f)} = \langle f^{n+1}, \delta^n \rangle = j \circ e.
    \] 
    As $e$ is an epimorphism, we may cancel it on the right, to arrive at $p \circ d = j$, which ends the proof.
\end{proof}

\begin{remark}
    As is evident from the proof by close inspection we have only used two facts about the class of all injective objects of $\A$ -- that every object of $\A$ embeds into some injective object and that the class of injectives is closed under finite direct sums. This will later on be used in Subsection \ref{Subsubsection: Derived functors of left exact functors}, when constructing a right derived functor of a given functor $F$ in the presence of an $F$-adapted class and also in the proof of Proposition \ref{In Db(X) everything is a complex of vector bundles}.
\end{remark}

\begin{theorem}
    \label{D+ = K+ and injectives}
    \emph{\cite[\S 2, Proposition 2.40]{huybrechts2006fouriermukai}}
    Assume $\A$ contains enough injectives and let $\J \subseteq \A$ denote the full subcategory on injecitve objects of $\A$. Then the inclusion $\kplus{\J} \hookrightarrow \kplus{\A}$ induces an equivalence of triangulated categories
    \[
        \kplus{\J} \iso \derplus{\A}.
    \]
\end{theorem}

\begin{remark}
    When $\A$ contains enough injectives, Theorem \ref{D+ = K+ and injectives} in particular shows that $\derplus{\A}$ is a locally small category, \ie all its Hom-sets are in fact sets.  
    % \info{maybe I replace "usual" with locally small everywhere this comes up.}
\end{remark}

\begin{lemma}
    \label{acyclic to injective is nulhomotopic}
    Let $A^\bullet$ be any acyclic complex in $\kplus{\A}$ and $I^\bullet$ a complex of injectives from $\kplus{\J}$. Then
    \[
        \hom_{\kplus{\A}}(A^\bullet, I^\bullet) = 0.
    \]
    In other words every morphism from an acyclic complex to an injective one is null-homotopic.
\end{lemma}

\begin{proof}
    We will show that $f \htpy 0$ for any chain map $f \colon A^\bullet \to I^\bullet$ by inductively constructing an appropriate homotopy $h$. For simplicity assume $A^i = 0$ and $I^i = 0$ for $i < 0$ and define $h^i \colon A^i \to I^{i-1}$ to be $0$ for $i \leq 0$. 
    
    As $A^\bullet$ is acyclic, the first non-trivial differential $d^0\colon A^0 \to A^1$ is a monomorphism. From $I^0$ being injective, we obtain a morphism $h^1\colon A^1 \to I^0$, satisfying
    \[
        f^0 = h^1 \circ d^0.
    \]
    Note that the above can actually be rewritten to $f^0 = h^1d^0 + \delta^{-1}h^0$, as $h^0 = 0$.

    For the induction step assume that we have already constructed $h^i \colon A^i \to I^{i-1}$, with $f^{i-1} = h^id^{i-1} + \delta^{i-2}h^{i-1}$ for all $i \leq n$. We are aiming to construct $h^{n+1} \colon A^{n+1} \to I^n$, for which $f^n = h^{n+1} d^n + \delta^{n-1}h^n$ holds. First, expand the differential $d^n$ into a composition of the canonical epimorphism $e \colon A^n \to \coker d^{n-1}$, followed by $j \colon \coker d^{n-1} \to A^{n+1}$ induced by the universal property of $\coker d^{n-1}$ by the map $d^n \colon A^n \to A^{n+1}$. Morphism $j$ is actually a monomorphism by acyclicity of $A^\bullet$, since we know that
    \[
        0 = H^n(A^\bullet) \iso \ker(j \colon \coker d^{n-1} \to A^{n+1}).
    \]
    Next, we see that by the universal property of $\coker d^{n-1}$, morphism $f^n - \delta^{n-1}h^n \colon A^n \to I^n$ induces a morphism ${g \colon \coker d^{n-1} \to I^n}$, because
    \begin{align*}
        (f^n - \delta^{n-1}h^n)d^{n-1} &= f^nd^{n-1} - \delta^{n-1}h^nd^{n-1} \\
        &= f^nd^{n-1} + \delta^{n-1}\delta^{n-2}h^{n-1} - \delta^{n-1}f^{n-1} \\
        &= 0.
    \end{align*}
    The second equality follows from the inductive hypothesis and the third one from $f$ being a chain map and $\delta^{n-1}$, $\delta^{n-2}$ being a differentials. 

    Lastly, for $I^n$ is injecive and $j$ is a monomorphism, there is a morphism $h^{n+1} \colon A^{n+1} \to I^n$, satisfying $h^{n+1} \circ j = g$, thus after precomposing both sides with $e$, we arrive at $h^{n+1}d^n = f^n - \delta^{n-1}h^n$, which can be rewritten as
    \[
        f^n = h^{n+1}d^n + \delta^{n-1}h^n.
        \qedhere  
    \]
\end{proof}

\begin{lemma}
    \label{D+ K+ aux lemma 1}
    % Let $A^\bullet$ and $B^\bullet$ be complexes belonging to $\kplus{\A}$ and $I^\bullet$ a complex of injectives from $\kplus{\J}$.
    Let $A^\bullet$ and $B^\bullet$ belong to $\kplus{\A}$ and $I^\bullet \in \kplus{\J}$. Let $f \colon B^\bullet \to A^\bullet$ be a quasi-isomorphism, then 
    \[
        \Hom_{\kplus{\A}}(A^\bullet, I^\bullet) \xrightarrow{ \ f^* \ } \Hom_{\kplus{\A}}(B^\bullet, I^\bullet) 
    \]
    is an isomorphism of $k$-modules.
\end{lemma}

\begin{proof}
    In $\kplus{\A}$ we have a distinguished triangle $B^\bullet \to A^\bullet \to C(f)^\bullet \to B[1]^\bullet$, which induces a long exact sequence (of $k$-modules)
    \begin{multline*}
        \cdots \longrightarrow \Hom_{\kplus{\A}}(C(f)^\bullet, I^\bullet) \longrightarrow \\
        \longrightarrow \Hom_{\kplus{\A}}(A^\bullet, I^\bullet) \xrightarrow{ \ f^* \ }
        \Hom_{\kplus{\A}}(B^\bullet, I^\bullet) \longrightarrow \\
        \longrightarrow \Hom_{\kplus{\A}}(C(f)[-1]^\bullet, I^\bullet) \longrightarrow \cdots,
    \end{multline*}
    since $\Hom_{\kplus{\A}}(-, I^\bullet)$ is a cohomological functor by Example \ref{partial homs are cohomological functors}. As $f$ is a quasi-isomorphism, the cone $C(f)^\bullet$ is acyclic along with all its shifts, thus showing that
    \[
        \Hom_{\kplus{\A}}(C(f)[i]^\bullet, I^\bullet) = 0,
    \]
    for all $i \in \Z$ by Lemma \ref{acyclic to injective is nulhomotopic}, since $I^\bullet$ is a complex of injectives. From the long exact sequence it then clearly follows that $f^*$ is an isomorphism. 
\end{proof}

\begin{lemma}
    \label{D+ K+ aux lemma 2}
    Let $A^\bullet$ belong to $\kplus{\A}$ and $I^\bullet$ to $\kplus{\J}$. Then the morphism action 
    \begin{equation}
        \label{eq: morphism action of Q}
        \Hom_{\kplus{\A}}(A^\bullet, I^\bullet) \to \Hom_{\derplus{\A}}(A^\bullet, I^\bullet)
    \end{equation}
    of the localization functor $Q$ is an isomorphism of $k$-modules. 
\end{lemma}

\begin{proof}
    We already know this is a $k$-module homomorphism, thus it is enough to show that it is bijective. This is accomplished by constructing its inverse. Let $\phi \colon A^\bullet \to I^\bullet$ be a morphism in $\derplus{\A}$ and let $A^\bullet \xleftarrow{ \ s \ } B^\bullet \xrightarrow{ \ f \ } I^\bullet$ be its left roof representative. The morphism $s$ being a quasi-isomorphism implies that $s^* \colon \Hom_{\kplus{\A}}(A^\bullet, I^\bullet) \to \Hom_{\kplus{\A}}(B^\bullet, I^\bullet)$ is bijective by Lemma \ref{D+ K+ aux lemma 1}, so there is a unique morphism $g\colon A^\bullet \to I^\bullet$, such that $f = g \circ s$ in $\kplus{\A}$. The inverse to \eqref{eq: morphism action of Q} is then defined by sending $\phi$ to $g$. 
    
    First let's argue why this map is well defined, \ie independent of the choice of a left roof representative for $\phi$. We pick two left roof representatives $A^\bullet \xleftarrow{ \ s_0 \ } B_0^\bullet \xrightarrow{ \ f_0 \ } I^\bullet$ and $A^\bullet \xleftarrow{ \ s_1 \ } B_1^\bullet \xrightarrow{ \ f_1 \ } I^\bullet$ for $\phi$ and let $g_0$ and $g_1$ be such that $f_i = g_i \circ s_i$ for both $i$. As the roofs are equivalent there are quasi-isomorphisms $t_0 \colon C^\bullet \to B^\bullet_0$ and $t_1 \colon C^\bullet \to B^\bullet_1$, which fit into the following commutative diagram.
    \[\begin{tikzcd}[row sep = small, column sep = small]
        % https://q.uiver.app/#q=WzAsNSxbMCwyLCJBIl0sWzEsMSwiQl8wIl0sWzQsMiwiSSJdLFszLDEsIkJfMSJdLFsyLDAsIkMiXSxbMSwyLCJmXzAiLDIseyJsYWJlbF9wb3NpdGlvbiI6NjB9XSxbMywwLCJzXzEiLDAseyJsYWJlbF9wb3NpdGlvbiI6NjB9XSxbMywyLCJmXzEiXSxbNCwzLCJ0XzEiXSxbMSwwLCJzXzAiLDJdLFs0LDEsInRfMCIsMl1d
        && C^\bullet \\
        & {B^\bullet_0} && {B^\bullet_1} \\
        A^\bullet &&&& I^\bullet
        \arrow["{t_0}"', from=1-3, to=2-2]
        \arrow["{t_1}", from=1-3, to=2-4]
        \arrow["{s_0}"', from=2-2, to=3-1]
        \arrow["{s_1}"{pos=0.6}, from=2-4, to=3-1]
        \arrow["{f_1}", from=2-4, to=3-5]
        \arrow["{f_0}"'{pos=0.6}, from=2-2, to=3-5, crossing over]
    \end{tikzcd}\]
    Therefore $g_0s_0t_0 = f_0t_0 = f_1t_1 = g_1s_1t_1 = g_1s_0t_0$. As $s_0t_0$ is a quasi-isomorphism, we see that $g_0 = g_1$ by applying Lemma \ref{D+ K+ aux lemma 1}.
    %  follows by applying lemma \ref{D+ K+ aux lemma 1} as the composition $s_0t_0$ is a quasi-isomorphism.
    
    % The following diagram shows why the constructed map is a left inverse to \eqref{eq: morphism action of Q}
    % \[
    %     \left[ A^\bullet \xleftarrow{ \ s \ } B^\bullet \xrightarrow{ \ f \ } I^\bullet \right] \longmapsto 
    %     \left( g\colon A^\bullet \to I^\bullet\right) \longmapsto 
    %     \left[A^\bullet \xleftarrow{ \ \id{A^\bullet} } A^\bullet \xrightarrow{ \ g \ } I^\bullet\right] = 
    %     \left[ A^\bullet \xleftarrow{ \ s \ } B^\bullet \xrightarrow{ \ f \ } I^\bullet \right]
    % \]

    % \begin{align*}
    %     \Hom_{\derplus{\A}}(A^\bullet, I^\bullet) \to \Hom_{\kplus{\J}}(A^\bullet, I^\bullet) \to \Hom_{\derplus{\A}}(A^\bullet, I^\bullet) \\
    %     \left[ A^\bullet \xleftarrow{ \ s \ } B^\bullet \xrightarrow{ \ f \ } I^\bullet \right] \longmapsto 
    %     \left( g\colon A^\bullet \to I^\bullet\right) \longmapsto 
    %     \left[A^\bullet \xleftarrow{ \ \id{A^\bullet} } A^\bullet \xrightarrow{ \ g \ } I^\bullet\right] = 
    %     \left[ A^\bullet \xleftarrow{ \ s \ } B^\bullet \xrightarrow{ \ f \ } I^\bullet \right]
    % \end{align*}

%     \[\begin{tikzcd}[column sep = small , row sep = small]
%         % https://q.uiver.app/#q=WzAsNixbMCwwLCJcXEhvbV97RF4rKFxcQSl9KEFeXFxidWxsZXQsIEleXFxidWxsZXQpIl0sWzQsMCwiXFxIb21fe0ReKyhcXEEpfShBXlxcYnVsbGV0LCBJXlxcYnVsbGV0KSJdLFsyLDAsIlxcSG9tX3tLXisoXFxKKX0oQV5cXGJ1bGxldCwgSV5cXGJ1bGxldCkiXSxbMCwxLCJcXGxlZnRbIEFeXFxidWxsZXQgXFx4bGVmdGFycm93eyBcXCBzIFxcIH0gQl5cXGJ1bGxldCBcXHhyaWdodGFycm93eyBcXCBmIFxcIH0gSV5cXGJ1bGxldCBcXHJpZ2h0XSJdLFsyLDEsIlxcbGVmdCggZ1xcY29sb24gQV5cXGJ1bGxldCBcXHRvIEleXFxidWxsZXRcXHJpZ2h0KSJdLFs0LDEsIlxcbGVmdFtBXlxcYnVsbGV0IFxceGxlZnRhcnJvd3sgXFwgXFxpZHtBXlxcYnVsbGV0fSB9IEFeXFxidWxsZXQgXFx4cmlnaHRhcnJvd3sgXFwgZyBcXCB9IEleXFxidWxsZXRcXHJpZ2h0XSA9ICAgICAgICAgIFxcbGVmdFsgQV5cXGJ1bGxldCBcXHhsZWZ0YXJyb3d7IFxcIHMgXFwgfSBCXlxcYnVsbGV0IFxceHJpZ2h0YXJyb3d7IFxcIGYgXFwgfSBJXlxcYnVsbGV0IFxccmlnaHRdIl0sWzAsMl0sWzIsMV0sWzMsNCwiIiwyLHsic3R5bGUiOnsidGFpbCI6eyJuYW1lIjoibWFwcyB0byJ9fX1dLFs0LDUsIiIsMix7InN0eWxlIjp7InRhaWwiOnsibmFtZSI6Im1hcHMgdG8ifX19XV0=
% 	{\Hom_{\derplus{\A}}(A^\bullet, I^\bullet)} && {\Hom_{\kplus{\J}}(A^\bullet, I^\bullet)} && {\Hom_{\derplus{\A}}(A^\bullet, I^\bullet)} \\
% 	{\left[ A^\bullet \xleftarrow{ \ s \ } B^\bullet \xrightarrow{ \ f \ } I^\bullet \right]} && {\left( g\colon A^\bullet \to I^\bullet\right)} && {\left[A^\bullet \xleftarrow{ \ \id{A^\bullet} } A^\bullet \xrightarrow{ \ g \ } I^\bullet\right] =          \left[ A^\bullet \xleftarrow{ \ s \ } B^\bullet \xrightarrow{ \ f \ } I^\bullet \right]}
% 	\arrow[from=1-1, to=1-3]
% 	\arrow[from=1-3, to=1-5]
% 	\arrow[maps to, from=2-1, to=2-3]
% 	\arrow[maps to, from=2-3, to=2-5]
% \end{tikzcd}\]

    % and why it is also a right inverse to \eqref{eq: morphism action of Q}
    % \[
    %     \left(  g\colon A^\bullet \to I^\bullet \right) \longmapsto
    %     \left[A^\bullet \xleftarrow{ \ \id{A^\bullet} } A^\bullet \xrightarrow{ \ g \ } I^\bullet\right] \longmapsto
    %     \left(  g\colon A^\bullet \to I^\bullet \right).
    %     \qedhere
    % \]


    % https://q.uiver.app/#q=WzAsNyxbMCwwLCJcXEhvbV97RF4rKFxcQSl9KEFeXFxidWxsZXQsIEleXFxidWxsZXQpIl0sWzIsMCwiXFxIb21fe0ReKyhcXEEpfShBXlxcYnVsbGV0LCBJXlxcYnVsbGV0KSJdLFsxLDAsIlxcSG9tX3tLXisoXFxKKX0oQV5cXGJ1bGxldCwgSV5cXGJ1bGxldCkiXSxbMCwxLCJcXGxlZnRbIEFeXFxidWxsZXQgXFx4bGVmdGFycm93eyBcXCBzIFxcIH0gQl5cXGJ1bGxldCBcXHhyaWdodGFycm93eyBcXCBmIFxcIH0gSV5cXGJ1bGxldCBcXHJpZ2h0XSJdLFsxLDEsIlxcbGVmdCggZ1xcY29sb24gQV5cXGJ1bGxldCBcXHRvIEleXFxidWxsZXRcXHJpZ2h0KSJdLFsyLDEsIlxcbGVmdFtBXlxcYnVsbGV0IFxceGxlZnRhcnJvd3sgXFwgXFxpZHtBXlxcYnVsbGV0fSB9IEFeXFxidWxsZXQgXFx4cmlnaHRhcnJvd3sgXFwgZyBcXCB9IEleXFxidWxsZXRcXHJpZ2h0XSJdLFszLDEsIlxcbGVmdFsgQV5cXGJ1bGxldCBcXHhsZWZ0YXJyb3d7IFxcIHMgXFwgfSBCXlxcYnVsbGV0IFxceHJpZ2h0YXJyb3d7IFxcIGYgXFwgfSBJXlxcYnVsbGV0IFxccmlnaHRdIl0sWzAsMl0sWzIsMV0sWzMsNCwiIiwyLHsic3R5bGUiOnsidGFpbCI6eyJuYW1lIjoibWFwcyB0byJ9fX1dLFs0LDUsIiIsMix7InN0eWxlIjp7InRhaWwiOnsibmFtZSI6Im1hcHMgdG8ifX19XSxbNSw2LCIiLDIseyJsZXZlbCI6Miwic3R5bGUiOnsiaGVhZCI6eyJuYW1lIjoibm9uZSJ9fX1dXQ==
% \[\begin{tikzcd}[row sep = 0.5em, column sep = 0.75em]
% 	{\Hom_{\derplus{\A}}(A^\bullet, I^\bullet)} & {\Hom_{\kplus{\J}}(A^\bullet, I^\bullet)} & {\Hom_{\derplus{\A}}(A^\bullet, I^\bullet)} \\
% 	{\left[ A^\bullet \xleftarrow{ \ s \ } B^\bullet \xrightarrow{ \ f \ } I^\bullet \right]} & {\left( g\colon A^\bullet \to I^\bullet\right)} & {\left[A^\bullet \xleftarrow{ \ \id{A^\bullet} } A^\bullet \xrightarrow{ \ g \ } I^\bullet\right]} & {\left[ A^\bullet \xleftarrow{ \ s \ } B^\bullet \xrightarrow{ \ f \ } I^\bullet \right]}
% 	\arrow[from=1-1, to=1-2]
% 	\arrow[from=1-2, to=1-3]
% 	\arrow[maps to, from=2-1, to=2-2]
% 	\arrow[maps to, from=2-2, to=2-3]
% 	\arrow[equals, from=2-3, to=2-4]
% \end{tikzcd}\]

% \[\begin{tikzcd}[row sep = 0.5em, column sep = 0.75em]
%     % https://q.uiver.app/#q=WzAsNixbMCwwLCJcXEhvbV97RF4rKFxcQSl9KEFeXFxidWxsZXQsIEleXFxidWxsZXQpIl0sWzIsMCwiXFxIb21fe0ReKyhcXEEpfShBXlxcYnVsbGV0LCBJXlxcYnVsbGV0KSJdLFsxLDAsIlxcSG9tX3tLXisoXFxKKX0oQV5cXGJ1bGxldCwgSV5cXGJ1bGxldCkiXSxbMCwxLCJcXGxlZnRbIEFeXFxidWxsZXQgXFx4bGVmdGFycm93eyBcXCBzIFxcIH0gQl5cXGJ1bGxldCBcXHhyaWdodGFycm93eyBcXCBmIFxcIH0gSV5cXGJ1bGxldCBcXHJpZ2h0XSJdLFsxLDEsIlxcbGVmdCggZ1xcY29sb24gQV5cXGJ1bGxldCBcXHRvIEleXFxidWxsZXRcXHJpZ2h0KSJdLFsyLDEsIlxcbGVmdFtBXlxcYnVsbGV0IFxceGxlZnRhcnJvd3sgXFwgXFxpZHtBXlxcYnVsbGV0fSB9IEFeXFxidWxsZXQgXFx4cmlnaHRhcnJvd3sgXFwgZyBcXCB9IEleXFxidWxsZXRcXHJpZ2h0XSJdLFswLDJdLFsyLDFdLFszLDQsIiIsMix7InN0eWxlIjp7InRhaWwiOnsibmFtZSI6Im1hcHMgdG8ifX19XSxbNCw1LCIiLDIseyJzdHlsZSI6eyJ0YWlsIjp7Im5hbWUiOiJtYXBzIHRvIn19fV1d
% 	{\Hom_{\derplus{\A}}(A^\bullet, I^\bullet)} & {\Hom_{\kplus{\J}}(A^\bullet, I^\bullet)} & {\Hom_{\derplus{\A}}(A^\bullet, I^\bullet)} \\
% 	{\left[ A^\bullet \xleftarrow{ \ s \ } B^\bullet \xrightarrow{ \ f \ } I^\bullet \right]} & {\left( g\colon A^\bullet \to I^\bullet\right)} & {\left[A^\bullet \xleftarrow{ \ \id{A^\bullet} } A^\bullet \xrightarrow{ \ g \ } I^\bullet\right]}
% 	\arrow[from=1-1, to=1-2]
% 	\arrow[from=1-2, to=1-3]
% 	\arrow[maps to, from=2-1, to=2-2]
% 	\arrow[maps to, from=2-2, to=2-3]
% \end{tikzcd}\]


Following the diagram below, it is clear why the constructed map is a right inverse to the morphism action of $Q$ \eqref{eq: morphism action of Q} 
\[\begin{tikzcd}[row sep = 0.5em, column sep = 0.75em]
    % https://q.uiver.app/#q=WzAsNixbMSwwLCJcXEhvbV97RF4rKFxcQSl9KEFeXFxidWxsZXQsIEleXFxidWxsZXQpIl0sWzIsMCwiXFxIb21fe0teKyhcXEopfShBXlxcYnVsbGV0LCBJXlxcYnVsbGV0KSJdLFsyLDEsIlxcbGVmdCggZ1xcY29sb24gQV5cXGJ1bGxldCBcXHRvIEleXFxidWxsZXRcXHJpZ2h0KSJdLFswLDAsIlxcSG9tX3tLXisoXFxKKX0oQV5cXGJ1bGxldCwgSV5cXGJ1bGxldCkiXSxbMCwxLCJcXGxlZnQoIGdcXGNvbG9uIEFeXFxidWxsZXQgXFx0byBJXlxcYnVsbGV0XFxyaWdodCkiXSxbMSwxLCJcXGxlZnRbQV5cXGJ1bGxldCBcXHhsZWZ0YXJyb3d7IFxcIFxcaWR7QV5cXGJ1bGxldH0gfSBBXlxcYnVsbGV0IFxceHJpZ2h0YXJyb3d7IFxcIGcgXFwgfSBJXlxcYnVsbGV0XFxyaWdodF0iXSxbMCwxXSxbNCw1LCIiLDIseyJzdHlsZSI6eyJ0YWlsIjp7Im5hbWUiOiJtYXBzIHRvIn19fV0sWzUsMiwiIiwyLHsic3R5bGUiOnsidGFpbCI6eyJuYW1lIjoibWFwcyB0byJ9fX1dLFszLDBdXQ==
	{\Hom_{\kplus{\A}}(A^\bullet, I^\bullet)} & {\Hom_{\derplus{\A}}(A^\bullet, I^\bullet)} & {\Hom_{\kplus{\A}}(A^\bullet, I^\bullet)} \\
	{\left( g\colon A^\bullet \to I^\bullet\right)} & {\left[A^\bullet \xleftarrow{ \ \id{A^\bullet} } A^\bullet \xrightarrow{ \ g \ } I^\bullet\right]} & {\left( g\colon A^\bullet \to I^\bullet\right).}
	\arrow["Q", from=1-1, to=1-2]
	\arrow[from=1-2, to=1-3]
	\arrow[maps to, from=2-1, to=2-2]
	\arrow[maps to, from=2-2, to=2-3]
\end{tikzcd}\]
Lastly we see that it is also a left inverse by the following diagram \[\begin{tikzcd}[row sep = 0.5em, column sep = 0.75em]
    % https://q.uiver.app/#q=WzAsNixbMCwwLCJcXEhvbV97RF4rKFxcQSl9KEFeXFxidWxsZXQsIEleXFxidWxsZXQpIl0sWzIsMCwiXFxIb21fe0ReKyhcXEEpfShBXlxcYnVsbGV0LCBJXlxcYnVsbGV0KSJdLFsxLDAsIlxcSG9tX3tLXisoXFxKKX0oQV5cXGJ1bGxldCwgSV5cXGJ1bGxldCkiXSxbMCwxLCJcXGxlZnRbIEFeXFxidWxsZXQgXFx4bGVmdGFycm93eyBcXCBzIFxcIH0gQl5cXGJ1bGxldCBcXHhyaWdodGFycm93eyBcXCBmIFxcIH0gSV5cXGJ1bGxldCBcXHJpZ2h0XSJdLFsxLDEsIlxcbGVmdCggZ1xcY29sb24gQV5cXGJ1bGxldCBcXHRvIEleXFxidWxsZXRcXHJpZ2h0KSJdLFsyLDEsIlxcbGVmdFtBXlxcYnVsbGV0IFxceGxlZnRhcnJvd3sgXFwgXFxpZHtBXlxcYnVsbGV0fSB9IEFeXFxidWxsZXQgXFx4cmlnaHRhcnJvd3sgXFwgZyBcXCB9IEleXFxidWxsZXRcXHJpZ2h0XSJdLFswLDJdLFsyLDFdLFszLDQsIiIsMix7InN0eWxlIjp7InRhaWwiOnsibmFtZSI6Im1hcHMgdG8ifX19XSxbNCw1LCIiLDIseyJzdHlsZSI6eyJ0YWlsIjp7Im5hbWUiOiJtYXBzIHRvIn19fV1d
	{\Hom_{\derplus{\A}}(A^\bullet, I^\bullet)} & {\Hom_{\kplus{\A}}(A^\bullet, I^\bullet)} & {\Hom_{\derplus{\A}}(A^\bullet, I^\bullet)} \\
	{\left[ A^\bullet \xleftarrow{ \ s \ } B^\bullet \xrightarrow{ \ f \ } I^\bullet \right]} & {\left( g\colon A^\bullet \to I^\bullet\right)} & {\left[A^\bullet \xleftarrow{ \ \id{A^\bullet} } A^\bullet \xrightarrow{ \ g \ } I^\bullet\right]}
	\arrow[from=1-1, to=1-2]
	\arrow["Q", from=1-2, to=1-3]
	\arrow[maps to, from=2-1, to=2-2]
	\arrow[maps to, from=2-2, to=2-3]
\end{tikzcd}\]
along with observing that $A^\bullet \xleftarrow{ \ \id{A^\bullet} } A^\bullet \xrightarrow{ \ g \ } I^\bullet$ and $A^\bullet \xleftarrow{ \ s \ } B^\bullet \xrightarrow{ \ f \ } I^\bullet$ are equivalent roofs.
\end{proof}

\begin{proof}[Proof of Theorem \ref{D+ = K+ and injectives}]
    % We start of with a

    % \noindent
    % % \textbf{Preliminary claim.} Let $A^\bullet$ and $B^\bullet$ be complexes belonging to $\kplus{\A}$ and $I^\bullet$ a complex of injectives from $\kplus{\J}$.
    % \textbf{Preliminary claim.} Let $A^\bullet$ and $B^\bullet$ belong to $\kplus{\A}$ and $I^\bullet \in \kplus{\J}$. Let $f \colon B^\bullet \to A^\bullet$ be a quasi-isomorphism, then 
    % \[
    %     \Hom_{\kplus{\A}}(A^\bullet, I^\bullet) \xrightarrow{ \ - \circ f \ } \Hom_{\kplus{\A}}(B^\bullet, I^\bullet) 
    % \]
    % is a bijection.
    As $\kplus{\J} \to \derplus{\A}$ is a triangulated functor, we only need to show that it is fully faithful and essentially surjective. Let $I^\bullet$ and $J^\bullet$ be objects of $\kplus{\J}$. Then
    \[
        \Hom_{\kplus{\J}}(I^\bullet, J^\bullet) \xrightarrow{\ = \ } \Hom_{\kplus{\A}}(I^\bullet, J^\bullet) \xrightarrow{\ \eqref{D+ K+ aux lemma 2} \ } \Hom_{\derplus{\A}}(I^\bullet, J^\bullet)
    \]
    is a bijection, showing fully faithfulness (the last map is a bijection by Lemma \ref{D+ K+ aux lemma 2}). Essential surjectivity is clear from the existence of injective resolutions (\cf Proposition \ref{injective resolution}), because quasi-isomorphisms now play the role of isomorphisms in $\derplus{\A}$.
\end{proof}

Within the scope of an abelian category $\A$ with enough injectives, Theorem \ref{D+ = K+ and injectives} yields several well-known classical results of homological algebra. 

\begin{corollary}
    Let $\A$ be an abelian category with enough injectives.
    \begin{enumerate}[label = (\roman*)]
        \item{Any two injective resolutions of a complex $A^\bullet$ of $\kplus{\A}$ are homotopically equivalent, \ie isomorphic in $\kplus{\A}$}
        \item{For any morphism of complexes $f \colon A^\bullet \to B^\bullet$ in $\kplus{\A}$ and any injective resolutions $A^\bullet \to I^\bullet_A$ and $B^\bullet \to I^\bullet_B$ of $A^\bullet$ and $B^\bullet$, respectively, there is a chain map $I^\bullet_A \to I^\bullet_B$, unique up to homotopy, for which the square below commutes in $\kplus{\A}$.
        % \[\begin{tikzcd}[row sep = normal]
        %     % https://q.uiver.app/#q=WzAsNCxbMiwwLCJBXlxcYnVsbGV0Il0sWzIsMiwiQl5cXGJ1bGxldCJdLFswLDAsIkleXFxidWxsZXRfQSJdLFswLDIsIkleXFxidWxsZXRfQiJdLFswLDEsImYiXSxbMCwyXSxbMSwzXSxbMiwzLCIiLDAseyJzdHlsZSI6eyJib2R5Ijp7Im5hbWUiOiJkYXNoZWQifX19XV0=
        %     {I^\bullet_A} && {A^\bullet} \\
        %     \\
        %     {I^\bullet_B} && {B^\bullet}
        %     \arrow[dashed, from=1-1, to=3-1]
        %     \arrow[from=1-3, to=1-1]
        %     \arrow["f", from=1-3, to=3-3]
        %     \arrow[from=3-3, to=3-1]
        % \end{tikzcd}\]
        \[\begin{tikzcd}[column sep = 3em, row sep = 2.7em]
            % https://q.uiver.app/#q=WzAsNCxbMSwwLCJBXlxcYnVsbGV0Il0sWzEsMSwiQl5cXGJ1bGxldCJdLFswLDAsIkleXFxidWxsZXRfQSJdLFswLDEsIkleXFxidWxsZXRfQiJdLFswLDEsImYiXSxbMCwyXSxbMSwzXSxbMiwzLCIiLDAseyJzdHlsZSI6eyJib2R5Ijp7Im5hbWUiOiJkYXNoZWQifX19XV0=
            {I^\bullet_A} & {A^\bullet} \\
            {I^\bullet_B} & {B^\bullet}
            \arrow[dashed, from=1-1, to=2-1]
            \arrow[from=1-2, to=1-1]
            \arrow["f", from=1-2, to=2-2]
            \arrow[from=2-2, to=2-1]
        \end{tikzcd}\]
        }
    \end{enumerate}
\end{corollary}

% \info{add a little argument for why injective res. are unique up to htpy.}

% We leave out the proof for why this operation is associative, but refer the intersted reader to \cite{milicic-dercat} or \cite[\S 7]{kashiwara2006categories} for a more detailed and general approach astablishing the theory of localization of categories.
% \begin{remark}
%     This lemma reminds us of the Ore condition in the context of localization in non-commutative algebra as so analogously there is also a dual version of the lemma. 
% \end{remark}



% ============================================================== %

\subsection{Derived functors}

The main goal of this section is to assign to a sensible $k$-linear functor $F \colon \A \to \B$ between two abelian categories an appropriate \emph{triangulated} functor on the level of derived categories, called a \emph{derived functor}. We will first deal with the easiest case, when $F$ is exact and then move on to the case, where $F$ will only be assumed to be left or right exact and instead require the domain category $\A$ to posses some nice properties (\eg contain enough injectives or contain an $F$-adapted class of objects). In practice the additional conditions $\A$ is required to satisfy are not very restrictive. 

\subsubsection{Derived functors of exact functors}
\label{subsection: Derived functors of exact functors}

We start with some establishing lemmas.

\begin{lemma}
    \label{qis iff cone acyclic}
    Let $f\colon A^\bullet \to B^\bullet$ be a chain map. Then $f$ is a quasi-isomorphism if and only if its associated cone $C(f)^\bullet$ is acyclic. 
\end{lemma}

% \info{proof can reference a proposition from section on triangulated cats...}

\begin{proof}
    One only needs to consider the distinguished triangle $A^\bullet \xrightarrow{f} B^\bullet \to C(f)^\bullet \to A[1]^\bullet$ in $\kcat{\A}$ and its associated long exact sequence in cohomology.
\end{proof}

\begin{lemma}
    \label{F preserves qis iff preserves acyclic}
    Let $F\colon \kcat{\A} \to \kcat{\B}$ be a triangulated functor, then the following two conditions are equivalent. 
    % then $F$ preserves quasi-isomorphisms if and only if it preserves acyclic complexes.
    % \begin{enumerate}
    %     \item[\emph{(i)}] $F$ preserves quasi-isomorphisms
    %     \item[\emph{(ii)}] $F$ preserves acyclic complexes.
    % \end{enumerate}
    \begin{enumerate}[label = (\roman*)]
        \item For every quasi-isomorphism $s$ in $\kcat{\A}$, $F(s)$ is a quasi-isomorphism in $\kcat{\B}$.
        \item For every acyclic complex $A^\bullet$ of $\kcat{\A}$, $F(A^\bullet)$ is acyclic.
    \end{enumerate}
\end{lemma}

\begin{proof}
    $(\Rightarrow)$ Let $A^\bullet$ be an acyclic complex. Then the zero morphism $A^\bullet \to 0$ is a quasi-isomorphism and is by assumption mapped to a quasi-isomorphism $F(A^\bullet) \to 0$ by $F$. This means $F(A^\bullet)$ is acyclic.
    
    $(\Leftarrow)$ Let $f\colon A^\bullet \to B^\bullet$ be a quasi-isomorphism. The image of a distinguished triangle $A^\bullet \to B^\bullet \to C(f)^\bullet \to A^\bullet[1]$ in $\kcat{\A}$ by the triangulated functor $F$ is a distinguished triangle $FA^\bullet \to FB^\bullet \to F(C(f)^\bullet) \to F(A^\bullet)[1]$ in $\kcat{\B}$. By Lemma \ref{qis iff cone acyclic} $C(f)^\bullet$ is acyclic, implying $F(C(f)^\bullet)$ is acyclic by the assumption (ii). Hence again by Lemma \ref{qis iff cone acyclic} $F(f)$ is a quasi-isomorphism.
    % The cone $F(C(f)^\bullet)$ is then acyclic, because $C(f)^\bullet$ was acyclic by lemma \ref{qis iff cone acyclic}, thus showing $Ff$ is a quasi-isomorphism.  
\end{proof}

\begin{proposition}
    \label{Exact functor induces a derived functor}
    \emph{\cite[\S 2, Lemma 2.44]{huybrechts2006fouriermukai}}
    % Let $F\colon \kcat{\A} \to \kcat{\B}$ be a triangulated functor and assume it satisfies one of the equivalent conditions of Lemma \ref{F preserves qis iff preserves acyclic}. Then there exists a triangulated functor $\dercat{\A} \to \dercat{\B}$ on the level of derived categories, for which the following diagram of triangulated functors commutes.
    Let $F\colon \kcat{\A} \to \kcat{\B}$ be a functor and assume it satisfies one of the equivalent conditions of Lemma \ref{F preserves qis iff preserves acyclic}. Then there exists a unique functor $F_0 \colon \dercat{\A} \to \dercat{\B}$ on the level of derived categories, for which the following diagram of functors commutes. 
    \begin{equation}
        \label{eq: exact functor derives}
        \begin{tikzcd}[column sep = 1.5em]
        % https://q.uiver.app/#q=WzAsNCxbMCwwLCJLKFxcQSkiXSxbMiwwLCJLKFxcQikiXSxbMCwyLCJEKFxcQSkiXSxbMiwyLCJEKFxcQikiXSxbMCwxLCJGIl0sWzAsMl0sWzIsM10sWzEsM11d
        {\kcat{\A}} && {\kcat{\B}} \\
        \\
        {\dercat{\A}} && {\dercat{\B}}
        \arrow["F", from=1-1, to=1-3]
        \arrow["{Q_\A}"', from=1-1, to=3-1]
        \arrow["{Q_\B}", from=1-3, to=3-3]
        \arrow["{F_0}", from=3-1, to=3-3]
        \end{tikzcd}
    \end{equation}
    Uniqueness of the functor $F_0 \colon \dercat{\A} \to \dercat{\B}$ and commutativity of the square are both meant up to natural isomorphism. If $F \colon \kcat{\A} \to \kcat{\B}$ is assumed to be $k$-linear or triangulated, the induced functor $F_0 \colon \dercat{\A} \to \dercat{\B}$ is as well. 
\end{proposition}

\begin{proof}
    The claim follows from the universal property of $\dercat{\A}$ (\cf Proposition \ref{universal property for D(A) proposition}), as $Q_\B \circ F$ sends quasi-isomorphisms of $\kcat{\A}$ to isomorphisms of $\dercat{\B}$.
\end{proof}

\begin{example}
    This proposition gives us another way of defining the translation functor on the derived category $\dercat{\A}$. The translation functor $T \colon \kcat{\A} \to \kcat{\A}$ clearly satisfies the equivalent conditions of Lemma \ref{F preserves qis iff preserves acyclic}, thus it induces a functor on the level of derived categories $T \colon \dercat{\A} \to \dercat{\A}$. The fact that the induced translation functor on $\dercat{\A}$ is an autoequivalence follows from the assertion about uniqueness.  
\end{example}

% The induced functor $\dercat{\A} \to \dercat{\B}$ will be called the \emph{derived functor of $F$}, often being denoted with $F$ as well.

Every $k$-linear functor $F\colon \A \to \B$ induces a triangulated functor on the level of homotopy categories $\kcat{\A} \to \kcat{\B}$, defined as follows.
\begingroup
\renewcommand{\arraystretch}{1.5}
\begin{center}
    \begin{tabular}{r l}
        \textsl{Objects:} & $A^\bullet \longmapsto \bigl(\cdots \to F(A^i) \xrightarrow{F(d^i)} F(A^{i+1}) \to \cdots\bigr)$. \\
        \textsl{Morphisms:} & $\left[f\colon A^\bullet \to B^\bullet\right] \longmapsto \left[\left(F(f^i) \colon F(A^i) \to F(B^i)\right)_{i \in \Z}\right]$.
    \end{tabular}
\end{center}
\endgroup
 If moreover $F$ is assumed to be exact, then $\kcat{\A} \to \kcat{\B}$ satisfies the assumptions of Lemma \ref{Exact functor induces a derived functor} and thus induces a triangulated functor $\dercat{\A} \to \dercat{\B}$ on the level of derived categories. We call this the \emph{derived functor of $F$} and often denote it with $F$ as well. We emphasise again that this convention will be used only in the case when $F\colon \A \to \B$ is an \emph{exact} functor between abelian categories.


% ============================================================== %

\subsubsection{Derived functors of left (\resp right) exact functors}
\label{Subsubsection: Derived functors of left exact functors}
% \subsubsection*{$F$-adapted classes}

Assume $F \colon \A \to \B$ is a l\emph{eft} exact functor between abelian categories $\A$ and $\B$. The theory for right exact functors is obtained in parallel by dualizing everything. A crucial downside of functors which are no longer exact, is that their induced functors on the category of complexes no longer satisfy the two equivalent conditions of Lemma \ref{F preserves qis iff preserves acyclic}. The same idea with the universal property of Definition \ref{universal property of D(A)}, which worked in the case of exact functors, therefore cannot be applied in this case. Instead, we will present two other more restrictive methods of constructing right derived functors, which are the following:
\begin{enumerate}[label = $\triangleright$]
    \item Assuming $\A$ has enough injectives, construct a \emph{right derived functor} 
    \[
        {\rderived{}{F} \colon \derplus{\A} \to \derplus{\B}} 
    \]
    for any left exact functor $F \colon \A \to \B$.
    \item Fix a left exact functor $F \colon \A \to \B$ and assume the category $\A$ contains a so-called \emph{$F$-adapted class}, then construct a \emph{right derived functor} $\rderived{}{F} \colon \derplus{\A} \to \derplus{\B}$.
\end{enumerate}
% These methods are however no less restrictive than the case of exact functors with our applications in mind.
These methods are however no less restrictive than the case of exact functors, when viewed through the perspective of our applications later on. 
Lastly we note, that the first method is in some sense a special instance of the second one, which will also be explained in this subsection.
% considered in the context of our applications later on.

% \vspace{0.3 cm}

\noindent
\textsc{Method 1.} 
\label{method 1}
Let $F\colon \A \to \B$ be a left exact functor between abelian categories with $\A$ having enough injectives. By Theorem \ref{D+ = K+ and injectives} we know that the composition of $\kplus{\J} \to \kplus{\A}$ followed by the localization functor $Q_\A \colon \kplus{\A} \to \derplus{\A}$ is an equivalence of triangulated categories. The \emph{right derived functor of $F$} is then defined to be the composition
\begin{equation}
    \label{eq: construction of RF with injectives}
    \rderived{}{F}\colon \quad \derplus{\A} \xrightarrow{ \ \natiso \ } \kplus{\J} \hookrightarrow \kplus{\A} \xrightarrow{ \ F \ } \kplus{\B} \xrightarrow{\ Q_\B \ } \derplus{\B}.
\end{equation}
The functor $\rderived{}{F}$ is clearly triangulated for it is a composition of triangulated functors. Note that in this case the functor $\rderived{}{F}$ does not necessarily make the square \eqref{eq: exact functor derives} commutative, but rather satisfies another kind of universal property, which we will mention towards the end of this subsection.

We will use this construction many times throughout the course of the thesis, so we reiterate the above definition by providing a simple recipe for computing the image of $\rderived{}{F}$ on a complex $A^\bullet$ of $\derplus{\A}$.   
% a complex $A^\bullet$ via the functor $\rderived{}{F}$.
% and conceptually computing the image of a right derived functor $\rderived{}{F}$ on a complex $A^\bullet$ from $\kplus{\A}$ will be achieved using the next two steps. 
\begin{enumerate}
    \item Pick an injective resolution $A^\bullet \to I_A^\bullet$ of $A^\bullet$.
    \item Apply the functor $F$ to the complex $I^\bullet_A$ term-wise to obtain $F(I^\bullet_A)$. As complexes we have
    \[
        \rderived{}{F}(A^\bullet) \iso F(I^\bullet_A).
    \]
\end{enumerate}

\begin{remark}
    \label{Deriving homotopy functors}
    In the slightly more general situation, if one is already given a triangulated functor $F \colon \kplus{\A} \to \kplus{\B}$ at the level of homotopy categories and assumes that $\A$ contains enough injectives, procedure \eqref{eq: construction of RF with injectives} again yields a right derived functor $\rderived{}{F} \colon \derplus{\A} \to \derplus{\B}$. This remark will be especially useful when considering the \emph{$\Hom^\bullet$ complex} functor of Subsection \ref{subsection: Ext functors}, which does not appear as a functor induced by some left exact functor $\A \to \B$ on the level of abelian categories.
\end{remark}

\begin{remark}
    When considering a right exact functor $F \colon \A \to \B$, we instead assume the category $\A$ contains enough projectives. In this case we use the dualized version of Theorem \ref{D+ = K+ and injectives}, stating that $\kminus{\mathcal P} \natiso \derminus{\A}$, where $\mathcal P$ denotes the full subcategory of $\A$, consisting of projective objects of $\A$. The left derived functor is then defined as the composition
    % \[
    %     \lderived{}{F} \colon \derminus{\A} \to \derminus{\B}.
    % \]
    \[
        \lderived{}{F}\colon \quad \derminus{\A} \xrightarrow{ \ \natiso \ } \kminus{\mathcal P} \hookrightarrow \kminus{\A} \xrightarrow{ \ F \ } \kminus{\B} \xrightarrow{\ Q_\B \ } \derminus{\B}.
    \]
    Again, as a composition of triangulated functors, we see that $\lderived{}{F}$ is triangulated as well.
\end{remark}

\noindent
\textsc{Method 2.}
Unfortunately some of our categories will \emph{not} contain enough injectives, as can already be seen with the category of coherent sheaves $\coherent{X}$ of a scheme $X$ (see Example \ref{example of cats w/ and w/o injectives}).
% differing from a point.
% which is not a point. 
In this case \textsc{Method 1} of constructing right derived functors will not work. Luckily however, there exists another way of obtaining a right derived functor $\rderived{}{F} \colon \derplus{\A} \to \derplus{\B}$, assigned to a left exact functor $F: \A \to \B$, when $\A$ does not contain enough injectives, but instead possesses a broader and less restrictive class of objects. To this end we first introduce \emph{$F$-adapted classes}.

\begin{definition}
    \label{F-adapted}
    A class of objects $\I_F \subseteq \A$ is \emph{adapted} to a \textit{left exact}\footnote{
        Dually, a class of objects $\mathcal P_F$ of $\A$ is \emph{adapted} to a \textit{right exact} functor $F \colon \A \to \B$ if it is closed under finite direct sums, every object of $\A$ is a quotient of some object of $\mathcal P_F$, and for every acyclic complex $P^\bullet$ in $\kminus{\A}$, with $P^i \in \mathcal P_F$, its image $F(P^\bullet)$ under $F$ is also acyclic.} 
    functor $F \colon \A \to \B$ if the following three conditions are satisfied.
    \begin{enumerate}[label = (\roman*)]
        \item{$\I_F$ is stable under finite direct sums. 
        % \info{I started calling biproducts direct sums now...}
        } \label{F-adapted i} 
        \item{Every object $A$ of $\A$ \emph{embeds} into some object of $\I_F$, \ie there exists an object $I \in \I_F$ and a monomorphism $A \hookrightarrow I$.} \label{F-adapted ii}
        \item{For every acyclic complex $I^\bullet$ in $\kplus{\A}$, with $I^i \in \I_F$ for all $i \in \Z$, its image $F(I^\bullet)$ under $F$ is also acyclic.} \label{F-adapted iii}
    \end{enumerate}
\end{definition}

\begin{example}
    We observe that whenever $\A$ contains enough injectives, the class of all injective objects forms an $F$-adapted class for \emph{every} left exact functor ${F\colon \A \to \B}$. Points (i) and (ii) are clearly true for the class of injective objects and point (iii) follows from Lemma \ref{acyclic to injective is nulhomotopic}. Indeed, it tells us that an acyclic complex of injectives $I^\bullet$ is null-homotopic, which implies $F(I^\bullet)$ is null-homotopic as well, therefore, in particular, also acyclic.
\end{example}

In the presence of an $F$-adapted class $\I$ we will now construct the right derived functor of $F$. Firstly, we upgrade the class of objects $\I$ to a full $k$-linear subcategory of $\A$, also denoted by $\I$, having its class of objects be precisely the $F$-adapted class $\I$. 
% This is done by declaring $\hom_\J(X, Y) := \hom_\A(X, Y)$ for all objects $X$, $Y$ of the class $\I$. 
Then extending $F$ term-wise to a functor on the level of homotopy categories, also denoted by $F \colon \kplus{\I} \to \kplus{\B}$, utilizing the procedure outlined at the end of Subsection \ref{subsection: Derived functors of exact functors}, we obtain a functor, which maps acyclic complexes to acyclic complexes because by definition the $F$-adapted class $\I$ satisfies condition \ref{F-adapted iii} of Definition \ref{F-adapted}. By Proposition \ref{Exact functor induces a derived functor} the functor $F$ descends to a well-defined functor on the level of derived categories
\begin{equation}
    \label{eq: D+(I) -> D+(B)}
    \derplus{\I} \to \derplus{\B}.
\end{equation}
Since we want to define the derived functor $\rderived{}{F}$, whose domain is $\derplus{\A}$, it remains to construct a functor $\derplus{\A} \to \derplus{\I}$, which we can then post-compose with the functor $\derplus{\I} \to \derplus{\B}$ to obtain $\rderived{}{F}$. What we will show instead is the following.

\begin{proposition}
    \label{D+(I) = D+(A) F-adapted}
    The inclusion of categories $\I \hookrightarrow \A$ induces an equivalence of triangulated categories 
    \[
        \derplus{\I} \iso \derplus{\A}.
    \]
\end{proposition}

The proof of Proposition \ref{D+(I) = D+(A) F-adapted} hinges on the following lemma, closely resembling Lemma \ref{injective resolution}. In the same spirit, we call the complex $I^\bullet$ belonging to $\kplus{\I}$ together with a quasi-isomorphism $A^\bullet \to I^\bullet$ an \emph{$\I$-resolution} of the complex $A^\bullet$.

\begin{lemma}
    \label{F-acyclic resolutions}
    Every complex $A^\bullet$ of $\kplus{\A}$ has an $\I$-resolution.
    % For every complex $A^\bullet$ in $\kplus{\A}$ there is a quasi-isomorphism $A^\bullet \to I^\bullet$, where $I^\bullet$ is a complex in $\kplus{\I}$.
\end{lemma}

\begin{proof}
    A close inspection of the proof of Proposition \ref{injective resolution} reveals that formally only properties \ref{F-adapted i} and \ref{F-adapted ii} of Definition \ref{F-adapted}, which the class of all injective objects clearly satisfies, were used. Thus the same proof can be copied for this lemma.
    % By inspecting the proof of Proposition \ref{injective resolution}, we see, that only conditions \ref{F-adapted i} and \ref{F-adapted ii} of Definition \ref{F-adapted} were actually used, so the proof of this lemma is just a copy 
\end{proof}

\begin{proof}[Proof of Proposition \ref{D+(I) = D+(A) F-adapted}]
    First, there exists a triangulated functor $\derplus{\I} \to \derplus{\A}$ by Proposition \ref{Exact functor induces a derived functor}, since the inclusion $\I \hookrightarrow \A$ induces an inclusion $\kplus{\I} \hookrightarrow \kplus{\A}$ sending acyclic complexes to acyclic ones. We claim that functor $\derplus{\I} \to \derplus{\A}$ is an equivalence, so we will show only that it is full and faithful, since essential surjectivity is already taken care of by Lemma \ref{F-acyclic resolutions}
    
    % Next we show that it is faithful \ie that the morphism action
    % \begin{equation}
    %     \label{eq: morphism action of D(J) - D(A)}
    %     \Hom_{\derplus{\I}}(I^\bullet, J^\bullet) \to \Hom_{\derplus{\A}}(I^\bullet, J^\bullet)
    % \end{equation}

    % Lastly, Lemma \ref{F-acyclic resolutions} asserts that our functor is essentially surjective.

    First, we consider the morphism action 
    \begin{equation}
        \label{eq: morphism action of D(J) - D(A)}
        \Hom_{\derplus{\I}}(I^\bullet, J^\bullet) \to \Hom_{\derplus{\A}}(I^\bullet, J^\bullet)
    \end{equation}
    and show it to be injective for all objects $I^\bullet$, $J^\bullet$ of $\derplus{\I}$.
    % is injective for all objects $I^\bullet$, $J^\bullet$ of $\derplus{\I}$. 
    Suppose two morphisms $\phi_0$, $\phi_1 \colon I^\bullet \to J^\bullet$, represented by \emph{right} roofs $I^\bullet \to I_0^\bullet \xleftarrow{\sim} J^\bullet$ and $I^\bullet \to I_1^\bullet \xleftarrow{\sim} J^\bullet$, are sent to the same morphism in $\Hom_{\derplus{\A}}(I^\bullet, J^\bullet)$, meaning that there is the following commutative diagram in $\kplus{\A}$.
    \[\begin{tikzcd}[row sep = small, column sep = small]
        % https://q.uiver.app/#q=WzAsNSxbMSwxLCJJXzAiXSxbMywxLCJBIl0sWzAsMiwiSSJdLFs0LDIsIkoiXSxbMiwwLCJCIl0sWzIsMF0sWzMsMCwiXFxzaW0iXSxbMiwxXSxbMywxLCJcXHNpbSIsMl0sWzAsNF0sWzEsNCwiXFxzaW0iLDJdXQ==
        && A^\bullet \\
        & {I_0^\bullet} && I_1^\bullet \\
        I^\bullet &&&& J^\bullet
        \arrow[from=2-2, to=1-3]
        \arrow["\sim"', from=2-4, to=1-3]
        \arrow[from=3-1, to=2-2]
        \arrow[from=3-1, to=2-4]
        \arrow["\sim"{pos=0.3}, from=3-5, to=2-2, crossing over]
        \arrow["\sim"', from=3-5, to=2-4]
    \end{tikzcd}\]
    Then simply extending the diagram by a quasi-isomorphism $A^\bullet \to I_2^\bullet$, whose existence is ensured by Lemma \ref{F-acyclic resolutions}, proves that $\phi_0$ and $\phi_1$ were in fact equal and that the action \eqref{eq: morphism action of D(J) - D(A)} is faithful.

    Next, we show that the action on morphisms is also surjective, \ie the homomorphism \eqref{eq: morphism action of D(J) - D(A)} is surjective. Pick any $\psi\colon I^\bullet \to J^\bullet$ in $\derplus{\A}$, which is represented by a right roof $I^\bullet \to A^\bullet \xleftarrow{\sim} J^\bullet$. As before, extending the roof by a quasi-isomorphism $A^\bullet \to I^\bullet_0$ from Lemma \ref{F-acyclic resolutions}, leaves us with a representative $I^\bullet \to I_0^\bullet \xleftarrow{\sim} J^\bullet$ of some morphism $\phi\colon I^\bullet \to J^\bullet$ in $\derplus{\I}$.
    \[\begin{tikzcd}[column sep = small, row sep = small]
        % https://q.uiver.app/#q=WzAsNSxbMSwxLCJJXzAiXSxbMywxLCJBIl0sWzAsMiwiSSJdLFs0LDIsIkoiXSxbMiwwLCJJXzAiXSxbMiwwXSxbMywwLCJcXHNpbSIsMCx7ImxhYmVsX3Bvc2l0aW9uIjo0MH1dLFsyLDFdLFszLDEsIlxcc2ltIiwyXSxbMCw0LCIiLDIseyJsZXZlbCI6Miwic3R5bGUiOnsiaGVhZCI6eyJuYW1lIjoibm9uZSJ9fX1dLFsxLDQsIlxcc2ltIiwyXV0=
        && {I_0^\bullet} \\
        & {I_0^\bullet} && A^\bullet \\
        I^\bullet &&&& J^\bullet
        \arrow[equals, from=2-2, to=1-3]
        \arrow["\sim"', from=2-4, to=1-3]
        \arrow[from=3-1, to=2-2]
        \arrow[from=3-1, to=2-4]
        \arrow["\sim"{pos=0.3}, from=3-5, to=2-2, crossing over]
        \arrow["\sim"', from=3-5, to=2-4]
    \end{tikzcd}\]
    % \[\begin{tikzcd}[column sep = small, row sep = 1em]
    %     % https://q.uiver.app/#q=WzAsNSxbMSwyLCJJXzAiXSxbMywyLCJBIl0sWzAsNCwiSSJdLFs0LDQsIkoiXSxbMiwwLCJJXzAiXSxbMiwwXSxbMywwLCJcXHNpbSIsMCx7ImxhYmVsX3Bvc2l0aW9uIjo0MH1dLFsyLDFdLFszLDEsIlxcc2ltIiwyXSxbMCw0LCIiLDIseyJsZXZlbCI6Miwic3R5bGUiOnsiaGVhZCI6eyJuYW1lIjoibm9uZSJ9fX1dLFsxLDQsIlxcc2ltIiwyXV0=
    %     && {I_0^\bullet} \\
    %     \\
    %     & {I_0^\bullet} && A^\bullet \\
    %     \\
    %     I^\bullet &&&& J^\bullet
    %     \arrow[equals, from=3-2, to=1-3]
    %     \arrow["\sim"', from=3-4, to=1-3]
    %     \arrow[from=5-1, to=3-2]
    %     \arrow[from=5-1, to=3-4]
    %     \arrow["\sim"{pos=0.4}, from=5-5, to=3-2]
    %     \arrow["\sim"', from=5-5, to=3-4]
    % \end{tikzcd}\]
    The above diagram then shows, that $\phi$ is sent to $\psi$ and thus the action on morphisms is surjective.
\end{proof}

\begin{remark}
    Upon close inspection, we recognise Theorem \ref{D+ = K+ and injectives} as a special case of Proposition \ref{D+(I) = D+(A) F-adapted}, namely take the $F$-adapted class $\I$ to be the class of all injectives of $\A$, provided there is enough of them. Indeed, it can be shown that any quasi-isomorphism $f\colon I^\bullet \to J^\bullet$ between two bounded below complexes of injectives is an isomorphism in $\kplus{\I}$. This is done by showing both $f^*\colon \Hom_{\kplus{\I}}(J^\bullet, I^\bullet) \to \Hom_{\kplus{\I}}(I^\bullet, I^\bullet)$ and $f_* \colon \Hom_{\kplus{\I}}(J^\bullet, I^\bullet) \to \Hom_{\kplus{\I}}(J^\bullet, J^\bullet)$ are isomorphisms. Lemma \ref{D+ K+ aux lemma 1} already shows us that $f^*$ is an isomorphism and by a very similar argument as in the proof of the same lemma $f_*$ is as well.
\end{remark}

We can now define the \emph{right derived functor} $\rderived{}{F}$ as the composition
\[
    \rderived{}{F} \colon \quad \derplus{\A} \xrightarrow{\ \iso \ } \derplus{\I} \longrightarrow D^+(B),
\] 
where $\derplus{\A} \xrightarrow{\ \iso \ } \derplus{\I}$ denotes a quasi-inverse to the equivalence $\derplus{\I} \iso \derplus{\A}$ established in Proposition \ref{D+(I) = D+(A) F-adapted} and $\derplus{\I} \to \derplus{\B}$ is the functor \eqref{eq: D+(I) -> D+(B)}.
The procedure for computing the right derived functor $\rderived{}{F}(A^\bullet)$ of a complex $A^\bullet$ in $\derplus{\A}$ is then very similar to \textsc{Method 1}. One takes an $\I$-resolution $A^\bullet \to I^\bullet_A$ of $A^\bullet$ and then applies the functor $F$ to $I^\bullet_A$, \ie
\[
    \rderived{}{F}(A^\bullet) = F(I^\bullet_A).
\]

% \begin{remark}
%     \label{F-adapted subcategory}
%     $F$-adapted subcat?
% \end{remark}

We mention that both methods give rise to right derived functors satisfying the following defining universal property. We only state this here, but refer the reader to \cite[\S III.6]{gelfand2002methods} for the claim and its proof or \cite[\S 13.3, \S 10.3, \S 7.3]{kashiwara2006categories} for a more comprehensive and high level treatment using the formalism of localization. 

\begin{definition}
    \label{Universal property for RF}
    % [\text{\cite[\S III.6, Definition 6]{gelfand2002methods}}]
    Let $F \colon \A \to \B$ be a left exact functor between abelian categories. The \emph{right derived functor of $F$} is a triangulated functor $\rderived{}{F} \colon \derplus{\A} \to \derplus{\B}$ together with a natural transformation $\varepsilon \colon Q_\B \circ F \Longrightarrow \rderived{}{F} \circ Q_\A$, such that for any triangulated functor $G \colon \derplus{\A} \to \derplus{\B}$ and any natural transformation $\theta \colon Q_\B \circ F \Longrightarrow G \circ Q_\A$, there exists a unique natural transformation $\eta \colon \rderived{}{F} \Longrightarrow G$, for which
    \[\begin{tikzcd}[column sep = small]
        % https://q.uiver.app/#q=WzAsMyxbMCwxLCJhIl0sWzEsMCwiYiJdLFsyLDEsImMiXSxbMSwwLCJ4IiwyLHsibGV2ZWwiOjJ9XSxbMSwyLCJ5IiwwLHsibGV2ZWwiOjJ9XSxbMCwyLCJ6IiwwLHsibGV2ZWwiOjJ9XV0=
        & Q_\B \circ F \\
        \rderived{}{F} \circ Q_\A && G \circ Q_\A
        \arrow["\varepsilon"', Rightarrow, from=1-2, to=2-1]
        \arrow["\theta", Rightarrow, from=1-2, to=2-3]
        \arrow["\eta Q_\A", Rightarrow, from=2-1, to=2-3]
    \end{tikzcd}\]
    is commutative in the functor category $\operatorname{Fct}(\kplus{\A}, \derplus{\B})$.  
\end{definition}

% \noindent
% \textsc{Method 2$^*$.} \info{I think I will delete this method 2$^*$.} There is one instance in this thesis in which the two described methods will not suffice, namely in deriving the \emph{differentialy graded} $\Hom^\bullet$-functor. In this case our starting point will be a triangulated functor already on the level of homotopy categories $F \colon \kplus{\A} \to \kplus{\B}$, which does not originate from any functor at the level of abelian categories $\A \to \B$. Analogous to Definition \ref{F-adapted}, we introduce the following.
% \begin{definition}
%     \label{F-adapted for homotopy category}
%     Let $F \colon \kplus{\A} \to \kplus{\B}$ be a triangulated functor. A full triangulated subcategory $\mathcal K_F \subseteq \kplus{\A}$ is siad to be \emph{$F$-adapted} if the following two conditions hold.
%     \begin{enumerate}[label = (\roman*)]
%         \item{For every complex $A^\bullet$ in $\kplus{\A}$, there is a } \label{F-adapted for homotopy category i}
%         \item{For every acyclic complex $A^\bullet$ of $\mathcal K_F$, the complex $F(A^\bullet)$ is acyclic as well.} \label{F-adapted for homotopy category ii}
%     \end{enumerate}
% \end{definition}

% In order to construct the right derived functor $\rderived{}{F} \colon \derplus{\A} \to \derplus{\B}$, we assume the existence of an $F$-adapted category $\mathcal K_F \subseteq \kplus{\A}$ and introduce the full additive subcategory $\D_F \subseteq \derplus{\A}$ spanned on objects of $\mathcal K_F$. Proposition \ref{triangulated subcategory proposition} guarantees that $\D_F$ is a full triangulated subcategory of $\derplus{\A}$ and is moreover equivalent to the latter by \ref{F-adapted for homotopy category i} of Definition \ref{F-adapted for homotopy category}. Then we can extend the funtor $F$ to a functor $\D_F \to \derplus{\B}$ given by the assignment
% \begin{center}
%     \begin{tabular}{r r c l}
%         \textsl{Objects:} & $A^\bullet$ & $\longmapsto$ & $F(A^\bullet)$. \\
%         \textsl{Morphisms:} & $[A^\bullet \xleftarrow{s} C^\bullet \xrightarrow{f} B^\bullet]$ & $\longmapsto$ & $Q_\B F(f) \circ Q_\B F(s)\inv$.
%     \end{tabular}
% \end{center}
% This is a well-defined functor because condition \ref{F-adapted for homotopy category ii} of Definition \ref{F-adapted for homotopy category} ensures $F$ sends quasi-isomorphisms to quasi-isomorphisms by Proposition \ref{F preserves qis iff preserves acyclic}.
% Lastly the right derived functor is then given as the composition
% \[
%     \rderived{}{F} \colon \quad \derplus{\A} \xrightarrow{\ \iso \ } \D_F \longrightarrow \derplus{\B}.
% \]
% \begin{example}
%     When $\A$ contains enough injectives, the category $\kplus{\J} \subseteq \kplus{\A}$ is $F$-adapted for any triangulated functor $F \colon \kplus{\A} \to \kplus{\B}$.
% \end{example}

% \vspace{3cm}

For two composable right exact functors the following proposition tells us that first deriving and then composing, or first composing and then deriving gives essentially the same result.

\begin{proposition}
    \label{composition of derived functors}
    \textsl{\cite[\S III.7, Theorem 1]{gelfand2002methods}}
    % Let $F \colon \kcat{\A} \to \kcat{\B}$ and $G \colon \kcat{\B} \to \kcat{\mathcal C}$ be triangulated. 
    Let $F \colon \A \to \B$ and $G \colon \B \to \mathcal{C}$ be left exact functors between abelian categories. Suppose category $\A$ admits an $F$-adapted class $\I_F$, category $\B$ admits a $G$-adapted class $\I_G$ and assume that every object of $\I_F$ is sent to $\I_G$ by $F$. Then $\rderived{}{F}$, $\rderived{}{G}$ and $\rderived{}{(G \circ F)}$ exist and there is a natural isomorphism of functors
    \[
        \rderived{}{(G \circ F)} \natiso \rderived{}{G} \circ \rderived{}{F} \colon \quad \derplus{\A} \to \derplus{\mathcal C}.
    \] 
\end{proposition}

\begin{proof}
    Right derived functors $\rderived{}{F} \colon \derplus{\A} \to \derplus{\B}$ and $\rderived{}{G} \colon \derplus{\B} \to \derplus{\mathcal C}$ exist because $\A$ (\resp $\B$) admits an $F$-adapted (\resp $G$-adapted) class. To see that $\rderived{}{(G \circ F)}$ exists we show that $\I_F$ is also an $(G\circ F)$-adapted class. Indeed, $\I_F$ is stable under finite sums and for every object $A$ of $\A$ there is an embedding $A \hookrightarrow I$ into some object $I$ belonging to $\I_F$. Further, assuming $I^\bullet$ is an acyclic complex with $I^i$ from $\I_F$ for all $i \in \Z$, we see that $F(I^\bullet)$ is acyclic, because $\I_F$ is $F$-adapted. For all $i\in \Z$ we know $F(I^i)$ belong to $\I_G$ by assumption, therefore we see that $G(F(I^\bullet))$ is acyclic, because $\I_G$ is $G$-adapted.

    Showing that $\rderived{}{(G \circ F)}$ and $\rderived{}{G} \circ \rderived{}{F}$ are naturally isomorphic will be an exercise in using the universal property of Definition \ref{Universal property for RF}. First denote the following natural transformations 
    \[
        \alpha \colon Q_\B \circ F \implies \rderived{}{F}\circ Q_\A, \qquad \beta \colon Q_{\mathcal C}\circ G \implies \rderived{}{G}\circ Q_\B,
    \]   
    \[
        \varepsilon \colon Q_{\mathcal C} \circ (G\circ F) \implies \rderived{}{(G \circ F)} \circ Q_\A.
    \]
    Define a natural transformation $\theta := \rderived{}{G}\alpha \circ \beta F$, this means $\theta_{A^\bullet} = \rderived{}{G}(\alpha_{A^\bullet})\circ \beta_{F(A^\bullet)}$ for any object $A^\bullet$ of $\kplus{\A}$.
    By the universal property of $\rderived{}{(G\circ F)}$ there exists a natural transformation $\eta \colon \rderived{}{(G \circ F)} \implies \rderived{}{G} \circ \rderived{}{F}$ such that the following diagram of natural transformations commutes.
     \[\begin{tikzcd}[column sep = small]
        % https://q.uiver.app/#q=WzAsMyxbMCwxLCJhIl0sWzEsMCwiYiJdLFsyLDEsImMiXSxbMSwwLCJ4IiwyLHsibGV2ZWwiOjJ9XSxbMSwyLCJ5IiwwLHsibGV2ZWwiOjJ9XSxbMCwyLCJ6IiwwLHsibGV2ZWwiOjJ9XV0=
        & Q_\mathcal{C} \circ (G \circ F) \\
        \rderived{}{(G \circ F)} \circ Q_\A && (\rderived{}{G} \circ \rderived{}{F}) \circ Q_\A
        \arrow["\varepsilon"', Rightarrow, from=1-2, to=2-1]
        \arrow["\theta", Rightarrow, from=1-2, to=2-3]
        \arrow["\eta Q_\A", Rightarrow, from=2-1, to=2-3]
    \end{tikzcd}\]
    We will show that $\eta$ is an isomorphism and we will do so by showing that $\eta_{A^\bullet}$ is an isomorphism for every $A^\bullet$ of $\derplus{\A}$. Since $A^\bullet$ has an $\I_F$-resolution $A^\bullet \to I^\bullet$ and $\eta$ is a natural transformation, we may show that for any $I^\bullet$ belonging to $\derplus{\I_F}$, $\eta_{I^\bullet}$ is an isomorphism instead. In that case we have the following commutative diagram in $\derplus{\mathcal C}$.
    \begin{equation*}
        \begin{tikzcd}[column sep = 1.5em]
        % https://q.uiver.app/#q=WzAsNCxbMCwwLCJLKFxcQSkiXSxbMiwwLCJLKFxcQikiXSxbMCwyLCJEKFxcQSkiXSxbMiwyLCJEKFxcQikiXSxbMCwxLCJGIl0sWzAsMl0sWzIsM10sWzEsM11d
        {(Q_{\mathcal C}\circ G\circ F)(I^\bullet)} && {(\rderived{}{G}\circ Q_\B)(F(I^\bullet))} \\
        \\
        {\rderived{}{(G \circ F)}(Q_\A(I^\bullet))} && {(\rderived{}{G}\circ \rderived{}{F})(Q_\A(I^\bullet))}
        \arrow["\beta_{F(I^\bullet)}", from=1-1, to=1-3]
        \arrow["{\varepsilon_{I^\bullet}}"', from=1-1, to=3-1]
        \arrow["{\rderived{}{G}(\alpha_{I^\bullet})}", from=1-3, to=3-3]
        \arrow["\eta_{I^\bullet}", from=3-1, to=3-3]
        \end{tikzcd}
    \end{equation*}
    From the construction of the ``structural'' natural transformations $\alpha$, $\beta$ and $\varepsilon$ \cite[\S III, 6.4]{gelfand2002methods} one sees that whenever these are applied to 
    appropriately addapted resolutions, they become isomorphisms.
    % complexes consisting of terms belonging to adapted classes, they become isomorphisms. 
    In our case, since $\I_F$ is both $F$-adapted and $(G \circ F)$-adapted and $\I_G$ is $G$-adapted, this means that $\varepsilon_{I^\bullet}$, $\beta_{F(I^\bullet)}$ and $\alpha_{I^\bullet}$ are all isomorphisms, allowing us to conclude that $\eta_{I^\bullet}$ is an isomorphism as well.  \qedhere

    % By our construction of $\rderived{}{F}$ and $\rderived{}{G}$ and the construction of natural transformations $\alpha$ and $\beta$ from \cite[\S III, 6.4]{gelfand2002methods}, we see that $\alpha_{I^\bullet}$ and $\beta_{J^\bullet}$ are isomorphisms for any complexes $I^\bullet$ and $J^\bullet$ belonging to the adapted classes $\I_F$ and $\I_G$, respectively.


    % To show $\rderived{}{(G \circ F)} \natiso \rderived{}{G} \circ \rderived{}{F}$ one considers the following diagram of functors.
    % \[\begin{tikzcd}[row sep = small]
    %     % https://q.uiver.app/#q=WzAsOCxbMCw1LCJhIl0sWzEsNCwiXFxJX0YiXSxbMiwzLCJhIl0sWzMsMiwiYiJdLFs0LDAsImIiXSxbMiwwLCJcXElfRyJdLFs0LDMsImMiXSxbNSw0LCJjIl0sWzAsMSwiXFxzaW0iXSxbMSwyXSxbMiwzLCJGIl0sWzMsNCwiUV97XFxCfSIsMl0sWzQsNV0sWzUsMywiXFxzaW0iLDJdLFszLDYsIkciXSxbNiw3LCJRX3tcXG1hdGhjYWwgQ30iXV0=
    %     && \kplus{\I_G} && \derplus{\B} \\
    %     \\
    %     &&& \kplus{\B} \\
    %     && \kplus{\A} && \kplus{\mathcal C} \\
    %     & {\I_F} &&&& \derplus{\mathcal C} \\
    %     \derplus{\A}
    %     \arrow["\sim"', from=1-3, to=3-4]
    %     \arrow[from=1-5, to=1-3]
    %     \arrow["{Q_{\B}}"', from=3-4, to=1-5]
    %     \arrow["G", from=3-4, to=4-5]
    %     \arrow["F", from=4-3, to=3-4]
    %     \arrow["{Q_{\mathcal C}}", from=4-5, to=5-6]
    %     \arrow[from=5-2, to=4-3]
    %     \arrow["\sim", from=6-1, to=5-2]
    % \end{tikzcd}\]
    % https://q.uiver.app/#q=WzAsOCxbMCwzLCJhIl0sWzEsMywiXFxJX0YiXSxbMiwzLCJhIl0sWzMsMiwiYiJdLFs0LDAsImIiXSxbMiwwLCJcXElfRyJdLFs0LDMsImMiXSxbNSwzLCJjIl0sWzAsMSwiXFxzaW0iXSxbMSwyXSxbMiwzLCJGIl0sWzMsNCwiUV97XFxCfSIsMCx7ImN1cnZlIjoyfV0sWzQsNSwiIiwwLHsiY3VydmUiOjJ9XSxbNSwzLCJcXHNpbSIsMCx7ImN1cnZlIjoyfV0sWzMsNiwiRyJdLFs2LDcsIlFfe1xcbWF0aGNhbCBDfSJdXQ==
% \[\begin{tikzcd}[row sep = small]
% 	&& \kplus{\I_G} & & \derplus{\B} \\
% 	&&& (\bigcirc ) \\
% 	&&& \kplus{\B} \\
% 	\derplus{\A} & \kplus{\I_F} & \kplus{\A} && \kplus{\mathcal C} & \derplus{\mathcal C}
% 	\arrow["", curve={height=12pt}, from=1-3, to=3-4]
% 	\arrow["\sim", curve={height=20pt}, from=1-5, to=1-3]
% 	\arrow["{Q_{\B}}", curve={height=12pt}, from=3-4, to=1-5]
% 	\arrow["G", from=3-4, to=4-5]
% 	\arrow["\sim", from=4-1, to=4-2]
% 	\arrow[from=4-2, to=4-3]
% 	\arrow["F", from=4-3, to=3-4]
% 	\arrow["{Q_{\mathcal C}}", from=4-5, to=4-6]
% \end{tikzcd}\]
% By definition computing $\rderived{}{(G \circ F)}$ amounts to following the left-to-right arrows at the bottom, while computing $\rderived{}{G} \circ \rderived{}{F}$ requires one to also go around the loop $(\bigcirc)$ once. Since starting at $\kplus{\B}$ and doing the loop $(\bigcirc)$ is naturally isomorphic to the functor $\id{\kplus{\B}}$, the result follows.
\end{proof}

\subsubsection{Higher derived functors}

Classically derived functors of a left exact functor $F \colon \A \to \B$ first appeared as a sequence of $k$-linear functors $\A \to \B$. We introduce them in the following definition, using our established viewpoint. 

\begin{definition}
    For a right derived functor $\rderived{}{F} \colon \derplus{\A} \to \derplus{\B}$ of a left exact functor $F \colon \A \to \B$, we define the \emph{higher derived functors of} $F$ as compositions
    \[
        \rderived{i}{F} := H^i \circ \rderived{}{F} \colon \quad \derplus{\A} \to \B
    \]
    for all $i \in \Z$. 
\end{definition}

Frequently we will be interested in the image of a higher derived functor $\rderived{i}{F}$ of just a single object $A$ belonging to category $\A$. In that case we will always interpret it as a bounded complex concentrated in degree $0$ according to the fully faithful functor $\A \to \dercat{\A}$ of Proposition \ref{A -> D(A) fully faithful}. 

\begin{example}
    For an object $A$ of $\A$, equipped with an $F$-adapted class $\I$, we can compute that $\rderived{i}{F}(A) \iso 0$, for $i < 0$, and $\rderived{0}{F}(A) \iso F(A)$. 
    % Indeed, considering $A$ as a complex concentrated in degree $0$, let $I^\bullet$ be its $\I$-resolution. 
    Indeed, consider an $\I$-resolution $I^\bullet$ of $A$.
    Then, as $I^i = 0$ for $i < 0$, we have that $\rderived{i}{F}(A) = H^i(F(I^\bullet)) \iso 0$ for $i < 0$. For the second claim, because $I^\bullet$ is quasi-isomorphic to $A$, we have $\ker(I^0 \to I^1) \iso A$, and since $F$ is left exact, we conclude that $F(A) \iso \ker(F(I^0) \to F(I^1)) \iso H^0(F(I^\bullet)) = \rderived{0}{F}(A)$.
\end{example}

\begin{definition}
    \label{F-acyclic}
    Suppose the right derived functor $\rderived{}{F}$ exists. We say an object $A$ of $\A$ is \emph{$F$-acyclic} if $\rderived{i}{F}(A) \iso 0$ for all $i \neq 0$.
\end{definition}

A useful tool for gathering information about the action of the higher derived functors of a left exact functor $F$ on objects will be the long exact sequence associated to $F$ introduced in the following proposition. In view of this proposition we also interpret higher derived functors of $F$ as measuring the extent to which $F$ fails to be exact. 

\begin{proposition}
    \label{LES for right derived functor}
    Let $F \colon \A \to \B$ be a left exact functor and let $0 \to A \to B \to C \to 0$ be a short exact sequence in $\A$. Then there exists a long exact sequence
    \begin{multline}
        \label{eq: LES for left exact functor F}
        0 \to F(A) \to F(B) \to F(C) \to \rderived{1}{F}(A) \to \rderived{1}{F}(B) \to \rderived{1}{F}(C) \to \cdots \\
        \cdots \to \rderived{i}{F}(A) \to \rderived{i}{F}(B) \to \rderived{i}{F}(C) \to \rderived{i+1}{F}(A) \to \cdots.
    \end{multline}
\end{proposition}

\begin{proof}
    First, we describe how the short exact sequence gives rise to a distinguished triangle $A^\bullet \to B^\bullet \to C^\bullet \to A[1]^\bullet$ in $\derplus{\A}$, where $A^\bullet$, $B^\bullet$ and $C^\bullet$ are objects $A$, $B$ and $C$, respectively, considered as complexes concentrated in degree $0$. First denote the morphisms $f \colon A \to B$ and $g \colon B \to C$. Then the cone $C(f)^\bullet$ can be used as a peak for a roof spanned by $C^\bullet$ and $A[1]^\bullet$ as follows.
    % \[\begin{tikzcd}
    %     % https://q.uiver.app/#q=WzAsMjEsWzAsMiwiQyhmKSJdLFswLDEsIkMiXSxbMCwzLCJBWzFdIl0sWzEsMSwiXFxkb3RzIl0sWzIsMSwiMCJdLFszLDEsIkMiXSxbNCwxLCIwIl0sWzUsMSwiXFxkb3RzIl0sWzEsMiwiXFxkb3RzIl0sWzIsMiwiQSJdLFsxLDMsIlxcZG90cyJdLFszLDIsIkIiXSxbMiwzLCJBIl0sWzQsMiwiMCJdLFs1LDIsIlxcY2RvdHMiXSxbMywzLCIwIl0sWzQsMywiMCJdLFs1LDMsIlxcY2RvdHMiXSxbMiwwLCItMSJdLFszLDAsIjAiXSxbNCwwLCIxIl0sWzAsMV0sWzAsMl0sWzksMTJdLFs5LDRdLFsxMSw1LCJnIiwyXSxbMTEsMTVdLFsxMywxNl0sWzEzLDZdLFs5LDExLCJmIl0sWzgsOV0sWzMsNF0sWzQsNV0sWzUsNl0sWzExLDEzXSxbNiw3XSxbMTMsMTRdLFsxNiwxN10sWzE1LDE2XSxbMTIsMTVdLFsxMCwxMl1d
    %     & & &  {\color{linkcolor}{\mathtt{-1}}} & {\color{linkcolor}{\mathtt{0}}} & {\color{linkcolor}{\mathtt{1}}} & \\
    % \end{tikzcd}\]
    % \vspace{-0.8cm}
    \[\begin{tikzcd}
        % https://q.uiver.app/#q=WzAsMTgsWzAsMSwiQyhmKSJdLFswLDAsIkMiXSxbMCwyLCJBWzFdIl0sWzEsMCwiXFxjZG90cyJdLFsyLDAsIjAiXSxbMywwLCJDIl0sWzQsMCwiMCJdLFs1LDAsIlxcY2RvdHMiXSxbMSwxLCJcXGNkb3RzIl0sWzIsMSwiQSJdLFsxLDIsIlxcY2RvdHMiXSxbMywxLCJCIl0sWzIsMiwiQSJdLFs0LDEsIjAiXSxbNSwxLCJcXGNkb3RzIl0sWzMsMiwiMCJdLFs0LDIsIjAiXSxbNSwyLCJcXGNkb3RzIl0sWzAsMV0sWzAsMl0sWzksMTJdLFs5LDRdLFsxMSw1LCJnIiwyXSxbMTEsMTVdLFsxMywxNl0sWzEzLDZdLFs5LDExLCJmIl0sWzgsOV0sWzMsNF0sWzQsNV0sWzUsNl0sWzExLDEzXSxbNiw3XSxbMTMsMTRdLFsxNiwxN10sWzE1LDE2XSxbMTIsMTVdLFsxMCwxMl1d
        C^\bullet & \cdots & 0 & C & 0 & \cdots \\
        {C(f)^\bullet} & \cdots & A & B & 0 & \cdots \\
        {A[1]^\bullet} & \cdots & A & 0 & 0 & \cdots
        \arrow[from=1-2, to=1-3]
        \arrow[from=1-3, to=1-4]
        \arrow[from=1-4, to=1-5]
        \arrow[from=1-5, to=1-6]
        \arrow[from=2-1, to=1-1]
        \arrow[from=2-1, to=3-1]
        \arrow[from=2-2, to=2-3]
        \arrow[from=2-3, to=1-3]
        \arrow["f", from=2-3, to=2-4]
        \arrow["\id{A}"', from=2-3, to=3-3]
        \arrow["g"', from=2-4, to=1-4]
        \arrow["d^0", from=2-4, to=2-5]
        \arrow[from=2-4, to=3-4]
        \arrow[from=2-5, to=1-5]
        \arrow[from=2-5, to=2-6]
        \arrow[from=2-5, to=3-5]
        \arrow[from=3-2, to=3-3]
        \arrow[from=3-3, to=3-4]
        \arrow[from=3-4, to=3-5]
        \arrow[from=3-5, to=3-6]
    \end{tikzcd}\]
    % \[\begin{tikzcd}
    %     % https://q.uiver.app/#q=WzAsMjEsWzAsMiwiQyhmKSJdLFswLDEsIkMiXSxbMCwzLCJBWzFdIl0sWzEsMSwiXFxkb3RzIl0sWzIsMSwiMCJdLFszLDEsIkMiXSxbNCwxLCIwIl0sWzUsMSwiXFxkb3RzIl0sWzEsMiwiXFxkb3RzIl0sWzIsMiwiQSJdLFsxLDMsIlxcZG90cyJdLFszLDIsIkIiXSxbMiwzLCJBIl0sWzQsMiwiMCJdLFs1LDIsIlxcY2RvdHMiXSxbMywzLCIwIl0sWzQsMywiMCJdLFs1LDMsIlxcY2RvdHMiXSxbMiwwLCItMSJdLFszLDAsIjAiXSxbNCwwLCIxIl0sWzAsMV0sWzAsMl0sWzksMTJdLFs5LDRdLFsxMSw1LCJnIiwyXSxbMTEsMTVdLFsxMywxNl0sWzEzLDZdLFs5LDExLCJmIl0sWzgsOV0sWzMsNF0sWzQsNV0sWzUsNl0sWzExLDEzXSxbNiw3XSxbMTMsMTRdLFsxNiwxN10sWzE1LDE2XSxbMTIsMTVdLFsxMCwxMl1d
    %     % & \color{linkcolor}\cdots &  {\color{linkcolor}{\mathtt{-1}}} & {\color{linkcolor}{\mathtt{0}}} & {\color{linkcolor}{\mathtt{1}}} & \color{linkcolor}\cdots \\
    %     & & & & \\
    %     C & \cdots & 0 & C & 0 & \cdots \\
    %     {C(f)} & \cdots & A & B & 0 & \cdots \\
    %     {A[1]} & \cdots & A & 0 & 0 & \cdots
    %     \arrow[from=2-2, to=2-3]
    %     \arrow[from=2-3, to=2-4]
    %     \arrow[from=2-4, to=2-5]
    %     \arrow[from=2-5, to=2-6]
    %     \arrow[from=3-1, to=2-1]
    %     \arrow[from=3-1, to=4-1]
    %     \arrow[from=3-2, to=3-3]
    %     \arrow[from=3-3, to=2-3]
    %     \arrow["f", from=3-3, to=3-4]
    %     \arrow[from=3-3, to=4-3]
    %     \arrow["g"', from=3-4, to=2-4]
    %     \arrow[from=3-4, to=3-5]
    %     \arrow[from=3-4, to=4-4]
    %     \arrow[from=3-5, to=2-5]
    %     \arrow[from=3-5, to=3-6]
    %     \arrow[from=3-5, to=4-5]
    %     \arrow[from=4-2, to=4-3]
    %     \arrow[from=4-3, to=4-4]
    %     \arrow[from=4-4, to=4-5]
    %     \arrow[from=4-5, to=4-6]
    % \end{tikzcd}\]
    Then the morphism $C(f)^\bullet \to C^\bullet$ is a quasi-isomorphism. Indeed, since $f$ is a monomorphism, we have $H^{-1}(C(f)^\bullet) \iso 0$, thus the induced map on cohomology in degree $-1$ is the trivial morphism, which in this case an isomorphism. Since $B \iso \ker d^0$, we have $H^0(C(f)^\bullet) \iso \coker f$. As $0 \to A \to B \to C \to 0$ is exact, we know that $g$ induces an isomorphism $\coker f \to C$, but this is precisely the morphism $H^0(C(f)^\bullet \to C^\bullet)$. Therefore $C(f)^\bullet \to C^\bullet$ is a quasi-isomorphism and the left roof
    \[
        C^\bullet \longleftarrow C(f)^\bullet \longrightarrow A[1]^\bullet
    \] 
    represents a morphism $\phi \colon C^\bullet \to A[1]^\bullet$ in $\derplus{\A}$. In $\derplus{\A}$ we thus obtain a commutative diagram, where the vertical arrows are isomorphisms.
    \[\begin{tikzcd}[column sep = normal, row sep = 2.5em]
        % https://q.uiver.app/#q=WzAsOCxbMSwwLCJCXlxcYnVsbGV0Il0sWzIsMCwiQyhmKV5cXGJ1bGxldCJdLFsxLDEsIkJeXFxidWxsZXQiXSxbMiwxLCJDKHMpXlxcYnVsbGV0Il0sWzAsMSwiQV5cXGJ1bGxldCJdLFszLDEsIkFbMV1eXFxidWxsZXQiXSxbMywwLCJBWzFdXlxcYnVsbGV0Il0sWzAsMCwiQV5cXGJ1bGxldCJdLFswLDFdLFsyLDMsIlxcdGF1Il0sWzQsMl0sWzMsNSwiXFxwaGkiXSxbMSw2XSxbNywwXSxbMSwzLCJcXHNpbSIsMl0sWzYsNSwiIiwxLHsibGV2ZWwiOjIsInN0eWxlIjp7ImhlYWQiOnsibmFtZSI6Im5vbmUifX19XSxbMCwyLCIiLDEseyJsZXZlbCI6Miwic3R5bGUiOnsiaGVhZCI6eyJuYW1lIjoibm9uZSJ9fX1dLFs3LDQsIiIsMSx7ImxldmVsIjoyLCJzdHlsZSI6eyJoZWFkIjp7Im5hbWUiOiJub25lIn19fV1d
        {A^\bullet} & {B^\bullet} & {C(f)^\bullet} & {A[1]^\bullet} \\
        {A^\bullet} & {B^\bullet} & {C^\bullet} & {A[1]^\bullet}
        \arrow[from=1-1, to=1-2]
        \arrow[equals, from=1-1, to=2-1]
        \arrow[from=1-2, to=1-3]
        \arrow[equals, from=1-2, to=2-2]
        \arrow[from=1-3, to=1-4]
        \arrow["\sim"', from=1-3, to=2-3]
        \arrow[equals, from=1-4, to=2-4]
        \arrow[from=2-1, to=2-2]
        \arrow["", from=2-2, to=2-3]
        \arrow["\phi", from=2-3, to=2-4]
    \end{tikzcd}\]
    This shows that the bottom triangle is distinguished. 
    
    We can now succesively apply the triangulated functor $\rderived{}{F}$ and the cohomological functor $H^0$ over the distinguished triangle $A^\bullet \to B^\bullet \to C^\bullet \to A[1]^\bullet$ to obtain the following long exact sequence in cohomology by Proposition \ref{LES in cohomology}.
    % After first applying the triangulated functor $\rderived{}{F}$ and then the cohomological functor $H^0$ to the triangle $A \to B \to C \to A[1]$, we obtain a long exact sequence 
    \begin{multline*}
        0 \to H^0(\rderived{}{F}(A)) \to H^0(\rderived{}{F}(B)) \to H^0(\rderived{}{F}(C)) \to H^0(\rderived{}{F}(A)[1]) \to \cdots \\
        \cdots \to H^0(\rderived{}{F}(A)[i]) \to H^0(\rderived{}{F}(B)[i]) \to H^0(\rderived{}{F}(C)[i]) \to H^0(\rderived{}{F}(A)[i+1]) \to \cdots
    \end{multline*}
    The terms of this sequence are then readily seen to be isomorphic to the terms of \eqref{eq: LES for left exact functor F}. \qedhere
\end{proof}

The rest of this section is devoted to proving some useful results relating $F$-acyclic and $F$-adapted objects. 

\begin{lemma}
    \label{F-adapted is F-acyclic}
    Every object of an $F$-adapted class is also $F$-acyclic.
\end{lemma}

\begin{proof}
    We only have to recall Definition \ref{F-acyclic}. An object $I$ belonging to an $F$-adapted class $\I$ trivially forms its own $\I$-resolution $I^\bullet = (\cdots \to 0 \to I \to 0 \to \cdots)$, which allows us to compute $\rderived{i}{F}(I) = H^i(\rderived{}{F}(I^\bullet)) = H^i(F(I^\bullet)) \iso 0$, whenever $i \neq 0$ (and also $F(I)$ for $i = 0$).
\end{proof}

\begin{proposition}
    \label{F-acyclic is F-adapted}
    Let $F : \A \to \B$ be a left exact functor between two abelian categories. Let $\I \subset \A$ be an $F$-adapted class of objects in $\A$. Then the class of all $F$-acyclic objects of $\A$, denoted by $\I_F$, is also $F$-adapted. 
\end{proposition}

\begin{proof}
    We verify conditions (i)--(iii) of Definition \ref{F-adapted}. (i) Since the higher derived functors $\rderived{i}{F}$ are $k$-linear, $\I_F$ is stable under finite sums. (ii) Holds by Lemma \ref{F-adapted is F-acyclic}. (iii) Suppose $A^\bullet$ is an acyclic complex in $K^+(\A)$ with $A^i \in \I_F$ for all $i$. Then using acyclicity of $A^\bullet$, we may break this complex up into a series of short exact sequences 
    % as follows.
    % \begin{align*}
    %     0 \to A^0 \to & A^1 \to \ker d^2 \to 0 \\
    %     0 \to \ker d^2 \to & A^2 \to \ker d^3 \to 0 \\
    %     & \vdots
    % \end{align*} 
    \begin{equation}
        \label{eq: SES from acyclic cx}
        0 \to \ker d^i \to A^i \to \ker d^{i+1} \to 0 \quad \text{ for $i \in \Z$.}
    \end{equation}
    As our complex $A^\bullet$ is supported on $\N$, the first interesting short exact sequence happens at index $i = 1$, where $\ker d^1 = \im d^0 \iso A^0$
    It yields the following long exact sequence associated to $F$
    \begin{multline*}
        0 \to F(A^0) \to F(A^1) \to F(\ker d^2) \to \rderived{1}{F}(A^0) \to \cdots \\ \cdots \to \rderived{i}{F}(A^0) \to \rderived{i}{F}(A^1) \to \rderived{i}{F}(\ker d^2) \to \rderived{i+1}{F}(A^0) \to \cdots.
    \end{multline*}
    Exactness of this sequence together with $F$-acyclicity of $A^0$ and $A^1$ shows that $\ker d^2$ is also $F$-acyclic and by only considering the beginning few terms of the sequence, we obtain the short exact sequence 
    % From the fact that $A^0$ and $A^1$ are $F$-acyclic we see that $\ker d^2$ is $F$-acyclic and we obtain a short exact sequence 
    \[
        0 \to F(A^0) \to F(A^1) \to F(\ker d^2) \to 0.
    \]
    After inductively applying the same reasoning for each of the original short exact sequences of \eqref{eq: SES from acyclic cx} for $i > 1$, we are left with short exact sequences\footnote{To be more precise, we have used that $F$ is left exact in order to swap the order of $F$ and $\ker$ to reach \[\ker Fd^i \iso F(\ker d^i).\]}
    \[
        0 \to \ker Fd^i \to F(A^i) \to \ker Fd^{i+1} \to 0 \quad \text{ for $i \geq 1$.}
    \]
    Collecting them all back together we conclude that the complex $F(A^\bullet)$ is acyclic.
    % \info{There are inconsistencies with $F(A^\bullet) = K^+(F)(A^\bullet)$ and so on. $\to$ I am going to abuse notation and use $F(A^\bullet)$, bc this is also done in practise \eg $f_*\F^\bullet$, not $K^+(f_*)(\F^\bullet)$} \qedhere
\end{proof}

% \[\begin{tikzcd}[column sep = small, row sep = 1.8em]
%     % https://q.uiver.app/#q=WzAsNSxbMCwwLCJcXGtwbHVze1xcSn0iXSxbMiwwLCJcXGtwbHVze1xcQX0iXSxbMiwyLCJcXGRlcnBsdXN7XFxBfSJdLFs0LDAsIlxca3BsdXN7XFxCfSJdLFs0LDIsIlxcZGVycGx1c3tcXEJ9Il0sWzEsMiwiUV9cXEEiXSxbMiwwLCJcXG5hdGlzbyIsMl0sWzAsMSwiXFxpb3RhIiwwLHsic3R5bGUiOnsidGFpbCI6eyJuYW1lIjoiaG9vayIsInNpZGUiOiJ0b3AifX19XSxbMSwzLCJGIl0sWzMsNCwiUV9cXEIiXV0=
% 	{\kplus{\J}} && {\kplus{\A}} && {\kplus{\B}} \\
% 	\\
% 	&& {\derplus{\A}} && {\derplus{\B}}
% 	\arrow["\iota", hook, from=1-1, to=1-3]
% 	\arrow["F", from=1-3, to=1-5]
% 	\arrow["{Q_\A}", from=1-3, to=3-3]
% 	\arrow["{Q_\B}", from=1-5, to=3-5]
% 	\arrow["\natiso"', from=3-3, to=1-1]
% \end{tikzcd}\]

% In th subsection we will deal with the case, when $F$ is only left exact. The theory dually runs in parallel, when $F$ is right exact. 

\subsubsection{Ext functors}
\label{subsection: Ext functors}

As a first application of the established theory of derived functors we consider the $\Hom$-functor
\[
    \Hom_\A(A, -) \colon \A \to \mod{k},
\]
where $A$ is an object of the abelian category $\A$. Assume that category $\A$ contains enough injectives. By Example \ref{Homs are left exact functors}, we know that $\Hom_\A(A, -)$ is a left exact functor, so we can right derive it, to obtain the functor
% so there exists its right derived functor
\[
    \mathbf{R}\Hom_\A(A, -) \colon \derplus{\A} \to \derplus{\mod{k}}.
\] 
Taking the $i$-th cohomology, we obtain higher derived functors of $\Hom_\A(A, -)$, called the \emph{Ext-functors}
\[
    \Ext^i_\A(A, -) := \rderived{i}{\Hom_\A(A, -)} = H^i(\rderived{}{\Hom_\A(A, -)}).
\]
There exists a very beautiful connection relating the Ext-functors of category $\A$ and the Hom-functors of the bounded below derived category $\derplus{\A}$ captured in the next proposion, given without proof, for in a moment we will present its generalized version.

\begin{proposition}
    \emph{\cite[\S 2, Proposition 2.56]{huybrechts2006fouriermukai}}
    Let $\A$ be an abelian category with enough injectives and let $A$ and $B$ belong to $\A$. Then for all $i \in \Z$ there exist isomorphisms
    \[
        \Ext^i_\A(A, B) \iso \Hom_{\derplus{\A}}(A, B[i]).
    \]
\end{proposition}

As a preliminary we introduce the \emph{graded objects} of $\A$ to be $\Z$-indexed sequences $A^\bullet = (A^j)_{j \in \Z}$ of objects $A^j$ of $\A$. A \emph{morphism $f \colon A^\bullet \to B^\bullet$ of degree $i \in \Z$} is a sequence $(f^j \colon A^j \to B^{i+j})_{j \in \Z}$ of morphisms of $\A$. The set of all degree $i$ morphisms will be denoted with $\Hom^i_\A(A^\bullet, B^\bullet)$. Bundling up all this data, we obtain the $\Z$-\emph{graded category of $\A$} denoted by $\A^\Z$, consisting of graded objects of $\A$ and morphisms of all integer degrees, thus
\[
    \Hom_{\A^\Z}(A^\bullet, B^\bullet) = \prod_{i \in \Z}\Hom^i_\A(A^\bullet, B^\bullet).
\]  
The category $\A^\Z$ can also be thought of as the functor category from the free category of $\Z$ to $\A$, hence the notation $\A^\Z$. 
% \info{This is introduced so I can say the differential is a morphism of degree $1$, homotopy is a morphism of degree $-1$, and makes defining $\Hom^\bullet$ just slightly easier.}

\begin{definition}
    \label{definition of Hom complex}
Let $A^\bullet$ and $B^\bullet$ be complexes of $\A$. We define the \emph{$\Hom^\bullet$ complex} $\Hom^\bullet(A^\bullet, B^\bullet)$ of $k$-modules given by terms
\[
    \Hom^i_\A(A^\bullet, B^\bullet) = \prod_{j \in \Z}\Hom_\A(A^j, B^{i + j}),
\]
\ie the set of all degree $i$ maps from $A^\bullet$ to $B^\bullet$, considered as graded objects of $\A^\Z$, and differentials 
\[
    \delta^i \colon \Hom^i_\A(A^\bullet, B^\bullet) \to \Hom^{i+1}_\A(A^\bullet, B^\bullet), \quad 
    \delta^i(f) = \left(d^{i+j}_B \circ f^j + (-1)^{i+1} f^{j+1} \circ d^j_A\right)_{j \in \Z}.
\]
In a moment, when specifying the dependence of the differential $\delta^i$ on the complex $B^\bullet$ will become important, we will denote it by $\delta^i_B$ instead. 
\end{definition}
% This differential will also be denoted by $\delta^i_B$.

Set $\Hom^\bullet(A^\bullet, -)(B^\bullet) = \Hom^\bullet(A^\bullet, B^\bullet)$ and for a chain map $u\colon B^\bullet \to C^\bullet$ in $\com(\A)$ define $u_* = \Hom^\bullet(A^\bullet, -)(u) \colon \Hom^\bullet(A^\bullet, B^\bullet) \to \Hom^\bullet(A^\bullet, C^\bullet)$ to be the chain map given by the collection $(u_*^i)_{i \in \Z}$, where $u_*^i$ sends $f = (f^j)_{j \in \Z} \in \Hom^i(A^\bullet, B^\bullet)$ to $(u^{i+j} \circ f^j)_{j \in \Z}$. Then we obtain a functor
\[
    \Hom^\bullet(A^\bullet, -) \colon \com(\A) \to \com(\mod{k}).
\]
Indeed, clearly $u_*$ is a graded morphism of degree $0$, which moreover commutes with the differentials, according to the following computation
\begin{align*}
    u^{i+1}_*\left(\delta^i_{B}(f)\right) &= u^{i+1}_*\left(\left(d^{i+j}_B f^j + (-1)^{i+1} f^{j+1} d^j_A\right)_j\right) \\
    &= \left(u^{i+j+1} d^{i+j}_B f^j + (-1)^{i+1} u^{i+j+1} f^{j+1} d^j_A\right)_j \\
    &= \left(d^{i+j}_C u^{i+j} f^j + (-1)^{i+1} u^{i+j+1} f^{j+1} d^j_A\right)_j \\
    &= \delta^i_{C}\left(\left(u^{i + j}f^j\right)_j\right) \\
    &= \delta^i_{C}\left(u^i_*(f)\right),
\end{align*}
for all $f \in \Hom^i(A^\bullet, B^\bullet)$. While not difficult to check, the verification of functoriality is omitted. 

Next, we would like $\Hom^\bullet(A^\bullet, -)$ to descend to a homotopy functor $\kcat{\A} \to \kcat{\mod{k}}$. This is in fact so, as a null-homotopic chain map $u\colon B^\bullet \to C^\bullet$ is sent to a null-homotopic chain map $u_* \htpy 0$. Indeed, suppose homotopy $h \in \Hom^{-1}(B^\bullet, C^\bullet)$ witnesses $u \htpy 0$, then for all $i \in \Z$, we have
\[
    u^i = d^{i-1}_C \circ h^i + h^{i+1} \circ d^i_B.
\]
We let $h^i_*$ denote the morphism $\Hom^i(A^\bullet, B^\bullet) \to \Hom^{i-1}(A^\bullet, C^\bullet)$, given by sending $f$ to $(h^{i+j} \circ f^j)_{j \in \Z}$, and then compute
\begin{align*}
    (\delta^{i-1}_C \circ h^{i}_* + h^{i+1}_* \circ \delta^i_B)(f) &= 
    \delta^{i-1}_C\left((h^{i+j}f^j)_j\right) + h^{i+1}_*\left((d_B^{i+j} f^j + (-1)^{i+1}f^{j+1}d^j_A)_j\right) \\
    &= \left(d^{i+j-1}_C h^{i+j} f^j + (-1)^{i} h^{i+j+1} f^{j+1} d^j_A \right)_j  + \\ 
    & \qquad \qquad + \left(h^{i+j+1}d_B^{i+j}f^j + (-1)^{i+1} h^{i+j+1} f^{j+1} d^j_A \right)_j \\
    &= \left(d^{i+j-1}_C h^{i+j} f^j + h^{i+j+1}d_B^{i+j}f^j\right)_j \\
    &= \left((d^{i+j-1}_C h^{i+j} + h^{i+j+1}d_B^{i+j})\circ f^j\right)_j \\
    &= \left(u^{i+j} \circ f^j\right)_j \\
    &= u_*^{i}(f).
\end{align*}
Thus the homotopy $h_* = (h^i_*)_{i \in \Z}$ whitnesses $u_* \htpy 0$, allowing us to conclude that 
\[
    \Hom^\bullet(A^\bullet, -) \colon \kcat{\A} \to \kcat{\mod{k}}
\] 
is well defined. The preceding functor is also triangulated, because it commutes with the translation functors and it sends distinguished triangles of $\kcat{\A}$ to distinguished triangles of $\kcat{\mod{k}}$. This is because of the following isomorphism of chain complexes of $k$-modules,
\[
    C(u_*)^\bullet = C(\Hom^\bullet(A^\bullet, -)(u))^\bullet \iso \Hom^\bullet(A^\bullet, -)(C(u)^\bullet) = \Hom^\bullet(A^\bullet, C(u)^\bullet),
\]
which holds for any chain map $u \colon B^\bullet \to C^\bullet$. Indeed, we first have for all $i \in \Z$ the following isomorphisms going between the terms of each complex
% \begin{align*}
%     C(u_*)^i 
%     &= \Hom^{i+1}(A^\bullet, B^\bullet) \oplus \Hom^i(A^\bullet, C^\bullet) \\
%     &= \prod_{j \in \Z}\Hom_\A(A^j, B^{i+j+1}) \oplus \prod_{j \in \Z}\Hom_\A(A^j, C^{i+j}) \\
%     &\iso \prod_{j \in \Z} \left(\Hom_\A(A^j, B^{i+j+1}) \oplus \Hom_\A(A^j, C^{i+j})\right) \\
%     &\iso \prod_{j \in \Z}\Hom_\A(A^j, B^{i+j+1} \oplus C^{i+j}) \\
%     &= \prod_{j \in \Z}\Hom_\A(A^j, C(u)^{i+j}) \\
%     &= \Hom^i(A^\bullet, C(u)^\bullet).
% \end{align*}
\begin{align*}
    \Hom^i(A^\bullet, C(u)^\bullet) &=
    \prod_{j \in \Z}\Hom_\A(A^j, C(u)^{i+j}) \\
    &\iso \prod_{j \in \Z}\Hom_\A(A^j, B^{i+j+1} \oplus C^{i+j}) \\
    &\iso \prod_{j \in \Z} \left(\Hom_\A(A^j, B^{i+j+1}) \oplus \Hom_\A(A^j, C^{i+j})\right) \\
    &= \prod_{j \in \Z}\Hom_\A(A^j, B^{i+j+1}) \oplus \prod_{j \in \Z}\Hom_\A(A^j, C^{i+j}) \\
    &= \Hom^{i+1}(A^\bullet, B^\bullet) \oplus \Hom^i(A^\bullet, C^\bullet) \\
    &= C(u_*)^i.
\end{align*}
To show that this collection of isomorphisms witnesses two complexes being isomorphic, we have to show that they are also chain maps, \ie they commute with the differentials.  
% Under the above isomorphisms this element correcponds to the element $()$ of $\Hom^i(A^\bullet, C(u)^\bullet)$
% For this pick $f \in \Hom^i(A^\bullet, C(u)^\bullet)$, which corresponds to a pair of $f_B \in \Hom_\A^{i+1}(A^\bullet, B^\bullet)$ and $f_C \in \Hom_\A^{i}(A^\bullet, C^\bullet)$ under the chain of isomorphisms above. 
Let
\[
    \delta_{C(u)}^i \colon \Hom^i(A^\bullet, C(u)^\bullet) \to \Hom^{i+1}(A^\bullet, C(u)^\bullet)
\]
denote the differential of $\Hom^\bullet(A^\bullet, C(u)^\bullet)$ and let
\[
    d^i \colon C(u_*)^i \to C(u_*)^{i+1} \qquad d^i = \begin{pmatrix}
        -\delta^{i+1}_B & 0 \\
        u_*^{i+1} & \delta^i_C
    \end{pmatrix}
\]
be the differential of $C(u_*)^\bullet$. 
Pick an element $f = ((f_B^j, f_C^j))_{j \in \Z} \in \Hom^i(A^\bullet, C(u)^\bullet)$. Under the isomorphism above $f$ is sent to the pair $(f_B, f_C) \in C(u_*)^i$, where
\[
    f_B = (f_B^j \colon A^j \to B^{i+j+1})_{j \in \Z} \quad \text{and} \quad f_C = (f_C^j \colon A^j \to C^{i+j})_{j \in \Z}.
\] 
Then the computation below proves that the isomorphisms commute with the differentials, because $\delta^i_{C(u)}(f)$ and $d^i((f_B, f_C))$ coincide at every coordinate $j \in \Z$. 
% $f = (f_B, f_C) \in C(u_*)^i$. Applying the above isomorphism amounts to reparenthesizing to obtain $((f_B^j, f_C^j))_{j \in \Z} \in \Hom^i(A^\bullet, C(u)^\bullet)$.
% We will now verify that for each $i \in \Z$ the $j$-th component of tuple $\delta^i_{C(u)}(f)$ after reparenthesizing coincides with $d^i(((f_B^j, f_C^j))_{j \in \Z} )$.
\begin{align*}
    [\delta^i_{C(u)}(f)]_j &=
    d^{i+j}_{C(u)} \circ f^j + (-1)^{i+1} f^{j+1} \circ d^j_A \\
    & = \begin{pmatrix}
        d_{B[1]}^{i+j} & 0 \\
        u^{i+j+1} & d_C^{i+j}
    \end{pmatrix} \circ
    \begin{pmatrix}
        f^j_B \\
        f^j_C
    \end{pmatrix} + (-1)^{i+1}
    \begin{pmatrix}
        f^{j+1}_B \\
        f^{j+1}_C
    \end{pmatrix} \circ d^j_A \\
    & = \begin{pmatrix}
        d_{B[1]}^{i+j} f^j_B + (-1)^{i+1}f_B^{j+1}d^j_A \\
        u^{i+j+1}f^j_B + d^{i+j}_Cf^j_C + (-1)^{i+1}f_C^{j+1}d^j_A
    \end{pmatrix} \\
    & = \begin{pmatrix}
        [\delta^i_{B[1]}(f_B)]_j \\
        [u^{i+1}_*(f_B)]_j + [\delta^i_C(f_C)]_j
    \end{pmatrix} \\
    & = \left[ \begin{pmatrix}
        -\delta^{i+1}_B & 0 \\
        u_*^{i+1} & \delta^i_C
    \end{pmatrix}
    \begin{pmatrix}
        f_B \\
        f_C
    \end{pmatrix}
    \right]_j.
\end{align*}
Here we have used the notation $[-]_j$ to mean the $j$-th component of a tuple. 

% We will now verify that for each $i \in \Z$ the $j$-th components of tuples $\delta_{C(u_*)}^i(f)$ and $...$ coincide. 
% where these isomorphisms also commute with the differentials. Pick $f \in \Hom^i(A^\bullet, C(u)^\bullet)$ which is given by a pair $f$


% \vspace{0.3cm}



At last, for $A^\bullet$ belonging to $\kminus{\A}$, \ie bounded from \emph{above}\footnote{
    This assumption cannot be omitted, for otherwise we cannot claim, that the functor $\Hom^\bullet(A^\bullet, -)$ maps into the bounded \emph{below} homotopy category $\kplus{\mod{k}}$.
}, we have a triangulated functor $\Hom^\bullet(A^\bullet, -) \colon \kplus{\A} \to \kplus{\mod{k}}$, which we may derive, in accordance with Remark \ref{Deriving homotopy functors}, of course assuming $\A$ contains enough injectives, to obtain
\[
    \rderived{}{\Hom^\bullet(A^\bullet, -)} \colon \derplus{\A} \longrightarrow \derplus{\mod{k}}.
\]
Its higher derived counterpart will be denoted by $\Ext^i_\A(A^\bullet, - ) := \rderived{i}{\Hom^\bullet(A^\bullet, -)}$.

% \begin{remark}
%     There is also a generalization of this result providing a way to identify the hom-functors in $\derplus{\A}$ applied not only to objects of $\A$, but to bounded below complexes. In that case, on the Ext-side, one considers the \emph{differentialy graded hom-functors} of $\kplus{\A}$, namely $\Hom^\bullet(-,-) \colon \kminus{\A}^\op \times \kplus{\A} \to \kcat{\mod{k}}$, for which $\Hom^\bullet(A^\bullet, -)$ turns out to be right derivable inducing its higher derived functors $\Ext^i(A^\bullet, -) = \rderived{i}{\Hom^\bullet(A^\bullet, -)}$. Then for all $i \in \Z$ and all $A^\bullet$ and $B^\bullet$ of $\derplus{\A}$ there are isomorphisms
%     \[
%         \Ext^i(A^\bullet, B^\bullet) \iso \Hom_{\derplus{\A}}(A^\bullet, B[i]^\bullet).
%     \]
%     \info{this has to be expaned, because I need it for Serre functors, naturality in $A^\bullet$ and $B^\bullet$ is important.}
% \end{remark}

\begin{remark}
    \label{bifunctors on the homotopy category}
    Behind the scenes of this whole procedure one actually consideres a $k$-linear bifunctor $F \colon \A \times \A' \to \A''$, with $\A$, $\A'$ and $\A''$ abelian over $k$, where the latter is assumed to contain countable products. From this bifunctor we induce another $k$-linear bifunctor on the level of chain complexes $F_{\pi} \colon \com(\A) \times \com(\A') \to \com(\A'')$ via the total complex of a double complex as described in \cite[\S 11.6]{kashiwara2006categories}.
    %  with terms $F^\bullet_\pi(X^\bullet, Y^\bullet) = \operatorname{tot_\pi}(F(X^\bullet, Y^\bullet))$, where $\operatorname{tot_\pi}(Z^{\bullet, \bullet})^n = \prod_{p+q = n} Z^{p,q}$. 
    It then turns out that $F_\pi$ induces a well defined triangulated bifunctor on the level of the homotopy category 
    \[
        \kcat{\A} \times \kcat{\A'} \to \kcat{\A''}.
    \]
    % Additionally 
    % $\kplus{\A} \times \kplus{\A'} \to \kplus{\A''}$, $\kminus{\A} \times \kminus{\A'} \to \kminus{\A''}$, $\kb{\A} \times \kcat{\A'} \to \kcat{\A''}$ and $\kcat{\A} \times \kb{\A'} \to \kcat{\A''}$.
\end{remark}

% \begin{proposition}
%     \label{Homs in derived category}
%     Let $\A$ be an abelian category with enough injectives. Then for every $A^\bullet$ of $\kminus{\A}$ and $B^\bullet$ belonging to $\kplus{\A}$ there natural isomorphisms
%     \[
%         \alpha_{A^\bullet, B^\bullet} : \Ext^i_\A(A^\bullet, B^\bullet) \iso \Hom_{\dercat{\A}}(A^\bullet, B[i]^\bullet).
%     \]
% \end{proposition}
% \info{I think this is true, but I won't be able to show it as I need to use $\derplus{\A} \natiso \kplus{\J}$, since $A^\bullet$ is not bounded from below.}

\begin{theorem}
    \label{Homs in derived category}
    \emph{\cite[\S 2, Remark 2.57]{huybrechts2006fouriermukai}}
    Let $\A$ be an abelian category with enough injectives. Then for any two bounded complexes $A^\bullet$ and $B^\bullet$ belonging to $\derb{\A}$ there are natural isomorphisms
    \[
        \Ext^i_\A(A^\bullet, B^\bullet) \iso \Hom_{\derb{\A}}(A^\bullet, B[i]^\bullet).
    \]
\end{theorem}

\begin{remark}
    Notice that in the theorem we require the complex $A^\bullet$ to be bounded even though we have so far only assumed it to be bounded from above. This is necessary as we will use Lemma \ref{D+ K+ aux lemma 2}, which requires $A^\bullet$ to be bounded from below. 
\end{remark}

\begin{lemma}
    \label{Homotopy hom = cohomology of inner hom}
    Let $A^\bullet$ and $B^\bullet$ be chain complexes in $\kcat{\A}$ and let $i \in \Z$. Then
    \[
        H^i(\Hom^\bullet(A^\bullet, B^\bullet)) = \Hom_{\kcat{\A}}(A^\bullet, B[i]^\bullet).
    \] 
\end{lemma}

\begin{proof}
    This is practically the definition of $\Hom_{\kcat{\A}}(A^\bullet, B[i]^\bullet)$. The cohomology $k$-module $H^i(\Hom^\bullet(A^\bullet, B^\bullet))$ is defined to be the quotient $\ker \delta^i_B / \im \delta^{i-1}_B$.
    % Here $\delta$ denotes the differential of $\Hom^\bullet(A^\bullet, B^\bullet)$.
    Thus let us identify $\ker \delta^i_B$ and $\im \delta^{i-1}_B$. Notice, that for $f \in \Hom^i(A^\bullet, B^\bullet)$ we have
    \[
        \delta^i(f) = \left(d^{i+j}_B \circ f^j + (-1)^{i+1} f^{j+1} \circ d^j_A\right)_{j \in \Z} = (-1)^i \left(d^j_{B[i]} \circ f^j - f^{j+1} \circ d^j_A\right)_{j \in \Z}.
    \]
    Hence $\ker \delta^i_B$ is the set of all chain maps $A^\bullet \to B[i]^\bullet$, and $\im \delta^{i-1}_B$ the set of all null-homotopic chain maps $A^\bullet \to B[i]^\bullet$.
\end{proof}

\begin{proof}[Proof of Proposition \ref{Homs in derived category}]
    Let $B^\bullet \to I^\bullet$ be an injective resolution for $B^\bullet$. We follow the chain of isomorphisms
    \begin{align*}
        \Hom_{\derb{\A}}(A^\bullet, B[i]^\bullet) &= \Hom_{\derplus{\A}}(A^\bullet, B[i]^\bullet)  \\
        &\iso \Hom_{\derplus{\A}}(A^\bullet, I[i]^\bullet)  \\
        &\iso \Hom_{\kplus{\A}}(A^\bullet, I[i]^\bullet) &\text{(Lemma \ref{D+ K+ aux lemma 2})}\\
        &= H^i(\Hom^\bullet(A^\bullet, I^\bullet)) &\text{(Lemma \ref{Homotopy hom = cohomology of inner hom})} \\
        &= \Ext_\A^i(A^\bullet, B^\bullet). & \quad & \qedhere
    \end{align*}
\end{proof}
\newpage

\section{Derived categories in geometry}

\subsection{Derived category of coherent sheaves}

One of the main invariants of a scheme $X$ over $k$ is its category of coherent sheaves $\coherent{X}$. One can think of this category as a slight expansion of the category of $k$-vector bundles on $X$ in the sense that $\coherent{X}$ is abelian, where as the former very often is not. In this chapter we restrict ourself to noetherian schemes and our primary object of study will be the bounded derived category of coherent sheaves
\[
    D^b(X) := D^b(\coherent{X}).
\]

\subsection{Derived functors in algebraic geometry}

\subsubsection{Global sections functor}
\subsubsection{Push-forward $f_*$}
\subsubsection{Pull-back $f^*$}
\subsubsection{Inner hom}
\subsubsection{Tensor product }

\newpage

\section{Fourier--Mukai transforms}
\label{Chapter: Fourier-Mukai transforms}

Fourier--Mukai transforms are named after two prominent mathematicians -- Fourier and Mukai. 
% Mukai named them after Fourier, and in the process of studying them, his name also became attached to them. 
% The second one naming them after the first and in the process of studying them also getting his name attached to the term.
In his paper \cite{Mukai1981} Mukai introduced these transforms to study bounded derived categories of coherent sheaves on abelian varieties and their duals, in the process discovering non-isomorphic varieties which share equivalent bounded derived categories. He named his transforms after Fourier because of their stark resemblance to classical Fourier transforms of analysis.
%  and also provided a dictionary to pass between the two sides of his analogy.

In this chapter we will first define Fourier--Mukai transforms on the level of derived categories, explore some of their properties and state an important theorem of Orlov, which will be indispensable later on. The remaining part of the chapter will be devoted to passing Fourier--Mukai transforms first to the level of $K$-groups and lastly to rational cohomology. We will examine the interactions of these three views with the use of Mukai vectors.
% and then examining their interactions via Mukai vectors. 

% \vspace{0.3cm}
\noindent
\textsl{Notation.}
Let $X$ and $Y$ be smooth projective varieties over $k$ and denote the canonical projections 
\[
    p \colon X \times Y \to X \quad \text{and} \quad q \colon X \times Y \to Y.
\]
% We will also transition from using $\F^\bullet$ for denoting complexes of sheaves to instead using a touch lighter notation $\F$ to denote either a single sheaf or a complex of sheaves.
We will also transition from using $\F^\bullet$ to denote complexes of sheaves to the lighter notation $\F$, which may refer either to a single sheaf or to a complex of sheaves.

\begin{definition}
    The \emph{Fourier--Mukai transform} associated to a complex $\E$ of $\derb{X \times Y}$ is defined to be the functor
    \[
        \fm{\E} \colon \derb{X} \to \derb{Y} \qquad \fm{\E} := \rderived{}{q_*}(\derivedtensor{\E}{\lderived{}{p^*}(-)}{\struct{X \times Y}}).
    \]
    The complex $\E$ is called the \emph{kernel} of $\fm{\E}$.
\end{definition}

\begin{remark}
    We see that $\fm{\E}$ is a composition of three triangulated functors,
    \[
        \fm{\E} \colon \quad \derb{X} \xrightarrow{\ \lderived{}{p^*} \ } \derb{X \times Y} \xrightarrow{\derivedtensor{\E}{(-)}{\struct{X \times Y}}} \derb{X \times Y} \xrightarrow{\ \rderived{}{q_*} \ } \derb{Y}
    \]
    meaning it is also triangulated itself.  
\end{remark}

Note that as the projection $p \colon X \times Y \to X$ is flat, the pull-back functor $p^*$ is exact and therefore need not be derived. Many examples of known functors are seen to be of Fourier--Mukai type.

\begin{example}
    \label{Identifying fm transforms}
    % In this example we notice and identity some common functors from algebraic geometry appear as Fourier-Mukai transforms. 
    In the following examples we will write derived functors only wherever truly necessary. For instance tensoring with a vector bundle, pushing-forward along a closed embedding, pulling-back along a flat morphism are all exact functors, which need not be derived.
    % 
    % only the functors which are necessary to be derived will be derived. 
\begin{enumerate}[label = (\roman*)]
    \item{
    $\id{\derb{X}} \iso \fm{\struct{\Delta}}$. Recall that $\struct{\Delta}$ is defined to be the push-forward $\Delta_*\struct{X}$, of the structure sheaf of $X$ along the diagonal embedding $\Delta \colon X \to X \times X$. Then for any object $\F$ of $\derb{X}$ we compute 
    \begin{align*}
        \fm{\struct{\Delta}}(\F) &= \rderived{}{q_*}(\derivedtensor{\struct{\Delta}}{p^*(\F)}{\struct{X \times X}}) \\
        &\iso \rderived{}{q_*}(\derivedtensor{\Delta_*\struct{X}}{{p^*}(\F)}{\struct{X \times X}}) \\
        &\iso \rderived{}{q_*}(\Delta_*(\struct{X} \otimes_{\struct{X}}\lderived{}{\Delta^*}({p^*}(\F)))) \\
        &\iso \rderived{}{(q \circ \Delta)_*}(\lderived{}{(p \circ \Delta)^*}(\F)) \\
        &\iso \F.
    \end{align*}
    % \begin{align*}
    %     \fm{\struct{\Delta_X}} &= \rderived{}{p_*}(\derivedtensor{\struct{\Delta_X}}{\lderived{}{p^*}(E)}{\struct{X \times X}}) \\
    %     &\iso \rderived{}{p_*}(\derivedtensor{\Delta_*\struct{X}}{\lderived{}{p^*}(E)}{\struct{X \times X}}) \\
    %     &\iso \rderived{}{p_*}(\derivedtensor{\Delta_*\struct{X}}{\lderived{}{\Delta^*}(\lderived{}{p^*}(E))}{\struct{X \times X}}) \\
    % \end{align*}
    % \begin{align*}
    %     \fm{\struct{\Delta}} &= p_*(\struct{\Delta} \otimes_{\struct{X \times X}} p^*(\F)) \\
    %     &= p_*(\Delta_*\struct{X} \otimes_{\struct{X \times X}} p^*(\F)) \\
    %     &\iso p_*(\Delta_*(\struct{X} \otimes_{\struct{X}} \Delta^*p^*(\F))) \\
    %     &\iso (p \circ \Delta)_*((p \circ \Delta)^* \F) \\
    %     % &\iso \struct{X} \otimes E \\
    %     &\iso \F.
    % \end{align*}
    All isomorphisms are natural in $\F$, thus proving $\id{\derb{X}} \iso \fm{\struct{\Delta}}$.
    }
    \item{$(-)[1] \iso \fm{\struct{\Delta}[1]}$. While noting that all the functors involved are triangulated, a similar computation as above identifies the translation functor as a Fourier--Mukai transform with kernel $\struct{\Delta}[1]$.} \label{shift is FM}
    \item{
    $\mathcal L \otimes (-) \iso \fm{\Delta_*\mathcal L}$. Let $\mathcal L$ denote a line bundle on $X$. Then utilizing the diagonal embedding $\Delta \colon X \to X \times X$ again, we can compute for any complex $\F$ of $\derb{X}$
    \begin{align*}
        \fm{{\Delta_*\mathcal L}}(\F) &= \rderived{}{q_*}(\derivedtensor{\Delta_*\mathcal L}{p^*(\F)}{\struct{X \times X}}) \\
        % &\iso \rderived{}{q_*}(\derivedtensor{\Delta_*\struct{X}}{{p^*}(\F)}{\struct{X \times X}}) \\
        &\iso \rderived{}{q_*}(\Delta_*(\mathcal L \otimes_{\struct{X}}\lderived{}{\Delta^*}({q^*}(\F)))) \\
        &\iso \rderived{}{(q \circ \Delta)_*}(\mathcal L \otimes_{\struct{X}} \lderived{}{(p \circ \Delta)^*}(\F)) \\
        &\iso \mathcal L \otimes_{\struct{X}} \F.
    \end{align*}
    Since all the isomorphisms are natural, we have $\mathcal L \otimes_{\struct{X}} (-) \iso \fm{\Delta_*\mathcal L}$.
    } \label{Tensoring by a line bundle is FM}
    \item{$S_X \iso \fm{\Delta_*\omega_X[n]}$. As a consequence of \ref{shift is FM} and \ref{Tensoring by a line bundle is FM}, we see that the Serre functor $S_X = (-) \otimes \omega_X[n]$, where $n$ denotes the dimension of $X$, is a Fourier--Mukai transform with kernel $\Delta_*\omega_X[n]$. 
    }
    \item{$\rderived{}{f_*} \iso \fm{\struct{\Gamma_f}}$. For a morphism $f \colon X \to Y$, let $\struct{\Gamma_f}$ denote the push-forward of the sheaf $\struct{X}$ along the morphism $i\colon X \to X \times Y$ induced by the identity $\id{X}\colon X \to X$ and $f\colon X \to Y$.  Then for any object $\F$ of $\derb{X}$ we have
    % \begin{align*}
    %     \fm{\struct{\Gamma_f}} &= p_*(\struct{\Gamma_f} \otimes_{\struct{X \times Y}} q^*(\F)) \\
    %     &= p_*(i_*\struct{X} \otimes_{\struct{X \times X}} q^*(\F)) \\
    %     &\iso p_*(i_*(\struct{X} \otimes_{\struct{X}} i^*q^*(\F))) \\
    %     % &\iso \struct{X} \otimes E \\
    %     &\iso f_*\F.
    % \end{align*}
    % \begin{align*}
    %     \fm{\struct{\Gamma_f}} &= \rderived{}{p_*}(\struct{\Gamma_f} \otimes_{\struct{X \times Y}} \lderived{}{q^*}(\F)) \\
    %     &= \rderived{}{p_*}(i_*\struct{X} \otimes_{\struct{X \times X}} \lderived{}{q^*}(E)) \\
    %     &\iso \rderived{}{p_*}(\rderived{}{i_*}(\struct{X} \otimes_{\struct{X}} \lderived{}{i^*}\lderived{}{q^*}(E))) \\
    %     % &\iso \struct{X} \otimes E \\
    %     &\iso \rderived{}{f_*}(E).
    % \end{align*}
    \begin{align*}
        \fm{\struct{\Gamma_f}}(\F) &= \rderived{}{q_*}(\derivedtensor{\struct{\Gamma_f}}{p^*(\F)}{\struct{X \times Y}}) \\
        &\iso \rderived{}{q_*}(\derivedtensor{i_*\struct{X}}{{p^*}(\F)}{\struct{X \times Y}}) \\
        &\iso \rderived{}{q_*}(i_*(\struct{X} \otimes_{\struct{X}}\lderived{}{i^*}({p^*}(\F)))) \\
        &\iso \rderived{}{(q \circ i)_*}(\lderived{}{(p \circ i)^*}(\F)) \\
        &\iso \rderived{}{f_*}\F.
    \end{align*}
    Note that we have used $p \circ i = \id{X}$ and $q \circ i = f$ passing to the last line. As all the isomorphisms are natural, we have $\rderived{}{f_*} \iso \fm{\struct{\Gamma_f}}$.
    }
\end{enumerate}
\end{example}

\begin{remark}
    From now on we will consider all our functors on derived categories to be derived and follow the convention of not explicitly denoting them as being derived. For example we will write just $f_*$ to mean $\rderived{}{f_*}$ etc.
\end{remark}

In his paper \cite{Mukai1981} Mukai discovered that every Fourier--Mukai transform admits left and right adjoints. 

\begin{proposition}
    \label{Adjoints of FM transforms}
    \emph{\cite{Mukai1981}}
    Let $\fm{\E} \colon \derb{X} \to \derb{Y}$ be a Fourier--Mukai transform associated to an object $\E$ of $\derb{X \times Y}$. Then $\fm{\E}$ admits left and right adjoints
    \[
        \fm{\ladj{\E}} \colon \derb{Y} \to \derb{X} \quad \text{and} \quad \fm{\radj{\E}} \colon \derb{Y} \to \derb{X},
    \]
    which are evidently also Fourier--Mukai with their respective kernels
    \[
        \ladj{\E} = \E^\vee \otimes_{\struct{X \times Y}} q^*\omega_Y[\dim Y] \quad \text{and} \quad 
        \radj{\E} = \E^\vee \otimes_{\struct{X \times Y}} p^*\omega_X[\dim X].
    \]
\end{proposition}

\begin{proof}
    See \cite[\S 5, Proposition 5.9]{huybrechts2006fouriermukai}. This follows from a highly non-trivial fact that the derived push-forward $\rderived{}{f_*} \colon \derb{X} \to \derb{Y}$ along a proper map $f \colon X \to Y$ admits a \emph{right} adjoint $f^! = \lderived{}{f^*}(-) \otimes_\struct{X} \omega_f[\dim f]$, where $\omega_f = \omega_X \otimes_\struct{X} f^*\omega_Y^*$ and $\dim f = \dim Y - \dim X$. This is a consequence of Grothendieck--Verdier duality \cite[\S VII, Corollary 4.3]{Hartshorne1966}. Since $X$ and $Y$ are projective the projection $p \colon X \times Y \to X$ is proper.
\end{proof}

\begin{remark}
    \label{another description of adjoints of FM}
    Using Serre functors we are able to write down an alternative description of left and right adjoints of $\fm{\E}$, namely
    \begin{equation}
        \label{eq: fm adjoints with serre functors}
        \fm{\ladj{\E}} = \fm{\E^\vee} \circ S_Y \quad \text{and} \quad 
        \fm{\radj{\E}} = S_X \circ \fm{\E^\vee}.
    \end{equation}
    Indeed, a straightforward computation, using the fact that pull-backs commute with tensor products, shows
    \begin{align*}
        \fm{\ladj{E}}(\F) &= p_*(\ladj{\E} \otimes_{\struct{X \times Y}} q^*\F) \\
        &= p_*(\E^\vee \otimes_{\struct{X \times Y}} q^*\omega_Y[\dim Y] \otimes_{\struct{X \times Y}} q^*\F) \\
        &\iso p_*(\E^\vee \otimes_{\struct{X \times Y}} q^*(\F \otimes_{\struct{Y}} \omega_Y[\dim Y])) \\
        &= \fm{\E^\vee}(S_Y(\F))
    \end{align*}
    for any $\F$ of $\derb{Y}$. For the other, using the projection formula \eqref{eq: projection formula}, we have
    \begin{align*}
        \fm{\radj{E}}(\F) &= p_*(\radj{\E} \otimes_{\struct{X \times Y}} q^*\F) \\
        &= p_*(\E^\vee \otimes_{\struct{X \times Y}} p^*\omega_X[\dim X] \otimes_{\struct{Y}} q^*\F) \\
        & \iso p_*(\E^\vee \otimes_{\struct{X \times Y}} q^*\F) \otimes_\struct{X} \omega_X[\dim X] \\
        % &\iso p_*(\E^\vee \otimes_{\struct{X \times Y}} q^*(\F \otimes_{\struct{Y}} \omega_Y[\dim Y])) \\
        &= S_X(\fm{\E^\vee}(\F))
    \end{align*}
    for any $\F$ of $\derb{X}$. The equations \eqref{eq: fm adjoints with serre functors} also agree with the more general method of Proposition \ref{Serre functors and adjoints} for obtaining a right adjoint of a functor provided that a left adjoint is known, when Serre functors are present.
    %  in the presence of Serre functors 
\end{remark}

\begin{proposition}
    \label{Composition of fm is fm}
    \emph{\cite[Proposition 1.3]{Mukai1981}}
    % Let $\fm{\E^\bullet} \colon \derb{X} \to \derb{Y}$ and $\fm{\F^\bullet} \colon \derb{Y} \to \derb{Z}$ be Fourier-Mukai transforms with kernels $\E^\bullet$ and $\F^\bullet$ belonging to $\derb{X \times Y}$ and $\derb{Y \times Z}$ respectively. Then the composition $\fm{\F^\bullet} \circ \fm{\E^\bullet} \colon \derb{X} \to \derb{Z}$ is also a Fourier-Mukai transform with kernel $\mathcal R^\bullet = ...$, which is sometimes also called the \emph{convolution} of $\E^\bullet$ and $\F^\bullet$.
    Let $\E$ and $\F$ belong to $\derb{X \times Y}$ and $\derb{Y \times Z}$, respectively. Consider Fourier--Mukai transforms $\fm{\E} \colon \derb{X} \to \derb{Y}$ and ${\fm{\F} \colon \derb{Y} \to \derb{Z}}$.
    % Let $\fm{\E} \colon \derb{X} \to \derb{Y}$ and ${\fm{\F} \colon \derb{Y} \to \derb{Z}}$ be Fourier-Mukai transforms with kernels $\E$ and $\F$ belonging to $\derb{X \times Y}$ and $\derb{Y \times Z}$ respectively. 
    Then their composition $\fm{\F} \circ \fm{\E} \colon \derb{X} \to \derb{Z}$ is also a Fourier--Mukai transform with kernel
    \[
        \G = (\pi_{XZ})_*(\pi_{XY}^*\E \otimes \pi_{YZ}^*\F),
    \]
    which is sometimes also called the \emph{convolution} of $\E^\bullet$ and $\F^\bullet$. Morphisms $\pi_{XY}$, $\pi_{YZ}$ and $\pi_{XZ}$ denote the canonical projections from $X \times Y \times Z$ to $X \times Y$, $Y \times Z$ and $X \times Z$, respectively.
\end{proposition}

\begin{proof}
    See \cite[\S 5, 5.10]{huybrechts2006fouriermukai}. The proof utilizes the projection formula \eqref{eq: projection formula} and the flat base change formula \eqref{eq: flat base change formula}. Essentially this means that it is far less sophisticated than the proof of the existence of adjoints.    
\end{proof}

The following is a celebrated theorem by Orlov, which shed a new light on the topic of triangulated functors between bounded derived categories of coherent sheaves on smooth projective varieties. Its main advantage lies in the fact that instead of studying fully faithful triangulated functors on their own, one can study them in simpler terms using their Fourier--Mukai kernels. 
% One finds this theorem in \cite[Theorem 3.2.2]{Orlov2003}.
% in the fact that we can study 
% It can be interpreted to provide a way 

\begin{theorem}
    \label{Orlov's theorem}
    \emph{\cite[Theorem 3.2.2]{Orlov2003}}
    Let $X$ and $Y$ be smooth projective varieties over a field $k$. Suppose $F \colon \derb{X} \to \derb{Y}$ is a triangulated fully faithful functor admitting a left (or right) adjoint functor. Then there exists up to an isomorphism a unique object $\E$ of $\derb{X \times Y}$ such that $F$ and $\fm{\E}$ are naturally isomorphic.
\end{theorem}

\begin{remark}
    \begin{enumerate}[label = (\roman*)]
        \item Theorem \ref{Orlov's theorem} may be applied to any triangulated equivalence $F$ between bounded derived categories $\derb{X}$ and $\derb{Y}$. This is what we will do most often. 
        \item Due to bounded derived categories being equipped with Serre functors (Proposition \ref{Serre functor on Db(X)}), the assumptions that $F$ admits a left or a right adjoint are equivalent by Proposition \ref{Serre functors and adjoints}.
        \item{Even though this theorem was generally believed to be false without the fully faithfulness assumption, a non-Fourier--Mukai triangulated functor between bounded derived categories of coherent sheaves of two smooth projective varieties was exemplified only relatively recently \cite{RizzardoVanDenBerghNeeman2019}.}
    \end{enumerate}
\end{remark}

% Once equipped with Orlov's Theorem \ref{Orlov's theorem}, we are able to prove a beautiful 
We say that varieties $X$ and $Y$ are \emph{derived equivalent} if there exists a triangulated equivalence $\derb{X} \iso \derb{Y}$. Due to the above theorem by Orlov, varieties $X$ and $Y$ are in that case called \emph{Fourier--Mukai partners.}
This leads us to the following beautiful application of Orlov's theorem. It is also an important statement by itself, providing evidence for why the bounded derived category of coherent sheaves is a nicely behaved invariant for smooth projective varieties. In particular it says that $\derb{-}$ detects the dimension and the triviality of the canonical bundle.  
% showing that the bounded derived categories of coherent sheaves on smooth and projective varieties over $k$ detect dimension and triviality of the canonical bundle. 
\begin{theorem}
    \label{Db detects dimension and triviality of canonical bundle}
    % \emph{\cite[\S 4, Proposition 4.1]{huybrechts2006fouriermukai}}
    \emph{\cite[Lemma 2.1]{bridgeland2001complex}}
    Suppose $X$ and $Y$ are smooth projective varieties over $k$. Assume
    \[
        \derb{X} \natiso \derb{Y}.
    \]
    Then $\dim X = \dim Y$ and if the canonical bundle $\omega_X$ of $X$ is trivial, the canonical bundle $\omega_Y$ of $Y$ is trivial as well.
\end{theorem}

\begin{proof}
    By Orlov's theorem \ref{Orlov's theorem} the equivalence $\derb{X} \natiso \derb{Y}$ is witnessed by a Fourier--Mukai transform $\fm{\E}$ with a kernel $\E$ belonging to $\derb{X \times Y}$. Since $\fm{\E}$ is an equivalence its left and right adjoints agree. As they are both given by Fourier--Mukai transforms their respective kernels $\ladj{\E}$ and $\radj{\E}$, introduced in Proposition \ref{Adjoints of FM transforms}, must be isomorphic in $\derb{X \times Y}$. Writing this out we see that
    \[
        \E^\vee \iso \E^\vee \otimes (q^*\omega_X \otimes p^*\omega_Y)[\dim Y - \dim X].
    \]
    As tensoring by a line bundle, namely $q^*\omega_X \otimes p^*\omega_Y$ is exact, both tensor products are underived, but more importantly, the cohomology of complexes $\E^\vee$ and ${\E^\vee \otimes (q^*\omega_X \otimes p^*\omega_Y)}$ agrees in all degrees. Since $\E$ is a bounded complex of coherent sheaves, its derived dual $\E^\vee = \rderived{}{\localhom[X \times Y]^\bullet(\E, \struct{X \times Y})}$ is bounded as well. Then comparing cohomology intermediately\footnote{As $\fm{\E}$ is an equivalence, the kernel $\E$ cannot be isomorphic to the zero object in $\derb{X \times Y}$, implying that $\E^\vee$ has non-trivial cohomology at least in some degrees.} gives 
    \[
        \dim X = \dim Y.
    \]
    
    For the second part, the assumption $\omega_X \iso \struct{X}$, simplifies the Serre functor $S_X$ of $\derb{X}$ to 
    \[
        S_X \colon \derb{X} \to \derb{X} \qquad S_X \natiso (-)[n], 
    \]
    where $n$ denotes the dimension of both $X$ and $Y$. Since equivalences of triangulated categories are known to commute with Serre functors by Proposition \ref{Serre functors commute with equivalences}, we deduce from $\fm{\E} \circ S_X \natiso S_Y \circ \fm{\E}$ that the Serre functor $S_Y$ of $\derb{Y}$ also simplifies to $(-)[n]$. By Definition \ref{Serre functor on Db(X)} this may be expressed as
    \[
        (-)[n] \natiso S_Y = (-) \otimes_{\struct{Y}} \omega_Y[n].
    \]
    Plugging in the structure sheaf $\struct{Y}$, we see that $\omega_Y \iso \struct{Y}$ holds in $\coherent{Y}$. The last assertion 
    % about $\coherent{Y}$ 
    follows from fully faithfulness of the inclusion $\coherent{Y} \to \derb{Y}$ according to Proposition \ref{A -> D(A) fully faithful}. 
\end{proof}

\subsection{Fourier--Mukai transforms on \emph{K}-groups}
\label{Subsection: FM transform on K-theory}

We introduce the $K$-theoretic Fourier--Mukai transforms as a passageway between categorical and cohomological Fourier--Mukai transforms. Throughout this section we assume our schemes are smooth projective complex varieties and often that our morphisms are proper or flat. We do this so that we have for example access to propositions like \ref{In Db(X) everything is a complex of vector bundles}, \ref{push-forward on Db(X)}, etc.

Given a smooth projective complex variety $X$ we define its \emph{$K$-group} as follows. Consider the free abelian group generated by the isomorphism classes of coherent sheaves\footnote{
    An attentive reader might express a set-theoretic concern at this point -- does the collection of all isomorphism classes of coherent sheaves on $X$ even form a set? On every noetherian scheme $X$ this is indeed the case. As was pointed out to me by Domen Zevnik this vaguely follows from the fact that $X$ can be covered by a (finite) set of affine schemes of some noetherian rings and over noetherian rings there are only set-many isomorphism classes of finitely generated modules, because all such modules are finitely presented. 
} on $X$. The quotient of this group by its subgroup generated by expressions $[\E] - [\F] + [\G]$, when there is a short exact sequence of the form $0 \to \E \to \F \to \G \to 0$, is defined to be the $K$-group of $X$. We denote it by $\kgroup{X}$. Extending this notion of additivity on short exact sequences, we define for any bounded complex $\F^\bullet$ of $\derb{X}$ 
\[
    [\F^\bullet] := \sum_{i \in \Z}(-1)^i[\F^i] \in \kgroup{X}.
\]
\begin{remark}
    Observe that any complex $\F^\bullet$ may be split up into a series of short exact sequences of the form
    \[
        0 \to \ker d^{i}_\F \to \F^i \to \im d^i_\F \to 0.
    \]
    In $\kgroup{X}$ this means that $[\F^i] = [\ker d^{i}_\F] + [\im d^i_\F]$. From the short exact sequence defining the cohomology sheaf  $\mathcal H^i(\F^\bullet)$ we also see that $[\mathcal H^i(\F^\bullet)] = [\ker d^{i}_\F] - [\im d^{i-1}_\F]$. Using these two sets of formulas one reparenthesizes the definition of $[\F^\bullet]$ to obtain
    \begin{equation}
        \label{eq: K-class alternating sum of cohomology}
        [\F^\bullet] = \sum_{i \in \Z}(-1)^i[\mathcal H^i(\F^\bullet)].
    \end{equation}
    In particular this shows that the operation $[-]$ of forming a class in $\kgroup{X}$ out of a complex $\F^\bullet$ of $\derb{X}$ is an invariant of its isomorphism class in $\derb{X}$.
\end{remark}


The group $\kgroup{X}$ also comes equipped with a multiplication operation. On isomorphism classes of locally free sheaves $\E$ and $\E'$ this operation is given by
\[
    [\E]\cdot[\E'] = [\E \otimes_{\struct{X}} \E'].
\]
For coherent sheaves $\F$ and $\G$ it turns out that one has to take the \emph{derived} tensor product $[\F]\cdot[\G] = [\derivedtensor{\F}{\G}{\struct{X}}]$ to accommodate the relations in $\kgroup{X}$ coming from short exact sequences. It turns out that extending this operation to chain complexes $\F^\bullet$ and $\G^\bullet$ of $\derb{X}$ also amounts to taking the derived tensor product but now of complexes
\begin{equation}
    \label{eq: K-theory product compatibility}
    [\F^\bullet]\cdot [\G^\bullet] = [\derivedtensor{\F^\bullet}{\G^\bullet}{\struct{X}}].
\end{equation}

\begin{remark}
    The above is seen to be true by replacing both complexes with their respective isomorphic complexes of locally free coherent sheaves, whose existence is due to Proposition \ref{In Db(X) everything is a complex of vector bundles}.
\end{remark}

\begin{definition}
    Let $f \colon X \to Y$ be a morphism between smooth projective varieties over $\C$. The \emph{pull-back map} is defined to be
    \[
        f^* \colon \kgroup{Y} \to \kgroup{X}, \qquad f^*([\G]) = \sum_{i \in \Z}(-1)^i[\lderived{i}{f^*}\G].
    \]
\end{definition}

\begin{remark}
    First, all the higher inverse image sheaves $\lderived{i}{f^*}\G$ are coherent. This is because $\lderived{}{f^*}$ may be computed using locally free resolutions and by Proposition \ref{pull-back of coherent sheaves}, $f^*\E$ is coherent for any coherent sheaf $\E$. Second, the existence of a long exact sequence for derived functors associated to a short exact sequence of coherent sheaves (Proposition \ref{LES for right derived functor}), makes $f^*$ well defined.    
    
    In many occasions the morphism $f \colon X \to Y$ will be flat, which means that it induces an \emph{exact} pull-back functor $f^* \colon \coherent{Y} \to \coherent{X}$ on the level of categories of coherent sheaves. In this case the above definition greatly simplifies to $f^*([\G]) = [f^*\G]$. This will for instance be the case for the canonical projection $X \times Y \to X$, which is known to be flat.  
    % Second, every flat morphism $f \colon X \to Y$ induces an \emph{exact} pull-back functor $f^* \colon \coherent{Y} \to \coherent{X}$, thus $f^* \colon \kgroup{Y} \to \kgroup{X}$ is well defined. 
\end{remark}

\begin{definition}
    Let $f \colon X \to Y$ be a proper morphism between smooth projective varieties over $\C$. 
    % Let $\F$ be a coherent sheaf on $X$. 
    The \emph{push-forward map} defined on equivalence classes of coherent sheaves is
    \[
        f_! \colon \kgroup{X} \to \kgroup{Y}, \qquad f_!([\F]) = \sum_{i \in \Z}(-1)^i[\rderived{i}{f_*}\F].
    \]
    By the following remark this definition is sound.
\end{definition}

\begin{remark}
    First, by \cite[\S III.3.2, Theorem 3.2.1]{EGA} all the higher direct images $\rderived{i}{f_*}\F$ are coherent.
    Second, the existence of a long exact sequence for right derived functors associated to a short exact sequence of coherent sheaves (Proposition \ref{LES for right derived functor}) implies that $f_!$ is well defined. See also \cite[\S II, Lemma 6.2.6]{Weibel2013}.
\end{remark}

The following proposition captures compatibilities of categorical derived push-forward and derived pull-back functors with their respective $K$-theoretical counterparts.  

\begin{proposition}
    \label{push-forward and pull-back compatibilities}
    \emph{\cite[\S 5.2]{huybrechts2006fouriermukai}}
    Let $f \colon X \to Y$ be a proper morphism between smooth projective complex varieties. For every bounded complex $\F^\bullet$ of $\derb{X}$ and $\G^\bullet$ of $\derb{Y}$, we have
    \[
        f_!([\F^\bullet]) = [\rderived{}{f_*}\F^\bullet] \qquad \text{and} \qquad f^*([\G^\bullet]) = [\lderived{}{f^*}\G^\bullet].
    \]
\end{proposition}

\begin{proof}
    We prove the first equality by applying the \emph{Leray spectral sequence}, which is a special instance of the spectral sequence \eqref{eq: special case of Grothendieck spectral sequence}
    \[
        E^{p,q}_2 = \rderived{p}{f_*}\mathcal{H}^q(\F^\bullet) \implies \rderived{p+q}{f_*}\F^\bullet = E^{p+q}.
    \]
    Observe that 
    \[
        f_!([\F^\bullet]) = \sum_{j\in \Z}(-1)^j f_!([\mathcal{H}^j(\F^\bullet)]) = \sum_{j \in \Z}(-1)^j \sum_{i \in \Z}(-1)^i[\rderived{i}{f_*}\mathcal{H}^j(\F^\bullet)] = \sum_{i, j \in \Z}(-1)^{i+j}[E^{i,j}_2]
    \]  
    and
    \[
        [\rderived{}{f_*}\F^\bullet] = \sum_{n \in \Z}(-1)^n[\rderived{n}{f_*}\F^\bullet] = \sum_{n \in \Z}(-1)^n[E^n].
    \]
    Thus we only need to show that
    \[
        \sum_{i, j \in \Z}(-1)^{i+j}[E^{i,j}_2] = \sum_{n \in \Z}(-1)^n[E^n].
    \]
    After recalling that $H^{p,q}(E_r) = \ker(d^{p,q}_r)/\im(d_r^{p-r, q+r-q})$, and realizing that each page $E_r$ of the spectral sequence consists of a collection of chain complexes, one utilizes \eqref{eq: K-class alternating sum of cohomology} to inductively compute 
    \[
        \sum_{i,j \in \Z}(-1)^{i+j}[E^{i,j}_2] = \sum_{i,j \in \Z}(-1)^{i+j}[E^{i,j}_3] = \cdots = \sum_{i,j \in \Z}(-1)^{i+j}[E^{i,j}_\infty]. 
    \]
    Since $E^{i,j}_\infty \iso F^iE^{i+j}/F^{i+1}E^{i+j}$, we see that
    \begin{align*}
        \sum_{i,j \in \Z}(-1)^{i+j}[E^{i,j}_\infty] &=
        \sum_{i,j \in \Z}(-1)^{i+j}([F^{i}E^{i+j}] - [F^{i+1}E^{i+j}]) \\
        &= \sum_{p,n \in \Z}(-1)^{n}([F^{p}E^{n}] - [F^{p+1}E^{n}]) \\
        &= \sum_{n \in \Z}(-1)^n[E^n],
    \end{align*}
    where the last equation is seen to follow by telescoping along the filtration $(F^pE^n)_p$ of $E^n$.

    We see the second equality holds true, because $\G^\bullet$ may be replaced by an isomorphic (in $\derb{X}$) complex $\E^\bullet$ of vector bundles, guaranteed to exist by Proposition \ref{In Db(X) everything is a complex of vector bundles}. Since vector bundles are $f^*$-acyclic by Corollary \ref{vector bundles f^* acyclic}, we compute
    \begin{equation*}
        f^*([\G^\bullet]) = f^*([\E^\bullet]) = \sum_{i \in \Z}(-1)^i[\lderived{i}{f^*}\E^i] = \sum_{i \in \Z}(-1)^i[f^*\E^i] = [f^*\E^\bullet] = [\lderived{}{f^*}\G^\bullet]. \qedhere
    \end{equation*}
    % For the second equation it is enough to verify it for a single coherent sheaf $\F$ on $Y$, because this time both sides of the equation are clearly additive in $\F$.
\end{proof}

% In particular we will utilize the Grothendieck-Riemann-Roch theorem, which will later on help us relate the two in Section \ref{Subsection: FM transform on cohomology}.

\begin{definition}
    Let $X$ and $Y$ be smooth projective complex varieties. The \emph{$K$-theoretic Fourier--Mukai transform} with kernel $\xi \in \kgroup{X \times Y}$ is defined to be
    \[
        \fmkt{\xi} \colon \kgroup{X} \to \kgroup{Y}, \qquad \fmkt{\xi} := q_!(\xi \cdot p^*(-)).
    \]
\end{definition}

\begin{corollary}
    \emph{\cite[\S 5.2]{huybrechts2006fouriermukai}}
    Let $X$ and $Y$ be smooth projective complex varieties and $\E^\bullet$ a bounded complex of $\derb{X \times Y}$. Then
    \[
        \fmkt{[\E^\bullet]}([-]) = [\fm{\E^\bullet}(-)].
    \]
\end{corollary}

\begin{proof}
    This is a direct consequence of the compatibility between the product on $\kgroup{X}$ and the derived tensor product on $\derb{X}$, according to \eqref{eq: K-theory product compatibility}, and Proposition \ref{push-forward and pull-back compatibilities}.
\end{proof}

\subsection{Fourier--Mukai transforms on rational cohomology}
\label{Subsection: FM transform on cohomology}

In order to understand how a Fourier--Mukai transform descends to rational cohomology, we must first explain what is meant by \emph{rational cohomology}. We will take it to be the singular cohomology with rational coefficients, but applying it directly to the underlying topological spaces of a smooth projective variety over $\C$ is \emph{not} the appropriate notion one has to consider. Instead we have to resort to the famous \emph{GAGA principle} of Serre \cite{Serre1956}, which allows one to identify a smooth projective variety $X$ over $\C$ with its complex analytic space $X\an$, whose underlying set of points consists of precisely the $\C$-points of $X$. In the case that $X$ is a smooth projective variety over $\C$, the associated complex analytic space $X\an$ is a complex projective\footnote{
    A complex manifold $X$ is \emph{projective} if it admits an embedding into complex projective space $\C\mathrm{P}^n$ for some $n \in \N$.
} manifold. Moreover this identification is functorial, meaning that it provides an equivalence between the category of smooth projective varieties over $\C$, considered as a full subcategory of $\sch{\C}$, and the category of complex projective manifolds together with holomorphic maps. Moreover it also states that the categories of coherent sheaves on $X$ and the category of \emph{analytic coherent sheaves} on the corresponding complex analytic space $X\an$ are equivalent. A good reference for this topic is \cite{GriffithsAdams1984}. Henceforth whenever we will be dealing with smooth projective varieties $X$ over $\C$ and considering the rational cohomology groups $\cohom{n}{X}{\Q}$, we will implicitly pass to their complex analytic spaces $X\an$ and interpret the cohomology groups as $\cohom{n}{X\an}{\Q}$.

% are dealing with singular cohomology 

To introduce a Fourier--Mukai type map between the rational cohomology modules of $X$ and $Y$ associated to any cohomology class $\alpha \in \cohom{*}{X \times Y}{\Q}$, we must first define the correct analog of the ``push-forward map'' on cohomology. Again we are assuming that $X$ and $Y$ are smooth projective varieties over $\C$ and we identify them with their respective complex analytic spaces, which are compact connected complex manifolds.
% A Fourier--Mukai type map exists also between the rational cohomology modules of $X$ and $Y$ associated to any cohomology class $\alpha \in \cohom{*}{X \times Y}{\Q}$. In order to introduce it we first need to define the correct analog of the ``push-forward map'' on cohomology. We are again assuming that $X$ and $Y$ are smooth and projective varieties over $\C$ and we identify them with their respective complex analytic spaces, which are in this case compact connected complex manifolds. 
As such they are orientable as real manifolds, allowing Poincaré duality to relate their homology and cohomology. Let $n$ denote the \emph{real} dimension of $X$ and let $[X] \in H_n(X, \Q)$ denote the (rational) fundamental class of $X$. By Poincaré duality \cite[\S VI, Theorem 8.3]{Bredon1993} the homomorphism
\[
    [X] \smallfrown - \colon \cohom{k}{X}{\Q} \to H_{n-k}(X, \Q)
\]
is an isomorphism and we denote its inverse with $D_X \colon H_{n-k}(X, \Q) \to \cohom{k}{X}{\Q}$.

\begin{definition}
    \label{Definition of umkehr map}
    Let $f \colon X \to Y$ be a continuous map between two compact connected complex manifolds. Let $n$ and $m$ denote the real dimensions of $X$ and $Y$, respectively.
    %  $m$ the dimension of $Y$.
    %  of real dimension $n$ and $m$ respectively.
    The \emph{Gysin map}\footnote{This map is sometimes also called the \emph{umkehr map}. The German word ``umkehr'' pointing out the fact that the map goes in the opposite direction of the standard pull-back.} $f_! \colon \cohom{k}{X}{\Q} \to \cohom{m-n+k}{Y}{\Q}$ is defined to be the composition
    \[
        f_! \colon \quad \cohom{k}{X}{\Q} \xrightarrow{[X] \smallfrown -} H_{n-k}(X, \Q) \xrightarrow{\ f_* \ } H_{n-k}(Y, \Q) \xrightarrow{D_Y} \cohom{m-n+k}{Y}{\Q}.
    \]
\end{definition}

\begin{remark}
    In other words $f_!$ is precisely the map, which fits into the commutative diagram below
    \[\begin{tikzcd}
        % https://q.uiver.app/#q=WzAsNCxbMCwwLCJcXGJ1bGxldCJdLFsyLDAsIlxcYnVsbGV0Il0sWzAsMiwiXFxidWxsZXQiXSxbMiwyLCJcXGJ1bGxldCJdLFswLDEsIltYXSBcXHNtYWxsZnJvd24gLSJdLFsyLDMsIltZXSBcXHNtYWxsZnJvd24gLSJdLFsxLDMsImZfKiJdLFswLDIsImZfISIsMl1d
        \cohom{k}{X}{\Q} && H_{n-k}(X, \Q) \\
        \\
        \cohom{m-n+k}{X}{\Q} && H_{n-k}(X, \Q).
        \arrow["{[X] \smallfrown -}", from=1-1, to=1-3]
        \arrow["{f_!}"', from=1-1, to=3-1]
        \arrow["{f_*}", from=1-3, to=3-3]
        \arrow["{[Y] \smallfrown -}", from=3-1, to=3-3]
    \end{tikzcd}\]
\end{remark}

\begin{remark}
    For a topological space $X$ we will distinguish between the collection of all cohomology groups, considered concisely as a graded object $\cohom{\bullet}{X}{\Z} = \bigoplus_{i\in \Z} \cohom{i}{X}{\Z}$, and its (graded) cohomology ring, which we will denote by $\cohom{*}{X}{\Z}$. In that regard $f^* \colon \cohom{\bullet}{Y}{\Q} \to \cohom{\bullet}{X}{\Q}$ is a graded map of degree $0$ and $f_! \colon \cohom{\bullet}{X}{\Q} \to \cohom{\bullet}{Y}{\Q}$ is a graded map of degree $m - n$.
\end{remark}

\begin{proposition}
    \label{cohomological projection formula}
    \emph{\cite[\S VI, Proposition 14.1]{Bredon1993}}
    Let $f \colon X \to Y$ be a continuous map between closed orientable manifolds $X$ and $Y$. Then for all $\alpha \in \cohom{*}{X}{\Z}$ and $\beta \in \cohom{*}{Y}{\Z}$
    \begin{equation}
        \label{eq: cohomological projection formula}
        f_!(\alpha) \smallsmile \beta = f_!(\alpha \smallsmile f^*\beta).
    \end{equation}
\end{proposition}

\begin{remark}
    In the proof of the above proposition we use the following formulas, which are easily seen to follow by unwrapping the definitions of operations $\smallsmile$ and $\smallfrown$ on singular cohomology and homology. They may also be found in \cite[\S VI, Theorem 5.2]{Bredon1993}.
    \begin{enumerate}[label = (\roman*)]
        \item{Let $\eta \in \cohom{k}{X}{\Z}$, $\theta \in \cohom{\ell}{X}{\Z}$ and $\sigma \in H_{k+\ell+m}(X, \Z)$, then 
        \[
            (\sigma \smallfrown \eta) \smallfrown \theta = 
            \sigma \smallfrown (\eta \smallsmile \theta)
        \]
        holds in $H_m(X, \Z)$.}
        \item{Let $\sigma \in H_{k + \ell}(X, \Z)$ and $\eta \in \cohom{k}{Y}{\Z}$ then 
        \[
            f_*(\sigma \smallfrown f^*\eta) = f_*\sigma \smallfrown \eta
        \]
        holds in $H_\ell(Y, \Z)$.
        }
    \end{enumerate}
    Formula \eqref{eq: cohomological projection formula} is often called the \emph{(cohomological) projection formula}.
\end{remark}

\begin{proof}[Proof of Proposition \ref{cohomological projection formula}]
    % Let $n$ and $m$ denote the real dimensions of $X$ and $Y$ respectively.
    We compute that
    \begin{align*}
        f_!(\alpha \smallsmile f^*\beta) &=
        D_Y(f_*([X] \smallfrown (\alpha \smallsmile f^*\beta))) \\
        &= D_Y(f_*(([X] \smallfrown \alpha) \smallfrown f^*\beta)) \\
        &= D_Y(f_*([X] \smallfrown \alpha) \smallfrown \beta) \\
        &= D_Y(([Y] \smallfrown f_!(\alpha)) \smallfrown \beta) \\
        &= D_Y([Y] \smallfrown (f_!(\alpha) \smallsmile \beta)) \\
        &= f_!(\alpha) \smallsmile \beta. \qedhere
    \end{align*}
\end{proof}

\begin{proposition}[Cohomological base change formula]
    Consider the following pull-back square of closed orientable manifolds and assume $g$ and $v$ are fibrations.
    \[\begin{tikzcd}[row sep = 3em]
        % https://q.uiver.app/#q=WzAsNCxbMCwwLCJYJyJdLFswLDEsIlgiXSxbMSwxLCJZIl0sWzEsMCwiWSciXSxbMSwyLCJmIl0sWzMsMiwiZyJdLFswLDEsInYiLDJdLFswLDMsInUiXSxbMCwyLCIiLDEseyJzdHlsZSI6eyJuYW1lIjoiY29ybmVyIn19XV0=
        {X'} & {Y'} \\
        X & Y
        \arrow["u", from=1-1, to=1-2]
        \arrow["v"', from=1-1, to=2-1]
        \arrow["\lrcorner"{anchor=center, pos=0.125}, draw=none, from=1-1, to=2-2]
        \arrow["g", from=1-2, to=2-2]
        \arrow["f", from=2-1, to=2-2]
    \end{tikzcd}\]
    Then
    \begin{equation}
        \label{eq: cohomological base change}
        f^* \circ g_! = v_! \circ u^* 
    \end{equation}
    as homomorphisms $\cohom{\bullet}{Y'}{\Q} \to \cohom{\bullet}{X}{\Q}$.
\end{proposition}

\begin{proof}
    See \cite[\S 6.1, Theorem 6.2]{Fulton1998}.
\end{proof}

As before let $p \colon X \times Y \to X$ and $q \colon X \times Y \to Y$ denote the canonical projections, now in the category of complex manifolds. 

\begin{definition}
    The \emph{cohomological Fourier--Mukai transform} associated to a cohomology class $\alpha \in \cohom{\bullet}{X \times Y}{\Q}$ is defined to be the homomorphism
    \[
        f^\alpha \colon \cohom{\bullet}{X}{\Q} \longrightarrow \cohom{\bullet}{Y}{\Q} \qquad \fmcoh{\alpha} := q_!(\alpha \smallsmile p^*(-))
        % \beta \longmapsto q_!(\alpha \smallsmile p^*\beta).
    \]
\end{definition}

We would now like to relate this cohomological Fourier--Mukai transform back to the $K$-theoretic and the categorical one. This will be achieved by applying the famous Grothendieck--Riemann--Roch theorem, 
% \ref{Grothendieck-Riemann-Roch}, 
for which we first need to introduce the Chern character and the Todd class. 

\subsubsection*{Grothendieck--Riemann--Roch theorem}

In order to relate Fourier--Mukai functors on bounded derived categories of smooth projective varieties over $\C$ with their rational cohomology counterparts we will employ the famous Grothendieck--Riemann--Roch theorem. For us to be able to state it, we first introduce the Chern character and the Todd class of a smooth complex variety.

\begin{definition}
    \label{Definition of Chern character}
    Let $X$ be a smooth projective variety over $\C$. The \emph{Chern character} is a ring homomorphism
    \[
        \ch[] \colon \kgroup{X} \to \cohom{*}{X}{\Q}
    \] 
    enjoying many desirable properties, such as
    \begin{enumerate}[label = (\roman*)]
        \item{For a line bundle $\mathcal L$ on $X$: $\ch[](\mathcal L) = e^{\cclass[1]{\mathcal L}} = 1 + \cclass[1]{\mathcal L} + \frac{1}{2!}\cclass[1]{\mathcal L}^2 + \frac{1}{3!}\cclass[1]{\mathcal L}^3 + \cdots$.
        % \sum_{k=0}^{\infty}\frac{1}{k!}\cclass[1]{\mathcal{L}}^k$.
        } \label{chern character on line bundle}
        \item{\footnote{
            See \ref{Definition of v vee} for the definiton of $(-)^\vee$ and \cite[Lemma C.7]{vanBree2020} for a proof.
        }For any coherent sheaf $\E$ on $X$: $\ch[]{\E^\vee} = \ch[]{\E}^\vee$.} \label{chern character and dual}
        \item{Naturality for pull-back: $f^*\ch[]{e} = \ch[]{f^*e}$ for any proper morphism $f \colon X \to Y$.}
        \item{The Chern character takes values in $\cohom{2\bullet}{X}{\Q}$, \ie cohomology groups of \emph{even} degree. We let $\ch[i](e)$ denote the component of $\ch[]{e}$ living in $\cohom{2i}{X}{\Q}$.}
    \end{enumerate} 
\end{definition}

\begin{remark}
    Such a homomorphism in fact exists and a construction may be found in \cite[\S 3.2]{Fulton1998}. The Chern character is usually constructed by first specifying it on line bundles as in \ref{chern character on line bundle}. Then utilizing the so-called \emph{Splitting principle} \cite[\S 3.2, Remark 3.2.3]{Fulton1998} it is possible to extend the Chern character to arbitrary vector bundles on $X$. Further, one defines the Chern character of a coherent sheaf $\E$ on $X$ using a bounded locally free resolution, mentioned already in Proposition \ref{In Db(X) everything is a complex of vector bundles}. Lastly, one extends the Chern character to arbitrary bounded complexes of coherent sheaves by additivity, \ie
    \[
        \ch[](\F^\bullet) = \sum_{i \in \Z}(-1)^i\ch[](\F^i).
    \] 
    % one uses the fact that every coherent sheaf on $X$ has a bounded resolution with vector bundles, mentioned already in the proof of Proposition \ref{In Db(X) everything is a complex of vector bundles}.
\end{remark}

Along with the Chern character also come Chern classes of vector bundles. For a vector bundle $\E$ on $X$ the $i$-th Chern class will be an element $\cclass[i](\E)$ of $\cohom{2i}{X}{\Z}$. The $0$-th Chern class $\cclass[0](\E)$ equals the rank of the vector bundle $\E$ and is also denoted by $\rk{\E}$. For a construction and overview on Chern classes we again refer the reader to \cite[\S 3.2]{Fulton1998}.
We introduced Chern classes to get a more practical description of the Chern character. Indeed, it turns out that the Chern character evaluated on a vector bundle $\E$ is expressible in terms of the Chern classes $\cclass[i] = \cclass[i](\E)$ of $\E$ as the following series to be found in \cite[\S 3, Example 3.2.3]{Fulton1998}
\begin{align}
    \label{eq: chern character series expansion}
    \ch[]{\E} = \rk{\E} + \cclass[1]{\E} + &\frac{1}{2}\Bigl(\cclass[1]{\E}^2 - 2\cclass[2]{\E}\Bigr) + \notag
    \\ &+ \frac{1}{6}\Bigl(\cclass[1]{\E}^3 - 3 \cclass[1]{\E}\cclass[2]{\E} + 3\cclass[3]{\E}\Bigr) + \cdots.
\end{align}
The Chern classes of $X$ are defined to be the Chern classes $\cclass[i](\tangent{X})$ of the tangent sheaf $\tangent{X} = \Omega_X^\vee$ of $X$ and denoted simply by $\cclass[i](X)$.
% The Chern character exists 



The \emph{Todd class} of a smooth complex projective variety $X$ is a cohomology class, living in even degrees, denoted by
\[
    \tdclass{X} \in \cohom{2\bullet}{X}{\Q}.
\]
% will be denoted by $\tdclass{X}$. 
As with the Chern character and Chern classes we refrain from properly defining or constructing $\tdclass{X}$, but give the first few terms of its series expansion in terms of Chern classes of 
% the tangent sheaf $\tangent{X}$ of 
$X$. It may be found in \cite[\S 3, Example 3.2.4]{Fulton1998} and goes as follows
\begin{equation}
    \label{eq: todd class}
    \tdclass{X} = 1 + \frac{1}{2}\cclass[1](X) + \frac{1}{12}\Bigl(\cclass[1]{X}^2 + \cclass[2]{X}\Bigr) + \frac{1}{24}\cclass[1](X)\cclass[2](X) + \cdots.
\end{equation} 
% Here $\cclass[i](X)$ denotes the $i$-th Chern class of the tangent sheaf $\tangent{X}$ of $X$. 
% The reader can consult \cite[\S 3.2]{Fulton1998} for more details.

% \begin{definition}
%     The \emph{Todd class} of a smooth complex variety $X$ is defined to be
%     \[
%         \tdclass{X} = 
%     \]
%     where $\tangent{X}$ is the tangent sheaf of $X$, \ie $\tangent{X} = \Omega_X^*$.  
%     We define this to also be the Todd class of the corresponding compact complex manifold $X$ considered as an analytic space. 
% \end{definition}



% \begin{remark}
%     From $\Omega_{X \times Y} \iso p^*\Omega_X \oplus q^*\Omega_Y$, we also see that $\tangent{X \times Y}$ so
%     \[
%         \tdclass{X \times Y} = p^*\tdclass{X} \smallsmile q^*\tdclass{Y}
%     \]
%     Since the degree $0$ part of the Todd class is $1$, we see that it is invertible in the ring $\cohom{*}{X}{\Q}$ and its inverse can be given by a power series.
% \end{remark}

\begin{theorem}[Grothendieck--Riemann--Roch, \text{\cite[\S 15, Theorem 15.2]{Fulton1998}}]
    \label{Grothendieck-Riemann-Roch}
    Let $X$ and $Y$ be smooth complex projective varieties and ${f \colon X \to Y}$ a proper morphism. Then for all $e \in \kgroup{X}$, equation 
    \begin{equation}
        \label{eq: GRR}
        \ch[]{f_!(e)} \smallsmile \tdclass{Y} = f_!(\ch[]{e} \smallsmile \tdclass{X})
    \end{equation}
    holds in $\cohom{*}{Y}{\Q}$.
\end{theorem}

% \begin{proof}
%     A proof may be found in \cite[\S 15]{Fulton1998}.
%     % The statement is \cite[\S 15, Theorem 15.2]{Fulton1998} and the proof may be found in Chapter 15 of this reference.
% \end{proof}

\begin{remark}
    In the formulation of the Grothendieck--Riemann--Roch theorem in \cite[\S 15, Theorem 15.2]{Fulton1998} the equation \eqref{eq: GRR} holds inside the so-called (rational) \emph{Chow ring} $A(Y)_\Q = A(Y) \otimes_{\Z} \Q$ instead of $\cohom{*}{Y}{\Q}$. We will not define this ring, but mention that there exists a push-forward map also on these Chow rings $f_* \colon A(X) \to A(Y)$ and homomorphisms $\operatorname{cl} \colon A(X) \to \cohom{*}{X}{\Z}$, for each smooth complex variety $X$, called \emph{cycle maps}, which are covariant in $X$ and compatible with Chern classes. This enables us to state the theorem in the way that we did. More on this topic can be found in \cite[\S 19]{Fulton1998}.
    % such that the diagram below commutes.
    % \[\begin{tikzcd}
    %     % https://q.uiver.app/#q=WzAsNCxbMCwwLCJcXGJ1bGxldCJdLFsyLDAsIlxcYnVsbGV0Il0sWzAsMiwiXFxidWxsZXQiXSxbMiwyLCJcXGJ1bGxldCJdLFswLDEsIltYXSBcXHNtYWxsZnJvd24gLSJdLFsyLDMsIltZXSBcXHNtYWxsZnJvd24gLSJdLFsxLDMsImZfKiJdLFswLDIsImZfISIsMl1d
    %     A(X)_\Q && H^{*}(X, \Q) \\
    %     \\
    %     A(Y)_\Q && H^{*}(Y, \Q)
    %     \arrow["", from=1-1, to=1-3]
    %     \arrow["{f_*}"', from=1-1, to=3-1]
    %     \arrow["{f_!}", from=1-3, to=3-3]
    %     \arrow["", from=3-1, to=3-3]
    % \end{tikzcd}\]
    % This enables us to state the theorem in the way that we did. 
\end{remark}

For a coherent sheaf $\E$ on $X$ set $h^i(X, \E) = \dim_\C H^i(X, \E)$, which we know to be finite by Theorem \ref{Serre finitness}. We also know that $\cohom{i}{X}{\E} = 0$ for $i > \dim X$ by a vanishing result of Grothendieck \ref{Grothendieck vanishig}.
%  we already know that $\cohom{i}{X}{\E} = 0$ for $i > \dim X$ by a finiteness result of Serre (Theorem \ref{Serre finitness}). 
The concept of an \emph{Euler characteristic} of $\E$, set out to be
\[
    \chi(X, \E) = \sum_{i \in \Z} (-1)^i h^i(X, \E),
\] 
is then well defined. From the existence of a long exact sequence in cohomology \eqref{eq: LES in cohomology of SES of coherent sheaves}, we see that $\chi(X, -)$ is additive on short exact sequences, \ie for a short exact sequence of coherent sheaves $0 \to \E \to \F \to \G \to 0$, we have
\[
    \chi(X, \F) = \chi(X, \E) + \chi(X, \G).
\]
Precisely this property allows us to extend $\chi(X,-)$ to a group homomorphism on $\kgroup{X}$.
\begin{theorem}[Hirzebruch--Riemann--Roch, \text{\cite[\S 15, Corollary 15.2.1]{Fulton1998}}]
    \label{Hirzebruch-Riemann-Roch}
    Let $X$ be a smooth complex projective variety and $\E$ a coherent sheaf on $X$. Then
    \begin{equation}
        \label{eq: Hirzebruch-Riemann-Roch formula}
        \chi(X, \E) = \int_X \ch{\E}\tdclass{X}.
    \end{equation}
\end{theorem}

% Evaluated against a fundamental class $[X]$ 

\begin{remark}
    The integral on the right hand side we will take to be $\int_X \ch{\E}\tdclass{X} = \pairing{(\ch{\E}\tdclass{X})_n}{[X]}$, where $n$ denotes the real dimension of $X$ and $(-)_n$ the $n$-th degree component of a cohomology class. 
    % In other words to compute the right side we only calculate the top degree contributions of the product $\ch{\E}\td{X}$
    % Letting $n$ denote the real dimension of $X$, 
    Computing the integral then amounts to calculating the $n$-th degree contributions of the product $\ch{\E}\tdclass{X}$ and evaluating them into a rational number through the isomorphism $\cohom{n}{X}{\Q} \iso \Q$, given by the choice of a fundamental class $[X]$.
\end{remark}

\begin{proof}
    We will use the Grothendieck--Riemann--Roch formula for the structure morphism $\varepsilon \colon X \to \spec{\C}$. The Chern character on a point is completely determined by its image of the trivial line bundle $\mathcal L = \struct{\spec{\C}}$. This is because $\coherent{\spec{\C}}$ is equivalent to the category of finite dimensional $\C$-vector spaces and $\ch[]$ is additive. In this case we have
    \[
        \ch[]{\mathcal L} = e^{\cclass[1]{\mathcal L}} = e^0 = 1.
    \]
    Therefore after identifying the ring $\cohom{*}{\spec{\C}}{\Q}$ with $\Q$, we see that $\ch[](-) = \dim_\C(-)$. We also note that $\tdclass{\spec{\C}} = 1$ in this cohomology ring. The left-hand side of \eqref{eq: GRR} is then
    \[
        \ch[]{\varepsilon_!(\E)}\smallsmile 1 = \ch[]{\sum_{i \in \Z}(-1)^i[\rderived{i}{\varepsilon_*}\E]} = \sum_{i\in \Z}(-1)^i h^i({X},{\E}) = \chi(X, \E).
    \]  
    Evaluating the right-hand side only amounts to interpreting the meaning of $\varepsilon_!$ according to the remark before this proof.
    % Whereas the right-hand side is 
    % \[
    %     \varepsilon_!(\ch[]{\E} \smallsmile \tdclass{X}) = \langle \ch[]{\E} \smallsmile \tdclass{X}, [X] \rangle = \int_X \ch[]{\E} \smallsmile \tdclass{X}.
    % \]
\end{proof}

\begin{remark}
    \label{HRR is an equality of homomorphisms}
    Notice that since the left-hand side of equation \eqref{eq: Hirzebruch-Riemann-Roch formula} is an integer, the integral on the right must be as well. 
    We also remark that since both sides of \eqref{eq: Hirzebruch-Riemann-Roch formula} are additive in $\E$, the formula actually describes an equality of two group homomorphisms $\kgroup{X} \to \Z$. This is no surprise, as this same behaviour was already present in the Grothendieck--Riemann--Roch formula, from which we have seen the Hirzebruch--Riemann--Roch formula follows.
\end{remark}

\subsubsection*{Mukai vector}

Of special importance for relating the $K$-theoretic Fourier--Mukai transforms with their cohomological counterparts is the Mukai vector. 

\begin{definition}
    The \emph{Mukai vector} of a class $e \in \kgroup{X}$ is defined to be
    \[
        v(e) := \ch[](e)\sqrt{\tdclass{X}}.
    \]
    The Mukai vector of a complex $\E^\bullet$ of $\derb{X}$ is defined to be the Mukai vector of the corresponding class $[\E^\bullet] \in \kgroup{X}$, so $v(\E^\bullet) := v([\E^\bullet])$.
\end{definition}

\begin{remark}
    \label{sqrt of cohomology class}
    Recall that for compact manifolds $X$ the cohomology ring $\cohom{*}{X}{\Q}$ is finite dimensional. Thus we can get away with defining the square root of a cohomology class of the form $1 + \alpha \in \cohom{*}{X}{\Q}$, where $\alpha$ contains no degree $0$ homogeneous term, by a power series
    % As a consequence we may for example define the square root of a cohomology class $\alpha \in \cohom{*}{X}{\Q}$ by a power series
    \begin{equation}
        \label{eq: sqrt power series}
        \sqrt{1 + \alpha} = 1 + \tfrac{1}{2}\alpha - \tfrac{1}{8}\alpha^2 + \tfrac{1}{16}\alpha^3 + \cdots \in \cohom{*}{X}{\Q}.
    \end{equation}
    In particular this lets us define $\sqrt{\tdclass{X}}$.
    % Defining the exponential $e^\alpha \in \cohom{*}{X}{\Q}$ is also done in a similar manner. 
    % \info{I see problems with this now. What if $\alpha$ is not homogeneous?}
\end{remark}

\begin{example}
    \label{Mukai vector of skyscraper}
    Let us compute the Mukai vector of a skyscraper sheaf $k(x)$ associated to a closed point $x$ of a smooth $n$-dimensional projective variety $X$ over $\C$. Let $x \colon \spec{\C} \to X$ also denote the corresponding morphism and let
    % For a smooth complex projective variety $X$, we 
    $\mu_X \in \cohom{n}{X}{\Q} \iso \Q$ denote the generator for which $\pairing{\mu_X}{[X]} = 1$. By the Grothendieck--Riemann--Roch formula we compute
    \[
        \ch[](k(x))\tdclass{X} = \ch[]{x_*\struct{\spec{\C}}} \tdclass{X} = x_!(\ch[]{\struct{\spec{\C}}}\tdclass{\spec{\C}}) = x_!(1) = \mu_X.
    \]
    Thus we see that
    \[
        v(k(x)) = \mu_X \smallsmile \sqrt{\tdclass{X}}\inv.
    \]
\end{example}

\begin{proposition}
    \label{Mukai vector FM transform interaction}
    \emph{\cite[\S 5.2, Corollary 5.29]{huybrechts2006fouriermukai}}
    For each kernel $\xi \in \kgroup{X \times Y}$ and any $e \in \kgroup{X}$ we have
    \[
        \fmcoh{v(\xi)}(v(e)) = v(\fmkt{\xi}(e))
    \]  
\end{proposition}

\begin{proof}
    This equation follows from the following computation
    \begin{align*}
        \fmcoh{v(\xi)}(v(e)) &=
        q_!(v(\xi) \smallsmile p^*v(e)) \\
        &= q_!(\ch[](\xi)\sqrt{\tdclass{X \times Y}} \smallsmile p^*(\ch[]{e}\sqrt{\tdclass{X}})) \\
        &= q_!(\ch[](\xi)p^*\sqrt{\tdclass{X}} \smallsmile q^*\sqrt{\tdclass{Y}} \smallsmile \ch[]{p^*(e)}p^*\sqrt{\tdclass{X}}) & \text{(Def. \ref{Definition of Chern character})} \\
        &= q_!(\ch[](\xi \cdot p^*(e))\tdclass{X \times Y} \smallsmile q^*\sqrt{\tdclass{Y}}\inv) & \text{(Formula \eqref{eq: projection formula})}\\
        &= q_!(\ch[](\xi \cdot p^*(e))\tdclass{X \times Y}) \smallsmile \sqrt{\tdclass{Y}}\inv & \text{(Formula \eqref{eq: GRR})} \\
        &= \ch[]{q_!(\xi \cdot p^*(e))}\sqrt{\tdclass{Y}} \\
        &= v(\fmkt{\xi}(e)). & & \qedhere
    \end{align*}
\end{proof}

From now on a cohomological Fourier--Mukai transform, whose kernel is a Mukai vector $v(\E)$ of some complex $\E$ of $\derb{X \times Y}$, will be denoted by $\fmcoh{\E} \colon \cohom{\bullet}{X}{\Q} \to \cohom{\bullet}{Y}{\Q}$ instead of $\fmcoh{v(\E)}$.   

\begin{proposition}
    \emph{\cite[\S 5.2, Lemma 5.32]{huybrechts2006fouriermukai}}
    \label{Composition of cohomological fm is fm}
    Let $\fmcoh{\E} \colon H^\bullet(X, \Q) \to H^\bullet(Y, \Q)$ and $\fmcoh{\F} \colon H^\bullet(Y, \Q) \to H^\bullet(Z, \Q)$ be cohomological Fourier--Mukai transforms for kernels $\E$ and $\F$ belonging to $\derb{X \times Y}$ and $\derb{Y \times Z}$ respectively. Then the composition $\fmcoh{\F} \circ \fmcoh{\E}$ is also a cohomological Fourier--Mukai transform given by the kernel $\G$ of Proposition \ref{Composition of fm is fm}
    \[
        \fmcoh{\G} = \fmcoh{\F} \circ \fmcoh{\E}.
    \]
\end{proposition}

\begin{remark}
    The proof of this result is analogous to the proof of Proposition \ref{Composition of fm is fm}, where instead we use the cohomological projection and base change formulas, \eqref{eq: cohomological projection formula} and \eqref{eq: cohomological base change}, respectively. We also mention that the Mukai vector $v(\G)$ turns out to be equal to the class $(\pi_{XZ})_!(\pi_{XY}^*v(\E) \smallsmile \pi_{YZ}^*v(\F))$ in $\cohom{*}{X \times Z}{\Q}$, which is also an ingredient we need for the computation above, but we omit the tedious verification.
    % We also note that in this case the Mukai vector $v(\G)$ is transformed to $(\pi_{XZ})_!(\pi_{XY}^*v(\E) \smallsmile \pi_{YZ}^*v(\F))$.
\end{remark}

\begin{definition}
    \label{Definition of v vee}
    For a vector $v \in \cohom{2\bullet}{X}{\Q} = \bigoplus_{i \in \Z} \cohom{2i}{X}{\Q}$ confined in \emph{even} degrees, with components $v = (v_0, v_1, v_2, v_3, \dots)$, we define $v^\vee = (v_0, -v_1, v_2, -v_3, \dots)$.
    % define
    % $v^\vee \in \cohom{2\bullet}{X}{\Q}$ with components $(v^\vee)_{2i} = (-1)^i v_{2i}$. In other words for $v = (v_0, v_1, v_2,\dots)$ 
\end{definition}

This operation is clearly natural for pull-backs and fixes all cohomology classes living in degrees, which are multiples of $4$.  

\begin{definition}
    \label{Definition of Mukai pairing}
    For any $v$, $w \in \cohom{2\bullet}{X}{\Q}$ coming from the even part, we define the \emph{Mukai pairing} on $\cohom{2\bullet}{X}{\Q}$ to be
    \[
        \pairing{v}{w} = \int_X v^\vee \smallsmile w.
    \]
\end{definition}

% Since the Mukai vector of any complex $\E^\bullet$ is spread out only in the even cohomology groups, we can pair their Mukai vectors with each other in this way. Another way of pairing up (bounded complexes of) coherent sheaves is given by the \emph{Euler pairing}, defined for $\E^\bullet$ and $\F^\bullet$ belonging to $\derb{X}$ to be the alternating sum
Since the Mukai vector of any coherent sheaf $\E$ is spread out only in the even cohomology groups, we can pair their Mukai vectors with each other in this way. Another way of pairing up coherent sheaves is given by the \emph{Euler pairing}, defined for coherent sheaves $\E$ and $\F$ on $X$ to be the alternating sum
% \begin{equation}
%     \label{Euler pairing}
%     \chi(\E^\bullet, \F^\bullet) = \sum_{i \in \Z} (-1)^i \dim_\C \Ext^i_\struct{X}(\E^\bullet, \F^\bullet).
% \end{equation}
\begin{equation}
    \label{Euler pairing}
    \chi(\E, \F) = \sum_{i \in \Z} (-1)^i \dim_\C \Ext^i_\struct{X}(\E, \F).
\end{equation}
The following proposition says that these two pairings of coherent sheaves are the same.
% basically the same, differing only by a sign. 

\begin{proposition}
    \label{equler pairing = mukai pairing}
    \emph{\cite[\S I, Lemma 3.1.7]{caldararu2000derived}}
    Let $\E$ and $\F$ be coherent sheaves on $X$. Then
    \begin{equation}
        \label{eq: euler pairing = mukai pairing}
        \pairing{v(\E)}{v(\F)} = \chi(\E, \F).
    \end{equation}
\end{proposition}
% \begin{proposition}
%     \label{equler pairing = mukai pairing}
%     Let $\E^\bullet$ and $\F^\bullet$ be complexes of coherent sheaves belonging to $\derb{X}$. Then
%     \[
%         \pairing{v(\E^\bullet)}{v(\F^\bullet)} = \chi(\E^\bullet, \F^\bullet)
%     \]
% \end{proposition}

\begin{proof}
    We first prove this equality when $\E$ is a vector bundle and then argue why this is sufficient. First, we note that $\localhom[X](\E, \F) \iso \E^\vee \otimes_{\struct{X}} \F$ and this isomorphism is functorial in $\F$. Moreover, since $\E^\vee$ is a vector bundle, the functor $\E^\vee \otimes(-)$ is exact. Using Proposition \ref{composition of derived functors} we compute
    \begin{align*}
        \Ext^i_\struct{X}(\E, \F) &= H^i(\rderived{}{\Hom_{\struct{X}}(\E, -)}(\F)) \\
        &\iso H^i((\rderived{}{\Gamma} \circ \rderived{}{\localhom[X]}(\E, -))(\F)) \\
        &\iso H^i((\rderived{}{\Gamma}\circ(\E^\vee \otimes -))(\F)) \\
        &\iso \rderived{i}{\Gamma}(\E^\vee \otimes \F) \\
        &\iso H^i(X, \E^\vee \otimes \F).
    \end{align*}    
    Applying the Hirzebruch--Riemann--Roch formula \eqref{eq: Hirzebruch-Riemann-Roch formula}, we see that 
    \[
        \pairing{v(\E)}{v(\F)} = \int_X \ch[]{\E}^\vee\ch[]{\F}\tdclass{X} = \int_X \ch[]{\E^\vee \otimes \F}\tdclass{X} = \chi(X, \E^\vee \otimes \F) = \chi(\E, \F).
    \]
    Since both sides of the equation \eqref{eq: euler pairing = mukai pairing} are additive on short exact sequences in the variable $\E$, the general formula follows from this special case of vector bundles. This is because every coherent sheaf on $X$ has a bounded locally free resolution. 
\end{proof}

% \begin{proof}
%     This is a straight forward application of the Hirzebruch--Riemann--Roch formula \eqref{eq: Hirzebruch-Riemann-Roch formula}, once we realize that we have the following isomorphism in $\derb{X}$
%     \[
%         {\E^\bullet}^\vee \derivedtensor{}{}{\struct{X}} \F^\bullet \iso \rderived{}{\localhom[X]^\bullet(\E^\bullet, \F^\bullet)}.
%     \]
%     Indeed,
%     \begin{align*}
%         \pairing{v(\E^\bullet)}{v(\F^\bullet)} &= \int_X v(\E^\bullet)^\vee \smallsmile v(\F^\bullet) \\
%         &= \int_X \ch[]{{\E^\bullet}^\vee \derivedtensor{}{}{\struct{X}} \F^\bullet}\tdclass{X} \\
%         &= \chi(X, \rderived{}{\localhom[X]^\bullet(\E^\bullet, \F^\bullet)}) \\
%         &= \sum_{i \in \Z}(-1)^i \dim_\C \rderived{i}{\Gamma}(\rderived{}{\localhom[X]^\bullet(\E^\bullet, \F^\bullet)}) \\
%         &= \sum_{i\in \Z}(-1)^i \dim_\C \Ext^i_{\struct{X}}(\E^\bullet, \F^\bullet) \\
%         &= \chi(\E^\bullet, \F^\bullet).
%     \end{align*}
% \end{proof}

\subsubsection*{Hodge structures}

This short section serves as a quick recollection on the very basics of Hodge theory and ends with a proposition describing a relationship between a cohomological Fourier--Mukai transform and some Hodge decompositions. 

\begin{definition}
    A \emph{Hodge structure of weight $n \in \Z$} on a finitely generated free abelian group or a finite dimensional $\Q$-vector space $V$ is a direct sum decomposition of $V_\C = V \otimes_\Z \C$, consisting of subspaces $V^{p,q} \subseteq V_\C$ for $p$, $q \in \Z$ of the form
    \[  
        V_\C = \bigoplus_{p+q = n} V^{p,q},
    \]
    satisfying $\ols{V^{p,q}} = V^{q,p}$ for all pairs $p$, $q \in \Z$. If $V$ is a $\Z$-module, we say it is equipped with an \emph{integral} Hodge structure and if $V$ is a $\Q$-vector space, we say the Hodge structure is \emph{rational}.
    
    A $\Z$-linear (\resp $\Q$-linear) homomorphism $f \colon V \to W$ between two integral (\resp rational) Hodge structures of weight $n$ is said to \emph{preserve the Hodge structures} if
    \[
        f_\C(V^{p,q}) \subseteq W^{p,q}
    \]  
    for all $p$, $q \in \Z$, with $p + q = n$, where $f_\C$ denotes the complexification $f \otimes \id{\C}$.
\end{definition}

\begin{remark}
    Complex conjugation on $V_\C = V \otimes_\Z \C$ is defined on elementary tensors as $\ols{v \otimes \lambda} = v \otimes \ols{\lambda}$.
\end{remark}

A prototypical example of a Hodge structure of weight $n$ comes from complex geometry and is given by the following decomposition of the $n$-th cohomology group $\cohom{n}{X}{\C}$ of a complex projective manifold $X$
\begin{equation}
    \label{eq: Hodge decompositon on complex manifold}
    H^n(X, \Q)_\C = \bigoplus_{p+q = n} H^{p,q}(X).
\end{equation}
The groups $H^{p,q}(X)$ are defined to be the sheaf cohomology groups $\cohom{q}{X}{\Omega_X^p}$. 
% which turn out to be isomorphic to the \emph{Dolbeault cohomology groups}.
% may also be identified with \emph{Dolbeault cohomology groups}. 
Here $\Omega_X^p$ denotes the sheaf of holomorphic $p$-forms or more precisely the $p$-th exterior power $\bigwedge^p \Omega_X$ of the cotangent sheaf $\Omega_X$ on $X$. 
The groups $H^{p,q}(X)$ turn out to be isomorphic to the so-called \emph{Dolbeault cohomology groups} $H^{p,q}_{\olsi{\partial}}(X)$, which are in turn easier to identify with certain subspaces of globally defined $(p,q)$-forms, which are called \emph{harmonic} forms. This shift of perspective, from cohomology classes to globally defined differential forms, then makes the origin of a Hodge decomposition of $\cohom{n}{X}{\C}$ at least somewhat reasonable and we mention that it arises as a result of the \emph{Hodge decomposition theorem} \cite[\S 0.6]{GriffithsHarris1994}.
% The Hodge decomposition of $\cohom{n}{X}{\Q}$ then arises from the \emph{Hodge decomposition theorem} \cite[text]{keylist} through identifying the Dolbeault cohomology groups with certain subspaces of globally defined $(p,q)$-forms, called \emph{harmonic} forms. 
We collect a few facts, which one is able to deduce from this viewpoint. They can be found in standard texts like \cite{GriffithsHarris1994} or \cite{Voisin2002}.
\begin{itemize}[label = $\vartriangleright$ ]
    \item{For any $p$, $q \in \Z$: $\olsi{H^{p,q}(X)} = H^{q,p}(X)$ \cite[\S II, Corollary 6.12]{Voisin2002}.}
    \item{The exterior product of a $(p,q)$-form $\alpha$ with a $(r,s)$-form $\beta$ is a $(p+r, q+s)$-form $\alpha \wedge \beta$. Consequently, if $n$ denotes the complex dimension of $X$ and if either $p+r > n$ or $q + s > n$, then $\alpha \wedge \beta = 0$ \cite[\S II, Corollary 6.15]{Voisin2002}}.
    \item{For any holomorphic map $f \colon X \to Y$, the pull-back $f^* \colon \cohom{\bullet}{Y}{\Q} \to \cohom{\bullet}{X}{\Q}$ preserves the Hodge structures, meaning that if $\beta$ is a $(p,q)$-from on $Y$, $f^*\beta$ is a $(p,q)$-from on $X$. See \cite[\S 7.3.2]{Voisin2002}.}
    \item{Let $n$ and $m$ denote the \emph{complex} dimensions of compact complex manifolds $X$ and $Y$, respectively and set $k = m-n$. Then for any holomorphic map $f \colon X \to Y$, the Gysin map $f_! \colon \cohom{\bullet}{X}{\C} \to \cohom{\bullet}{Y}{\C}$ is of bi-degree $(k,k)$, meaning that a $(p,q)$-form $\alpha$ on $X$ is sent to a $(p+k, q+k)$-form $f_!\alpha$ on $Y$. In that regard $f_!$ does not preserve the Hodge structure unless $X$ and $Y$ are equidimensional. See \cite[\S 7.3.2]{Voisin2002}.}
    \item{Characteristic classes like the Chern character and the Todd class are \emph{algebraic}, meaning they live in
    % belong to rational cohomology, they are invariant under complex conjugation, meaning they live in the \emph{algebraic part} of $\cohom{\bullet}{X}{\C}$, \ie 
    \[
        \bigoplus_{p \in \Z} H^{p,p}(X) \cap \cohom{2p}{X}{\Q}.
    \]} 
\end{itemize} 

A useful invariant related to the Hodge decomposition \eqref{eq: Hodge decompositon on complex manifold} are the \emph{Hodge numbers}
\[
    h^{p,q}(X) = \dim_{\C} H^{p,q}(X).
\]

\begin{remark}
    In this section the meaning of cohomology groups $\cohom{n}{X}{\C}$ is a bit blurred. At times we take it to be the singular cohomology with complex coefficients, this makes it easy to understand how $\cohom{n}{X}{\Q}$ might be a subgroup of $\cohom{n}{X}{\C}$. At other times we take it to be the complex de Rham cohomology, which is better suited for Hodge decompositions. In any case our worries may be remedied by the complex analogue of \emph{de Rham's theorem}, which states that for compact complex manifolds $X$ there is a natural isomorphism of graded rings
    \[
        H^*_{\operatorname{dR}}(X, \C) \to H^*_{\operatorname{sing}}(X, \C).
    \]
    For a proof using a sheaf-theoretic argument see \cite[p.\ 43--45]{GriffithsHarris1994}.
\end{remark}

% This viewpoint also allows one to see why
% \[
%     \olsi{H^{p,q}(X)} = H^{q,p}(X) 
% \]
% holds for any $p$, $q$.


% that the $\smallsmile$-product on the cohomology ring $\cohom{*}{X}{\C}$ defines  


% We will describe the groups $H^{p,q}(X)$ using our established language of derived functors but refer the reader to consult \cite[text]{keylist} for many of the details left out.

% For example in the case of a $2$-dimensional complex manifold  

Cohomological Fourier--Mukai transforms unfortunately do not preserve the Hodge structures in general\footnote{
    In the same breath we have to mention that for K3 surfaces Fourier--Mukai transforms actually \emph{do} preserve the Hodge structures, as on can deduce from Lemma \ref{isometry of Mukai lattices preserving H20}. However note that one uses also a certain lattice structure on the cohomology ring to prove this fact.
}, but they interact with Hodge structures in a particular way. At least when they arise from equivalences of the corresponding bounded derived categories, they preserve another kind of grading.

\begin{proposition}
    \label{Hodge lattice, fm transform interaction}
    \emph{\cite[\S 5, Proposition 5.39]{huybrechts2006fouriermukai}}
    Let $X$ and $Y$ be complex projective varieties over $\C$. Consider the cohomological Fourier--Mukai transform $\fmcoh{\E} \colon \cohom{\bullet}{X}{\Q} \to \cohom{\bullet}{Y}{\Q}$ associated to a kernel $\E$ belonging to $\derb{X \times Y}$, for which $\fm{\E} \colon \derb{X} \to \derb{Y}$ is an equivalence.
    % Then for every pair of integers $(p,q)$ 
    Then for every $i \in \Z$, transform $\fmcoh{\E}$ induces isomorphisms
    % \begin{equation*}
    %     % \label{eq: Hodge lattice, fm transform}
    %     \fmcoh{\E}_\C\Biggl(\  \bigoplus_{p-q = i}H^{p,q}(X)\Biggr) \subseteq \bigoplus_{r - s = i} H^{r,s}(Y).
    % \end{equation*}
    \begin{equation*}
        % \label{eq: Hodge lattice, fm transform}
        \bigoplus_{p-q = i}H^{p,q}(X) \iso \bigoplus_{p - q = i} H^{p,q}(Y).
    \end{equation*}
\end{proposition}

\begin{proof}
    First we show that $\fmcoh{\E}$ is an isomorphism. As $\fm{\E}$ is an equivalence, its left adjoint $\fm{\ladj{\E}}$ is its quasi-inverse, so there is a natural isomorphism of functors $\fm{\ladj{\E}} \circ \fm{\E} \natiso \id{\derb{X}}$. On the other hand, by Proposition \ref{Composition of fm is fm}, we know that the composition $\fm{\ladj{\E}} \circ \fm{\E}$ is also a Fourier--Mukai transform, given by some kernel $\G$ of $\derb{X \times X}$. Hence by Orlov's theorem \ref{Orlov's theorem}, stipulating uniqueness of kernels of Fourier--Mukai equivalences, we obtain $\G \iso \struct{\Delta}$, because $\id{\derb{X}} \natiso \fm{\struct{\Delta}}$. On the level of cohomology we therefore see that $\fmcoh{\ladj{\E}} \circ \fmcoh{\E} = \fmcoh{\struct{\Delta}}$. Borrowing Lemma \ref{cohomological fm of diagonal} from a future chapter and noting that the exact same proof works in our slightly more general case, we get $\fmcoh{\struct{\Delta}} = \id{\cohom{\bullet}{X}{\Q}}$. This means that $\fmcoh{\E}$ is an isomorphism.
    % natural equivalence $\fm{\ladj{E}} \circ \fm{\E} \natiso \id{\derb{X}}$ implies that
    % We borrow a result from a future chapter saying that in the case that $\fm{\E}$ is an equivalence, $\fmcoh{\E}$ is an isomorphism. See the \textsl{Bijection} part of the proof of Theorem \ref{Derived Torelli} on page \pageref{proof of ->}. 
    Now it is enough to prove that for each $p$, $q \in \Z$ the following inclusion holds
    \begin{equation}
        \label{eq: Hodge lattice, fm transform}
        \fmcoh{\E}_\C\left(H^{p,q}(X)\right) \subseteq \bigoplus_{r - s = p - q} H^{r,s}(Y).
    \end{equation}
    The Mukai vector $v(\E) = \ch[]{\E}\sqrt{\tdclass{X \times Y}}$ being a product of two algebraic classes is also algebraic, meaning that it is a sum of classes of type $(p,p)$, \ie
    \[
        v(\E) \in \bigoplus_{p \in \Z} H^{p,p}(X \times Y) \cap \cohom{2p}{X \times Y}{\Q}.
    \] 
    % We are now going to express $v(\E)$ in terms of two types of decompositions into classes belonging to $H^{p,q}(X)$ and $H^{r,s}(Y)$.
    % The first decomposition is given the \emph{Künneth formula}, which allows us to decompose $\cohom{n}{X \times Y}{\Q}$ into a direct sum
    % \[
    %     \cohom{n}{X \times Y}{\Q} \iso \bigoplus_{i+j = n} \cohom{i}{X}{\Q} \otimes_\Q \cohom{j}{Y}{\Q}.
    % \]
    % Further, by the Hodge decomposition, each cohomology group $\cohom{i}{X}{\Q}$ decomposes into $\bigoplus_{p+q = i}H^{p,q}(X) \cap \cohom{i}{X}{\Q}$ and similarly for the cohomology groups $\cohom{j}{Y}{\Q}$. The Mukai vector $v(\E) \in \cohom{\bullet}{X \times Y}{\Q}$ can therefore be expressed as
    % \[
    %     v(\E) = \sum_{\substack{p,q,r,s \in \Z \\ p+r = q+s}} p^*\alpha^{p,q} \smallsmile q^*\beta^{r,s},
    % \] 
    % where $\alpha^{p,q} \in H^{p,q}(X)$ and $\beta^{r,s} \in H^{r,s}(Y)$ and the condition $p+r = q+s$ is there because $v(\E)$ is algebraic. 
    % Let $n$ and $m$ denote the dimensions of $X$ and $Y$, respectively, and set $k = m-n$. 
    Consider the complexified\footnote{
        The universal coefficients theorem allows us to just swap $\Q$ for $\C$ everywhere, because the isomorphisms $\cohom{n}{X}{\Q}\otimes \C \iso \cohom{n}{X}{\C}$ and $H_n(X, \Q)\otimes \C \iso H_n(X, \C)$ are natural. 
    } cohomological Fourier--Mukai transform $\fmcoh{\E}$ as the composition
    \[  
        \fmcoh{\E}_\C \colon \quad \cohom{\bullet}{X}{\C} \xrightarrow{\ p^* \ } \cohom{\bullet}{X \times Y}{\C} \xrightarrow{v(\E) \smallsmile -} \cohom{\bullet}{X \times Y}{\C} \xrightarrow{\ q_! \ } \cohom{\bullet}{Y}{\C}.
    \]
    Pick any class $\theta \in H^{k, \ell}(X)$. Since $p^*$ is of bi-degree $(0,0)$, $p^*\theta \in H^{k,\ell}(X \times Y)$. Multiplying by $v(\E)$ results in a sum of classes of type $(k + t, \ell + t)$ for $t \in \Z$, \ie
    \[
        v(\E) \smallsmile p^*\theta \in \bigoplus_{t \in \Z}H^{k + t, \ell + t}(X \times Y)
    \]
    Lastly, since the push-forward $q_!$ is of bi-degree $(n,n)$, where $n$ denotes the dimension of $X$, we end up with a sum of classes of type $(k + t + n, \ell + t + n)$ for $t \in \Z$. Re-indexing shows that $\fmcoh{\E}(\theta)$ is a sum of classes of type $(r,s)$, where $r$, $s$ run over the integers, satisfying $r - s = k - \ell$, \ie
    \[
        \fmcoh{\E}(\theta) = q_!(v(\E) \smallsmile p^*\theta) \in \bigoplus_{\substack{r,s\in \Z \\ r-s = k - \ell}}H^{r,s}(Y). \qedhere
    \]
\end{proof}
\newpage

\section{K3 surfaces}
\label{Chapter: K3}

This chapter serves as a gentle introduction and overview of the main properties of K3 surfaces. 

We will start off by defining algebraic K3 surfaces and their complex counterparts. Next, using a combination of both viewpoints we will derive the most important invariants of K3 surfaces.
We will start with computing the Chern and Todd classes, which will enable us to compute the so-called Hodge diamond of a K3 surface. Using some topological methods we will identify all the integral cohomology groups. Along the way we will introduce the Picard and Néron--Severi group and see that they are isomorphic. We will conclude the chapter with examining the intersection pairing and finally stating the global Torelli theorem at the end characterizing K3 surfaces through their Hodge lattices. 
% staring with computing their integral cohomology groups using the exponential sequence. We will shed some light on the Hodge decomposition and compute the so-called Hodge diamond of a K3 surface. We will conclude the chapter with examining the intersection pairing and introducing the Picard and Neron-Severi groups. 

% \vspace{0.15cm}
% \noindent
% Throughout this chapter we will be accessing the language of lattice theory. For its main themes we refer the reader to Appendix \ref{appendix B}.

\subsection{Algebraic and complex}

K3 surfaces come in two flavours, algebraic and complex, essentially equivalent, but still formally different to allow us to exploit both the methods of algebraic geometry as well as topology in combination with complex geometry. We start with the algebraic ones.
% but formally allowing us to effortlessly pass between the realms of algebraic and complex geometry. 

\begin{definition}
    A smooth projective surface $X$ over a field $k$ is a \emph{K3 surface}, if 
    \[
        \omega_X \iso \struct{X} \quad \text{and} \quad \cohom{1}{X}{\struct{X}} = 0.
    \]
\end{definition}

\begin{example}
    Let $X$ be a smooth quartic in $\PP^3$, meaning a smooth projective variety given by some homogeneous polynomial equation of degree four.
    We will show that any such variety is a K3 surface. Let $f \in \Gamma(\PP^3, \struct{\PP^3}(4))$ be the defining homogeneous polynomial for $X$ of degree four.
    % We will now show that any such smooth quartic $X$ in $\PP^3$, defined by some homogeneous degree $4$ polynomial $f \in \Gamma(X, \struct{\PP^3}(4))$, is a K3 surface. 
    Since $X$ is a hypersurface in $\PP^3$ it is of dimension two \ie a surface. Let $i \colon X \hookrightarrow \PP^3$ denote the closed embedding of $X$ into $\PP^3$ and consider the following short exact sequence of coherent sheaves on $\PP^3$
    \[
        0 \to \struct{\PP^3}(-4) \to \struct{\PP^3} \to i_*\struct{X} \to 0.
    \]
    The first non-trivial morphism is given by multiplication by $f$.
    %  and is clearly injective. 
    The second non-trivial morphism arises as the unit of the adjunction $i\inv \dashv i_*$, since by definition we set $\struct{X}$ to be $i\inv\struct{\PP^3}$.
    % It is then also clear that this map is surjective by performing the verification on stalks. The exactness in the middle is again verified on stalks and follows from the facts that $f_x \in \struct{\PP^3,x}$ is invertible and $(i_*\struct{X})_x = 0$ if $x \notin X$ and that 
    % $f_x \in \struct{\PP^3,x}$ is either invertible if $x \notin X$ or $0$ if $x \in X$ and the fact that the unit $\struct{\PP^3} \to i_*\struct{X}$ is an isomorphism if $x \in X$ and $0$ otherwise.
    First consider a portion of the long exact sequence in cohomology \eqref{eq: LES in cohomology of SES of coherent sheaves}
    \[
        \cdots \to \cohom{1}{\PP^3}{\struct{\PP^3}} \to \cohom{1}{\PP^3}{i_*\struct{X}} \to \cohom{2}{\PP^3}{\struct{\PP^3}(-4)} \to \cdots.
    \]
    Since the groups on the edge are known to vanish by \cite[\S III, Theorem 5.1]{Hartshorne1977}, the middle one does as well. Noting that the push-forward $i_*$ is exact, since $i$ is an embedding, we see that Proposition \ref{composition of derived functors} yields
    \begin{align*}
        \cohom{1}{\PP^3}{i_*\struct{X}} &=
        H^1(\rderived{}{\Gamma_{\PP^3}}(i_*\struct{X})) \\
        &\iso H^1((\rderived{}{\Gamma_{\PP^3} \circ i_*})(\struct{X})) \\
        &\iso H^1(\rderived{}{\Gamma_X}(\struct{X})) \\
        & = \cohom{1}{X}{\struct{X}}.
    \end{align*}
    Therefore $\cohom{1}{X}{\struct{X}} = 0$. Triviality of the canonical bundle $\omega_X$ is seen from the \emph{conormal sequence}
    \[
        0 \to \struct{X}(-4) \to i^*\Omega_{\PP^3} \to \Omega_X \to 0.
    \]
    After taking determinants we obtain
    \[
        i^*\omega_{\PP^3} \iso \struct{X}(-4) \otimes \omega_X.
    \]
    Upon tensoring with the inverse of the pull-back of $\omega_{\PP^3} \iso \struct{\PP^3}(-4)$, we conclude with $\omega_X \iso \struct{X}$.
\end{example}

\begin{example}        
    A very concrete example of a smooth quartic in $\PP^3$ is the \emph{Fermat quartic}\footnote{It is easily seen to be smooth by the Jacobian criterion for smoothness.}, which is a surface cut out by the homogeneous polynomial equation 
    \[
        x_0^4 + x_1^4 + x_2^4 + x_3^4 = 0.
    \]
\end{example}

Complex analytic K3 surfaces come with essentially the same definition as the algebraic ones. Of course we are now interpreting the structure sheaf $\struct{X}$ as the sheaf of \emph{holomorphic} functions on $X$ and $\omega_X$ as the holomorphic canonical bundle.  

\begin{definition}
    \label{Definition of complex K3}
    A connected complex projective manifold $X$ of dimension two is a \emph{K3 surface}, if  
     \[
        \omega_X \iso \struct{X} \quad \text{and} \quad \cohom{1}{X}{\struct{X}} = 0.
    \]
\end{definition}

\begin{remark}
    To give a little intuition about the condition $\cohom{1}{X}{\struct{X}} = 0$ in the definition, we will shortly see that also $\cohom{1}{X}{\Z} = 0$. The latter is a necessary condition for $X$ to be simply connected, however not sufficient. But in the case of complex analytic K3 surfaces all of them turn out to be simply connected. This is a deep fact we will not go into, but mention that it is proven using another profound result, namely that all analytic K3 surfaces are diffeomorphic and then showing that the Fermat quartic is simply connected. More on this can be found in \cite[\S 7.1]{Huybrechts2016}.
\end{remark}

As already mentioned in the introduction to Section \ref{Subsection: FM transform on cohomology} one can pass between the algebraic and the analytic categories using Serre's GAGA principle \cite{Serre1956}. We also mentioned the existence of an equivalence between categories of coherent sheaves on the algebraic side and analytic coherent sheaves on the complex analytic side preserving cohomology. This passage is also well behaved in the sense that the associated analytic coherent sheaf to the structure sheaf $\struct{X}$ is the sheaf of holomorphic functions $\struct{X\an}$ on $X\an$ and the sheaf associated to the sheaf of Kähler differentials $\Omega_X$ on $X$ is the sheaf of holomorphic 1-forms $\Omega^1_{X\an}$ on $X\an$. From this we see that the complex analytic space associated to an algebraic K3 surface is a complex K3 surface in the sense of Definition \ref{Definition of complex K3}.

\begin{remark}
    The \emph{projective} assumption is usually not present in the definition. In fact it is replaced by a more general assumption of compactness. We made this choice, because we will only be dealing only with algebraic K3 surfaces and thus every complex analytic K3 surface will have its algebraic counterpart. It is also worth noting that there do exist non-projective complex K3 surfaces and that they show up in important places. One of which is in a proof of the global Torelli theorem \ref{Classical Torelli theorem}, which will make an appearance at the end of this chapter, but whose proof we omit, instead referring to \cite[\S 7]{Huybrechts2016}. 
\end{remark}









\noindent
\textsl{Notation.}
K3 surfaces are complex manifolds therefore orientable, implying $\cohom{4}{X}{\Z} \iso \Z$. Until the rest of the chapter we fix a generator $\mu_X \in \cohom{4}{X}{\Z}$ for which $\pairing{\mu_X}{[X]} = 1$, where $[X] \in H_4(X, \Z)$ is a chosen fundamental class of $X$. Throughout the remainder of this chapter we will be accessing the language of lattice theory. For its main themes we refer the reader to Appendix \ref{appendix B}.

\subsection{Main invariants}

The first vital tool for our examination is Serre duality \eqref{eq: Classic Serre duality}. Given a vector bundle $\E$ on $X$ it gives us isomorphisms $\Ext_{\struct{X}}^{2-i}(\E, \struct{X}) \iso \cohom{i}{X}{\E}^*$ for all $i \in \Z$. Here we have already used a defining property of a K3 surface, namely that $\omega_X \iso \struct{X}$. The left side of the duality can be slightly modified to obtain 
\begin{equation}
    \label{eq: Serre duality on K3}
    \cohom{2-i}{X}{\E^\vee} \iso \cohom{i}{X}{\E}^*.
\end{equation}
% This already allows us to compute nearly all the Hodge numbers. We have 
Next we recall the general fact that the pairing
\[
    \Omega_X \otimes_\struct{X} \Omega_X \to \bigwedge^2\Omega_X
\]
is \emph{perfect}. By definition this means that the morphism $\Omega_X \to \localhom[X](\Omega_X, \bigwedge^2 \Omega_X)$, associated to the pairing above, is an isomorphism. Since the canonical bundle $\omega_X = \bigwedge^2 \Omega_X$ of a K3 surface $X$ is trivial, we see that 
\[
    \Omega_X \iso \Omega_X^\vee = \tangent{X}.
\]
This useful observation will be key in the computation of Chern and Todd classes of K3 surfaces. 

\begin{proposition}
    \label{characteristic classes of K3}
    Let $X$ be a complex K3 surface, then
    % and let $\mu_X \in \cohom{4}{X}{\Z}$ denote the generator for which $\pairing{\mu_X}{[X]} = 1$. Then
    \[
        \cclass[0](X) = 2 \qquad \cclass[1](X) = 0 \qquad \cclass[2](X) = 24\mu_X  \qquad \tdclass{X} = 1 + 2\mu_X.
    \]
\end{proposition}

\begin{proof}
    Tangent sheaf $\tangent{X}$ is a vector bundle of rank $2$, so $\cclass[0](X) = 2$. Since the sheaf of differentials $\Omega_X$ is self-dual, the tangent sheaf $\tangent{X}$ is as well. First, we see that $\ch[1](\tangent{X}) = \cclass[1](X)$. Then property \ref{chern character and dual} of Definition \ref{Definition of Chern character} applied to $\tangent{X} \iso \tangent{X}^\vee$ shows that $\cclass[1](X) = -\cclass[1](X)$. In Proposition \ref{cohomology of K3} we will see that $\cohom{2}{X}{\Z}$ is torsion-free, thus we reach $\cclass[1](X) = 0$. The second Chern class is computed using the Hirzebruch--Riemann--Roch formula applied to the trivial line bundle $\struct{X}$. Utilizing \eqref{eq: Serre duality on K3} we compute its Euler characteristic to be 
    \[
        \chi(X, \struct{X}) = h^0(X, \struct{X}) - h^1(X, \struct{X}) + h^2(X, \struct{X}) = 1 - 0 + 1 = 2.
    \]
    Noting that $\ch[]{\struct{X}} = 1$ and simplifying \eqref{eq: todd class} to $\td_X = 1 + \frac{1}{12}\cclass[2]{X}$, Theorem \ref{Hirzebruch-Riemann-Roch} implies
    \[
        2 = \chi(X, \struct{X}) = \int_X \ch[]{\struct{X}}\tdclass{X} = \pairing{\tfrac{1}{12}\cclass[2]{X}}{[X]}.
    \]
    Thus $\cclass[2](X) = 24 \mu_X$ and $\tdclass{X} = 1 + 2\mu_X$.
\end{proof}

For later we record the following K3 version of the Hirzebruch--Riemann--Roch formula. Let $\E$ be a coherent sheaf on $X$ then 
\begin{equation}
    \label{eq: HRR for K3}
    \chi(X, \E) = \pairing{\ch[2](\E)}{[X]} + 2\rk{\E} = \pairing{\tfrac{1}{2}(\cclass[1]{\E}^2 - 2\cclass[2]{\E})}{[X]} + 2 \rk{\E}.
\end{equation}
This already allows us to compute the Hodge numbers of a K3 surface $X$.

\begin{proposition}
    Let $X$ be a K3 surface. Arranged in a so-called \emph{Hodge diamond} the Hodge numbers of $X$ are the following:
    \begin{center}
        \begin{tabular}{c c c}
            \begin{tabular}{ccccc}
                &  & $h^{2,2}$ &  &  \\
                & $h^{2,1}$ &  & $h^{1,2}$ &  \\
                $h^{2,0}$ &  & $h^{1,1}$ &  & $h^{0,2}$ \\
                & $h^{1,0}$ &  & $h^{0,1}$ &  \\
                &  & $h^{0,0}$ &  &  \\
            \end{tabular} & \qquad &
            \begin{tabular}{ccccc}
                &  & $\ 1 \ $ &  &  \\
                & $\ 0 \ $ &  & $\ 0 \ $ &  \\
                $\ 1 \ $ &  & $20$ &  & $\ 1 \ $ \\
                & $0$ &  & $0$ &  \\
                &  & $1$ &  &  \\
            \end{tabular}
        \end{tabular}
    \end{center}
\end{proposition}

\begin{proof}
    Due to the symmetry $h^{p,q} = h^{q,p}$ coming from $\olsi{H^{p,q}(X)} = H^{q,p}(X)$, we will only compute $h^{p,q}$ for $p \geq q$. Starting from the bottom $h^{0,0}(X) = h^0(X, \struct{X}) = 1$. By definition we have $h^{1,0}(X) = h^1(X, \struct{X}) = 0$. By Serre duality \eqref{eq: Serre duality on K3} we get $h^{2,0}(X) = h^2(X, \struct{X}) = h^{0}(X, \struct{X}) = 1$ and $h^{2,2}(X) = h^{2}(X, \Omega_X^2) = h^2(X, \struct{X}) = h^0(X, \struct{X}) = 1$. Next $h^{2,1}(X) = h^{1,2}(X) = h^1(X, \Omega_X^2) = h^1(X, \struct{X}) = 0$. Finally, for the center of the diamond, we use the Hirzebruch--Riemann--Roch formula \eqref{eq: HRR for K3} applied to the tangent sheaf $\tangent{X} \iso \Omega_X^\vee \iso \Omega_X$, to obtain
    \begin{align*}
        h^{0,1}(X) - h^{1,1}(X) + h^{2,1}(X) 
        % &= \chi(X, \Omega_X) = \\ 
        =\pairing{\tfrac{1}{2}(\cclass[1]{\Omega_X}^2 - 2\cclass[2]{\Omega_X})}{[X]} + 2 \rk{\Omega_X} = -20.
    \end{align*}
    Thus, since $h^{0,1}(X) = h^{2,1}(X) = 0$, we get $h^{1,1}(X) = 20$.
\end{proof}

Notice that Proposition \ref{Hodge lattice, fm transform interaction}, says in particular that a Fourier--Mukai transform, which arises from an equivalence of derived categories, preserves columns of the Hodge diamond. As a consequence of the Hodge numbers just computed above, we prove that the bounded derived category $\derb{-}$ ``detects K3 surfaces''.
% As a consequence of the Hodge numbers just computed and Proposition \ref{Hodge lattice, fm transform interaction}, saying in particular that a Fourier--Mukai transform, which is an isomorphism, preserves columns of the Hodge diamond, we prove that $\derb{-}$ ``detects K3 surfaces''.

\begin{theorem}
    \label{Db(-) detects K3}
    Suppose $X$ and $Y$ are smooth projective varieties over $\C$. Assume $X$ is a K3 surface and
    \[
        \derb{X} \natiso \derb{Y}.
    \]
    Then $Y$ is a K3 surface as well. 
\end{theorem}

\begin{proof}
    By Theorem \ref{Db detects dimension and triviality of canonical bundle}, we already know $Y$ is a surface with a trivial canonical bundle. It is left to show that $\cohom{1}{Y}{\struct{Y}} = 0$. By Proposition \ref{Hodge lattice, fm transform interaction}, we see that 
    \[
        h^{0,1}(X) + h^{1,2}(X) = h^{0,1}(Y) + h^{1,2}(Y). 
    \]
    However $h^{0,1}(X) = 0$ and $h^{1,2}(X) = h^{2,1}(X) = 0$.
    %  \dim \cohom{1}{X}{\omega_X} = \dim \cohom{1}{X}{\struct{X}} = 0$. 
    Thus $h^{0,1}(Y) = 0$ proves that $Y$ is a K3 surface.
\end{proof}

\subsubsection*{Cohomology and intersection pairing}

Now we turn our attention to cohomology of K3 surfaces. The passage to complex K3 surfaces will unlock techniques of singular and Čech cohomology, which will turn out to be really useful here. Fluidly transitioning to the complex analytic side of things, let $X$ be a complex K3 surface, let $\struct{X}$ denote the sheaf of holomorphic functions on $X$ and define $\OO_X^*$ to be the sheaf of nowhere vanishing holomorphic functions on $X$. Consider the \emph{exponential sequence}
\[
    0 \to \Z \to \struct{X} \to \OO_X^* \to 0.
\]
The first term $\Z$ is seen as the sheaf of locally constant functions taking values in the ring of integers. The first non-trivial morphism is just the inclusion and the second one is given locally by $f \mapsto e^{2\pi i f}$. The associated long exact sequence in
%  Čech
cohomology then reads
\begin{equation}
    \label{eq: LES of exponential SES}
    \cdots \to \cohom{i}{X}{\Z} \to \cohom{i}{X}{\struct{X}} \to \cohom{i}{X}{\OO^*_X} \to \cohom{i+1}{X}{\Z} \to \cdots.
\end{equation}
% We will return to it momentarily, but presently focus on the connecting morphism
% \[
%     \cdots \to \cohom{1}{X}{\OO_X^*} \xrightarrow{\delta} \cohom{2}{X}{\Z} \to \cdots.
% \]

\begin{definition}
    The \emph{Picard group} of a complex K3 surface $X$ is defined to be the group of isomorphism classes of line bundles on $X$ with multiplication given by the tensor product of sheaves. We denote it with $\pic{X}$.
\end{definition}

\begin{remark}
    % Using \emph{Čech's} viewpoint on cohomology 
    Based on the cocycle representation of line bundles, one is able to construct a natural isomorphism from $\pic{X}$ to the first \emph{Čech cohomology} group $H^1(X,\OO_X^*)$ \cite[\S 4, Theorem 4.49]{Voisin2002}.
    %  with coefficients in the sheaf $\OO_X^*$ of nowhere vanishing holomorphic functions. 
    The Čech cohomology groups are also known to be isomorphic to the ususal sheaf cohomology groups we have been using all along \cite[\S 4, Theorem 4.44]{Voisin2002}.
\end{remark}

Based on the previous remark we are able to bring new meaning to the second connecting homomorphism $\cohom{1}{X}{\OO_X^*} \to \cohom{2}{X}{\Z}$ of \eqref{eq: LES of exponential SES}. 
% Under the composition isomorphism $\pic{X} \iso \cohom{1}{X}{\OO_X^*}$ it turns out to be precisely the first Chern class $\cclass[1]$
The value of a line bundle $\mathcal L$ on $X$ under the composition
\[
    \pic{X} \xrightarrow{\ \sim \ } \cohom{1}{X}{\OO_X^*} \longrightarrow \cohom{2}{X}{\Z}
\]
% turns out to result 
can actually be taken to be the definition of its \emph{first Chern class} $\cclass[1]{\mathcal L}$.
% is actually precisely the definition of its first Chern class $\cclass[1]{\mathcal L}$.
\begin{definition}
    The image of the connecting homomorphism $\cohom{1}{X}{\OO_X^*} \to \cohom{2}{X}{\Z}$ is defined to be the \emph{Néron--Severi group} of $X$, denoted by $\NS{X}$. 
\end{definition}

\begin{remark}
    By an already listed property of characteristic classes, when discussing Hodge decompositions in Section \ref{Subsection: FM transform on cohomology}, we note that the Néron--Severi group $\NS{X}$ can be seen as a subgroup of $H^{1,1}(X) \cap \cohom{2}{X}{\Z}$. The converse inclusion is also true and follows from the Lefschetz Theorem on $(1,1)$-classes \cite[\S 1.2]{GriffithsHarris1994}. Thus we have an equality
    \begin{equation}
        \label{eq: NS = 1,1 cap 2nd coh}
        \NS{X} = H^{1,1}(X) \cap \cohom{2}{X}{\Z}.
    \end{equation}
\end{remark}

\begin{proposition}
    \label{picard group torsion free}
    Let $X$ be a complex K3 surafce. The Picard group $\pic{X}$ and the Néron--Severi group $\NS{X}$ are isomorphic. They are torsion-free and carry a symmetric non-degenerate integral bilinear form called the intersection form. 
    % called the \emph{intersection pairing}. 
\end{proposition}

\begin{proof}
    Observe that $\cohom{1}{X}{\struct{X}} = 0$ in \eqref{eq: LES of exponential SES} shows that the connecting homomorphism is injective, therefore $\pic{X} \iso \NS{X}$. For the definition of the bilinear form on $\pic{X}$ and the rest see \cite[\S 1, Proposition 2.4]{Huybrechts2016}.
    % For a proof see \cite[\S 1, Proposition 2.4]{Huybrechts2016}.
    % See \cite[\S 1, Remark]{Huybrechts2016} for why $\pic{X}$ is torsion free and \cite[\S 1, Proposition 2.4]{Huybrechts2016} for the rest. 
\end{proof}

\begin{proposition}
    \label{cohomology of K3}
    Let $X$ be a complex K3 surface. The integral cohomology of $X$ is
    \begin{equation}
        \label{eq: cohomology of K3}
        \cohom{\bullet}{X}{\Z}: \qquad \Z \ \quad 0\ \quad \Z^{22} \quad 0\ \quad \Z.
    \end{equation}
\end{proposition}

\begin{proof}
    We already know $\cohom{0}{X}{\Z} \iso \Z$ and $\cohom{4}{X}{\Z} \iso \Z$ for $X$ is connected and orientable. Looking at the Hodge numbers
    we can already conclude that the cohomology groups are as in \eqref{eq: cohomology of K3}, but only up to torsion. 
    % we know what the integral cohomology groups are up to torsion. 
    We now show why they are in fact all torsion-free. For $\cohom{1}{X}{\Z}$ consider the following part of \eqref{eq: LES of exponential SES} 
    \[
        \cohom{0}{X}{\struct{X}} \to \cohom{0}{X}{\OO_X^*} \to \cohom{1}{X}{\Z} \to \cohom{1}{X}{\struct{X}}.
    \]
    The two groups on the left are $\C$ and $\C^*$, respectively, since $X$ is compact. The morphism relating them is the surjective exponential function, thus the connecting homomorphism going into $\cohom{1}{X}{\Z}$ is trivial, This makes $\cohom{1}{X}{\Z}$ into a subgroup of the trivial group $\cohom{1}{X}{\struct{X}} = 0$.
    % , therefore $\cohom{1}{X}{\Z}$ is trivial.
    Identifying $\pic{X} \iso \cohom{1}{X}{\OO_X^*}$ we have an exact sequence
    \[
        0 \to \pic{X} \to \cohom{2}{X}{\Z} \to \cohom{2}{X}{\struct{X}}.
    \]
    By Serre duality \eqref{eq: Serre duality on K3} we see that $\cohom{2}{X}{\struct{X}} \iso \C$. 
    % By Proposition \ref{picard group torsion free} $\pic{X}$ is torsion free. 
    Since $\pic{X}$ is torsion free by Proposition \ref{picard group torsion free}, two torsion free groups surround $\cohom{2}{X}{\Z}$ in an exact sequence, preventing it from having torsion either.
    By the universal coefficients theorem the torsion subgroup of $\cohom{3}{X}{\Z}$ is isomorphic to the torsion subgroup of the homology $H_2(X,\Z)$. As the latter is torsion free, being isomorphic to $\cohom{2}{X}{\Z}$ by Poincaré duality, the former must be torsion free as well. 
\end{proof}

The middle cohomology group $\cohom{2}{X}{\Z}$ now being free, enables us to introduce a lattice structure on it. The pairing it comes equipped with is called the \emph{intersection pairing} and is naturality induced by the cohomological $\smallsmile$-product. 
% It comes naturally equipped with the so-called \emph{intersection pairing} induced by the $\smallsmile$-product. 
\begin{definition}
    The \emph{intersection pairing} on $\cohom{2}{X}{\Z}$ is defined to be 
    \[
        \intersect{\alpha}{\beta} = \pairing{\alpha \smallsmile \beta}{[X]}.
    \] 
\end{definition}

\begin{proposition}
    \label{intersection pairing on K3 is even}
    Let $X$ be a K3 surface. Then the intersection pairing $\intersect{\cdot}{\cdot}$ makes $\cohom{2}{X}{\Z}$ into an even unimodular lattice of rank $22$ and signature $(3,19)$, making it isomorphic to the following orthogonal direct sum
    \begin{equation}
        \label{eq: intersection pairing on K3}
        \cohom{2}{X}{\Z} \iso E_8(-1)^{\oplus 2} \oplus \hyperbolic^{\oplus 3}.
    \end{equation}
    % given by the $\smallsmile$-product is an even lattice. 
\end{proposition}

\begin{proof}[Sketch of proof]
    % To see that the intersection pairing is unimodular we
    Since $\cohom{2}{X}{\Z}$ is free, the Kronecker pairing from the universal coefficients theorem together with Poincaré duality show that the intersection pairing is unimodular. To see that the intersection pairing is even one considers the mod $2$ reduction in the coefficients $\cohom{2}{X}{\Z} \to \cohom{2}{X}{\Z/2}$ and shows that the pairing on $\cohom{2}{X}{\Z/2}$ is always $0$. By Proposition \ref{characteristic classes of K3} we know that the first Chern class $\cclass[1](X)$ is trivial and we also know that it reduces to the second Stiefel--Whitney class $w_2(X)$ $\mathrm{(mod \ 2)}$, which is therefore also trivial. Using the Wu formula, one is able to show that
    \[
        x \smallsmile x = w_2(X) \smallsmile x
    \]
    holds for all classes $x \in \cohom{2}{X}{\Z/2}$ from which the desired result follows. Proposition \ref{cohomology of K3} shows that $\cohom{2}{X}{\Z}$ is free of rank $22$. The claim about the signature of this lattice is proven in \cite[\S 1, Proposition 3.5]{Huybrechts2016}. Lastly employing Milnor's classification result of even indefinite unimodular lattices (Theorem \ref{Classification of indefinite even unimodular lattices}), we obtain \eqref{eq: intersection pairing on K3}.
\end{proof}

\begin{remark}
    When $\NS{X}$ is equipped with the lattice structure of Proposition \ref{picard group torsion free}, the inclusion $\NS{X} \hookrightarrow \cohom{2}{X}{\Z}$ is an isometry. This makes it possible to interpret the intersection pairing
    of two (isomorphism classes of) line bundles $\mathcal L$ and $\mathcal L'$ on $X$ as 
    \[
        \intersect{\mathcal L}{\mathcal L'} = \intersect{\cclass[1](\mathcal L)}{\cclass[1](\mathcal{L'})}.
    \]
\end{remark}

\begin{corollary}
    Let $X$ be a K3 surface. The Chern character $\ch[](\E)$ of any coherent sheaf $\E$ on $X$ is integral, meaning that it lies in $\cohom{\bullet}{X}{\Z} \subseteq \cohom{\bullet}{X}{\Q}$.
\end{corollary}

\begin{proof}
    We see that the Chern character on a K3 surface reduces to the following
    \[
        \ch[]{\E} = \rk{\E} + \cclass[1]{\E} + \frac{1}{2}\Bigl(\cclass[1]{\E}^2 - 2\cclass[2]{\E}\Bigr).
    \]
    Since the intersection pairing is even, $\cclass[1]^2(\E)$ is twice a multiple of some cohomology class from $\cohom{4}{X}{\Z}$.
\end{proof}

\begin{remark}
The Chern character of a coherent sheaf $\E$ being integral implies also that its Mukai vector $v(\E)$ is integral. Indeed, using the power series expansion \eqref{eq: sqrt power series} (or just guessing), we see that 
\[
    \sqrt{\tdclass{X}} = 1 + \mu_X.
\]
% (the other possible choice being $1 - \mu_X)$
Thus, one easily sees that the Mukai vector on a K3 surface is integral because it has the following from
\begin{align}
    v(\E) &= \ch[0](\E) + \ch[1]{\E} + \bigl(\ch[2]{\E} + \ch[0]{\E}\mu_X\bigr)
    \notag \\
    &= \rk{\E} + \cclass[1]{\E} + \frac{1}{2}\Bigl(\cclass[1]{\E}^2 - 2\cclass[2]{\E}\Bigr) + \rk{\E}\mu_X. \notag
\end{align}
We also remark that on a K3 surface the inverse of the square root of the Todd class $(\sqrt{\tdclass{X}})\inv$ equals $1 - \mu_X$ and is therefore integral as well.
\end{remark}

With the Néron--Severi group seen as a sublattice of $\cohom{2}{X}{\Z}$ one can consider also its orthogonal coplement. The resulting group will play a very important role in the last chapter in characterizing derived equivalent K3 surfaces. 

\begin{definition}
    The \emph{transcendental lattice} of a complex K3 surface $X$ is defined to be the orthogonal complement
    \[
        \transc{X} = \NS{X}^\perp = \{ x \in \cohom{2}{X}{\Z} \mid \intersect{x}{y} = 0 \text{ for all $y \in \NS{X}$}\}.
    \]
\end{definition}

\begin{remark}
    By Lemma \ref{orthogonal complemet is a lattice} the transcendental lattice $\transc{X}$ is indeed a lattice with the bilinear form inherited from $\cohom{2}{X}{\Z}$. It is also a primitive sublattice of $\cohom{2}{X}{\Z}$ by Example \ref{orthogonal complement primitive}.
\end{remark}

\subsubsection*{Global Torelli theorem}

Lastly we mention the global Torelli theorem. It will be utilized in the last chapter to prove that a K3 surface $Y$ is isomorphic to a moduli space of certain sheaves on its Fourier--Mukai partner. We have already seen that $\cohom{2}{X}{\Z}$ comes equipped with a lattice structure given by the intersection pairing. Moreover due to $X$ being a complex projective manifold it also posseses a Hodge structure. We will call such lattices Hodge lattices and two Hodge lattices will be considered isomorphic, if there is a \emph{Hodge isometry} between them. 

\begin{definition}

    A \emph{Hodge isometry} between two Hodge lattices $\Lambda$ and $\Lambda'$ is a Hodge structure preserving isomorphism of lattices $f \colon \Lambda \xrightarrow{\sim} \Lambda'$.
    % , which also preserves the Hodge structures. 
\end{definition}

\begin{theorem}[Global Torelli theorem]
    \label{Classical Torelli theorem}
    Let $X$ and $Y$ be K3 surfaces over $\C$. Then $X$ and $Y$ are isomorphic if and only if there exists a Hodge isometry
    \[
        f\colon \cohom{2}{X}{\Z} \xrightarrow{\sim} \cohom{2}{Y}{\Z}.
    \]
\end{theorem}

\begin{proof}
    For a proof see \cite[\S 7, Theorem 5.3]{Huybrechts2016}.
\end{proof}

\begin{remark}
    The global Torelli theorem emphasises that the Hodge lattice $\cohom{2}{X}{\Z}$ of a K3 surface $X$ is a complete invariant of its isomorphism type. 
    % In particular this means that the set of all isomorphism classes of K3 surfaces is parametrized by the set of all Hodge structures on the abstract K3 lattice $E_8(-1)^{\oplus 2} \oplus \hyperbolic^{\oplus 3}$
\end{remark}

% By noting where the isomorphism sends the 

% ========= %


% \subsection{Two important theorems}

% When discussing K3 surfaces one can not pass two of the arguably most important theorems on K3 surfaces -- the Global Toreli theorem and the surjectivity of the period map. The first theorem will be utilized in the last chapter to prove that a K3 surface $Y$ is isomorphic to a moduli space of certain sheaves on its Fourier--Mukai partner. 
% % another, derived equivalent, K3 surface $X$.

% We have already seen that $\cohom{2}{X}{\Z}$ comes equipped with a lattice structure given by the intersection pairing. Moreover due to $X$ being a complex projective manifold it also posseses a Hodge structure. We will call such lattices Hodge lattices and two Hodge lattices will be considered isomorphic, if there is a \emph{Hodge isometry} between them. 

% \begin{definition}

%     A \emph{Hodge isometry} between two Hodge lattices $\Lambda$ and $\Lambda'$ is a Hodge structure preserving isomorphism of lattices $f \colon \Lambda \to \Lambda'$.
%     % , which also preserves the Hodge structures. 
% \end{definition}

% \begin{theorem}[Global Torelli theorem]
%     \label{Classical Torelli theorem}
%     Let $X$ and $Y$ be K3 surfaces over $\C$. Then $X$ and $Y$ are isomorphic if and only if there exists a Hodge isometry
%     \[
%         f\colon \cohom{2}{X}{\Z} \to \cohom{2}{Y}{\Z}.
%     \]
% \end{theorem}

% \begin{proof}
%     For a proof see \cite[\S 7, Theorem 5.3]{Huybrechts2016}.
% \end{proof}

% By Proposition \ref{intersection pairing on K3 is even} we know that the lattice $\cohom{2}{X}{\Z}$ of every complex K3 surface $X$ is abstractly isomorphic to $E_8(-1)^{\oplus 2} \oplus \hyperbolic^{\oplus 3}$, which we denote with $\LK$. 
\newpage

\section{Derived Torelli theorem}
\label{Chapter: Derived Torelli theorem}

\info{in this chapter sheaves will always be coherent.}

In this last chapter we will again be dealing with K3 surfaces over the field of complex numbers $\C$. In particular we will characterize when two K3 surfaces $X$ and $Y$ have equivalent bounded derived categories, through their internal structure, namely their so called \emph{transcendental lattices}. For a K3 surface $X$, its \emph{transcendental lattice} is defined to be the orthogonal complement $\NS{X}^\perp \subseteq \cohom{2}{X}{\Z}$ with respect to the intersection pairing $\pairing{-}{-}$ on $\cohom{2}{X}{\Z}$, introduced in the previous chapter. The transcendental lattice of $X$ will be denoted by $\transc{X}$.
Naturally, $\transc{X}$ also comes equipped with a weight $2$ Hodge structure induced from that of $\cohom{2}{X}{\Z}$. 

\info{todd class of K3}

\begin{theorem}
    \label{Derived Torelli}
    Let $X$ and $Y$ be K3 surfaces over the field of complex numbers $\C$. Then $\derb{X} \natiso \derb{Y}$ if and only if there exists a Hodge isometry $f\colon T_X \to T_Y$ between their transcendental lattices.
\end{theorem}

We will be using Fourier-Mukai transforms in an essential way. When applied to the case of K3 surfaces adjoints of Fourier-Mukai transforms take up an interesting form.

Thus we see that in the case of K3 surfaces every left and right adjoints of Fourier-Mukai transforms agree.
Here we have of course used one of the defining properties of K3 surfaces -- triviality of their canonical bundle.

% \subsection{Mukai lattice}

In order to consider all the cohomology groups of a K3 surface $X$ at once Mukai introduced the following pairing on $\cohom{\bullet}{X}{\Z}$. For 

For this lattice to nicely fit into our context of Hodge lattices, we also equip it with a compatible weight $2$ Hodge structure. This is achieved through the introduction of the \emph{Mukai lattice}. 

\begin{definition}
    The \emph{Mukai lattice} of a K3 surface $X$ over $\C$, denoted $\mukai{X}$, consists of the free abelian group $\cohom{\bullet}{X}{\Z}$ together with the bilinear pairing introduced above in \eqref{} equipped with a weight $2$ Hodge structure described by 
\end{definition}

\begin{center}
    \begin{tabular}{r c c}
        $\widetilde{H}^{2,0}(X)$ & $=$ & $H^{2,0}(X)$ \\
        $\widetilde{H}^{1,1}(X)$ & $=$ & $\cohom{0}{X}{\C} \oplus H^{1,1}(X) \oplus \cohom{4}{X}{\C}$ \\
        $\widetilde{H}^{0,2}(X)$ & $=$ & $H^{0,2}(X)$.
    \end{tabular}
\end{center}
\info{describe which parts are orthogonal and why}

% \subsection{Moduli space of sheaves on a K3 surface}
\subsection{Mukai lattice and one side of the proof}

\info{recall what the projections $p$ and $q$ are.}

\noindent
\textsc{Implication $\derb{X} \iso \derb{Y} \implies \mukai{X} \iso \mukai{Y}$.} After first establishing a few preliminary results, we will prove Proposition \ref{D equivalence implies Hodge isometry} by borrowing the kernel of a Fourier-Mukai transform appearing as the equivalence $\derb{X} \iso \derb{Y}$, which Orlov's result \ref{Orlov's theorem} enables us to do, and then show that its corresponding cohomological version gives a Hodge isometry $\mukai{X} \iso \mukai{Y}$.

\begin{proposition}
    \label{D equivalence implies Hodge isometry}
    Suppose there exists an equivalence of triangulated categories $\derb{X} \natiso \derb{Y}$, then there exists a Hodge isometry $\mukai{X} \iso \mukai{Y}$.
\end{proposition}

\begin{lemma}
    \label{Mukai vector is integral}
    For any object $\E^\bullet$ of $\derb{X \times Y}$, its Chern character $\ch{\E^\bullet}$ is integral, \ie belongs to $H^\bullet(X \times Y, \Z) \subseteq H^\bullet(X \times Y, \Q)$. Consequently the Mukai vector $v(\E^\bullet)$ is integral as well.
\end{lemma}

\begin{proof}
    We will use the Künneth formula in an essential way so we recall it here. Since $X$ and $Y$, as K3 surfaces, have torsion-free cohomology the $n$-th cohomology of the product $X \times Y$ decomposes as
    \[
        H^n(X \times Y, \Z) \iso \bigoplus_{p+q = n} H^p(X, \Z) \otimes_\Z H^q(Y, \Z),
    \] 
    for all $n \in \N$.
    
    First, rewrite the Chern character with respect to the ordinary grading on cohomology $H^\bullet(X \times Y, \Q) = \bigoplus_{i=0}^4 H^{2i}(X \times Y, \Q)$ as the vector
    \[
        \ch{\E^\bullet} = \left(\rk{\E^\bullet}, \cclass[1](\E^\bullet), \tfrac{1}{2}(\cclass[1]{\E^\bullet}^2 - 2\cclass[2]{\E^\bullet}), \ch[2]{\E^\bullet}, \ch[3]{\E^\bullet} \right).
    \]
    Components $\rk{\E^\bullet}$ and $\cclass[1]{\E^\bullet}$ are integral by definition. The first Chern class $\cclass[1]{\E^\bullet} \in H^2(X\times Y, \Z)$ may be rewritten, in accordance with the Künneth decomposition
    \[
        H^2(X\times Y, \Z) \iso H^2(X, \Z) \oplus H^2(Y, \Z),
    \]
    in the form $\cclass[1]{\E^\bullet} = p^*\alpha + q^*\beta$, for some $\alpha \in H^2(X, \Z)$ and $\beta \in H^2(Y, \Z)$. Therefore, as K3 surfaces $X$ and $Y$ have \emph{even} intersection pairings by Proposition \ref{intersection pairing on K3 is even}, we see that $\cclass[1]{\E^\bullet}^2 = p^*\alpha^2 + 2 p^*\alpha.q^*\beta + q^*\beta^2$ is an even multiple of some class from $\cohom{4}{X \times Y}{\Z}$. Along with $\cclass[2]{\E^\bullet}$ being integral this shows the $H^4(X \times Y, \Q)$-component of $\ch[](\E^\bullet)$ is integral.

\end{proof}

\begin{lemma}
    \label{Intersection pairing through push-forward of structure map}
    Let $X$ be a K3 surface over $\C$ and $\nu \colon X \to \spec{\C}$ its structure map. Then for all $\alpha$, $\alpha' \in \mukai{X}$
    \[
        \pairing{\alpha}{\alpha'} = \nu_*(\alpha \smallsmile \alpha')
    \] 
    and 
    \[
        \nu_*(\alpha^\vee) = \nu_*(\alpha). 
    \]
\end{lemma}

\begin{proof}

\end{proof}

\begin{proof}[Proof of Proposition \ref{D equivalence implies Hodge isometry}]
    By Orlov's Theorem \ref{Orlov's theorem}, the functor witnessing the equivalence $\derb{X} \natiso \derb{Y}$ is naturally isomorphic to a Fourier-Mukai transform $\fm{\E^\bullet}$ for some kernel $\E^\bullet$ of the category $\derb{X \times Y}$. We will show that the cohomological Fourier-Mukai transform $\fmcoh{\E^\bullet} \colon H^\bullet(X, \Q) \to H^\bullet(Y, \Q)$ of section \ref{Subsection: FM transform on cohomology} induces a Hodge isometry between the Mukai lattices
    \[
        \fmcoh{\E^\bullet} \colon \mukai{X} \to \mukai{Y}.
    \]

    First recall that by Definition \ref{}, the cohomologiacl Fourier-Mukai transform $\fmcoh{\E^\bullet}$ is defined to be
    \[
        \fmcoh{\E^\bullet} \colon H^\bullet(X, \Q) \to H^\bullet(Y, \Q) \qquad \alpha \mapsto p_*(v(\E^\bullet) \smallsmile q^*(\alpha)).
    \]
    By lemma \ref{Mukai vector is integral} the Mukai vector $v(\E^\bullet)$ lies in $H^\bullet(X \times Y, \Z) \subseteq H^\bullet(X \times Y, \Q)$, thus $\fmcoh{\E^\bullet}$ may be restricted to the integral parts, thus forming a homomorphism of abelian groups
    \[
        \fmcoh{\E^\bullet} \colon H^\bullet(X, \Z) \to H^\bullet(Y, \Z).
    \]

    \noindent
    \textsl{Bijection.}
    Next we verify that $\fmcoh{\E^\bullet}$ is bijective. We do so by constructing its right inverse. For $\fm{\E^\bullet}$ is an equivalence, the Fourier-Mukai transform associated to the the kernel $\E^\bullet_L$, of its left adjoint, is actually its quasi-inverse. Thus from $\fm{\E^\bullet_L} \circ \fm{\E^\bullet} \natiso \id{\derb{X}}$ along with uniqueness of kernels and Example \ref{Identifying fm transforms}, we see that $\struct{\Delta_X} \iso \E^\bullet_L * \E^\bullet$. As cohomological Fourier-Mukai transforms compose just like the categorical ones do, according to Proposition \ref{Composition of cohomological fm is fm}, we see that $\fmcoh{\E^\bullet_L} \circ \fmcoh{\E^\bullet} = \fmcoh{\struct{\Delta_X}}$, thus it suffices to show that $\fmcoh{\struct{\Delta_X}} = \id{H^\bullet(X, \Z)}$.
    
    We start by computing the Mukai vector of $\struct{\Delta_X}$. Consulting the help of Grothendieck-Riemann-Roch \ref{Grothendieck-Riemann-Roch}, we see that
    \begin{align*}
        \ch(\struct{\Delta_X}) \tdclass{X \times X} &= \ch(\Delta_*\struct{X})\tdclass{X \times X} = \\ &= \ch(\Delta_!\struct{X})\tdclass{X \times X} = \Delta_*(\ch(\struct{X}) \tdclass{X}) = \Delta_*\tdclass{X},
    \end{align*}
    holds true in $\cohom{*}{X \times X}{\Q}$. In the second equality, we have used the relation $\Delta_*\struct{X} = \Delta_!\struct{X}$ in $K(X \times X)$, because for any closed embedding, such as $\Delta \colon X \to X \times X$, its push-forward $\Delta_*$ is exact, and separately in the last equality $\ch(\struct{X}) = \exp(\cclass[1](\struct{X})) = 1$, because $\struct{X}$ is a line bundle. Next, we observe, that from $\tdclass{X \times X} = p^*\tdclass{X} \smallsmile p^*\tdclass{X}$, equation $\Delta^*\sqrt{\tdclass{X\times X}} = \tdclass{X}$ follows. By the cohomological projection formula (\cf Proposition \ref{cohomological projection formula}) for the map $\Delta$, we further compute
    \[
        \Delta_*\tdclass{X} = \Delta_*\Delta^*\sqrt{\tdclass{X \times X}} = \Delta_*(1)\sqrt{\tdclass{X \times X}},
    \] 
    implying that $v(\struct{\Delta_X}) = \Delta_*(1)$. 
    Lastly, for any $\alpha \in \cohom{\bullet}{X}{\Z}$, using the projection formula again, we arrive at
    \[
        \fmcoh{\struct{\Delta_X}}(\alpha) = p_*(v(\struct{\Delta_X})\smallsmile p^*\alpha) = p_*(\Delta_*(1)\smallsmile p^*\alpha) = p_*(\Delta_*(1\smallsmile \Delta^*p^*(\alpha))) = \alpha.
    \]
    % By letting $\mathcal R^\bullet$ denote the convolution of $\E^\bullet$ and $\E^\bullet_L$, we see that by uniqueness of kernels of Fourier-Mukai transforms it follows that 
    The map $\fmcoh{\E^\bullet}$ is now a homomorphism of free abelian groups admitting a right inverse, therefore it is an isomorphism of abelian groups. We are left to show that $\fmcoh{\E^\bullet}$ is an isometry and that it preserves the Hodge structure. 
    
    \vspace{0.5cm}
    \noindent
    \textsl{Isometry.}
    To see that $\fmcoh{\E^\bullet}$ is an isometry, we do two preliminary computations. Let $\alpha \in \mukai{X}$ and $\beta \in \mukai{Y}$ and let $\nu_X \colon X \to \spec{\C}$, $\nu_Y \colon Y \to \spec{\C}$ and $\nu_{X \times Y} \colon X \times Y \to \spec{\C}$ denote the structure maps, then by Lemma \ref{Intersection pairing through push-forward of structure map}
    \begin{align*}
        \pairing{\fmcoh{\E^\bullet}(\alpha)}{\beta}_Y &= \nu_{Y, *}(\beta^\vee \smallsmile \fmcoh{\E^\bullet}(\alpha))
        \\
        &= \nu_{Y, *}\left(
            \beta^\vee \smallsmile p_*\left(v(\E^\bullet) \smallsmile q^*\alpha\right)\right) 
        \\
        &= \nu_{Y, *}\left(
            \beta^\vee \smallsmile p_*\left(\ch{\E^\bullet}
            {\sqrt{\tdclass{X \times Y}}}
            \smallsmile q^*\alpha\right)
        \right) 
        \\
        &= \nu_{Y, *}\left( 
            p_*\left(
                p^*\beta^\vee \smallsmile \ch{\E^\bullet}
                {\sqrt{\tdclass{X \times Y}}}
                \smallsmile q^*\alpha
            \right)
        \right) \\
        &= \nu_{X \times Y, *}\left(
            p^*\beta^\vee \smallsmile \ch{\E^\bullet}
                {\sqrt{\tdclass{X \times Y}}}
                \smallsmile q^*\alpha
        \right)
    \end{align*}
    and similarly one computes
    \info{$\ch(E^\vee) = \ch(E)^\vee$?}
    \[
        \pairing{\alpha}{\fmcoh{\E^\bullet_L}(\beta)}_X = \nu_{X \times Y, *}\left(
            q^*\alpha^\vee \smallsmile \ch{\E^\bullet}^\vee
                {\sqrt{\tdclass{X \times Y}}}
                \smallsmile p^*\beta
        \right)
    \]

    Then clearly for all $\alpha$, $\alpha' \in \mukai{X}$
    \[
        \pairing{\fmcoh{\E^\bullet}(\alpha)}{\fmcoh{\E^\bullet}(\alpha')}_Y = 
        \pairing{\alpha}{\fmcoh{\E^\bullet_L}(\fmcoh{\E^\bullet}(\alpha'))}_X = 
        \pairing{\alpha}{\alpha'}_X,
    \]       
    proving that $\fmcoh{\E^\bullet}$ is an isometry. 

    \vspace{0.5cm}

    \noindent
    \textsl{Hodge structures.}
    Lastly, we show that $\fmcoh{\E^\bullet}$ preserves the Hodge structure \ie
    \[
        \fmcoh{\E^\bullet}_\C(\widetilde{H}^{p,q}(X)) \subseteq \widetilde{H}^{p,q}(Y) 
    \]
    for all $p$, $q$, with $p + q = 2$. 
    By Proposition \ref{Hodge lattice, fm transform interaction}, the condition \eqref{eq: Hodge lattice, fm transform} specializes to 
    \begin{equation}
        \label{eq: H20 is mapped to H20}
        f_\C(\widetilde{H}^{2,0}(X)) \subseteq \widetilde{H}^{2,0}(Y)
    \end{equation}
    % since the term on the right is the only non-trivial one of the direct sum decomposition.
    and in fact this turns out to be sufficient for $f$ to preserve the Hodge structures. This is a very special coincidence, which happens as a consequence of the Hodge structure on the Mukai lattice of a K3 surface taking a particular from.
    % In fact, we will show this inclusion holds only for $(p,q) = (2,0)$, as this turns out to be sufficient. 
    Indeed, if $f_\C(\widetilde{H}^{2,0}(X)) \subseteq \widetilde{H}^{2,0}(Y)$, then $f_\C(\widetilde{H}^{0,2}(X)) \subseteq \widetilde{H}^{0,2}(Y)$ is obtained through conjugation
    \[
        f_\C\left(\widetilde{H}^{0,2}(X)\right) = f_\C\left(\ols{\widetilde{H}^{2,0}(X)}\right) = \ols{f_\C\left(\widetilde{H}^{2,0}(X)\right)} \subseteq \ols{\widetilde{H}^{2,0}(Y)} = \widetilde{H}^{0,2}(Y).
    \]
    And, to see $f_\C(\widetilde{H}^{1,1}(X)) \subseteq \widetilde{H}^{1,1}(Y)$, we pick a non-zero class $x \in \widetilde{H}^{1,1}(X)$ and map it over to $y = f_\C(x) \in \mukai{X} \otimes \C$. Then for any $y^{2,0} \in \widetilde{H}^{2,0}(Y)$, we have $\pairing{y}{y^{2,0}} = 0$. Indeed, first as $f_\C$ is an isomorphism of $\C$-vector spaces and $\widetilde{H}^{2,0}(X)$ and $\widetilde{H}^{2,0}(Y)$ are one dimensional, the inclusion \eqref{eq: H20 is mapped to H20} is actually an equality, thus we may represent $y^{2,0}$ as $f_\C(x^{2,0})$ for some $x^{2,0} \in \widetilde{H}^{2,0}(X)$. As $\widetilde{H}^{2,0}(X) \perp \widetilde{H}^{1,1}(X)$ and $f_\C$ is an isometry, we obtain $0 = \pairing{x}{x^{2,0}} = \pairing{y}{y^{2,0}}$ to conclude $y \in \widetilde{H}^{2,0}(Y)^\perp$. Symmetrically one obtains $y \in \widetilde{H}^{0,2}(Y)^\perp$. 

    Finally, once we verify that $\widetilde{H}^{2,0}(Y)^\perp \cap \widetilde{H}^{0,2}(Y)^\perp \subseteq \widetilde{H}^{1,1}(Y)$, we may conclude that $y \in \widetilde{H}^{1,1}(Y)$ to prove our last claim.
    This is indeed the case for once we write a class $z \in \widetilde{H}^{2,0}(Y)^\perp \cap \widetilde{H}^{0,2}(Y)^\perp$ according to the Hodge decomposition as $z = z^{2,0} + z^{1,1} + z^{0,2}$, we see that $z^{2,0} = z^{0,2} = 0$, by non-degeneracy of the pairing. For example for any $w = w^{2,0} + w^{1,1} + w^{0,2} \in \mukai{Y}$ we have
    \[
        \pairing{z^{2,0}}{w} = \pairing{z^{2,0}}{w^{2,0}} + \pairing{z^{2,0}}{w^{1,1}} + \pairing{z^{2,0}}{w^{0,2}} = 0,
    \]
    where $\pairing{z^{2,0}}{w^{2,0}} = 0$, because $z^{2,0} = z - z^{1,1} - z^{0,2}$ shows $z^{2,0} \in \widetilde{H}^{0,2}(Y)^\perp$ and $\pairing{z^{2,0}}{w^{1,1}} = 0$ together with $\pairing{z^{2,0}}{w^{0,2}} = 0$ follow from orthogonality relations \eqref{}. \qedhere
    
    % arising from the fact that $y^{2,0}$ is of the form $f_\C(x^{2,0})$ for some $x^{2,0} \in \widetilde{H}^{2,0}(X)$
    
    
    % As $f_\C$ is an isomorphism of $\C$-vector spaces, the class $y$ is non-zero. 
    
    % $\widetilde{H}^{2,0}(Y) \oplus \widetilde{H}^{1,1}(Y) \oplus \widetilde{H}^{0,2}(Y)$
\end{proof}

\subsection{Moduli spaces of sheaves on K3 surfaces and the other side of the proof}

The last subsection is devoted to proving the right to left implication of Theorem \ref{Derived Torelli}, captured in the preceding proposition.

\begin{proposition}
    \label{Hodge isometry implies D equivalence}
    Let $X$ and $Y$ be K3 surfaces over $\C$ and suppose there exists a Hodge isometry 
    \[
        f \colon \mukai{X} \to \mukai{Y}
    \]
    of their Mukai lattices. Then there exists an equivalence of their bounded derived categories of coherent sheaves $\derb{X} \iso \derb{Y}$.
    % If there is a Hodge isometry of Mukai lattices $\mukai{X} \iso \mukai{Y}$, there is an equivalence of triangulated categories $\derb{X} \iso \derb{Y}$.
\end{proposition}
The method of our proof will first lead us to introduce another K3 surface $M$, which will turn out to be isomorphic to $Y$ through an application of the classical Torelli theorem \ref{Classical Torelli theorem}, and whose derived category we will show is equivalent to $\derb{X}$. Not only will the K3 surface $M$ itself be important, but also how it is constructed or what it \emph{represents}. We will expand on this viewpoint to some extent, but will have to refer the reader to \cite[text]{keylist} for details and proofs, as this part is unfortunately outside the scope of our thesis. This approach will serve as an invaluable tool for obtaining a kernel for a Fourier-Mukai transform of the from $\derb{M} \to \derb{X}$, which we will later prove to be an equivalence. To prove the latter claim we will take a bit of an alternative route, than what is presented in \cite{Orlov-K3}, and use the abstract machinery of triangulated categories -- indecomposability and spanning classes, established towards the latter parts of Section \ref{Subsection: Triangulated categories}, together with results of Section \ref{Subsection: Derived category of coherent sheaves}. 


% using the abstract machinery of triangulated categories and spanning classes, established towards the latter parts of subsection \ref{}. Through an application of the classical Torelli theorem \ref{Classical Torelli}, the K3 surface $M$ will turn out to be isomorphic to $Y$.

% A few more words are required to introduce how the K3 surface $M$ comes about. Not only is the K3 surface $M$ important, but also how it is constructed or what it \emph{represents}. This viewpoint will serve as an invaluable tool for obtaining a kernel for a Fourier-Mukai transform of the from $\derb{M} \to \derb{X}$, which we will later prove to be an equivalence. 

Recall that any object $X$ of a category $\mathcal C$ gives rise to a functor 
\[
    h_X \colon \mathcal C^\op \to \Set,
\]
which is given by $h_X = \Hom_{\mathcal C}(-, X)$. A functor $F \colon \mathcal C^\op \to \Set$ is siad to be \emph{representable} if there is some object $X$ of $\mathcal C$, for which $F$ and $h_X$ are naturally isomorphic. Moreover the Yoneda lemma \cite[text]{keylist} then asserts for every object $X$ of $\mathcal C$ and any functor $F \colon \mathcal C ^\op \to \Set$ the existence of an isomorphism 
\begin{equation}
    \label{eq: nat iso of Yoneda}
    \Hom_{\operatorname{Fct(\mathcal C^\op, \Set)}}(h_X, F) \natiso F(X),
\end{equation}
which is natural both in $X$ and $F$.

From now on we focus on a fixed K3 surface $X$ over $\C$ and consider the category $\sch{\C}$ of schemes over $\C$. The main idea we are trying to is to consider sheaves on $X$. These sheaves will

appear in families ``indexed'' by some other scheme $S$ over $\C$. Such families may be encoded as single sheaves on the product $\F \in \coherent{X \times S}$. 


As this would be an absolute overkill for there are too many coherent sheaves, we restrict ourselves by imposing some additional constraints on our sheaves.

For any scheme $S$ over $\C$, a sheaf $\F \in \coherent{X \times S}$ is said to be \emph{flat over $S$}, if $\F$ is a flat $\pi_S\inv \struct{S}$-module, where $\pi_S \colon X \times S \to S$ is the canonical projection. Let $s \in S$ be a closed point, to which we know there is a corresponding morphism of schemes $\spec{\C} \to S$ also denoted with $s$. Then we define $i_s \colon X \to X \times S$ to be the composition
\[
    i_s \colon \quad X \longrightarrow X \times \spec{\C} \xrightarrow{\id{X} \times s} X \times S
\]
and introduce $\F|_s$ to mean the pull-back sheaf $i_s^*\F$ on $X$.


 

\begin{definition}
    \label{Definition of moduli functor}
    Let $X$ be a K3 surface over $\C$ with a prescribed Mukai vector $v \in \mukai{X}$. The \emph{moduli functor}
    \[
        \M_v \colon \sch{\C}^\op \longrightarrow \Set
    \]
    is defined on objects as 
    \[
        S \longmapsto \left\{\F \in \coherent{X \times S} \ \middle| 
        \begin{array}{l}
            \text{$\F$ is flat over $S$, and for each closed point $s \in S$,} \\
            \text{$\F|_s$ is a stable sheaf on $X$, with $v(\F|_s) = v$.}
        \end{array}
        \right\}_{/\sim},
    \]
    where $\F \sim \F'$ is defined to hold precisely when there exist a line bundle $\mathcal L$ on $S$, for which $\F \iso \F' \otimes_{\struct{X \times S}} \pi_S^*\mathcal L$, and on morphisms as
    \[
        (f\colon S' \to S) \longmapsto \left( (\id{X} \times f)^* \colon
            \begin{array}{r l}
                & \M_v(S) \to \M_v(S') \\
                & [\F] \mapsto [(\id{X} \times f)^*\F]
            \end{array}
            \right).
    \]
\end{definition}

\begin{remark}
    Our moduli functor is constrained by a prescribed Mukai vector as opposed to \cite[text]{keylist}, which prescribes the Chern character. As Mukai vectors and Chern characters determine each other on a K3 surface, the definitions are in fact equivalent. 
\end{remark}

\begin{theorem}
    \label{Representability of moduli functor}
    Let $v \in \mukai{X}$ be an isotropic vector with ... Then the moduli functor $\M_v \colon \sch{\C}^\op \to \Set$ of definition \ref{Definition of moduli functor} is representable and is represented by a K3 surface $M$ over $\C$.
\end{theorem}

\begin{example}
    The vector $(0,0,1)$ is an example of such a $v$ and so is any isometric image of it. 
\end{example}

\begin{remark}
    The moduli functor $\M_v$ is actually representable already even under no additional restrictions on the Mukai vector $v$. The conditions are there to ensure $M$ becomes a K3 surface. For example in general $M$ is smooth and has dimension $\pairing{v}{v} + 2$ (\cf \cite[\S 4.3, Propositon 4.20]{vanBree2020}).
\end{remark}

\begin{definition}
    The \emph{universal bundle (on $X$)} is the unique sheaf $\E \in \coherent{X \times M}$, to which the natural isomorphism \eqref{eq: nat iso of Yoneda} of the Yoneda lemma composed with the natural isomorphism of Theorem \ref{Representability of moduli functor}, witnessing representability of $\M_v$, assigns the identity natural transformation $\id{h_M}$ of the functor $h_M$
    \[
        \Hom_{\operatorname{Fct(\sch{\C}^\op, \Set)}}(h_M, h_M) \xrightarrow{\ \iso\ } h_M(M) \xrightarrow{\ = \ } \Hom_{\sch{\C}}(M, M) \xrightarrow{\ \iso\ } \M_v(M).
    \]
\end{definition}

As a consequence of a more general fact \cite[text]{keylist}, we obtain the following. 

\begin{theorem}
    The universal bundle $\E$ is actually a \emph{bundle} \ie a locally free $\struct{X \times M}$-module of finite rank. 
\end{theorem}


\noindent
With all the preparations now out of the way, we can begin with the proof.
% where we will exploit brilliant results of Mukai 
% In the definition of the moduli functor, we will use the term \emph{stable}, which we have not defined. 
\begin{proof}[Proof of Proposition \ref{Hodge isometry implies D equivalence}]
    \vspace{0.5 cm}
    \noindent
    \textsl{Identifying $M$ and $Y$.}

    \vspace{0.5 cm}
    \noindent
    \textsl{Fully faithfulness.}

    \vspace{0.5 cm}
    \noindent
    \textsl{Essential surjectivity.}
    
\end{proof}
\newpage

\appendix
\section{Spectral sequences and how to use them}

In this chapter we will first define cohomological spectral sequences and discuss their meaning through applications connected with derived categories.

At first, when encountering spectral sequences, one might think of them as just  bookkeeping devices encoding a tremendous amount of data, but we will soon see how elegantly one can infer certain properties related to derived categorical claims exploiting the fact that they can naturally be encoded with spectral sequences.
% from a spectral sequence associated to a given problem  that we can encode a problem in terms of such a sequence. 
% just by considering their form for example.
One of the most important spectral sequences, which is also very general within our scope of inspection, will be the Grothendieck spectral sequence relating the higher derived functors of two composable functors with the higher derived functors of their composition. Later on we will see that many useful and well-known spectral sequences occur as special cases of the Grothendieck spectral sequence. What follows was gathered mostly from \cite{}

% Huybrechts FMinAG : chapter on ss
% Gelfand, Manin : III.7
% McCleary 

For the time being we fix an abelian category $\mathcal A$ and start off with a definition.

\begin{definition}
    A \emph{cohomological spectral sequence} in an abelian category $\mathcal A$ consists of the following data on which we further impose two convergence conditions.
\begin{itemize}
    \item \emph{Sequence of pages.} 
    % A finite page at step $r \in \N$ is a bi-graded object $E^{\bullet, \bullet}_r$ together with differentials of bi-degree $(r, 1-r)$ \ie morphisms
    % \[
    %     d^{p,q}_r : E^{p,q}_r \to E^{p + r, q - r + 1}_r
    % \] 
    % satisfying
    % \[
    %     d^{p + r, q - r + 1}_r \circ d^{p,q}_r = 0,
    % \]
    % for each $p, q \in \Z$.
    A sequence of bi-graded objects $(E^{\bullet, \bullet}_r)_{r \in \N}$ equipped with differentials of bi-degree $(r, 1-r)$. The $r$-th term of this sequence is called the $r$-th \emph{page} and it consists of a lattice of objects $E^{p,q}_r$ of $\mathcal A$, for $p, q \in \Z$, and differentials
    \[
        d^{p,q}_r : E^{p,q}_r \to E^{p + r, q - r + 1}_r
    \] 
    satisfying
    \[
        d^{p + r, q - r + 1}_r \circ d^{p,q}_r = 0,
    \]
    for each $p, q \in \Z$.
    \item \emph{Isomorphisms.} A collection of isomorphisms
    \[
        \alpha^{p,q}_r : H^{p,q}(E_r) \xrightarrow{\sim} E^{p,q}_{r + 1},
    \]
    for all $p, q \in \Z$ and $r \in \N$, where 
    \[
        H^{p,q}(E_r) := \ker(d^{p,q}_r)/\image(d^{p-r, q + r - 1}_r),
    \]
    which allow us to turn the pages.
    \item \emph{Transfinite page.} A bi-graded object $E^{\bullet, \bullet}_\infty$. 
    % Without any differentials.
    \item \emph{Goal of computation.} A sequence of objects $(E^n)_{n \in \Z}$ of the category $\mathcal A$.
\end{itemize}
The above collection of data also has to satisfy the following two convergence conditions.
\begin{enumerate}[(a)]
    \item For each pair $(p,q)$, there exists $r_0 \geq 0$, such that for all $r \geq r_0$ we have
    \[
        d^{p,q}_r = 0 \quad \text{and} \quad d^{p + r, q - r + 1}_r = 0
    \]
    and the isomorphism $\alpha^{p,q}_r$ can be taken to be the identity. We then say that the $(p,q)$-term stabilizes after page $r_0$ and we denote $E^{p,q}_{r_0}$ (along with all the subsequent $E^{p,q}_{r}$ for $r \geq r_0$) by $E^{p,q}_\infty$.
    \item For each $n \in \Z$ there is a decreasing regular\footnote{In our case the filtration $(F^p E^n)_{p \in \Z}$ is \emph{regular}, whenever $\bigcap_p F^p E^n = \lim_p F^p E^n = 0$ and $\bigcup_p F^p E^n = \colim_p F^p E^n = E^n$.} filtration of $E^n$
    \[
        E^n \supseteq \dots \supseteq F^p E^n \supseteq F^{p + 1}E^n \supseteq \dots \supseteq 0
    \]
    and isomorphisms
    \[
        \beta^{p,q} : E^{p,q}_\infty \xrightarrow{\sim} F^p E^{p + q}/F^{p + 1}E^{p + q}
    \]
    for all $p, q \in \Z$.
\end{enumerate}

In this case we also denote the existence of such a spectral sequence by 
\[
    E^{p,q}_r \implies E^n.
\]
\end{definition}

\begin{remark}
    A few words are in order to justify us naming the sequence $(E^n)_{n \in \Z}$ our \emph{goal of computation}. Usually one is given a starting page or a small number of them and the first goal is to identify the transfinite page -- we are refering to convergence condition (a). Often one is able to infer the differentials degenerate after a number of turns of the pages from context or by observing the shape of the spectral sequence. For example \emph{first quadrant spectral sequences}, \ie the ones with non-trivial $E^{p,q}_r$ only for $(p,q)$ lying in the first quadrant, always satisfy condition (a).
    
    The second part of the computation is concerned with relating objects from the transfinte page $E^{\bullet, \bullet}_\infty$ with objects $E^n$. This is captured in the convergence condition (b), from which we can clearly observe that the intermediate quotients of the filtration $(F^p E^n)_{p \in \Z}$ for a fixed term $E^n$ lie on the anti-diagonal of the transfinite page passing through \eg $E^{n, 0}_\infty$.
    
    In condition (b) the existence of isomorphisms $\beta^{p,q}$ can also be restated by saying that $E^{p,q}_\infty$ fits into a short exact sequence
    \[
        0 \to F^{p + 1} E^n \to F^p E^n \to E^{p,q}_\infty \to 0.
    \]
    This observation becomes very fruitful when considering properties of objects of the category $\mathcal A$ which are closed under extensions, especially when the filtration of $E^n$ is finite.

    % In that case one can inductively reason about such a property of $E^n$ provided the filtration of $E^n$ is finite.
 \end{remark}

% Next we mention two ways of naturally obtaining a spectral sequence, the first one arises from a filtered chain complex. A filtration of a chain complex $K^\bullet \in \Ob(\com(\mathcal A))$ is a decreasing sequence of chain complexes 
% \[
%  K^\bullet \supseteq \dots \supseteq F^p K^\bullet \supseteq F^{p+1}K^\bullet \supseteq \dots \supseteq 0
% \]
% respecting the differential of $K^\bullet$ \ie $d(F^p K^n) \subseteq F^p K^{n+1}$. Using only this piece of information one can 

Thoku paper
\section{Lattice theory}
\label{appendix B}

This short chapter's purpose is mainly to establish the language of lattice theory. It contains two highly non-trivial results as well. One on extending certain isometries due to Nikulin \cite{Nikulin1980}, which we are generously utilizing in the proof of Proposition \ref{transcendental iso iff Mukai iso}. And another due to Milnor, one which the first one builds upon, classifying indefinite even unimodular lattices. The second one is not strictly necessary anywhere in this thesis, but serves as a pleasant way to make the intersection pairing on $\cohom{2}{X}{\Z}$ and later the Mukai pairing on $\mukai{X}$ of a K3 surface $X$ a bit more concrete. Our main sources were \cite[\S 14]{Huybrechts2016} and \cite{Nikulin1980}.

\begin{definition}
    A free finitely generated $\Z$-module $\Lambda$ together with a bilinear form $\intersect{\cdot}{\cdot} \colon \Lambda \times \Lambda \to \Z$, which is
    \begin{itemize}
        \item \emph{symmetric:} $\intersect{x}{y} = \intersect{y}{x}$, for all $x$, $y \in \Lambda$,
        \item \emph{non-degenerate:} if $\intersect{x}{y} = 0$ for all $y \in \Lambda$, then $x = 0$,
    \end{itemize}
    is called a \emph{lattice}. Lattice $\Lambda$ is said to be \emph{even} if $\intersect{x}{x}$ is an even integer for all $x \in \Lambda$. An element $x \in \Lambda$ is \emph{isotropic} if $\intersect{x}{x} = 0$.
    % a non-degenerate symmetric bilinear form $\intersect{\cdot}{\cdot} \colon \Lambda \times \Lambda \to \Z$ is called a \emph{lattice}. The lattice is said to be \emph{even} if $\intersect{x}{x}$ is an even integer for all $x \in \Lambda$.
\end{definition}

By picking an integral basis of $\Lambda$ one can represent a lattice by a symmetric integral matrix, called the \emph{intersection matrix}. If $x_1, \dots, x_n$ form an integral basis of $\Lambda$ the intersection matrix is
\[
    \bigl( (x_i.x_j)\bigr)_{1 \leq i, j \leq n}.
\]
The determinant of the intersection matrix is called the \emph{discriminant} of the lattice $\Lambda$, denoted $\operatorname{disc}\Lambda$, and is independent of the choice of the integral basis for $\Lambda$. Upon tensoring $\Lambda$ with $\R$ and $\R$-linearly extending the bilinear form $\intersect{\cdot}{\cdot}$, we obtain a non-degenerate symmetric bilinear form on $\Lambda \otimes_\Z \R$. These are known to be diagonalizable with only $1$ or $-1$ on the diagonal and we let $n_\pm$ denote the number of $\pm 1$ on the diagonal. The quantities $n_\pm$ are readily seen to be invariants of $\Lambda$. We call the pair $(n_+, n_-)$ the \emph{signature} of $\Lambda$ and the $n_+ + n_- = \rk\Lambda$ the \emph{rank} of $\Lambda$. The lattice $\Lambda$ is called \emph{definite} if either $n_+ = 0$ or $n_- = 0$ and \emph{indefinite} otherwise.



% Since the intersection matrix is symmetric, we may diagonalize it over $\R$. Because the bilinear form is non-degenerate the intersection matrix is non-singular thus the diagonalization process results in a matrix with only non-zero values on the diagonal. Denote with $n_+$ the number of positive values on the diagonal and with $n_-$ 

% Upon tensoring a lattice $\Lambda$ with $\R$ and $\R$-linearly extending the bilinear form $\intersect{\cdot}{\cdot}$,
% we obtain a finitely generated $\R$-vector space $\Lambda \otimes_\Z \R$ and $\R$-linearly extending the bilinear form $\intersect{\cdot}{\cdot}$, 
% we obtain a non-degenerate symmetric bilinear form on $\Lambda \otimes_\Z \R$. 
% In a given basis this form is represented by a symmetric matrix 
% We call the pair $(n_+, n_-)$ the \emph{signature} of $\Lambda$ and the quantity $n_+ + n_- = \rk{\Lambda}$ the \emph{rank} of $\Lambda$. The lattice $\Lambda$ is called \emph{definite}, if either $n_+ = 0$ or $n_- = 0$ and \emph{indefinite} otherwise.

\begin{definition}
    A lattice $\Lambda$ is called \emph{unimodular} if the morphism
    \[
        i_\Lambda \colon \Lambda \to \Lambda^* = \Hom_\Z(\Lambda, \Z), \qquad x \mapsto \intersect{x}{-}
    \]
    is an isomorphism of $\Z$-modules or equivalently\footnote{
        $\Lambda^*$ is also a free $\Z$-module and by non-degeneracy of $\intersect{\cdot}{\cdot}$, $i_\Lambda$ is an embedding. Then \eg using the Smith normal form allows one to see that $\abs{\operatorname{disc}\Lambda}$ equals the index $[\Lambda^* : \Lambda]$.
    } $\operatorname{disc}\Lambda = \pm 1$.
\end{definition}

\begin{remark}
    The morphism $i_\Lambda \colon \Lambda \to \Lambda^*$ is always injective, because the bilinear form $\intersect{\cdot}{\cdot}$ on $\Lambda$ is non-degenerate.
\end{remark}

\begin{example}
    \begin{enumerate}[label = (\roman*)]
        \item{The \emph{hyperbolic} lattice is defined by the intersection matrix
        \[
            \begin{pmatrix}
                0 & 1\\
                1 & 0
            \end{pmatrix}.
        \]
        We denote this indefinite unimodular lattice of rank two with $\hyperbolic$.}
        \item{The \emph{$E_8$-lattice} is given by the intersection matrix
        \[
            \begin{pmatrix}
                2 & -1 & & & & & & \\
                -1 & 2 & -1 & & & & & \\
                & -1 & 2 & -1 & -1 & & & \\
                & & -1 & 2 & 0 & & & \\
                & & -1 & 0 & 2 & -1 & & \\
                & & & &  -1 & 2 & -1 & \\
                & & & & &  -1 & 2 & -1 & \\
                & & & & & & -1 & 2
            \end{pmatrix}.
        \]
        The $E_8$-lattice is the unique even unimodular positive definite lattice of rank $8$ and we denote it by $E_8$.}
    \end{enumerate}
\end{example}

\begin{definition}
    The orthogonal direct sum of lattices $\Lambda$ and $\Lambda'$ is the free $\Z$-module $\Lambda \oplus \Lambda'$ equipped with the following bilinear form
    \[
        \intersect{(x,x')}{(y,y')} := \intersect{x}{y}_{\Lambda} + \intersect{x'}{y'}_{\Lambda'}.
    \]
    A \emph{twist} of a lattice $\Lambda$ by an integer $m \in \Z$ is the lattice $\Lambda(m)$ given by the same underlying $\Z$-module $\Lambda$, but equipped with the \emph{twisted} bilinear form
    \[
        \intersect{\cdot}{\cdot}_{\Lambda(m)} = m \cdot \intersect{\cdot}{\cdot}_{\Lambda}.
    \]
\end{definition}

\begin{definition}
    Let $\Lambda$ and $\Lambda'$ be two lattices. A morphism of lattices or an \emph{isometry} is a $\Z$-linear map $\phi \colon \Lambda \to \Lambda'$ satisfying
    \[
        \intersect{\phi(x)}{\phi(y)}_{\Lambda'} = \intersect{x}{y}_{\Lambda},
    \]
    for all $x$, $y \in \Lambda$. A lattice $L$ is called a \emph{sublattice} of $\Lambda$ if $L \subseteq \Lambda$ as underlying $\Z$-modules and the embedding $L\hookrightarrow \Lambda$ is an isometry. The \emph{orthogonal complement} of a sublattice $L$ in $\Lambda$ is defined to be
    \[
        L^\perp = \{x \in \Lambda \mid \intersect{x}{y} = 0, \text{ for all $y \in L$}\}.
    \]
\end{definition}

\begin{lemma}
    \label{orthogonal complemet is a lattice}
    Let $L$ be a sublattice of a lattice $\Lambda$. Then the orthogonal complement $L^\perp$ with the inherited bilinear form on $\Lambda$ is also a sublattice.
\end{lemma}

\begin{proof}
    First, $L^\perp$ is a subgroup of a free finite rank abelian group $\Lambda$, thus it is itself also free of finite rank. It is left to show that the inherited form on $L^\perp$ is also non-degenerate. For this we consider the rational extensions $L_\Q \subseteq \Lambda_\Q$ and show instead that the extended form on $L_\Q^\perp$ is non-degenerate. We denote with $W$ and $V$ the $\Q$-vector spaces $L_\Q$ and $\Lambda_\Q$, respectively. The adjoint map $V \to V^*$, given by $v \mapsto \intersect{v}{-}$, is injective and thus an isomorphism, therefore the linear map
    \[  
        V \longrightarrow W^*, \qquad v \mapsto \intersect{v}{-}|_W
    \]  
    is surjective. Its kernel is $W^\perp$, so $\dim W^\perp + \dim W = \dim V$. Thus $W^\perp \oplus W = V$ is an orthogonal direct sum decomposition. Assume now that $w \in W^\perp$ is such that $\intersect{w}{w'} = 0$ for all $w \in W^\perp$. Then $\intersect{w}{v} = 0$ for all $v \in V$ and by non-degeneracy of the form on $V$ this implies $w = 0$.
\end{proof}

\begin{definition}
    An embedding of lattices $L \hookrightarrow \Lambda$ is said to be \emph{primitive} if its cokernel $\coker(L \hookrightarrow \Lambda) = \Lambda/L$ is free. In that case we also call $L$ a \emph{primitive sublattice} of $\Lambda$.
\end{definition}

\begin{example}
    \label{orthogonal complement primitive}
    Let $L$ be any sublattice of a lattice $\Lambda$. Then the orthogonal complement $L^\perp$ is a primitive sublattice of $\Lambda$. Indeed, suppose $x \in \Lambda$ represents a torsion element in $\Lambda/L^\perp$. Then $n x \in L^\perp$ for some positive integer $n \in \Z$. But this implies that $x \in L^\perp$, so $x$ represents the zero element in $\Lambda/L^\perp$. Hence $\Lambda/L^\perp$ is free.
\end{example}

% \begin{proof}
%     First $L^\perp$ is a subgroup of a free finite rank abelian group $\Lambda$, thus it is itself also free of finite rank. 

    % To see that $L^\perp \hookrightarrow \Lambda$ is primitive, suppose $x \in \Lambda$ represents a torsion element in $\Lambda/L^\perp$. Then $n x \in L^\perp$ for some positive integer $n \in \Z$. But this implies that also $x \in L^\perp$ so $x$ represents the zero element in $\Lambda/L^\perp$.
% \end{proof}
    
    
    % Indeed, if $x \in \Lambda$ represented a torsion class in $\Lambda/L^\perp$, then $n x \in L^\perp$ for some positive integer $n \in \Z$. But this implies that $x \in L^\perp$, so its class in $\Lambda/L^\perp$ is trivial.

\begin{lemma}
    \label{composition of primitive embeddings}
    Let $i \colon L \hookrightarrow \Lambda$ and $i' \colon \Lambda \hookrightarrow \Lambda'$ be a pair of primitive embeddings. Then their composition $i' \circ i \colon L \hookrightarrow \Lambda'$ is also a primitive embedding.
\end{lemma}

\begin{proof}
    Since $\Lambda/L$ and $\Lambda'/\Lambda$ are free abelian, the short exact sequences $0 \to L \to \Lambda \to \Lambda/L \to 0$ and $0 \to \Lambda \to \Lambda' \to \Lambda'/\Lambda \to 0$ are split. This means that there are retracts $r \colon \Lambda \to L$ and $r' \colon \Lambda' \to \Lambda$, thus the short exact sequence $0 \to L \to \Lambda' \to \Lambda'/L \to 0$ also splits, which makes $\Lambda'/L$ free as a subgroup of $\Lambda'$.
\end{proof}

\begin{proposition}
    % [Nikulin, \cite{Nikulin1980}]
    \label{extending an isometry 1}
    \emph{\cite{Nikulin1980}}
    Let $L$ and $\Lambda$ be even lattices and assume $\Lambda$ is unimodular. Let $i_0$ and $i_1$ denote two primitive embeddings $L \hookrightarrow \Lambda$. Suppose the orthogonal complement $i_0(L)^\perp$ in $\Lambda$ contains a copy of the hyperbolic lattice $\hyperbolic$. Then there exists an isometry $\phi \colon \Lambda \to \Lambda$ such that $i_1 = \phi \circ i_0$.
    \[\begin{tikzcd}[row sep = 1em, column sep = 2em]
        % https://q.uiver.app/#q=WzAsMyxbMSwwLCJMIl0sWzAsMiwiXFxMYW1iZGEiXSxbMiwyLCJcXExhbWJkYSJdLFswLDEsImlfMCIsMix7InN0eWxlIjp7InRhaWwiOnsibmFtZSI6Imhvb2siLCJzaWRlIjoiYm90dG9tIn19fV0sWzAsMiwiaV8xIiwwLHsic3R5bGUiOnsidGFpbCI6eyJuYW1lIjoiaG9vayIsInNpZGUiOiJ0b3AifX19XSxbMSwyLCJcXHBoaSJdXQ==
        & L \\
        \\
        \Lambda && \Lambda
        \arrow["{i_0}"', hook', from=1-2, to=3-1]
        \arrow["{i_1}", hook, from=1-2, to=3-3]
        \arrow["\phi", from=3-1, to=3-3]
    \end{tikzcd}\]
\end{proposition}

\begin{proof}
    See \cite[\S 14, Theorem 1.12]{Huybrechts2016} and in particular point (iii) in the subsequent remark. For a more general statement, from which this one follows, consult \cite[Theorem 1.14.4]{Nikulin1980}.
\end{proof}

\begin{corollary}
    \label{extending an isometry 2}
    Let $i_0 \colon L_0 \hookrightarrow \Lambda_0$ and $i_1 \colon L_1 \hookrightarrow \Lambda_1$ denote two primitive embeddings of even lattices $L_0$ and $L_1$ into isomorphic even unimodular lattices $\Lambda_0$ and $\Lambda_1$, respectively. Suppose that there is an isomorphism of lattices $\psi \colon L_0 \to L_1$ and assume the orthogonal complement $i_0(L_0)^\perp$ in $\Lambda_0$ contains a copy of the hyperbolic lattice $\hyperbolic$. Then there is an isometry $\phi \colon \Lambda_0 \to \Lambda_1$, for which $\phi \circ i_0 = i_1 \circ \psi$.
    \[\begin{tikzcd}[column sep = 2em]
        % https://q.uiver.app/#q=WzAsNCxbMCwwLCJMXzAiXSxbMCwyLCJcXExhbWJkYSJdLFsyLDIsIlxcTGFtYmRhIl0sWzIsMCwiTF8xIl0sWzAsMSwiaV8wIiwyLHsic3R5bGUiOnsidGFpbCI6eyJuYW1lIjoiaG9vayIsInNpZGUiOiJib3R0b20ifX19XSxbMSwyLCJcXHBoaSJdLFszLDIsImlfMSIsMCx7InN0eWxlIjp7InRhaWwiOnsibmFtZSI6Imhvb2siLCJzaWRlIjoidG9wIn19fV0sWzAsMywiXFxwc2kiXV0=
        {L_0} && {L_1} \\
        \\
        \Lambda_0 && \Lambda_1
        \arrow["\psi", from=1-1, to=1-3]
        \arrow["{i_0}"', hook', from=1-1, to=3-1]
        \arrow["{i_1}", hook, from=1-3, to=3-3]
        \arrow["\phi", from=3-1, to=3-3]
    \end{tikzcd}\]
\end{corollary}

\begin{proof}
    Let $\theta \colon \Lambda_1 \to \Lambda_0$ be an isomorphism. Then $\theta \circ i_1 \circ \psi \colon L_0 \to \Lambda_0$ is a primitive embedding and we can use Proposition \ref{extending an isometry 1}.
\end{proof}

\begin{theorem}[Milnor, \text{\cite[Theorem 1.1.1]{Nikulin1980}}]
    \label{Milnor}
    Let $(n_+,n_-)$ be a pair of non-negative integers. Then there exists an even unimodular lattice of signature $(n_+, n_-)$ if and only if $n_+ - n_- \equiv 0 \ (\text{\emph{mod} }8)$. If $n_+$, $n_- > 0$, then the (indefinite) lattice is unique up to isomorphism.
\end{theorem}

% \begin{corollary}
%     Let $\Lambda_0$ and $\Lambda_1$ be positive definite even unimodular lattices with $\rk{\Lambda_0} = \rk{\Lambda_1}$, then
%     \[
%         \Lambda_0 \oplus \hyperbolic \iso \Lambda_1 \oplus \hyperbolic.
%     \]
% \end{corollary}

% \begin{proof}
%     Signature of $\Lambda_0$ and $\Lambda_1$ is $(\rk{\Lambda_0}, 0)$, thus the signature of $\Lambda_0 \oplus \hyperbolic$ and $\Lambda_1 \oplus \hyperbolic$ is $(\rk{\Lambda_0} + 1, 1)$, so the result follows by Milnor's Theorem \ref{Milnor}.
% \end{proof}

Its easy yet powerful corollary allow us to abstractly identify the lattice structures on $\cohom{2}{X}{\Z}$ and $\mukai{X}$.

\begin{theorem}
    \label{Classification of indefinite even unimodular lattices}
    Let $\Lambda$ be an indefinite even unimodular lattice of signature $(n_+, n_-)$. Setting $\tau = n_+ - n_-$ to be the index of $\Lambda$, then $\tau \equiv 0 \ (\text{\emph{mod }}8)$ and
    \begin{center}
        \begin{tabular}{r l}
            if $\tau \geq 0$, then & $\Lambda \iso E_8^{\oplus \frac{\tau}{8}} \oplus \hyperbolic^{\oplus n_-},$ \\
            if $\tau \leq 0$, then & $\Lambda \iso E_8(-1)^{\oplus \frac{-\tau}{8}} \oplus \hyperbolic^{\oplus n_+}.$
        \end{tabular}
    \end{center}
\end{theorem}

\begin{proof}
    The $E_8$ lattice has signature $(8,0)$, and the hyperbolic lattice $\hyperbolic$ has signature $(1,1)$, thus the signature of $E_8^{\oplus \frac{\tau}{8}} \oplus \hyperbolic^{\oplus n_-}$ equals $(8\cdot \tfrac{\tau}{8} + n_-,n_-) = (n_+, n_-)$. Now use uniqueness from Theorem \ref{Milnor}.
\end{proof}

\begin{remark}
    We know from Proposition \ref{intersection pairing on K3 is even} that for a K3 surface $X$ the Hodge lattice $\cohom{2}{X}{\Z}$ has signature $(3,19)$ and from Remark \ref{Mukai lattice structure} that $\mukai{X}$ has signature $(20,4)$, therefore Theorem \ref{Classification of indefinite even unimodular lattices} tells us that
    \[
        \cohom{2}{X}{\Z} \iso E_8(-1)^{\oplus 2} \oplus \hyperbolic^{\oplus 3} \quad \text{and} \quad
        \mukai{X} \iso E_8^{\oplus 2} \oplus \hyperbolic^{\oplus 4}.
    \]
\end{remark}
% \info{These three results are from Huybrects K3}

% \section{Test}
Text and math $a, b \in \Z$ Coh
\[
\int_{M} d\omega = \int_{\partial M } \omega
\]

\[
    D^b(\mathtt{Coh}(X))
\]

\end{document}