\documentclass[mat2, anglescina]{fmfdelo-mod}
% \documentclass[fin2, anglescina, tisk]{fmfdelo}
% \documentclass[isrm2, anglescina, tisk]{fmfdelo}
% \documentclass[ped, anglescina, tisk]{fmfdelo}
% Če pobrišete možnost tisk, bodo povezave obarvane,
% na začetku pa ne bo praznih strani po naslovu, …

%%%%%%%%%%%%%%%%%%%%%%%%%%%%%%%%%%%%%%%%%%%%%%%%%%%%%%%%%%%%%%%%%%%%%%%%%%%%%%%
% METAPODATKI
%%%%%%%%%%%%%%%%%%%%%%%%%%%%%%%%%%%%%%%%%%%%%%%%%%%%%%%%%%%%%%%%%%%%%%%%%%%%%%%

% - vaše ime
\avtor{Izak Jenko}

% - naslov dela v slovenščini
\naslov{K3 ploskve z vidika izpeljanih kategorij}

% - naslov dela v angleščini
\title{K3 surfaces from a derived categorical viewpoint}

% - ime mentorja/mentorice s polnim nazivom:
%   - doc.~dr.~Ime Priimek
%   - izr.~prof.~dr.~Ime Priimek
%   - prof.~dr.~Ime Priimek
%   za druge variante uporabite ustrezne ukaze
\mentor{.}
% \somentor{...}
% \mentorica{...}
% \somentorica{...}
% \mentorja{...}{...}
% \somentorja{...}{...}
% \mentorici{...}{...}
% \somentorici{...}{...}
% mentorja morate navesti še z angleškim nazivom:
% - doc.~dr. = Assist.~Prof.~Dr.
% - izr.~prof.~dr. = Assoc.~Prof.~Dr.
% - prof.~dr. = Prof.~Dr.
\supervisor{.}
% \supervisor{...}
% \cosupervisor{...}
% \supervisors{...}{...}
% \cosupervisors{...}{...}

% - leto magisterija
\letnica{2025}

% - povzetek v slovenščini
%   V povzetku na kratko opišite vsebinske rezultate dela. Sem ne sodi razlaga
%   organizacije dela, torej v katerem razdelku je kaj, pač pa le opis vsebine.
\povzetek{.}

% - povzetek v angleščini
\abstract{.}

% - klasifikacijske oznake, ločene z vejicami
%   Oznake, ki opisujejo področje dela, so dostopne na strani https://www.ams.org/msc/
\klasifikacija{.}

% - ključne besede, ki nastopajo v delu, ločene s \sep
\kljucnebesede{.}

% - angleški prevod ključnih besed
\keywords{.}

% - neobvezna zahvala
\zahvala{.}

% - ime datoteke z viri (vključno s končnico .bib), če uporabljate BibTeX
\literatura{references.bib}

% - ime datoteke z razširjenim povzetkom v slovenskem jeziku
\razsirjenipovzetek{razsirjenipovzetek.tex}

%%%%%%%%%%%%%%%%%%%%%%%%%%%%%%%%%%%%%%%%%%%%%%%%%%%%%%%%%%%%%%%%%%%%%%%%%%%%%%%
% DODATNE DEFINICIJE
%%%%%%%%%%%%%%%%%%%%%%%%%%%%%%%%%%%%%%%%%%%%%%%%%%%%%%%%%%%%%%%%%%%%%%%%%%%%%%%

% Packages

% Fonts
% \usepackage{newpxtext}

% \usepackage{concmath}
\usepackage[T1]{fontenc}

% \usepackage{fourier}
% \usepackage[T1]{fontenc}

% Additional packages
% \usepackage[parfill]{parskip}

\usepackage{enumerate}
\usepackage[shortlabels]{enumitem}

\usepackage{xparse}

% Colors
\usepackage{xcolor}

\usepackage{tikz}
\usetikzlibrary{cd, babel}
\usepackage{quiver}

% For comments in pdf
\usepackage{todonotes}
% \usepackage[pdftex,dvipsnames]{xcolor}
\newcommand{\info}[1]{
    \todo[linecolor=olive,backgroundcolor=olive!25,bordercolor=olive, size=\tiny]{#1}
}

% Krajšave v slovenščini
\newcommand{\ti}{t.~i.\ }
\newcommand{\tj}{tj.\ }
\newcommand{\oz}{oz.\ }

% Acronyms in english
\newcommand{\ie}{i.e.\ }
\newcommand{\eg}{e.g.\ }
\newcommand{\resp}{resp.\ }
\newcommand{\cf}{cf.\ }

% Shorthands
\newcommand{\R}{\mathbf R}
\newcommand{\N}{\mathbf Z_{\geq 0}}
\newcommand{\Z}{\mathbf Z}
\newcommand{\C}{\mathbf C}
\newcommand{\OO}{\mathcal O}
\newcommand{\F}{\mathcal F}
\newcommand{\G}{\mathcal G}
\newcommand{\A}{\mathcal A}
\newcommand{\B}{\mathcal B}
\newcommand{\I}{\mathcal I}
\newcommand{\J}{\mathcal J}
\newcommand{\D}{\mathcal D}
% \newcommand{\k}{\texttt k}


% Isos, equivalences, ...
\newcommand{\iso}{\simeq}
\newcommand{\htpy}{\simeq}
\newcommand{\natiso}{\simeq}

% Categorical constructions
\newcommand{\im}{\operatorname{im}}
\newcommand{\coim}{\operatorname{coim}}
\newcommand{\coker}{\operatorname{coker}}
\newcommand{\colim}{\operatorname{colim}}
\newcommand{\Ob}{\operatorname{Ob}}
\renewcommand{\hom}{\operatorname{Hom}}
\newcommand{\Hom}{\operatorname{Hom}}
\newcommand{\id}[1]{\operatorname{id}_{#1}}
\newcommand{\op}{\text{op}}

% Specific categories
\newcommand{\Set}{\mathsf{Set}}
\newcommand{\Ab}{\mathsf{Ab}}
\renewcommand{\mod}[1]{\mathsf{Mod}_{#1}}
\newcommand{\coherent}[1]{\mathsf{coh}(#1)}
\newcommand{\quasicoh}[1]{\mathsf{qcoh}(#1)}
\newcommand{\com}{\operatorname{Ch}}
\newcommand{\K}{\operatorname{K}}

% Derived categories
\newcommand{\dercat}[1]{\operatorname{D}(#1)}
\newcommand{\derplus}[1]{\operatorname{D}^+(#1)}
\newcommand{\derminus}[1]{\operatorname{D}^-(#1)}
\newcommand{\derb}[1]{\operatorname{D}^b(#1)}

% Derived functors
\newcommand{\rderived}[2]{\mathbf{R}^{#1}#2}
\newcommand{\lderived}[2]{\operatorname{L}^{#1}#2}

% \NewDocumentCommand{\rderived}{o m}{
%     \IfValueTF{#1}{%
%         R^{#1} #2%
%     }{%
%         R #2%
%     }%
% }

\newcommand{\derivedtensor}[3]{#1 \otimes_{#3}^{\operatorname{L}} #2}
\newcommand{\localtor}[4]{\mathcal{T}or^{#4}_{#1}(#2, #3)}
\newcommand{\localext}[4]{\mathcal{E}xt^{#1}_{#4}(#2, #3)}

% Lattices
\newcommand{\hyperbolic}{\mathcal U}


\newcommand{\inv}{^{-1}}
\newcommand{\abs}[1]{\left\lvert #1 \right\rvert}


% Quotient
% \newcommand{\bigslant}[2]{{\raisebox{.2em}{#1}\left/\raisebox{-.2em}{#2}\right.}}

%%%%%%%%%%%%%%%%%%%%%%%%%%%%%%%%%%%%%%%%%%%%%%%%%%%%%%%%%%%%%%%%%%%%%%%%%%%%%%%
% ZAČETEK VSEBINE
%%%%%%%%%%%%%%%%%%%%%%%%%%%%%%%%%%%%%%%%%%%%%%%%%%%%%%%%%%%%%%%%%%%%%%%%%%%%%%%

\begin{document}

\section{\textsf{redistribute: }Triangulated categories}

In this section we try to make a brief account of the relevant prerequisites needed to discuss derived categories.

\subsection{Additive, $k$-linear and abelian categories}

We mainly follow \cite{kashiwara2006categories}.

A categorical \emph{biproduct} of objects $X$ and $Y$ in a category $\mathcal C$ is an object $X \oplus Y$ together with morphisms
\begin{center}
    \begin{tabular}{c c}
        $p_X\colon X \oplus Y \to X$ & $p_Y\colon X \oplus Y \to Y$ \\
        $i_X\colon X \to X \oplus Y$ & $i_Y\colon Y \to X \oplus Y$ \\
    \end{tabular}
\end{center}
for which the pair $p_X, p_Y$ is the categorical product of $X$ and $Y$ and the pair $i_X, i_Y$ is the categorical coproduct. For a pair of morphisms $f_X \colon Z \to X$ and $f_Y \colon Z \to Y$ the unique induced morphism into the product $Z \to X \oplus Y$ is denoted by $(f_X, f_Y)$ and for a pair of morphisms $g_X \colon X \to Z$ and $g_Y \colon Y \to Z$ the unique induced morphism from the coproduct $X \oplus Y \to Z$ is denoted by $\langle g_X, g_Y \rangle$. For morphisms $f_0 \colon X_0 \to Y_0$ and $f_1 \colon X_1 \to Y_1$ we introduce notation $f_0 \oplus f_1 \colon X_0 \oplus X_1 \to Y_0 \oplus Y_1$ to mean either of the two (equal) morphisms 
\[
    \left(\langle f_0, 0 \rangle, \langle 0, f_1 \rangle\right) 
    \quad \text{ or } \quad 
    \left\langle (f_0, 0), (0, f_1)\right\rangle
\]
also depicted in matrix notation as
\[  
    \begin{pmatrix}
        f_0 & 0 \\
        0 & f_1
    \end{pmatrix}.
\]

Let $k$ denote either a field or the ring of integers $\Z$.

% $X \xleftarrow{p_X} X \oplus Y \xrightarrow{p_Y} Y$ is the categorical product of $X$ and $Y$ and $X \xrightarrow{i_X} X \oplus Y \xleftarrow{i_Y} Y$.

\begin{definition}
    A category $\A$ is additive (resp.\ $k$-linear) 
    \info{Decide on which terminology to use (everything is covered by $k$-linear...)}
    if all the hom-sets carry the structure of abelian groups (resp.\ $k$-modules) and the following axioms are satisfied
    \begin{enumerate}
        \item[\textbf{A1}] For all all objects $X$, $Y$ and $Z$ of $\A$ the composition
        \[
            \circ \colon \hom_{\A}(Y, Z) \times \hom_{\A}(X, Y) \to \hom_{\A}(X, Z)
        \] 
        is bilinear.
        \item[\textbf{A2}] There exists a \emph{zero object} $0$, for which $\Hom_\A(0,0) = 0$.
        \item[\textbf{A3}] For any two objects $X$ and $Y$ there exists a categorical biproduct of $X$ and $Y$. 
    \end{enumerate} 
\end{definition}

\begin{remark}
    \begin{enumerate}[(i)]
        \item The zero object $0$ of a $k$-linear category $\A$ is both the initial and terminal object of $\A$.
        \item One can recognise $k$-linear categories as the categories enriched over the category $\mod{k}$ of $k$-modules and $k$-linear maps. 
    \end{enumerate}
\end{remark}

% Whenever it will be clear from context, we will sometimes omit the subindex $\A$ from $\Hom_\A(X, Y)$ for clarity and write just $\Hom(X, Y)$ instead.

\begin{definition}
    A functor $F \colon \A \to \A'$ between two additive (resp.\ $k$-linear) categories $\A$ and $\A'$ is \emph{additive} (\resp \emph{$k$-linear}), if its action on morphisms
    \[
        \hom_\A(X, Y) \to \hom_{\A'}(F(X), F(Y))
    \] 
    is a group homomorphism (resp.\ $k$-linear map).
\end{definition}

Traditionally the term \emph{additive} is reserved for $\Z$-linear categories and $\Z$-linear functors between such categories.

% Traditionally $\Z$-linear categories and functors between such categories are called \emph{additve}.

For a morphism $f\colon X \to Y$ in an additive category $\A$ recall, that the \emph{kernel} of $f$ is the equalizer of $f$ and $0$ in $\A$, if it exists, and, dually, the \emph{cokernel} of $f$ is the coequalizer of $f$ and $0$. It is well-known and easy to verify, that the structure maps $\ker f \hookrightarrow X$ and $Y \twoheadrightarrow \coker f$ are monomophim and epimorphism respectively. We also define the \emph{image} and the \emph{coimage} of $f$ to be
\begin{align*}
    (\im f \to Y) &:= \ker(Y \to \coker f) \\
    (X \to \coim f) &:= \coker(\ker f \to X).
\end{align*}
Notice that the image and the coimage, just like the kernel and the cokernel, are defined to be morphisms, not only objects. Sometimes these are called \emph{structure morphisms}.

For a monomorphism $Y \hookrightarrow X$ we will sometimes by abuse of terminology call the cokernel $\coker(Y \to X)$ a \emph{quotient} and denote it by $X/Y$.

\begin{definition}
    A $k$-linear category $\A$ is \emph{abelian}, if it is closed under kernels and cokernels and satisfies axiom A4.
    \begin{enumerate}
        \item[\textbf{A4}] For any morphism $f\colon X \to Y$ in $\A$ the canonical morphism $\coim f \xrightarrow{\ \sim \ } \im f$ is an isomorphism.
        \[\begin{tikzcd}[column sep = small]
            % https://q.uiver.app/#q=WzAsNixbMSwwLCJYIl0sWzIsMCwiWSJdLFswLDAsIlxca2VyIGYiXSxbMywwLCJjb2tlciBmIl0sWzEsMSwiY29pbWYiXSxbMiwxLCJpbWYiXSxbMiwwXSxbMCwxLCJmIl0sWzEsM10sWzAsNF0sWzEsNV0sWzQsNSwiaCJdXQ==
            {\ker f} & X & Y & {\coker f} \\
            & \coim f & \im f
            \arrow[hook, from=1-1, to=1-2]
            \arrow["f", from=1-2, to=1-3]
            \arrow[two heads, from=1-2, to=2-2]
            \arrow[two heads, from=1-3, to=1-4]
            \arrow[hook', from=2-3, to=1-3]
            \arrow["\sim", from=2-2, to=2-3]
        \end{tikzcd}\] 
    \end{enumerate}
\end{definition}

Axiom A4 essentailly states that abelian categories are those additive categories possesing all kernels and cokernels in which the first isomorphism theorem holds.

\begin{remark}
    We obtain the morphism mentioned in axiom A4 in the following way. Due to $(\im f \to Y) = \ker(Y \to \coker f)$ and the composition $X \to Y \to \coker f$ being $0$, there is a unique morphism $X \to \im f$ by the universal property of kernels. The composition $\ker f \to X \to \im f$ then equals $0$, by the fact that $\im f \hookrightarrow Y$ is mono and $\ker f \to X \to Y$ equals $0$. From the universal property of cokernels we obtain a unique morphism $\coim f \to \im f$, since $(X \to \coim f) = \coker(\ker f \to X)$.
\end{remark}

\begin{example}
    The default examples of abelian categories are the category of abelian groups $\Ab$ or more generaly the category of $A$-modules $\mod{A}$ for a commutative ring $A$ and the categories of coherent and quasi-coherent sheaves $\coherent{X}$ and $\quasicoh{X}$ on a scheme $X$. On the other hand the category of (real or complex) vector bundles over a manifold of dimension at least $1$ is additive, but never abelian.
\end{example}

\begin{definition}
    Let
    \[
        X \xrightarrow{\ f \ } Y \xrightarrow{\ g\ } Z
    \]
    be a sequence of composable morphisms in an abelian category $\A$.  
    \begin{enumerate}[(i)]    
        \item We say this sequence is \emph{exact}, if $g \circ f = 0$ and the induced morphism $\im g \to \ker f$ is an isomorphism.

        \item Extending (i), a sequence $\cdots \to X^0 \to X^1 \to X^2 \to \cdots$ is \emph{exact}, if any subsequence $X^{i-1} \to X^i \to X^{i+1}$ for $i \in \Z$ is exact. 
        \item Exact sequences of the form $0 \to X \to Y \to Z \to 0$ are called \emph{short exact sequences}.
    \end{enumerate}
\end{definition}

To relate the definition of exactness with more primitive objects of an abelian category, namely kernels and cokernels, it is not difficult to show that
\[
    0 \to X \to Y \to Z \text{ is exact, if and only if } (X \to Y) = \ker(Y \to Z),
\]
and dually
\[
    X \to Y \to Z \to 0 \text{ is exact, if and only if } (Y \to Z) = \coker(X \to Y).
\]

In the context of abelian categories functors which preserve a bit more than just the $k$-linear structure are of interest. This brings us to

\begin{definition}
    Let $F \colon \A \to \B$ be an additive functor between abelian categories. 
    % Consider a short exact sequence $0 \to X \to Y \to Z \to 0$ in $\A$.
    \begin{enumerate}[(i)]
        \item $F$ is said to be \emph{left exact}, if $0 \to FX \to FY \to FZ$ is exact for any short exact sequence $0 \to X \to Y \to Z \to 0$ in $\A$.
        \item $F$ is said to be \emph{right exact}, if $FX \to FY \to FZ \to 0$ is exact for any short exact sequence $0 \to X \to Y \to Z \to 0$ in $\A$..
        \item $F$ is said to be \emph{exact}, if it is both left and right exact. 
    \end{enumerate}
\end{definition}

\begin{remark}
    Equivalently, one can also define left exact functors to be exactly those additive functors, which commute with kernels and dually define right exact functors to be additive functors commuting with cokernels.  
\end{remark}

\subsection{Triangulated categories}

In order to formulate the definition of a triangulated category more concisely, we introduce some preliminary notions. A \emph{category with translation} is a pair $(\D, T)$, where $\D$ is a category and $T$ is an auto-equivalence $T\colon \D \to \D$ called the \emph{translation functor}. If $\D$ is additive or $k$-linear, $T$ is moreover assumed to be additive or $k$-linear. We usually denote its action on objects $X$ with $X[1]$ and likewise its action on morphisms $f$ with $f[1]$.

A \emph{triangle} in a category with translation $(\D, T)$ is a triplet of composable morphisms $(f, g, h)$ of category $\D$ having the form 
\[
    X\xrightarrow{\ f \ } Y \xrightarrow{\ g \ } Z \xrightarrow{\ h \ } X[1].
\]
A \emph{morphism} of triangles $X\xrightarrow{f} Y \xrightarrow{g} Z \xrightarrow{h} X[1]$ and $X'\xrightarrow{f'} Y' \xrightarrow{g'} Z' \xrightarrow{h'} X'[1]$ is given by a triple of morphisms $(u, v, w)$ for which the diagram below commutes.
\[
\begin{tikzcd}
    % https://q.uiver.app/#q=WzAsOCxbMCwwLCJYIl0sWzEsMCwiWSJdLFsyLDAsIloiXSxbMywwLCJYWzFdIl0sWzAsMSwiWCciXSxbMSwxLCJZJyJdLFsyLDEsIlonIl0sWzMsMSwiWCdbMV0iXSxbMCwxXSxbMSwyXSxbMiwzXSxbNCw1XSxbNSw2XSxbNiw3XSxbMCw0LCJcXGFscGhhIiwyXSxbMSw1LCJcXGJldGEiLDJdLFsyLDYsIlxcZ2FtbWEiLDJdLFszLDcsIlxcYWxwaGFbMV0iLDJdXQ==
	X & Y & Z & {X[1]} \\
	{X'} & {Y'} & {Z'} & {X'[1]}
	\arrow[from=1-1, to=1-2]
	\arrow["u"', from=1-1, to=2-1]
	\arrow[from=1-2, to=1-3]
	\arrow["v"', from=1-2, to=2-2]
	\arrow[from=1-3, to=1-4]
	\arrow["w"', from=1-3, to=2-3]
	\arrow["{u[1]}"', from=1-4, to=2-4]
	\arrow[from=2-1, to=2-2]
	\arrow[from=2-2, to=2-3]
	\arrow[from=2-3, to=2-4]
\end{tikzcd}
\]
One can compose morphisms of triangles in the obvious way and the notion of an isomorphism of triangles is defined as usual. The following definition as stated is originally due to Verdier, who first introduced it in his thesis \cite{Verdier}.

\begin{definition}
    A \emph{triangulated category} is a $k$-linear category with translation $(\D, T)$ equipped with a class of \emph{distinguished triangles}, which is subject to the following four axioms.
    \begin{enumerate}
        \item[\textup{TR1}] \begin{enumerate}[(i)]
            \item Any triangle isomorphic to a distinguished triangle is also itself destinguished.
            \item For any $X$ the triangle
            \[
                X \xrightarrow{\id{X}} X \longrightarrow 0 \longrightarrow X[1]
            \] 
            is distinguished.
            \item For any morphism $f \colon X \to Y$ there is a distinguished triangle of the form
            \[
                X \xrightarrow{\ f \ } Y \longrightarrow Z \longrightarrow X[1].
            \]
            The object $Z$ is sometimes called the \emph{cone of $f$}. 
            \end{enumerate}
        \item[TR2] The triangle
            \[
                X\xrightarrow{\ f \ } Y \xrightarrow{\ g \ } Z \xrightarrow{\ h \ } X[1]
            \]
            is distinguished if and only if
            \[
                Y \xrightarrow{\ g \ } Z \xrightarrow{\ h \ } X[1] \xrightarrow{-f[1]} Y[1]
            \]
            is distinguished.
        \item[TR3] Given two distinguished triangles and morphisms $u\colon X \to X'$ and $v\colon Y \to Y'$, depicted in the solid diagram below 
        \[
        \begin{tikzcd}[column sep = normal, row sep = 2.5em]
            % https://q.uiver.app/#q=WzAsOCxbMCwwLCJYIl0sWzEsMCwiWSJdLFsyLDAsIloiXSxbMywwLCJYWzFdIl0sWzAsMSwiWCciXSxbMSwxLCJZJyJdLFsyLDEsIlonIl0sWzMsMSwiWCdbMV0iXSxbMCwxXSxbMSwyXSxbMiwzXSxbNCw1XSxbNSw2XSxbNiw3XSxbMCw0LCJcXGFscGhhIiwyXSxbMSw1LCJcXGJldGEiLDJdLFsyLDYsIlxcZ2FtbWEiLDJdLFszLDcsIlxcYWxwaGFbMV0iLDJdXQ==
            X & Y & Z & {X[1]} \\
            {X'} & {Y'} & {Z'} & {X'[1]}
            \arrow["f", from=1-1, to=1-2]
            \arrow["u"', from=1-1, to=2-1]
            \arrow["g", from=1-2, to=1-3]
            \arrow["v"', from=1-2, to=2-2]
            \arrow["h", from=1-3, to=1-4]
            \arrow["w"', dashed, from=1-3, to=2-3]
            \arrow["{u[1]}"', from=1-4, to=2-4]
            \arrow["{f'}", from=2-1, to=2-2]
            \arrow["{g'}", from=2-2, to=2-3]
            \arrow["{h'}", from=2-3, to=2-4]
        \end{tikzcd}
        \]
        satisfying $v \circ f = f' \circ u$, there exists a (in general non-unique) morphism $w\colon Z \to Z'$, for which $(u, v, w)$ is a morphism of triangles \ie the diagram above commutes. 
        \item[TR4] ...
    \end{enumerate} 
\end{definition}

Omitting the so-called \emph{octahedral} axiom TR4 we arrive at the definition of a \emph{pre-triangulated category}. These are essentailly the categories we will be working with, since we will never use nor verify the axiom TR4. We will nevertheless use the terminology \emph{``triangulated category''} in part to remain consistent with the existent literature and more importantly because our categories will be honest triangulated categories anyways.

% we will nevertheless use the terminology 

% \begin{definition}
%     A \emph{triangulated category} is an additive or $k$-linear category $\D$ together with two additional pieces of structure. 
%     \begin{enumerate}    
%         \item An additive or $k$-linear auto-equivalence
%         \[
%         T\colon \D \to \D,
%         \]
%         called a \emph{translation functor}. Usually we denote its action on an object $X$ of $\D$ with $X[1]$ and its action on a morphism $f$ with $f[1]$ instead of $T(X)$ and $T(f)$.
        
%         \item A collection of \emph{distinguished triangles}, which are triplets of composable morphisms $(u, v, w)$ of category $\D$ having the form 
%         \[
%             X\xrightarrow{u} Y \xrightarrow{v} Z \xrightarrow{w} X[1],
%             \]
%     \end{enumerate}
%     The above is constrained by the following axioms.

%     \begin{enumerate}
%         \item[\text{TR1}] 
%         \item[TR2] 
%     \end{enumerate}
% \end{definition}

\begin{remark}
    For a distinguished triangle
    \[
        X\xrightarrow{\ f \ } Y \xrightarrow{\ g \ } Z \xrightarrow{\ h \ } X[1]
    \]
    morphisms $f$ and $g$ are said to be of degree $0$ and morphism $h$ is said to be of degree $+1$. They are also sometimes diagramatically depicted as triangles, with the markings on morphisms describing their respective degrees.
    \[\begin{tikzcd}[column sep = small, row sep = 1.5em]
        % https://q.uiver.app/#q=WzAsMyxbMSwwLCJaIl0sWzAsMiwiWCJdLFsyLDIsIlkiXSxbMCwxLCIrMSIsMl0sWzEsMiwiMCJdLFsyLDAsIjAiLDJdXQ==
        & Z \\
        \\
        X && Y
        \arrow["{+1}"', from=1-2, to=3-1]
        \arrow["0", from=3-1, to=3-3]
        \arrow["0"', from=3-3, to=1-2]
    \end{tikzcd}\]
\end{remark}

\begin{definition}
    Let $\D$ and $\D'$ be triangulated categories with translation functors $T$ and $T'$ respectively. An additive functor $F \colon \D \to \D'$ is defined to be \emph{triangulated} or \emph{exact}, if the following two conditions are satisfied.
    \info{Maybe I will pick only triangulated, to not overload the term exact...}
    \begin{enumerate}[(i)]
        \item There exists a natural isomorphism of functors 
        \[
            \eta \colon F \circ T \iso T' \circ F.
        \]
        \item For every distinguished triangle
        \[
            X\xrightarrow{f} Y \xrightarrow{g} Z \xrightarrow{h} X[1]
        \]
        in $\D$, the triangle
        \[
            F(X) \xrightarrow{Ff} F(Y) \xrightarrow{Fg} F(Z) \to F(X)[1] 
        \]
        is distinguished in $\D'$, where the last morphism (of degree $1$) is obtained as the composition $F(Z) \xrightarrow{Fh} F(X[1]) \xrightarrow{\eta_X} F(X)[1]$.
    \end{enumerate}
\end{definition}

\begin{remark}
    The condition on $F$ being an \emph{additive} functor in the above definition is actually unnecessary and follows from conditions (i) and (ii) \cite[\href{https://stacks.math.columbia.edu/tag/05QY}{Tag 05QY}]{stacks-project}.
\end{remark}

\begin{definition}
    Triangulated categories $\D$ and $\D'$ are said to be \emph{equivalent} (as triangualated categories), if there are triangulated functors $F \colon \D \to \D'$ and $G \colon \D' \to \D$, such that $G \circ F \natiso \id{\D}$ and $F \circ G \natiso \id{\D'}$ and we call $F$ and $G$ \emph{triangulated equivalences}.
\end{definition}

\begin{proposition}
    Let $F \colon \D \to \D'$ be a triangualted functor, which is an equivalence of categories with a quasi-inverse $G \colon \D' \to \D$. Then $G$ is also a triangulated  functor. 
\end{proposition}

\begin{proof}
    This is a consequence of \cite[Proposition 1.41]{huybrechts2006fouriermukai}, as equivalences of categories are special instances of adjunctions. 
\end{proof}

As a consequence, two triangualted categories are equivalent (as triangulated categories) whenever there exists a fully faithful essentailly surjective triangulated functor from one to the other. 

\begin{definition}
    Let $H\colon \D \to \A$ be an additive functor from a triangulated category $\D$ to an abelian category $\A$. We say $H$ is a \emph{cohomological functor}, if for every distinguished triangle $X \to Y \to Z \to X[1]$ in $\D$, the induced long sequence in $\A$
    \begin{equation}
        \label{eq: long exact sequence of cohomological functor}
        \cdots \to H(X) \to H(Y) \to H(Z) \to H(X[1]) \to H(Y[1]) \to H(Z[1]) \to \cdots
    \end{equation}
    is exact.
\end{definition}

Construction of the long sequence \eqref{eq: long exact sequence of cohomological functor} is extremely simple as opposed to other known long exact sequences assigned to certain short exact sequences (\eg of sheaves or complexes) as all the complexity is actually captured within the distinguished triangle already. All one has to do is unwrap the triangle $X \to Y \to Z \to X[1]$ into the following chain of composable morphisms
\[
    \cdots \to Y[-1] \to Z[-1] \to X \to Y \to Z \to X[1] \to Y[1] \to Z[1] \to X[2] \to \cdots
\]
and apply functor $H$ over it. 

\begin{example}
    \label{partial homs are cohomological functors}
    For any object $W$ in a triangulated category $\D$ the functors 
    \[
    \Hom_\D(W, -) \colon \D \to \mod{k} \text{ and } \Hom_\D( - , W) \colon \D^\op \to \mod{k}
    \]
    are cohomological.
\end{example}

\newpage
% \chapter{Derived categories}
\section{Derived categories}

\subsection{Categories of complexes}
In order to define derived categories of an additive category $\A$ we first introduce the category of complexes and the homotopic category of complexes of $\A$. We will equip these with a triangulated structure and . Throughout this section $\A$ will be a fixed additeve or $k$-linear category and we also mention that we will be using the cohomological indexing convention.

As a preliminary we introduce graded objects in ??

\subsubsection*{{Category of complexes}}

By a \emph{chain complex} in $\A$ we mean a collection of objects and morphisms 
\[
    A^\bullet = \left((A^i)_{i \in \Z}, (d^i_A\colon A^i \to A^{i+1})_{i \in \Z}\right),
\] 
where $A^i$ are objects and $d^i$ are morphisms of $\A$, called \emph{differenitials}, subject to equations $d^{i+1}\circ d^i = 0$, for all $i \in \Z$. A complex is \emph{bounded from below} (\resp \emph{bounded from above}), if there exists $i_0 \in \Z$ for which $A^i = 0$ for all $i \leq i_0$ (\resp $i \geq i_0$) and is \emph{bounded}, if it is both bounded from below and bounded from above. A \emph{chain map} between two chain complexes $A^\bullet$ and $B^\bullet$ in $\A$ is a collection of morphisms in $\A$
\[
    f^\bullet = (f^i\colon A^i \to B^i)_{i \in \Z},
\]
for which $f^{i+1}\circ d^i_A = d^i_B \circ f^i$ holds for all $i \in \Z$. This may diagramatically be described by the following commutative ladder.

\[
\begin{tikzcd}
    % https://q.uiver.app/#q=WzAsMTAsWzEsMCwiQV57aS0xfSJdLFszLDAsIkFeaSJdLFs1LDAsIkFee2krMX0iXSxbMSwyLCJCXntpLTF9Il0sWzMsMiwiQl5pIl0sWzUsMiwiQl57aSsxfSJdLFs2LDIsIlxcY2RvdHMiXSxbNiwwLCJcXGNkb3RzIl0sWzAsMCwiXFxjZG90cyJdLFswLDIsIlxcY2RvdHMiXSxbMCwzLCJmXntpLTF9Il0sWzEsNCwiZl5pIl0sWzIsNSwiZl57aSsxfSJdLFswLDEsImRee2ktMX1fQSJdLFsxLDIsImReaV9BIl0sWzQsNSwiZF57aX1fQiJdLFszLDQsImRee2ktMX1fQiJdLFs1LDZdLFsyLDddLFs4LDBdLFs5LDNdXQ==
	\cdots & {A^{i-1}} && {A^i} && {A^{i+1}} & \cdots \\
	\\
	\cdots & {B^{i-1}} && {B^i} && {B^{i+1}} & \cdots
	\arrow[from=1-1, to=1-2]
	\arrow["{d^{i-1}_A}", from=1-2, to=1-4]
	\arrow["{f^{i-1}}", from=1-2, to=3-2]
	\arrow["{d^i_A}", from=1-4, to=1-6]
	\arrow["{f^i}", from=1-4, to=3-4]
	\arrow[from=1-6, to=1-7]
	\arrow["{f^{i+1}}", from=1-6, to=3-6]
	\arrow[from=3-1, to=3-2]
	\arrow["{d^{i-1}_B}", from=3-2, to=3-4]
	\arrow["{d^{i}_B}", from=3-4, to=3-6]
	\arrow[from=3-6, to=3-7]
\end{tikzcd}
\]

We then define the \emph{category of chain complexes} in $\A$, denoted by $\com(\A)$, to be the following additive category. 
\begin{center}
    \begin{tabular}{r l}
        \textsl{Objects:} & chain complexes in $\A$. \\
        \textsl{Morphims:} & $\Hom_{\com(\A)}(A, B)$ is the set of chain maps $A \to B$, \\ & equipped with a group structure inherited from $\A$ \\ & by applying operations componentwise.
    \end{tabular}
\end{center}
The composition law is defined componentwise and is clearly associative and bilinear, and the identity morphisms $1_{A^\bullet}$ are defined to be $(1_{A^i})_{i \in \Z}$. The complex $\cdots \to 0 \to 0 \to \cdots$ plays the role of the zero object in $\com(\A)$ and the biproduct of complexes $A$ and $B$ exists and is witnessed by the chain complex
\[
    A \oplus B = \left( (A^i \oplus B^i)_{i \in \Z}, (d^i_A \oplus d^i_B)_{i \in \Z}\right),
    % \quad \quad \cdots \to A^i \oplus B^i \xrightarrow{d^i_A \oplus d^i_B} A^{i+1} \oplus B^{i+1} \to \cdots,
\]
together with the cannonical projection and injection morphisms arising from biproducts componentwise.

Additionally, we also define the following full additive subcategories of $\com(\A)$.
\begin{center}
    \begin{tabular}{c l}
        $\com^+(\A)$ & \emph{Category of complexes bounded below}, spanned on \\ &complexes in $\A$ bounded below. \\
        $\com^-(\A)$ & \emph{Category of complexes bounded above}, spanned on \\ &complexes in $\A$ bounded above. \\
        $\com^b(\A)$ & \emph{Category of bounded complexes}, spanned on \\ &bounded complexes in $\A$. \\    
    \end{tabular}
\end{center}
\begin{remark}
    Whenever $\A$ is $k$-linear, all the categories of complexes $\com^*(\A)$ become $k$-linear as well in the obvious way.
\end{remark}

On all the categories of complexes mentioned above, we can now define the translation functor
\[
    T\colon \com^*(\A) \to \com^*(\A)
\]  
given by its action on objects and morphisms as follows.

\begin{center}
    \begin{tabular}{r l}
        \textsl{Objects:} & $T(A^\bullet) = A^\bullet[1]$ is the chain complex with $(A^\bullet[1])^i := A^{i+1}$ \\ & and differentials $d^i_{A[1]} = - d^{i+1}_{A}$. \\
        \textsl{Morphims:} & For a chain map $f^\bullet\colon A^\bullet \to B^\bullet$ we define $f^\bullet[1]$ to have \\ & component maps $(f^\bullet[1])^i = f^{i+1}$.
    \end{tabular}
\end{center}
The translation functor $T$ thus acts on a complex $A^\bullet$ by twisting its differential by a sign and shifting it one step to the \emph{left}, which is graphically pictured below.

\[\begin{tikzcd}[row sep = 1.2em]
    % https://q.uiver.app/#q=WzAsMTcsWzAsMSwiQV5cXGJ1bGxldFxcY29sb24iXSxbMCwyLCJBXlxcYnVsbGV0WzFdXFxjb2xvbiJdLFsxLDAsIlxcY2RvdHMiXSxbMywxLCJBXjAiXSxbNCwxLCJBXjEiXSxbMiwxLCJBXnstMX0iXSxbMiwyLCJBXjAiXSxbMywyLCJBXjEiXSxbNCwyLCJBXjIiXSxbNSwxLCJcXGNkb3RzIl0sWzUsMiwiXFxjZG90cyJdLFsxLDEsIlxcY2RvdHMiXSxbMSwyLCJcXGNkb3RzIl0sWzMsMCwiXFx0ZXh0dHR7MH0iXSxbNCwwLCJcXHRleHR0dHsxfSJdLFsyLDAsIlxcdGV4dHR0ey0xfSJdLFs1LDAsIlxcY2RvdHMiXSxbNiw3XSxbNyw4XSxbNSwzXSxbMyw0XSxbNCw5XSxbMTEsNV0sWzEyLDZdXQ==
	& \color{linkcolor}\cdots & {\color{linkcolor}{\mathtt{-1}}} & {\color{linkcolor}\small{\mathtt{0}}} & {\color{linkcolor}\small{\mathtt{1}}} & {\color{linkcolor}\small{\mathtt{2}}} & \color{linkcolor}\cdots \\
	{A^\bullet} & \cdots & {A^{-1}} & {A^0} & {A^1} & {A^2} & \cdots \\
	{A^\bullet[1]} & \cdots & {A^0} & {A^1} & {A^2} & {A^3} & \cdots
	\arrow[from=2-2, to=2-3]
	\arrow[from=2-3, to=2-4]
	\arrow[from=2-4, to=2-5]
	\arrow[from=2-5, to=2-6]
	\arrow[from=2-6, to=2-7]
	\arrow[from=3-2, to=3-3]
	\arrow[from=3-3, to=3-4]
	\arrow[from=3-4, to=3-5]
    \arrow[from=3-5, to=3-6]
    \arrow[from=3-6, to=3-7]
\end{tikzcd}\]

\begin{remark}
    We remark that the translation functor $T$ is clearly also additive or $k$-linear, whenever $\A$ is additive or $k$-linear.
\end{remark}

Since $T$ is an auto-equivalence there exists a quasi-inverse $T\inv$ to $T$, which is defined and unique up to a natural isomorphism. We may then speak of $T^{k}$ for any $k \in \Z$, whose action on a complex $A^\bullet$ is described by $(A^\bullet[k])^i = A^{i + k}$ with differential $d^{i}_{A[k]} = (-1)^k d^{i+k}_A$.

\subsubsection*{Homotopy category of complexes}

In this subsection we construct the homotopy category of chain complexes associated to a given additive category $\A$ and equip it with a triangulated structure. The main motivation for its introduction in this work is the fact that we will later on use it to construct the derived category of $\A$. In particular the homotopy category of $\A$, as opposed to the category of complexes\footnote{It is still possible to construct the derived category of $\A$ without passing through the homotopy category of complexes, however equipping it with a triangulated structure in this case becomes less elegant.} $\com(\A)$, can be enhanced with a triangulated structure which will afterwards descend to the level of derived categories.

\begin{definition}
    Let $f^\bullet$ and $g^\bullet$ be two chain maps in $\Hom_{\com(\A)}(A^\bullet, B^\bullet)$. We define $f^\bullet$ and $g^\bullet$ to be \emph{homotopic}, if there exists a collection of morphisms $\left( h^i\colon A^i \to B^{i-1} \right)_{i \in \Z}$, satisfying
    \[
        f^i - g^i = h^{i+1} \circ d^i_A + d^{i-1}_B \circ h^i
    \] 
    for all $i \in \Z$ and denote it by $f^\bullet \htpy g^\bullet$. We say $f^\bullet$ is \emph{nullhomotopic}, if $f^\bullet \htpy 0$.
    \[\begin{tikzcd}
        % https://q.uiver.app/#q=WzAsMTAsWzEsMCwiQV57aS0xfSJdLFszLDAsIkFeaSJdLFs1LDAsIkFee2krMX0iXSxbMSwyLCJCXntpLTF9Il0sWzMsMiwiQl5pIl0sWzUsMiwiQl57aSsxfSJdLFs2LDIsIlxcY2RvdHMiXSxbNiwwLCJcXGNkb3RzIl0sWzAsMCwiXFxjZG90cyJdLFswLDIsIlxcY2RvdHMiXSxbMCwzLCJmXntpLTF9Il0sWzEsNCwiZl5pIl0sWzIsNSwiZl57aSsxfSJdLFswLDEsImRee2ktMX1fQSJdLFsxLDIsImReaV9BIl0sWzQsNSwiZF57aX1fQiJdLFszLDQsImRee2ktMX1fQiJdLFs1LDZdLFsyLDddLFs4LDBdLFs5LDNdLFsxLDMsImheaSJdLFsyLDQsImhee2krMX0iXV0=
        \cdots & {A^{i-1}} && {A^i} && {A^{i+1}} & \cdots \\
        \\
        \cdots & {B^{i-1}} && {B^i} && {B^{i+1}} & \cdots
        \arrow[from=1-1, to=1-2]
        \arrow["{d^{i-1}_A}", from=1-2, to=1-4]
        \arrow["{f^{i-1}}"', from=1-2, to=3-2]
        \arrow["{d^i_A}", from=1-4, to=1-6]
        \arrow["{h^i}"', from=1-4, to=3-2]
        \arrow["{f^i}"', from=1-4, to=3-4]
        \arrow[from=1-6, to=1-7]
        \arrow["{h^{i+1}}"', from=1-6, to=3-4]
        \arrow["{f^{i+1}}"', from=1-6, to=3-6]
        \arrow[from=3-1, to=3-2]
        \arrow["{d^{i-1}_B}", from=3-2, to=3-4]
        \arrow["{d^{i}_B}", from=3-4, to=3-6]
        \arrow[from=3-6, to=3-7]
        \end{tikzcd}\] 
\end{definition}

\begin{lemma}
    \label{nullhomotopics form a subgroup}
    Let $A^\bullet$, $B^\bullet$ and $C^\bullet$ be complexes in $\com(\A)$, and let $f, f' \in \Hom(A^\bullet, B^\bullet)$ and $g, g' \in \Hom(B^\bullet, C^\bullet)$ be chain maps.
    \begin{enumerate}[(i)]
        \item The set of all nullhomotopic chain maps $A^\bullet \to B^\bullet$ forms a subgroup of $\Hom(A^\bullet, B^\bullet)$.
        \item If $f \htpy f'$ and $g \htpy g'$, then $g \circ f \htpy g' \circ f'$.
    \end{enumerate}
\end{lemma}

The homotopy category of complexes in $\A$, denoted by $K(\A)$, is defined to be the additive category consisting of
\begin{center}
    \begin{tabular}{r l}
        \textsl{Objects:} & chain complexes in $\A$. \\
        \textsl{Morphims:} & $\Hom_{K(\A)}(X, Y) := \Hom_{\com(\A)}(X, Y)/_\htpy$
    \end{tabular}
\end{center}
The composition law descends to the quotient by lemma \ref{nullhomotopics form a subgroup} (ii), \ie $[g]\circ[f] := [g \circ f]$, for composable $[f]$ and $[g]$, and for any $A^\bullet$ the identity morphism is defined to be $[1_A^\bullet]$. All the hom-sets $\Hom_{K(\A)}(X, Y)$ are abelian groups by lemma \ref{nullhomotopics form a subgroup} (i) and compositions are bilinear maps. 

As in the case of categories of complexes in $\A$, we can also define the following full additive subcategories of $K(\A)$.
\begin{center}
    \begin{tabular}{c l}
        $K^+(\A)$ & \emph{Homotopy category of complexes bounded below}, spanned on \\ &complexes in $\A$ bounded below. \\
        $K^-(\A)$ & \emph{Homotopy category of complexes bounded above}, spanned on \\ &complexes in $\A$ bounded above. \\
        $K^b(\A)$ & \emph{Homotopy category of bounded complexes}, spanned on \\ &bounded complexes in $\A$. \\    
    \end{tabular}
\end{center}

We now shift our focus to the construction of a triangulated structure on $K^*(\A)$. The translation functor $T\colon K^*(\A) \to K^*(\A)$ is defined on objects and morphisms in the following way.
\begin{center}
    \begin{tabular}{r l}
        \textsl{Objects:} & $A^\bullet \longmapsto A^\bullet[1]$. \\
        \textsl{Morphims:} & $[f^\bullet] \longmapsto [f^\bullet[1]]$.
    \end{tabular}
\end{center}


We define any triangle in $K(\A)$ isomorphic to a triangle of the form
\[
    A \xrightarrow{ \ f \ }  B \xrightarrow{\ \pi \ } C(f) \xrightarrow{\ \tau \ } A[1]
\]
to be distinguished.

\begin{proposition}
    The homotopy category $K(\A)$ together with the translation functor $T\colon K(\A) \to K(\A)$ and distinguished triangles defined above is a triangulated category.
\end{proposition}

\begin{remark}
    The proposition still holds true, if we replace $K(\A)$ with any of the categories $K^+(\A)$, $K^-(\A)$, $K^b(\A)$.
\end{remark}


\newpage
\begin{lemma}
    Let $f^\bullet \colon A^\bullet \to B^\bullet$ be a quasi-isomorphism and let $g^\bullet\colon C^\bullet \to B^\bullet$ be a morphism. Then there exist a quasi-isomorphism $u^\bullet\colon C_0^\bullet \to C^\bullet$ and a morphism $v^\bullet\colon C_0^\bullet \to A^\bullet$, such that $f \circ u = g \circ v$ \ie the diagram below commutes.
    \[
    \begin{tikzcd}
        {C_0^\bullet} && C^\bullet \\
        \\
        A^\bullet && B^\bullet
        \arrow["u^\bullet", dashed, from=1-1, to=1-3]
        \arrow["\sim"', draw=none, from=1-1, to=1-3]
        \arrow["v^\bullet"', dashed, from=1-1, to=3-1]
        \arrow["g^\bullet", from=1-3, to=3-3]
        \arrow["f^\bullet", from=3-1, to=3-3]
        \arrow["\sim"', draw=none, from=3-1, to=3-3]
    \end{tikzcd}
    \]
\end{lemma}

The above also holds in categories $K^+(\A)$, $K^-(\A)$, $K^b(\A)$.

\begin{proposition}
    \label{injective resolution}
    Suppose $\A$ contains enough injectives. Then every $A^\bullet$ in $K^+(\A)$ has an injective resolution.
\end{proposition}

\begin{proof}
    
\end{proof}

\begin{remark}
    As is evident from the proof by close inspection we have only used two facts about the class of all injective objects of $\A$ -- that every object of $\A$ embeds into some injective object and that the class of injectives is closed under finite direct sums. This will later on be used in subsection \ref{Subsection on F-adapted classes} when constructing a right derived functor of a given functor $F$ in the presence of an $F$-adapted class.
\end{remark}

\begin{lemma}
    Let $A^\bullet$ be any acyclic complex in $K^+(\A)$ and $I^\bullet$ a complex of injectives from $K^+(J)$. Then
    \[
        \hom_{K^+(\A)}(A^\bullet, I^\bullet) = 0.
    \]
    In other words every morphism from an acyclic complex to an injective one is nulhomotopic.
\end{lemma}

\begin{proof}
    
\end{proof}

\begin{theorem}
    Assume $\A$ contains enough injectives and let $\J \subseteq \A$ denote the full subcategory on injecitve objects of $\A$. Then the inclusion $K^+(\J) \hookrightarrow K^+(\A)$ induces an equivalence of categories
    \[
        K^+(\J) \iso D^+(\A).
    \]
\end{theorem}

\subsection{Derived functors}

\subsubsection{$F$-adapted classes}
\label{Subsection on F-adapted classes}

Unfortunately our categories at hand will sometimes not contain enough injectives, as can already be seen with the category of coherent sheaves $\coherent{X}$ on a scheme $X$, which is not a point. In this case our previously defined method of constructing right derived functors will not work. Luckly however, there exists a method of obtaining a right derived functor $\rderived{}{F} : D^+(\A) \to D^+(\B)$, given a left exact functor $F: \A \to \B$, even when $\A$ does not contain enough injectives. To this end we first introduce $F$-adapted classes.

\begin{definition}
    \label{F-adapted}
    A class of objects $\I \subset \A$ is \emph{adapted} to a left exact functor $F : \A \to \B$, if the following three conditions are satisfied.
    \begin{enumerate}[(i)]
        \item $\I$ is stable under finite sums. 
        \item Every object $A$ of $\A$ embeds into some object of $\I$, \ie there exists an object $I \in \I$ and a monomorphism $A \hookrightarrow I$.
        \item For every acyclic complex $I^\bullet$ in $K^+(\A)$, with $I^i \in \I$ for all $i \in \Z$, its image $F(I^\bullet)$ under $F$ is also acyclic.
    \end{enumerate}
\end{definition}

\begin{remark}
    We imediately observe that whenever $\A$ contains enough injectives, the class of all injective objects forms an $F$-adapted class for \emph{every} functor $F\colon \A \to \B$.
\end{remark}

In the presence of an $F$-adapted class $\I$ we will now construct the right derived functor of $F$. Firstly one can upgrade the class of objects $\I$ to a full additive subcategory $\J$ of $\A$ having its class of objects be preciesly $\I$. This is done by declaring $\hom_\J(X, Y) := \hom_\A(X, Y)$ for all objects $X$, $Y$ of the class $\I$. Then we can define a functor $K^+(F) \colon K^+(\J) \to K^+(\B)$ between triangulated categories, which acts on objects of $K^+(\J)$ as 
\[
    \left(\cdots \to A^i \xrightarrow{d^i} A^{i+1} \to \cdots\right) \longmapsto \left(\cdots \to F(A^i) \xrightarrow{F(d^i)} F(A^{i+1}) \to \cdots\right)
\]
and on morphisms as
\[
    f^\bullet = (f^i \colon A^i \to B^i)_{i \in \Z} \longmapsto (F(f^i))_{i \in \Z}.
\]
It is then clear that $K^+(F)$ is exact because of condition (ii) of definition \ref{F-adapted} and it descends to a well defined functor on the level of derived categories 
\[
    D^+(\J) \to D^+(\B).
\]
Since we want to define the derived functor $\rderived{}{F}$, whose domain is $D^+(\A)$, it remains to construct a functor $D^+(\A) \to D^+(\J)$, which we can then post-compose with $K^+(F)$ to obtain $\rderived{}{F}$. What we will show instead is that the inclusion of categories $\J \hookrightarrow \A$ induces an exact equivalence of triangulated categories $D^+(\J)$ and $D^+(\A)$.

It remains to be shown that the inclusion of categories $\J \hookrightarrow \A$ induces an the equivalence of categories $D^+(\J)$ and $D^+(\A)$. Firstly we remark that this inclusion clearly descends to a well-defined functor on the level of derived categories

This hinges on the following lemma.

\begin{lemma}
    \label{F-acyclic resolutions}
    For every complex $A^\bullet$ in $K^+(\A)$ there is a quasi-isomorphism $A^\bullet \to I^\bullet$ where $I^\bullet$ is a complex in $K^+(\J)$.
\end{lemma}

\begin{proof}
    By inspecting the proof of proposition \ref{injective resolution} we see that only conditions (i) and (ii) of definition \ref{F-adapted} were actually used, so the proof of this lemma follows mutatis mutandis from said proposition.
\end{proof}



% it is clear that the restriction of the functor $F$ to $\J$ descends to the right derived functor

% localization of K^+(\I) by qis equivalent to D^+(\A)

\begin{proposition}
    \label{F-acyclic is F-adapted}
    Let $F : \A \to \B$ be a left exact functor between two abelin categories. Let $\I \subset \A$ be an $F$-adapted class of objects in $\A$. Then the class of all $F$-acyclic objects of $\A$, denoted by $\I_F$, is also $F$-adapted. 
\end{proposition}

\begin{lemma}
    \label{F-adapted is F-acyclic}
    Every object of an $F$-adapted class is also $F$-acyclic.
\end{lemma}

\begin{proof}
    We only have to recall the definition 
\end{proof}

\begin{proof}[Proof of proposition \ref{F-acyclic is F-adapted}]
    We verify conditions (i)--(iii) of definition \ref{F-adapted}. (i) As the higher derived functors $R^iF$ are clearly additive, $\I_F$ is stable under finite sums. (ii) Holds by lemma \ref{F-adapted is F-acyclic}. (iii) Suppose $A^\bullet$ is an acyclic complex in $K^+(\A)$ with $A^i \in \I_F$ for all $i$. Then using acyclicity of $A^\bullet$, we may break this complex up into a series of short exact sequences as follows.
    \begin{align*}
        0 \to A^0 \to & A^1 \to \ker d^2 \to 0 \\
        0 \to \ker d^2 \to & A^2 \to \ker d^3 \to 0 \\
        & \vdots
    \end{align*} 
    The first short exact sequence yields the following long exact sequence associated to $F$.
    \begin{multline*}
        0 \to F(A^0) \to F(A^1) \to F(\ker d^2) \to \rderived{1}{F}(A^0) \to \cdots \\ \cdots \to \rderived{i}{F}(A^0) \to \rderived{i}{F}(A^1) \to \rderived{i}{F}(\ker d^2) \to \rderived{i+1}{F}(A^0) \to \cdots.
    \end{multline*}
    Exacness of the sequence together with $F$-acyclicity of $A^0$ and $A^1$ shows that $\ker d^2$ is also $F$-acyclic and by only considering the begining few terms of the sequence we obtain the  short exact sequence 
    % From the fact that $A^0$ and $A^1$ are $F$-acyclic we see that $\ker d^2$ is $F$-acyclic and we obtain a short exact sequence 
    \[
        0 \to F(A^0) \to F(A^1) \to F(\ker d^2) \to 0.
    \]
    After inductively applying the same reasoning for each of the original short exact sequences of \eqref{}, we are left with the following set of short exact sequences\footnote{To be more precise we have used that $F$ is left exact in order to swap the order of $F$ and $\ker$ to reach \[\ker Fd^i \simeq F(\ker d^i)\].}.
    \begin{align*}
        0 \to F(A^0) \to & F(A^1) \to \ker Fd^2 \to 0 \\
        0 \to \ker Fd^2 \to & F(A^2) \to \ker Fd^3 \to 0 \\
        & \vdots
    \end{align*}   
    Collecting them all together we can conclude that the complex $F(A^\bullet)$ is acyclic.\qedhere
\end{proof}
\newpage
\newpage

\section{Derived categories}

The first goal of this section is to construct the derived category of an abelian category $\A$ and equip it with a triangulated structure. Our arguments, although specialized to the homotopy category $\kcat{\A}$ and the class of quasi-isomorphisms, do not differ tremendously from the general theory of localization of categories. 
% Our arguments will try to be minimal and will not rely on the already established theory of localization of categories. Our approach might at times seem less elegant, but we include the exposition for completness nonetheless.
A more comprehensive and formal treatment of this topic is laid out in \cite[Chapters 7, 10, 13]{kashiwara2006categories} or \cite{milicic-dercat}.
The second part covers the construction of derived functors. We see how the established framework of derived categories nicely lends itself for the definition of derived functors and we also relate them back to the classical higher derived functors. To close the chapter, we prove a correspondence between the hom-sets of $\derplus{\A}$ and certain $\Ext$-modules.
% more general framework realtes back to the classical higher derived functors. 

\subsection{Derived categories of abelian categories}

In algebraic geometry cohomology of a geometric object, like a scheme or a variety, $X$ with respect to some coherent sheaf $\F$ plays a very important role. One way of computing $H^i(X, \F)$, which we shall also feature in section \ref{Derived categories in geometry}, involves the following. Instead of intrinsically \info{intrinsically might not be the best word as resolutions in certain cases do represent a sheaf intrinsically in the sense of generators and relations... (Hilbert syzygy thm)}
studying the sheaf $\F$, we represent it with a so called \emph{resolution}, which consists of a complex of sheaves $F^\bullet$, built up from sheaves $F^i$, for $i \in \Z$, belonging to some class of sheaves, which is well behaved under cohomology, and a quasi-isomorphism of the form $F^\bullet \to \F$ or $\F \to F^\bullet$. 
% After observing that using two different resolutions of $\F$ 
After noting that the sheaf $\F$ can be seen as a complex concentrated in degree $0$ and observing that replacing a resolution of $\F$ with another one
results in computing isomorphic cohomology groups, we are motivated  
not to distinguish the sheaf $\F$ from its resolutions in the ambient (homotopy) category of complexes any longer. 
% identify the sheaf $\F$ with the collection of all its resolutions. 
Taking a step back, we would like to modify the homotopy category $\kcat{\coherent{X}}$ in such a way that $\F$ is identified with all its resolutions, or in other words, we want all the quasi-isomorphisms of $\kcat{\coherent{X}}$ to turn into isomorphisms in a ``universal way''. We will take the latter to be our inspiration for the definition of the derived category $\dercat{\A}$ of a general abelian category $\A$. This is stated more formally in the form of the ensuing universal property.

% assigning the derived category $\dercat{\A}$ to a general abelian category $\A$. 

% is to consider a certain complex built from objects all originating from some special class, in a way representing $\F$, called its \emph{resolution}. In a way we would like to identify $\F$ with its resolution and the correct relation, which acheves this, is via quasi-isomorphisms. Inherently, computing the cohomology from said complex losses some information 

% One of the most important invariants of a geometric object $X$ in algebraic geometry is its cohomology $H^i(X, \F)$ with respect to some (coherent) sheaf $\F$. 


% The central idea in this part of algebraic geometry is 

% Our goal in general is then fairly simple, modify the homotopy category $\kcat{\A}$ in such a way that all the quasi-isomorphisms become isomorphisms in a ``universal way''. This is stated more formally in the form of the ensuing universal property.

% \begin{definition}
%     \label{universal property of D(A)}
%     Let $\A$ be an abelian category and $\kcat{\A}$ its homotopy category. A triangulated category $\dercat{\A}$ together with a triangulated functor $Q\colon \kcat{\A} \to \dercat{\A}$ is the \emph{derived category of $\A$}, if it satisfies:
%     \info{maybe I replace triangulated in this def. w/ $k$-linear and then add to proposion 2.8, that when $Q$ and $F$ are assumed to be triangulated $F_0$ is as well. This is so that I can use it directly to construct homology functors on $\dercat{\A}$}
%     \begin{enumerate}[label = (\roman*)]
%         \item For every quasi-isomorphism $s$ in $\kcat{\A}$, $Q(s)$ is an isomorphism in $\dercat{\A}$.
%         \item For any triangulated category $\D$ and any functor $F\colon \kcat{\A} \to \D$, sending quasi-isomorphisms $s$ in $\kcat{\A}$ to isomorphisms $F(s)$ in $\D$,
%         % for which $F(s)$ is an isomorphism in $\D$ for all quasi-isomorphisms $s$ in $\kcat{\A}$, 
%         there exists a triangulated functor $F_0 \colon \dercat{\A} \to \D$, which is unique up to a unique natural isomorphism, such that $F \natiso F_0 \circ Q$.
%         % , and moreover $F_0$ is required to be unique up to a unique natural isomorphism.
%         In other words, the diagram below commutes up to natural isomorphism. 
%         \[\begin{tikzcd}
%             % https://q.uiver.app/#q=WzAsMyxbMCwwLCJLKFxcQSkiXSxbMCwxLCJEKFxcQSkiXSxbMSwwLCJcXEQiXSxbMCwyLCJGIl0sWzAsMSwiUSIsMl0sWzEsMiwiRl9cXEEiLDIseyJzdHlsZSI6eyJib2R5Ijp7Im5hbWUiOiJkYXNoZWQifX19XV0=
%             {\kcat{\A}} & \D \\
%             {\dercat{\A}}
%             \arrow["F", from=1-1, to=1-2]
%             \arrow["Q"', from=1-1, to=2-1]
%             \arrow["{F_0}"', dashed, from=2-1, to=1-2]
%         \end{tikzcd}\]
%     \end{enumerate}
% \end{definition}

\begin{definition}
    \label{universal property of D(A)}
    Let $\A$ be an abelian category and $\kcat{\A}$ its homotopy category. A category $\dercat{\A}$ together with a functor $Q\colon \kcat{\A} \to \dercat{\A}$ is the \emph{derived category of $\A$}, if it satisfies:
    \begin{enumerate}[label = (\roman*)]
        \item For every quasi-isomorphism $s$ in $\kcat{\A}$, $Q(s)$ is an isomorphism in $\dercat{\A}$.
        \item For any category $\D$ and any functor $F\colon \kcat{\A} \to \D$, sending quasi-isomorphisms $s$ in $\kcat{\A}$ to isomorphisms $F(s)$ in $\D$,
        % for which $F(s)$ is an isomorphism in $\D$ for all quasi-isomorphisms $s$ in $\kcat{\A}$, 
        there exists a functor $F_0 \colon \dercat{\A} \to \D$, which is unique up to a unique natural isomorphism, such that $F \natiso F_0 \circ Q$.
        % , and moreover $F_0$ is required to be unique up to a unique natural isomorphism.
        In other words, the diagram below commutes up to natural isomorphism. 
        \[\begin{tikzcd}
            % https://q.uiver.app/#q=WzAsMyxbMCwwLCJLKFxcQSkiXSxbMCwxLCJEKFxcQSkiXSxbMSwwLCJcXEQiXSxbMCwyLCJGIl0sWzAsMSwiUSIsMl0sWzEsMiwiRl9cXEEiLDIseyJzdHlsZSI6eyJib2R5Ijp7Im5hbWUiOiJkYXNoZWQifX19XV0=
            {\kcat{\A}} & \D \\
            {\dercat{\A}}
            \arrow["F", from=1-1, to=1-2]
            \arrow["Q"', from=1-1, to=2-1]
            \arrow["{F_0}"', dashed, from=2-1, to=1-2]
        \end{tikzcd}\]
    \end{enumerate}
\end{definition}

\begin{remark}
    We recognize this definition as a special case of localization of categories \cite[\S 7, Definition 7.1.1.]{kashiwara2006categories} or \cite[\S III.2, Definition 1]{gelfand2002methods}. In particular it defines $\dercat{\A}$ to be the \emph{localization of a triangulated category} $\kcat{\A}$ by the family of all quasi-isomorphisms in $\kcat{\A}$.
\end{remark}

A naive way of constructing $\dercat{\A}$ out of $\kcat{\A}$ would be to artificially add the inverses to all the quasi-isomorphisms in $\kcat{\A}$ and then impose the correct collection of relations on the newly constructed class of morphisms. As this can quickly lead us to some set theoretic problems, we will construct a specific model, which achieves this, instead. Our construction is a priori not going to result in a locally small\footnote{A category $\mathcal C$ is called \emph{locally small} if for all objects $X$ and $Y$ of $\mathcal C$ the hom-sets $\Hom_\mathcal{C}(X, Y)$ are actual sets.} category, but as we shall soon see in practice all the categories we will be concerned with will be locally small. 
% still going to reach outside the scope of sets, but as we will se in practise 
% this can quickly become very tricky, we will construct a specific model instead.
% This will be achieved by constructing a specific model. 
% Instead we will construct a specific model, which achieves this. 

\subsubsection{Construction}

To start, we first need a technical lemma resembling the Ore condition from non-commuta\-tive algebra.

\begin{lemma}
    \label{Ore condition}
    Let $f \colon A^\bullet \to B^\bullet$ and $s\colon C^\bullet \to B^\bullet$ belong to the homotopy category $\kcat{\A}$, with $s$ being a quasi-isomorphism. Then there exists a quasi-isomorphism $u\colon C_0^\bullet \to A^\bullet$ and a morphism $g\colon C_0^\bullet \to C^\bullet$, such that the diagram below commutes in $\kcat{\A}$.
    \[\begin{tikzcd}[column sep = 2em]
        % https://q.uiver.app/#q=WzAsNCxbMCwyLCJBXlxcYnVsbGV0Il0sWzIsMiwiQl5cXGJ1bGxldCJdLFsyLDAsIkNeXFxidWxsZXQiXSxbMCwwLCJDXlxcYnVsbGV0XzAiXSxbMCwxLCJmIl0sWzIsMSwicyJdLFszLDIsImciLDAseyJzdHlsZSI6eyJib2R5Ijp7Im5hbWUiOiJkYXNoZWQifX19XSxbMywwLCJ1IiwyLHsic3R5bGUiOnsiYm9keSI6eyJuYW1lIjoiZGFzaGVkIn19fV0sWzMsMCwiXFxzaW0iLDAseyJzdHlsZSI6eyJib2R5Ijp7Im5hbWUiOiJub25lIn0sImhlYWQiOnsibmFtZSI6Im5vbmUifX19XSxbMiwxLCJcXHNpbSIsMix7InN0eWxlIjp7ImJvZHkiOnsibmFtZSI6Im5vbmUifSwiaGVhZCI6eyJuYW1lIjoibm9uZSJ9fX1dXQ==
        {C^\bullet_0} && {C^\bullet} \\
        \\
        {A^\bullet} && {B^\bullet}
        \arrow["g", dashed, from=1-1, to=1-3]
        \arrow["u", dashed, from=1-1, to=3-1]
        \arrow["\sim"', draw=none, from=1-1, to=3-1]
        \arrow["s", from=1-3, to=3-3]
        \arrow["\sim"', draw=none, from=1-3, to=3-3]
        \arrow["f", from=3-1, to=3-3]
    \end{tikzcd}\]
\end{lemma}

\begin{proof}
    \info{add proof}
\end{proof}

Equipped with the preceding lemma, we are now in a position to construct the derived category $\dercat{\A}$ of an abelian category $\A$. The derived category $\dercat{\A}$ will consist of
\[
    \textsl{Objects of $\dercat{\A}$:} \quad \text{chain complexes in $\A$},
\]
\ie the class of objects of $\kcat{\A}$ or $\com(\A)$, and a class of morphisms, which is quite intricate to define. 
% Its class of objects is formed by chain complexes in $\A$ \ie equals the class of objects of $\kcat{\A}$ and $\com(\A)$, but the definition of morphisms in this case is a bit more intricate. 
For fixed complexes $A^\bullet$ and $B^\bullet$ we define the hom-set\footnote{What will be defined here is a priori not necessarily a set, but a class, so $\dercat{\A}$, defined in this section, is not a category in the usual sense. We will however prove that in specific cases some variants of the derived category, especially concrete ones used later on, will from categories in the usual sense.} $\Hom_{\dercat{\A}}(A^\bullet, B^\bullet)$ in the following way.

\vspace{0.3 cm}

\noindent
\textsc{Hom-sets.}
A \emph{left roof spanned on $A^\bullet$ and $B^\bullet$} is a pair of morphisms $s \colon A^\bullet \to C^\bullet$ and $f\colon C^\bullet \to B^\bullet$ in the homotopy category $\kcat{\A}$, where $s$ is a quasi-isomorphism. This roof is depicted in the following diagram
\begin{equation}
    \label{eq: left roof}
    \begin{tikzcd}[row sep = small, column sep = small]
    % https://q.uiver.app/#q=WzAsMyxbMCwxLCJBIl0sWzIsMCwiQyJdLFs0LDEsIkIiXSxbMSwyLCJmIiwyXSxbMSwwLCJzIl0sWzEsMCwiXFxzaW0iLDJdXQ==
	&& C^\bullet \\
	A^\bullet &&&& B^\bullet
	\arrow["s", from=1-3, to=2-1]
	\arrow["\sim"', from=1-3, to=2-1]
	\arrow["f"', from=1-3, to=2-5]
    \end{tikzcd}
\end{equation}
and denoted by $(s,f)$.
Dually, one also obtains the notion of a \emph{right roof spanned on $A^\bullet$ and $B^\bullet$}, which is a pair of morphisms $g\colon A^\bullet \to C^\bullet$ and $u\colon B^\bullet \to C^\bullet$, where $u$ is a quasi-isomorphism, and is depicted below.
\[\begin{tikzcd}[row sep = small, column sep = small]
    % https://q.uiver.app/#q=WzAsMyxbMCwxLCJBIl0sWzIsMCwiQyJdLFs0LDEsIkIiXSxbMSwyLCJmIiwyXSxbMSwwLCJzIl0sWzEsMCwiXFxzaW0iLDJdXQ==
	&& C^\bullet \\
	A^\bullet &&&& B^\bullet
	\arrow["g"', to=1-3, from=2-1]
	\arrow["\sim"', to=1-3, from=2-5]
	\arrow["u", to=1-3, from=2-5]
\end{tikzcd}\] 
Our construction of $\Hom_{\dercat{\A}}(A^\bullet, B^\bullet)$ will be based on left roofs, for nothing is gained or lost by picking either one of the two. Both work just as well and are in fact equivalent (see \cite[Remark 7.1.18]{kashiwara2006categories}). Despite our arbitrary choice, it is still beneficial to consider both, as we will sometimes switch between the two whenever convenient. 

\begin{definition}
    \label{Definition of equivalence of roofs}
    Two left roofs $A^\bullet \xleftarrow{\ s_0 \ } C_0^\bullet \xrightarrow{\ f_0 \ } B^\bullet$ and $A^\bullet \xleftarrow{\ s_1 \ } C_1^\bullet \xrightarrow{\ f_1 \ } B^\bullet$ are defined to be \emph{equivalent}, if there exists a quasi-isomorphism $u\colon C^\bullet \to C_0^\bullet$ and a morphism ${g\colon C^\bullet \to C_1^\bullet}$ in $\kcat{\A}$, for which the diagram below commutes (in $\kcat{\A}$). 
    \begin{equation}
        \label{eq: equivalence of left roofs}
    \begin{tikzcd}[row sep = small, column sep = small]
        % [row sep = 1.5em, column sep = 2em]
        % https://q.uiver.app/#q=WzAsNSxbMCwyLCJBIl0sWzEsMSwiQ18wIl0sWzQsMiwiQiJdLFszLDEsIkNfMSJdLFsyLDAsIkMiXSxbMSwyXSxbMywwLCJcXHNpbSIsMCx7ImxhYmVsX3Bvc2l0aW9uIjo2MH1dLFszLDJdLFs0LDEsIlxcc2ltIiwyXSxbNCwzLCJnIiwyXSxbMSwwLCJcXHNpbSIsMl0sWzQsMSwidSJdXQ==
        && C^\bullet \\
        & {C_0^\bullet} && {C_1^\bullet} \\
        A^\bullet &&&& B^\bullet
        \arrow["\sim"', from=1-3, to=2-2]
        \arrow["u", from=1-3, to=2-2]
        \arrow["g"', from=1-3, to=2-4]
        \arrow["\sim"', from=2-2, to=3-1]
        \arrow["\sim"{pos=0.6}, from=2-4, to=3-1]
        \arrow[from=2-4, to=3-5]
        \arrow[from=2-2, to=3-5, crossing over]
    \end{tikzcd}
    \end{equation}
We denote this relation by $\equiv$.
\end{definition}

Note that since $C^\bullet \to C_0^\bullet \to A^\bullet$ is a quasi-isomorphism, the same is true for the composition ${C^\bullet \to C_1^\bullet \to A^\bullet}$, concluding that $g\colon C^\bullet \to C^\bullet_1$ is a quasi-isomorphism. Also observe that in the diagram \eqref{eq: equivalence of left roofs} we may find a new left roof, namely 
\begin{equation*}
    \begin{tikzcd}[row sep = small, column sep = small]
    % https://q.uiver.app/#q=WzAsMyxbMCwxLCJBIl0sWzIsMCwiQyJdLFs0LDEsIkIiXSxbMSwyLCJmIiwyXSxbMSwwLCJzIl0sWzEsMCwiXFxzaW0iLDJdXQ==
	&& C^\bullet \\
	A^\bullet &&&& B^\bullet,
	\arrow["s_0 \circ u"', from=1-3, to=2-1]
	\arrow["\sim", from=1-3, to=2-1]
	\arrow["f_1 \circ g", from=1-3, to=2-5]
    \end{tikzcd}
\end{equation*}
which is also equivalent to the two roofs we started with $(s_0, f_0)$ and $(s_1, f_1)$. 

\begin{lemma}
    The equivalence of left roofs on $A^\bullet$ and $B^\bullet$ is an equivalence relation.
\end{lemma}

\begin{proof}
    The relation is clearly reflexive. We take both $u$ and $g$ to be $\id{C^\bullet}$. By the note above it is also symmetric, for $g$ is a quasi-isomorphism. It remains to show transitivity. Suppose left roofs $A^\bullet \longleftarrow C_0^\bullet \longrightarrow B^\bullet$ and $A^\bullet \longleftarrow C_1^\bullet \longrightarrow B^\bullet$ are equivalent and left roofs $A^\bullet \longleftarrow C_1^\bullet \longrightarrow B^\bullet$ and $A^\bullet \longleftarrow C_2^\bullet \longrightarrow B^\bullet$ are equivalent. This is witnessed by the diagrams
    \begin{center}
        \begin{tabular}{l c r}
            \begin{tikzcd}[column sep = small, row sep = small]
                % https://q.uiver.app/#q=WzAsNSxbMCwyLCJBXlxcYnVsbGV0Il0sWzEsMSwiQ18wXlxcYnVsbGV0Il0sWzQsMiwiQl5cXGJ1bGxldCJdLFszLDEsIkNfMV5cXGJ1bGxldCJdLFsyLDAsIkReXFxidWxsZXQiXSxbMSwyXSxbMywwLCJcXHNpbSIsMCx7ImxhYmVsX3Bvc2l0aW9uIjo2MH1dLFszLDJdLFsxLDAsIlxcc2ltIiwyXSxbNCwxLCJcXHNpbSIsMl0sWzQsMywiZyIsMl0sWzQsMSwidSJdXQ==
                && {D_0^\bullet} \\
                & {C_0^\bullet} && {C_1^\bullet} \\
                {A^\bullet} &&&& {B^\bullet}
                \arrow["\sim"', from=1-3, to=2-2]
                \arrow["", from=1-3, to=2-2]
                \arrow[""', from=1-3, to=2-4]
                \arrow["\sim"', from=2-2, to=3-1]
                \arrow["\sim"{pos=0.7}, from=2-4, to=3-1]
                \arrow[from=2-4, to=3-5]
                \arrow[from=2-2, to=3-5, crossing over]
            \end{tikzcd} & 
            and &
            \begin{tikzcd}[column sep = small, row sep = small]
                % https://q.uiver.app/#q=WzAsNSxbMCwyLCJBXlxcYnVsbGV0Il0sWzEsMSwiQ18wXlxcYnVsbGV0Il0sWzQsMiwiQl5cXGJ1bGxldCJdLFszLDEsIkNfMV5cXGJ1bGxldCJdLFsyLDAsIkReXFxidWxsZXQiXSxbMSwyXSxbMywwLCJcXHNpbSIsMCx7ImxhYmVsX3Bvc2l0aW9uIjo2MH1dLFszLDJdLFsxLDAsIlxcc2ltIiwyXSxbNCwxLCJcXHNpbSIsMl0sWzQsMywiZyIsMl0sWzQsMSwidSJdXQ==
                && {D_1^\bullet} \\
                & {C_1^\bullet} && {C_2^\bullet} \\
                {A^\bullet} &&&& {B^\bullet.}
                \arrow["\sim"', from=1-3, to=2-2]
                \arrow["", from=1-3, to=2-2]
                \arrow[""', from=1-3, to=2-4]
                \arrow["\sim"', from=2-2, to=3-1]
                \arrow["\sim"{pos=0.7}, from=2-4, to=3-1]
                \arrow[from=2-4, to=3-5]
                \arrow[from=2-2, to=3-5, crossing over]
            \end{tikzcd}
        \end{tabular}
    \end{center}
    % \[\begin{tikzcd}
    %     % https://q.uiver.app/#q=WzAsOCxbMCwzLCJBIl0sWzEsMiwiQ18wIl0sWzYsMywiQiJdLFs1LDIsIkNfMiJdLFszLDIsIkNfMSJdLFsyLDEsIkRfMCJdLFs0LDEsIkRfMSJdLFszLDAsIkMiXSxbMSwyXSxbMywwLCJcXHNpbSIsMCx7ImxhYmVsX3Bvc2l0aW9uIjo2MH1dLFszLDJdLFsxLDAsIlxcc2ltIiwyXSxbNSwxLCJcXHNpbSIsMix7ImNvbG91ciI6WzI0MCw2MCw2MF19LFsyNDAsNjAsNjAsMV1dLFs1LDQsIiIsMCx7ImNvbG91ciI6WzI0MCw2MCw2MF19XSxbNCwwLCJcXHNpbSIsMix7ImxhYmVsX3Bvc2l0aW9uIjo3MH1dLFs0LDJdLFs2LDMsIiIsMCx7ImNvbG91ciI6WzMwMCw2MCw2MF19XSxbNiw0LCJcXHNpbSIsMix7ImNvbG91ciI6WzMwMCw2MCw2MF19LFszMDAsNjAsNjAsMV1dLFs3LDUsIlxcc2ltIiwyLHsic3R5bGUiOnsiYm9keSI6eyJuYW1lIjoiZGFzaGVkIn19fV0sWzcsNiwiIiwwLHsic3R5bGUiOnsiYm9keSI6eyJuYW1lIjoiZGFzaGVkIn19fV1d
    %     &&& C^\bullet \\
    %     && {D_0^\bullet} && {D_1^\bullet} \\
    %     & {C_0^\bullet} && {C_1^\bullet} && {C_2^\bullet} \\
    %     A^\bullet &&&&&& B^\bullet
    %     \arrow["\sim"', dashed, from=1-4, to=2-3]
    %     \arrow[dashed, from=1-4, to=2-5]
    %     \arrow["\sim"', color={rgb,255:red,92;green,92;blue,214}, from=2-3, to=3-2]
    %     \arrow[color={rgb,255:red,92;green,92;blue,214}, from=2-3, to=3-4]
    %     \arrow["\sim"', color={rgb,255:red,214;green,92;blue,214}, from=2-5, to=3-4]
    %     \arrow[color={rgb,255:red,214;green,92;blue,214}, from=2-5, to=3-6]
    %     \arrow["\sim"', from=3-2, to=4-1]
    %     \arrow[curve={height=12pt}, from=3-2, to=4-7]
    %     \arrow["\sim"'{pos=0.2}, from=3-4, to=4-1]
    %     \arrow[from=3-4, to=4-7]
    %     \arrow["\sim"{pos=0.6}, curve={height=-12pt}, from=3-6, to=4-1]
    %     \arrow[from=3-6, to=4-7]
    % \end{tikzcd}\]
    By lemma \ref{Ore condition} the right roof $D_0^\bullet \to C_1^\bullet \leftarrow D_1^\bullet$ may be completed to form a commutative square
    \[\begin{tikzcd}[column sep = 2em]
        % https://q.uiver.app/#q=WzAsNCxbMCwyLCJEXzBeXFxidWxsZXQiXSxbMiwyLCJDXzFcXGJ1bGxldCJdLFsyLDAsIkReXFxidWxsZXRfMSJdLFswLDAsIkNeXFxidWxsZXQiXSxbMCwxXSxbMiwxLCJcXHNpbSJdLFszLDAsIlxcc2ltIiwyLHsic3R5bGUiOnsiYm9keSI6eyJuYW1lIjoiZGFzaGVkIn19fV0sWzMsMiwiIiwwLHsic3R5bGUiOnsiYm9keSI6eyJuYW1lIjoiZGFzaGVkIn19fV1d
        {C^\bullet} && {D^\bullet_1} \\
        \\
        {D_0^\bullet} && {C_1^\bullet,}
        \arrow[dashed, from=1-1, to=1-3]
        \arrow["\sim"', dashed, from=1-1, to=3-1]
        \arrow["\sim", from=1-3, to=3-3]
        \arrow[from=3-1, to=3-3]
    \end{tikzcd}\]
    proving that $A^\bullet \leftarrow C_0^\bullet \rightarrow B^\bullet$ and $A^\bullet \leftarrow C_2^\bullet \rightarrow B^\bullet$ are equivalent.
\end{proof}

For a left roof \eqref{eq: left roof} we let $\left[s \setminus f\right]$ or $\left[A^\bullet \xleftarrow{\ s\ } C^\bullet \xrightarrow{\ f \ } B^\bullet\right]$ denote its equivalence class under $\equiv$. 

We then define $\Hom_{\dercat{\A}}(A^\bullet, B^\bullet)$ to be the class of left roofs spanned by $A^\bullet$ and $B^\bullet$, quotiented by the relation $\equiv$. That is
% \begin{equation*}
%     \Hom_{\dercat{\A}}(A^\bullet, B^\bullet) := 
%     \left\{ (s, f) \in \coprod_{C^\bullet \in \Ob(\kcat{\A})} \Hom_{\kcat{\A}}(C^\bullet, A^\bullet) \times \Hom_{\kcat{\A}}(C^\bullet, B^\bullet) \ \middle\vert \ \text{$s$ quasi-isomorphism} \right\}
% \end{equation*}
% \[
% \Hom_{\dercat{\A}}(A^\bullet, B^\bullet) := \\ 
% \left\{
% \left[
% \begin{array}{l c r}
% & C^\bullet & \\
% \quad \swarrow & & \searrow \quad \ \\
% A^\bullet & & B^\bullet
% \end{array}
% \right]_\equiv
% \, \middle| \,
% \begin{array}{l}
% C^\bullet \in \Ob \kcat{\A}, \\
% f \in \Hom_{\kcat{\A}}(C^\bullet, B^\bullet), \\
% s \in \Hom_{\kcat{\A}}(C^\bullet, A^\bullet) \text{ quasi-iso.}
% \end{array}
% \right\}.
% \]
\[
\Hom_{\dercat{\A}}(A^\bullet, B^\bullet) := 
\left\{
\left[
\begin{tikzcd}[row sep = small, column sep = 0.8em]
    % https://q.uiver.app/#q=WzAsMyxbMCwxLCJBIl0sWzIsMCwiQyJdLFs0LDEsIkIiXSxbMSwyLCJmIiwyXSxbMSwwLCJzIl0sWzEsMCwiXFxzaW0iLDJdXQ==
	&& C^\bullet \\
	A^\bullet &&&& B^\bullet
	\arrow["s", from=1-3, to=2-1]
	\arrow["\sim"', from=1-3, to=2-1]
	\arrow["f"', from=1-3, to=2-5]
\end{tikzcd}
\right]_\equiv
\, \middle| \,
\begin{array}{l}
C^\bullet \in \Ob \kcat{\A}, \\
f \in \Hom_{\kcat{\A}}(C^\bullet, B^\bullet), \\
s \in \Hom_{\kcat{\A}}(C^\bullet, A^\bullet) \text{ quasi-iso.}
\end{array}
\right\}.
\]

\vspace{0,3 cm}

\noindent
\textsc{Composition.} Next we define the composition operations
\[
    \circ \colon \Hom_{\dercat{\A}}(A_0^\bullet, A_1^\bullet) \times \Hom_{\dercat{\A}}(A_1^\bullet, A_2^\bullet) \to \Hom_{\dercat{\A}}(A_0^\bullet, A_2^\bullet),
\]
for all objects $A_0^\bullet$, $A_1^\bullet$ and $A_2^\bullet$ of $\dercat{\A}$. Let $\phi_0\colon A_0^\bullet \to A_1^\bullet$ and $\phi_1\colon A_1^\bullet \to A_2^\bullet$ be a pair of composable morphisms in $\dercat{\A}$. Next, pick their respective left roof representatives, $A_0^\bullet \xleftarrow{\ s_0\ } C_0^\bullet \xrightarrow{\ f_0 \ } A_1^\bullet$ and $A_1^\bullet \xleftarrow{\ s_1\ } C_1^\bullet \xrightarrow{\ f_1 \ } A_2^\bullet$, and concatenate them according to the solid zig-zag diagram below. 
\[\begin{tikzcd}[row sep = small, column sep = small]
    % https://q.uiver.app/#q=WzAsNixbMCwyLCJBXzAiXSxbMiwxLCJDXzAiXSxbNCwyLCJBXzEiXSxbOCwyLCJBXzIiXSxbNiwxLCJDXzEiXSxbNCwwLCJDIl0sWzEsMiwiZl8wIiwyXSxbMSwwLCJzXzAiXSxbMSwwLCJcXHNpbSIsMix7Im9mZnNldCI6LTEsInN0eWxlIjp7ImJvZHkiOnsibmFtZSI6Im5vbmUifSwiaGVhZCI6eyJuYW1lIjoibm9uZSJ9fX1dLFs0LDIsInNfMSJdLFs0LDMsImZfMSIsMl0sWzUsMSwidSIsMCx7InN0eWxlIjp7ImJvZHkiOnsibmFtZSI6ImRhc2hlZCJ9fX1dLFs1LDQsImciLDIseyJzdHlsZSI6eyJib2R5Ijp7Im5hbWUiOiJkYXNoZWQifX19XSxbNCwyLCJcXHNpbSIsMix7Im9mZnNldCI6LTEsInN0eWxlIjp7ImJvZHkiOnsibmFtZSI6Im5vbmUifSwiaGVhZCI6eyJuYW1lIjoibm9uZSJ9fX1dLFs1LDEsIlxcc2ltIiwyLHsib2Zmc2V0IjotMSwic3R5bGUiOnsiYm9keSI6eyJuYW1lIjoibm9uZSJ9LCJoZWFkIjp7Im5hbWUiOiJub25lIn19fV1d
	&&&& C^\bullet \\
	&& {C_0^\bullet} &&&& {C_1^\bullet} \\
	{A_0^\bullet} &&&& {A_1^\bullet} &&&& {A_2^\bullet}
	\arrow["u", dashed, from=1-5, to=2-3]
	\arrow["\sim"', shift left, draw=none, from=1-5, to=2-3]
	\arrow["g"', dashed, from=1-5, to=2-7]
	\arrow["{s_0}", from=2-3, to=3-1]
	\arrow["\sim"', shift left, draw=none, from=2-3, to=3-1]
	\arrow["{f_0}"', from=2-3, to=3-5]
	\arrow["{s_1}", from=2-7, to=3-5]
	\arrow["\sim"', shift left, draw=none, from=2-7, to=3-5]
	\arrow["{f_1}"', from=2-7, to=3-9]
\end{tikzcd}\]
By Lemma \ref{Ore condition} there are morphisms $u\colon C^\bullet \to C_0^\bullet$ and $g\colon C^\bullet \to C_1^\bullet$, depicted with dashed arrows, completing the diagram in $\kcat{\A}$. In this way we obtain a left roof, formed by a quasi-isomorphism $s_0 \circ u$ and a morphism $f_1 \circ g$, the equivalence class of which we define to be the composition 
\[
    \phi_1 \circ \phi_0 :=
    \left[A_0^\bullet \xleftarrow{s_0 \circ u} C^\bullet \xrightarrow{f_1 \circ g} A_2^\bullet\right].
\]
It can be shown that this is a well defined composition, independent of the choice of left roof representatives of $\phi_0$ and $\phi_1$ and independent of the choice of the peak $C^\bullet$ and morphisms $u\colon C^\bullet \to C_0^\bullet$ and $g\colon C^\bullet \to C_1^\bullet$, for which the square at the top of the diagram commutes. It is also true that the operation is associative. We leave out the proof, because it is routine, but refer the reader to a very detailed account by Miličić \cite[Ch.~1.3]{milicic-dercat}. 

\vspace{0,3 cm}

\noindent
\textsc{Identities.}
The identity morphism on an object $A^\bullet$ of $\dercat{\A}$ is defined to be the equivalence class of $A^\bullet \xleftarrow{\id{A^\bullet}} A^\bullet \xrightarrow{\id{A^\bullet}} A^\bullet$. It is not difficult to check that these classes play the role of identity morphisms in $\dercat{\A}$. 

\vspace{0,3 cm}

\noindent
\textsc{Functor $Q\colon \kcat{\A} \to \dercat{\A}$.} The \emph{localization} functor $Q\colon \kcat{\A} \to \dercat{\A}$ is an identity on objects functor with the action on morphisms defined by the assignment 
\[\begin{tikzcd}[row sep = 0.3em, column sep = 0.75em]
    % https://q.uiver.app/#q=WzAsNCxbMSwwLCJcXEhvbV97XFxkZXJjYXR7XFxBfX0oQV5cXGJ1bGxldCwgSV5cXGJ1bGxldCkiXSxbMCwwLCJcXEhvbV97fShBXlxcYnVsbGV0LCBJXlxcYnVsbGV0KSJdLFswLDEsIlxcbGVmdCggZlxcY29sb24gQV5cXGJ1bGxldCBcXHRvIEleXFxidWxsZXRcXHJpZ2h0KSJdLFsxLDEsIlxcbGVmdFtBXlxcYnVsbGV0IFxceGxlZnRhcnJvd3sgXFwgXFxpZHtBXlxcYnVsbGV0fSB9IEFeXFxidWxsZXQgXFx4cmlnaHRhcnJvd3sgXFwgZiBcXCB9IEleXFxidWxsZXRcXHJpZ2h0XSJdLFsyLDMsIiIsMix7InN0eWxlIjp7InRhaWwiOnsibmFtZSI6Im1hcHMgdG8ifX19XSxbMSwwXV0=
	{\Hom_{\kcat{\A}}(A^\bullet, B^\bullet)} & {\Hom_{\dercat{\A}}(A^\bullet, B^\bullet)} \\
	{\left( f\colon A^\bullet \to B^\bullet\right)} & {\left[A^\bullet \xleftarrow{ \ \id{A^\bullet} } A^\bullet \xrightarrow{ \ f \ } B^\bullet\right].}
	\arrow["{Q}",from=1-1, to=1-2]
	\arrow[maps to, from=2-1, to=2-2]
\end{tikzcd}\]

% \vspace{0,3 cm}

\noindent
\textsc{$k$-linear structure.} To equip the hom-sets of $\dercat{\A}$ with a $k$-module structure, we consult the following lemma resembling finding a common denominator in the context of left roofs.

\begin{lemma}
    \label{common denominator for left roofs}
    Let $\phi_0 \colon A^\bullet \to B^\bullet$ and $\phi_1 \colon A^\bullet \to B^\bullet$ be two morphisms in $\dercat{\A}$ represented by left roofs
    \begin{center}
    \begin{tabular}{l c r}
        \begin{tikzcd}[row sep = small, column sep = small]
            % https://q.uiver.app/#q=WzAsMyxbMCwxLCJBIl0sWzIsMCwiQ18wIl0sWzQsMSwiQiJdLFsxLDIsImZfMCJdLFsxLDAsInNfMCIsMl0sWzEsMCwiXFxzaW0iLDAseyJzdHlsZSI6eyJib2R5Ijp7Im5hbWUiOiJub25lIn0sImhlYWQiOnsibmFtZSI6Im5vbmUifX19XV0=
            && {C_0^\bullet} \\
            A^\bullet &&&& B^\bullet
            \arrow["{s_0}"', from=1-3, to=2-1]
            \arrow["\sim", draw=none, from=1-3, to=2-1]
            \arrow["{f_0}", from=1-3, to=2-5]
        \end{tikzcd}
        & \quad and \quad &
        \begin{tikzcd}[row sep = small, column sep = small]
            % https://q.uiver.app/#q=WzAsMyxbMCwxLCJBIl0sWzIsMCwiQ18wIl0sWzQsMSwiQiJdLFsxLDIsImZfMCJdLFsxLDAsInNfMCIsMl0sWzEsMCwiXFxzaW0iLDAseyJzdHlsZSI6eyJib2R5Ijp7Im5hbWUiOiJub25lIn0sImhlYWQiOnsibmFtZSI6Im5vbmUifX19XV0=
            && {C_1^\bullet} \\
            A^\bullet &&&& B^\bullet
            \arrow["{s_1}"', from=1-3, to=2-1]
            \arrow["\sim", draw=none, from=1-3, to=2-1]
            \arrow["{f_1}", from=1-3, to=2-5]
        \end{tikzcd}
    \end{tabular}
    \end{center}
    Then there is a quasi-isomorphism $s \colon C^\bullet \to A^\bullet$ and morphisms $g_0$, $g_1 \colon C^\bullet \to B^\bullet$, such that 
    \begin{center}
    \begin{tabular}{l c r}
        \begin{tikzcd}[row sep = small, column sep = small]
            % https://q.uiver.app/#q=WzAsMyxbMCwxLCJBIl0sWzIsMCwiQ18wIl0sWzQsMSwiQiJdLFsxLDIsImZfMCJdLFsxLDAsInNfMCIsMl0sWzEsMCwiXFxzaW0iLDAseyJzdHlsZSI6eyJib2R5Ijp7Im5hbWUiOiJub25lIn0sImhlYWQiOnsibmFtZSI6Im5vbmUifX19XV0=
            && {C^\bullet} \\
            A^\bullet &&&& B^\bullet
            \arrow["{s}"', from=1-3, to=2-1]
            \arrow["\sim", draw=none, from=1-3, to=2-1]
            \arrow["{g_0}", from=1-3, to=2-5]
        \end{tikzcd}
        & \quad and \quad &
        \begin{tikzcd}[row sep = small, column sep = small]
            % https://q.uiver.app/#q=WzAsMyxbMCwxLCJBIl0sWzIsMCwiQ18wIl0sWzQsMSwiQiJdLFsxLDIsImZfMCJdLFsxLDAsInNfMCIsMl0sWzEsMCwiXFxzaW0iLDAseyJzdHlsZSI6eyJib2R5Ijp7Im5hbWUiOiJub25lIn0sImhlYWQiOnsibmFtZSI6Im5vbmUifX19XV0=
            && {C^\bullet} \\
            A^\bullet &&&& B^\bullet
            \arrow["{s}"', from=1-3, to=2-1]
            \arrow["\sim", draw=none, from=1-3, to=2-1]
            \arrow["{g_1}", from=1-3, to=2-5]
        \end{tikzcd}
    \end{tabular}
    \end{center}
    represent $\phi_0$ and $\phi_1$ respectively.
\end{lemma}

\begin{proof}
    \info{add proof}    
\end{proof}

Using notation from lemma \ref{common denominator for left roofs}, we define the addition operation on $\Hom_{\dercat{\A}}(A^\bullet, B^\bullet)$ by the rule
\[
    \phi_0 + \phi_1 := \left[ A^\bullet \xleftarrow{\ s\ } C^\bullet \xrightarrow{ g_0 + g_1 } B^\bullet \right].
\]
It is well defined, associative, commutative and has a neutral element $0 = Q(0)$. The action of scalars of the ring $k$ is defined as  
\[
    \lambda \phi_0 := \left[ A^\bullet \xleftarrow{\ s_0\ } C^\bullet \xrightarrow{ \lambda f_0 } B^\bullet \right].
\]
Moreover, the composition law $\circ$ is $k$-bilinear and the localization functor $Q \colon \kcat{\A} \to \dercat{\A}$ is an additive functor. 

The zero complex $0^\bullet$ plays the role of the zero object of $\dercat{\A}$ and the direct sum of two complexes is seen to exist and is induced from the direct sum in $\kcat{\A}$ by applying the localization functor $Q$.


\vspace{0,3 cm}

\noindent
\textsc{Triangulated structure.} The translation functor $T \colon \dercat{\A} \to \dercat{\A}$ is defined to be the usual translation functor on objects and a morphism represented by some roof is sent to the equivalence class of that roof on which we have acted with the translation functor of $\kcat{\A}$. This action respects the equivalence relations, which define the hom-sets of $\dercat{\A}$, and thus induces a well defined functor, which is also an additive auto-equivalence, the quasi-inverse being given by translation in the other direction.  

The class of distinguished triangles in $\dercat{\A}$ is defined to consist of all triangles $A^\bullet \to B^\bullet \to C^\bullet \to A[1]^\bullet$, for which there exists a distinguished triangle $A_0^\bullet \to B_0^\bullet \to C_0^\bullet \to A_0[1]^\bullet$ in $\kcat{\A}$, such that $Q(A_0^\bullet) \to Q(B_0^\bullet) \to Q(C_0^\bullet) \to Q(A_0[1]^\bullet)$ is isomorphic to $A^\bullet \to B^\bullet \to C^\bullet \to A[1]^\bullet$ in $\dercat{\A}$. Spelling this out in the case of left roofs, we see that this condition implies the existence of another distinguished triangle $A_1^\bullet \to B_1^\bullet \to C_1^\bullet \to A_1[1]^\bullet$ in $\kcat{\A}$, such that the following commutative diagram, where vertical arrows are all quasi-isomorphisms exists in $\kcat{\A}$.
\[\begin{tikzcd}[]
    % https://q.uiver.app/#q=WzAsMTIsWzAsMSwiQSJdLFsxLDEsIkIiXSxbMiwxLCJDIl0sWzMsMSwiQVsxXSJdLFswLDIsIlgnIl0sWzEsMiwiWSciXSxbMiwyLCJaJyJdLFszLDIsIlgnWzFdIl0sWzAsMCwiWCciXSxbMSwwLCJZJyJdLFsyLDAsIlonIl0sWzMsMCwiWCdbMV0iXSxbMCwxXSxbMSwyXSxbMiwzXSxbNCw1XSxbNSw2XSxbNiw3XSxbMCw0XSxbMSw1XSxbMyw3XSxbOCw5XSxbOSwxMF0sWzEwLDExXSxbMCw4XSxbMSw5XSxbMiwxMF0sWzMsMTFdLFsyLDZdXQ==
	{A^\bullet} & {B^\bullet} & {C^\bullet} & {A[1]^\bullet} \\
	{A_1^\bullet} & {B_1^\bullet} & {C_1^\bullet} & {A_1[1]^\bullet} \\
    {A_0^\bullet} & {B_0^\bullet} & {C_0^\bullet} & {A_0[1]^\bullet}
	\arrow[from=1-1, to=1-2]
	\arrow[from=1-2, to=1-3]
	\arrow[from=1-3, to=1-4]
	\arrow[from=2-1, to=1-1]
	\arrow[from=2-1, to=2-2]
	\arrow[from=2-1, to=3-1]
	\arrow[from=2-2, to=1-2]
	\arrow[from=2-2, to=2-3]
	\arrow[from=2-2, to=3-2]
	\arrow[from=2-3, to=1-3]
	\arrow[from=2-3, to=2-4]
	\arrow[from=2-3, to=3-3]
	\arrow[from=2-4, to=1-4]
	\arrow[from=2-4, to=3-4]
	\arrow[from=3-1, to=3-2]
	\arrow[from=3-2, to=3-3]
	\arrow[from=3-3, to=3-4]
\end{tikzcd}\]

Verification, that the above defines the structure of a triangulated category on $\dercat{\A}$, is left out, but we remark, that it follows naturally from the triangulated structure on $\kcat{\A}$ and refer the reader to \cite[Chapter 2, Theorem 1.6.1]{milicic-dercat} or \cite[Chapter 10, Theorem 10.2.3]{kashiwara2006categories}.

\begin{remark}
    All that has been defined and established in this section with left roofs can analogously also be done with right roofs.
\end{remark}

\begin{proposition}
    \label{universal property for D(A) proposition}
    The constructed derived category $\dercat{\A}$ of an abelian category $\A$ together with the localization functor $Q \colon \kcat{\A} \to \dercat{\A}$ satisfies the universal property of Definition \ref{universal property of D(A)}. Additionally, in notation of Definition \ref{universal property of D(A)}, if category $\D$ and the functor $F \colon \kcat{\A} \to \D$ are $k$-linear or triangulated, the induced functor $F_0$ is as well.
\end{proposition}

\begin{proof}
    For (i) suppose $s\colon A^\bullet \to B^\bullet$ is a quasi-isomorphism. Then the inverse of $Q(s)$ is a morphism $\phi\colon B^\bullet \to A^\bullet$ represented by the left roof $B^\bullet \xleftarrow{\ s \ } A^\bullet \xrightarrow{\id{A^\bullet}} A^\bullet$. Clearly $\phi \circ Q(s) = \id{A^\bullet}$, whereas $Q(s) \circ \phi$, represented by $B^\bullet \xleftarrow{s} A^\bullet \xrightarrow{s} B^\bullet$, can be seen to equal $\id{B^\bullet}$ by the following diagram.
    \[\begin{tikzcd}[column sep = small, row sep = small]
        % https://q.uiver.app/#q=WzAsNSxbMCwyLCJCXlxcYnVsbGV0Il0sWzEsMSwiQV5cXGJ1bGxldCJdLFs0LDIsIkJeXFxidWxsZXQiXSxbMywxLCJCXlxcYnVsbGV0Il0sWzIsMCwiQV5cXGJ1bGxldCJdLFsxLDIsInMiLDJdLFszLDAsIiIsMCx7ImxhYmVsX3Bvc2l0aW9uIjo2MCwibGV2ZWwiOjIsInN0eWxlIjp7ImhlYWQiOnsibmFtZSI6Im5vbmUifX19XSxbMywyLCIiLDAseyJsZXZlbCI6Miwic3R5bGUiOnsiaGVhZCI6eyJuYW1lIjoibm9uZSJ9fX1dLFsxLDAsInMiLDJdLFs0LDEsIiIsMix7ImxldmVsIjoyLCJzdHlsZSI6eyJoZWFkIjp7Im5hbWUiOiJub25lIn19fV0sWzQsMywicyIsMl1d
        && {A^\bullet} \\
        & {A^\bullet} && {B^\bullet} \\
        {B^\bullet} &&&& {B^\bullet}
        \arrow[equals, from=1-3, to=2-2]
        \arrow["s"', from=1-3, to=2-4]
        \arrow["s"', from=2-2, to=3-1]
        \arrow[equals, from=2-4, to=3-1]
        \arrow[equals, from=2-4, to=3-5]
        \arrow["s"'{pos=0.7}, from=2-2, to=3-5, crossing over]
    \end{tikzcd}\]% , where all the unmarked arrows are identities.
    For (ii) the functor $F_0 \colon \dercat{\A} \to \kcat{\A}$ is defined by the assignment on
    \begin{center}
    \begin{tabular}{r r c l}
        \textsl{Objects:} & $A^\bullet$ & $\longmapsto$ & $F(A^\bullet)$. \\
        \textsl{Morphisms:} & $[A^\bullet \xleftarrow{s} C^\bullet \xrightarrow{f} B^\bullet]$ & $\longmapsto$ & $F(f) \circ F(s)\inv$.
    \end{tabular}
    \end{center}
    This is well defined and functorial for the functor $F\colon \kcat{\A} \to \D$ sends quasi-isomorphisms of $\kcat{\A}$ to isomorphisms of $\D$. Moreover, $F_0$ is a $k$-linear functor, as can quickly be computed by an application of Lemma \ref{common denominator for left roofs}. Using the same notation as in the lemma, we have
    \begin{gather*}
        F_0(\phi_0 + \phi_1) = F(g_0 + g_1) \circ F(s)\inv = F(g_0) \circ F(s)\inv + F(g_1) \circ F(s)\inv = F_0(\phi_0) + F_1(\phi_1) \\
        \text{and} \quad F_0(\lambda \phi_0) = F(\lambda f_0) \circ F(s_0)\inv = \lambda (F(f_0) \circ F(s_0)\inv) = \lambda F_0(\phi_0).
    \end{gather*}
    It is clear from the definition, that $F_0 \circ Q = F$ and that $F_0$ is unique up to a natural isomorphism for which the diagram \eqref{universal property of D(A)} commutes. 

    Lastly, when $\D$ and $F$ are triangulated, the functor $F_0$ also commutes with the translation functors of $\dercat{\A}$ and $\D$, because $Q$ as the identity on objects and $F$ commutes with translation functors of $\kcat{\A}$ and $\D$. From the fact that distinguished triangles in $\dercat{\A}$ are by definition exactly those triangles, which are isomorphic to $Q$-images of distinguished triangles of $\kcat{\A}$, and the fact that $F$ maps distinguished triangles of $\dercat{\A}$ to distinguished triangles of $\D$, we can conclude that $F_0$ is triangulated as well. 
\end{proof}


% ============================================================== %


\subsubsection{Cohomology}

Since cohomology already appeared as an indispensable tool on the level of the homotopy category $\kcat{\A}$, it would be nice to also have it be accessible at the level of derived categories. 
% also have access to it at the level of derived categories. 
We can in fact define cohomology functors on $\dercat{\A}$ by inducing them from the functors $H^i \colon \kcat{\A} \to \A$, which by definition send quasi-isomorphisms of $\kcat{\A}$ to isomorphisms of $\A$, using the universal property of Definition \ref{universal property of D(A)}. In this way we obtain $k$-linear functors 
\[
    H^i \colon \dercat{\A} \rightarrow \A, \qquad i \in \Z
\]
for $i \in \Z$, which as expected return the $i$-th cohomology of a complex $A^\bullet$ and return the morphism $H^i(f) \circ H^i(s)\inv$ when applied to a morphism in $\dercat{\A}$, represented by a left roof \eqref{eq: left roof}.


% ============================================================== %


\subsubsection{Subcategories of derived categories}

% We have already seen $\derplus{\A}$, $\derminus{\A}$ and $\derb{\A}$ be defined as full triangulated subcategories of $\dercat{\A}$ on the class of certain bounded complexes taking values in $\A$.  

As with the homotopy category of complexes $\kcat{\A}$, we also have the bounded versions of the derived category $\dercat{\A}$. They are constructed in the exact same way as $\dercat{\A}$ above, with the adjustment, that we replace every instance of the category $\kcat{\A}$ with either $\kplus{\A}$, $\kminus{\A}$ or $\kb{\A}$, to obtain $\derplus{\A}$, $\derminus{\A}$ or $\derb{\A}$, respectively. They are clearly all triangulated as well. All three bounded variants naturally come with the inclusion functors
\[
    \derplus{\A} \to \dercat{\A}, \qquad \derminus{\A} \to \dercat{\A}, \qquad \derb{\A} \to \dercat{\A},
\]
% \[\begin{tikzcd}[row sep = small]
%     % https://q.uiver.app/#q=WzAsNCxbMSwwLCJCIl0sWzAsMSwiQSJdLFsyLDEsIkMiXSxbMSwyLCJEIl0sWzEsMF0sWzAsMl0sWzEsM10sWzMsMl1d
% 	& \derplus{\A} \\
% 	\derb{\A} && \dercat{\A}, \\
% 	& \derminus{\A}
% 	\arrow[from=1-2, to=2-3]
% 	\arrow[from=2-1, to=1-2]
% 	\arrow[from=2-1, to=3-2]
% 	\arrow[from=3-2, to=2-3]
% \end{tikzcd}\]
defined to be the identity on objects and send a morphism, \ie equivalence class of some roof representative with respect to the ``bounded variant'' of the equivalence relation \ref{Definition of equivalence of roofs}, where, again, every instance of $\kcat{\A}$ is replaced by either $\kplus{\A}$, $\kminus{\A}$ or $\kb{\A}$, to the equivalence class of that same roof representative, but now under the original equivalence relation. All three functors are clearly also triangulated.  

% appearing as full triangulated subcategories of the unbounded variant $\dercat{\A}$ by proposition \ref{triangulated subcategory proposition}.
% As with the homotopy category of complexes $\kcat{\A}$, we also have the following bounded versions of the derived category $\dercat{\A}$, appearing as full triangulated subcategories of the unbounded variant $\dercat{\A}$ by proposition \ref{triangulated subcategory proposition}.

% \begin{center}
%     % \begin{tabular}{c l}
%     %     $\derplus{\A}$ & spanned on complexes in $\A$ bounded below. \\
%     %     $\derminus{\A}$ & spanned on complexes in $\A$ bounded above. \\
%     %     $\derb{\A}$ & spanned on bounded complexes in $\A$. \\    
%     % \end{tabular}
%     \begin{tabular}{c l}
%         $\derplus{\A}$ & spanned on objects of $\kplus{\A}$. \\
%         $\derminus{\A}$ & spanned on objects of $\kminus{\A}$. \\
%         $\derb{\A}$ & spanned on objects of $\kb{\A}$. \\    
%     \end{tabular}
% \end{center}

There is also another (equivalent) way of obtaining the categories mentioned above, which will sometimes be more convenient to work with. We introduce them in the form of the following proposition.

\begin{proposition}
    \label{Subcategories of D(A) but bigger}
    Let $\A$ be an abelian category. 
    \begin{enumerate}[label = (\roman*)]
        % \item{The functor $\derplus{\A} \to \dercat{\A}$ (\resp $\derminus{\A} \hookrightarrow \dercat{\A}$) is fully faithful and induces an equivalence between $\derplus{\A}$ and the full triangulated subcategory of $\dercat{\A}$ spanned on (possibly unbounded) complexes $A^\bullet$, for which there is an integer $n \in \Z$, such that $H^i(A^\bullet) \iso 0$ for all $i < n$ (\resp $i > n$).}
        \item{The functor $\derplus{\A} \to \dercat{\A}$ is fully faithful and induces an equivalence between $\derplus{\A}$ and the full triangulated subcategory of $\dercat{\A}$ spanned on (possibly unbounded) complexes $A^\bullet$, for which there is an integer $n \in \Z$, such that $H^i(A^\bullet) \iso 0$ for all $i < n$.}
        \item{The functor $\derminus{\A} \to \dercat{\A}$ is fully faithful and induces an equivalence between $\derminus{\A}$ and the full triangulated subcategory of $\dercat{\A}$ spanned on (possibly unbounded) complexes $A^\bullet$, for which there is an integer $n \in \Z$, such that $H^i(A^\bullet) \iso 0$ for all $i > n$.} 
        \item{The functor $\derb{\A} \to \dercat{\A}$ is fully faithful and induces an equivalence between $\derb{\A}$ and the full triangulated subcategory of $\dercat{\A}$ spanned on (possibly unbounded) complexes $A^\bullet$, for which there is an integer $n \in \Z$, such that $H^i(A^\bullet) \iso 0$ for all $\abs{i} > n$.} 
    \end{enumerate}
\end{proposition}

To make the proof of this proposition conceptually clearer, we introduce two complexes associated to a complex $A^\bullet$, called its \emph{right and left truncations}
\begin{align*}
    \rtruncation{n}(A^\bullet) &= (\cdots \to A^{n-2} \to A^{n-1} \to \ker d^{n} \to 0 \to \cdots) \text{ \ and } \\
    \ltruncation{n}(A^\bullet) &= (\cdots \to 0 \to \coker d^{n-1} \to A^{n+1} \to A^{n+2} \to \cdots).
\end{align*}
They come equipped with the natural inclusion map $j \colon \rtruncation{n}(A^\bullet) \to A^\bullet$ and the natural quotient map $q \colon A^\bullet \to \ltruncation{n}(A^\bullet)$, satisfying the following claims
\begin{align}
    H^i(j) \colon H^i(\rtruncation{n}(A^\bullet)) \to H^i(A^\bullet) \text{ is an isomorphism for all } i \leq n. \label{right truncation isos on cohomology} \\
    H^i(q) \colon H^i(A^\bullet) \to H^i(\ltruncation{n}(A^\bullet)) \text{ is an isomorphism for all } i \geq n. \label{left trunctation isos on cohomology}
\end{align}
Indeed, claim \eqref{right truncation isos on cohomology} is clearly true for $i < n$, with $i = n$ being the only interesting case. Here the right commutative square of the diagram below
% \[\begin{tikzcd}[column sep = small, row sep = 1.5em]
%     % https://q.uiver.app/#q=WzAsNCxbMCwwLCJcXGtlciBkXm4iXSxbMiwwLCIwIl0sWzAsMiwiQV57bn0iXSxbMiwyLCJBXntuKzF9Il0sWzIsMywiZF5uIl0sWzAsMSwiMCJdLFsxLDMsImpee24rMX0iXSxbMCwyLCJqXm4iLDJdXQ==
% 	{\ker d^n} && 0 \\
% 	\\
% 	{A^{n}} && {A^{n+1}}
% 	\arrow["0", from=1-1, to=1-3]
% 	\arrow["{j^n}"', from=1-1, to=3-1]
% 	\arrow["{j^{n+1}}", from=1-3, to=3-3]
% 	\arrow["{d^n}", from=3-1, to=3-3]
% \end{tikzcd}\]
\[\begin{tikzcd}[column sep = small, row sep = 1.5em]
    % https://q.uiver.app/#q=WzAsNixbMiwwLCJcXGtlciBkXm4iXSxbNCwwLCIwIl0sWzIsMiwiQV57bn0iXSxbNCwyLCJBXntuKzF9Il0sWzAsMCwiQV57bi0xfSJdLFswLDIsIkFee24tMX0iXSxbMiwzLCJkXm4iXSxbMCwxLCIwIl0sWzEsMywial57bisxfSJdLFswLDIsImpebiIsMl0sWzQsMCwiXFxkZWx0YSJdLFs0LDUsImpee24tMX0iLDJdLFs1LDIsImRee24tMX0iXV0=
	{A^{n-1}} && {\ker d^n} && 0 \\
	\\
	{A^{n-1}} && {A^{n}} && {A^{n+1}}
	\arrow["\delta", from=1-1, to=1-3]
	\arrow["{j^{n-1}}"', from=1-1, to=3-1]
	\arrow["0", from=1-3, to=1-5]
	\arrow["{j^n}"', from=1-3, to=3-3]
	\arrow["{j^{n+1}}", from=1-5, to=3-5]
	\arrow["{d^{n-1}}", from=3-1, to=3-3]
	\arrow["{d^n}", from=3-3, to=3-5]
\end{tikzcd}\]
induces the identity morphism between the kernels of the horizontal differentials and the left commutative square induces the identity morphism between the images of the horizontal differential. Consequently $j$ induces the identity morphism on the $n$-th cohomology $H^n(j) \colon H^n(\rtruncation{n}(A^\bullet)) \to H^n(A^\bullet)$.
Dually, the claim \eqref{left trunctation isos on cohomology} also holds. 
% In order to make the proof of this proposition conceptually clearer, we introduce truncation functors $\tau_{\geq n} \colon \kcat{\A} \to \kplus{\A}$, defined by
% \begin{center}
%     \begin{tabular}{r r c l}
%         \textsl{Objects:} & $A^\bullet$ & $\longmapsto$ & $(\cdots \to 0 \to \coker d^{n-1} \to A^{n+1} \to A^{n+2} \to \cdots)$. \\
%         \textsl{Morphisms:} & $(f\colon A^\bullet \to B^\bullet)$ & $\longmapsto$ & $ 
%         (\tau_{\geq n}(f)^i)_{i \in \Z}$, 
%         % \text{, where } \tau_{\geq n}(f)^i = 
%         % \begin{cases}
%         %     0, \ i < n \\
%         %     \coker d^{n-1}_A \to \coker d^{n-1}_B, \ i = n \\ 
%         %     f^i, \ i > n.
%         % \end{cases}$.
%     \end{tabular}
% \end{center}
% where $\tau_{\geq n}(f)^i$ is the zero morphism for $i < n$, the induced morphism $\coker d^{n-1}_A \to \coker d^{n-1}_B$ for $i = n$ and morphisms $f^i\colon A^i \to B^i$ for $i > n$. 
% % Naturally for every $A^\bullet$ there is a quotient map $q \colon A^\bullet \to \ltruncation{n}(A^\bullet)$.

% We also have $\tau_{\leq n} \colon \kcat{\A} \to \kminus{\A}$, defined by
% \begin{center}
%     \begin{tabular}{r r c l}
%         \textsl{Objects:} & $A^\bullet$ & $\longmapsto$ & $(\cdots \to A^{n-2} \to A^{n-1} \to \ker d^{n} \to 0 \to \cdots)$. \\
%         \textsl{Morphisms:} & $(f\colon A^\bullet \to B^\bullet)$ & $\longmapsto$ & $ 
%         (\tau_{\leq n}(f)^i)_{i \in \Z}$, 
%     \end{tabular}
% \end{center}
% where $\tau_{\leq n}(f)^i$ is $f^i \colon A^i \to B^i$ for $i < n$, the induced morphism $\ker d^n_A \to \ker d^n_B$ for $i = n$ and the zero morphism for $i > 0$.
% % In this case there are natural inclusions $j \colon \rtruncation{n}(A^\bullet) \to A^\bullet$ for every $A^\bullet$.

% \begin{lemma}
%     For every complex $A^\bullet$ the natural quotient map $q \colon A^\bullet \to \ltruncation{n}(A^\bullet)$ induces isomorphisms $H^i(q)$ for $i \geq n$ and the natural inclusion map $j \colon \rtruncation{n}(A^\bullet) \to A^\bullet$ induces isomorphisms $H^i(j)$ for $i \leq n$.
% \end{lemma}

\begin{proof}[Proof of Proposition \ref{Subcategories of D(A) but bigger}]
    For (i) the functor $\derplus{\A} \rightarrow \dercat{\A}$ is readily seen to be fully faithful by considering morphisms to be represented by \emph{right} roofs. For example in the right roof $A^\bullet \to C^\bullet \leftarrow B^\bullet$ spanned on complexes $A^\bullet$ and $B^\bullet$ of $\derplus{\A}$, we first observe that $C^\bullet$ has bounded cohomology from below, thus $q \colon C^\bullet \to \ltruncation{n}(C^\bullet)$ is a quasi-isomorphism for some $n \in \Z$. Extending the aforementioned right roof with quasi-isomorphism $q$, to obtain $A^\bullet \to \ltruncation{n}(C^\bullet) \leftarrow B^\bullet$, then yields an equivalent right roof, whose peak now belongs to $\derplus{\A}$ and the equivalence class of which is therefore sent to the morphism represented by $A^\bullet \to C^\bullet \leftarrow B^\bullet$.
    % taking the viewpoint of morphisms being represented by \emph{right} roofs and extending their peaks using the quotient map into their left truncations. For example in right roof $A^\bullet \to C^\bullet \leftarrow B^\bullet$ spanned on complexes $A^\bullet$ and $B^\bullet$ of $\derplus{\A}$, we first observe that $C^\bullet$ has bounded   
    % truncating their peaks by extening them with the quotient map $q$. 
    % so we verify only that it is essentially surjective onto the subcategory of complexes with cohomology bounded from below. 

    To verify this functor is essentially surjective onto the subcategory of complexes with cohomology bounded from below,
    suppose $A^\bullet$ satisfies $H^i(A^\bullet) \iso 0$, for all $i < n$. Then by \eqref{left trunctation isos on cohomology} the quotient map $q \colon A^\bullet \to \ltruncation{n}(A^\bullet)$ is a quasi-isomorphism, inducing an isomorphism $A^\bullet \iso \ltruncation{n}(A^\bullet)$ in $\dercat{\A}$. 
    
    The case (ii) is proven analogously, now considering left roofs and right truncations. For (iii), the functor $\derb{\A} \to \dercat{\A}$ is viewed as the composition $\derb{\A} \to \derplus{\A} \to \dercat{\A}$. Here $\derb{\A} \to \derplus{\A}$ is seen to be fully faithful via a minor modification of the argument for (ii) and $\derplus{\A} \to \dercat{\A}$ is fully faithful by (i). Essential surjectivity onto the required subcategory then follows as the roof $A^\bullet \to \ltruncation{n}(A^\bullet) \leftarrow \rtruncation{n}(\ltruncation{n}(A^\bullet))$, for some $n \in \Z$, witnesses an isomorphism in $\dercat{\A}$ between a complex $A^\bullet$ with bounded cohomology and a bounded complex $\rtruncation{n}(\ltruncation{n}(A^\bullet))$. \qedhere
    % we view the inclusion $\derb{\A} \hookrightarrow \dercat{\A}$ as the composition $\derb{\A} \hookrightarrow \derplus{\A} \hookrightarrow \dercat{\A}$
    
    % claim that the quotient map $q \colon A^\bullet \to \ltruncation{n}(A^\bullet)$ is actually a quasi-isomorphism.
    
    % This shows that there is an isomorphism in $\dercat{\A}$ between $A^\bullet$ and a complex $\ltruncation{n}(A^\bullet)$, which belongs to $\derplus{\A}$. 
\end{proof}

\info{Move footnote about $\A \to \kcat{\A}$ being fully faithful over here.}

\begin{proposition}
    \label{A -> D(A) fully faithful}
    The natural functor $\A \to \dercat{\A}$ is fully faithful and identifies $\A$ with the full subcategory of $\dercat{\A}$, spanned by complexes $A^\bullet$, satisfying $H^i(A^\bullet) \iso 0$, for all $i \neq 0$.
\end{proposition}

\begin{proof}
    For fully faithfulness see \cite[Chapter 13, Proposition 13.1.10]{kashiwara2006categories}. Essential surjectivity onto the required subcategory is \cite[Chapter 13, Proposition 13.1.12]{kashiwara2006categories} and follows from a similar argument as in the proof of Proposition \ref{Subcategories of D(A) but bigger} (iii) by taking $n$ to be $0$, thus showing that in $\dercat{\A}$ the complex $A^\bullet$ with trivial cohomology in non-zero degrees is isomorphic to $H^0(A^\bullet)$ concentrated in degree $0$.
\end{proof}
% \begin{proof}
%     We first show the morphism action
%     \begin{equation}
%         \label{eq: morphism action of A -> D(A)}
%         \Hom_\A(A, B) \to \Hom_{\dercat{\A}}(A, B)
%     \end{equation}
%     is injective. Suppose morphisms $f_0$ and $f_1 \colon A \to B$ are sent to equivalent roofs, described by the next diagram.
%     \[\begin{tikzcd}[row sep = small, column sep = small]
%         % [row sep = 1.5em, column sep = 2em]
%         % https://q.uiver.app/#q=WzAsNSxbMCwyLCJBIl0sWzEsMSwiQ18wIl0sWzQsMiwiQiJdLFszLDEsIkNfMSJdLFsyLDAsIkMiXSxbMSwyXSxbMywwLCJcXHNpbSIsMCx7ImxhYmVsX3Bvc2l0aW9uIjo2MH1dLFszLDJdLFs0LDEsIlxcc2ltIiwyXSxbNCwzLCJnIiwyXSxbMSwwLCJcXHNpbSIsMl0sWzQsMSwidSJdXQ==
%         && C^\bullet \\
%         & {A} && {A} \\
%         A &&&& B
%         \arrow["\sim", from=1-3, to=2-2]
%         \arrow["u"', from=1-3, to=2-2]
%         \arrow["g", from=1-3, to=2-4]
%         \arrow[equals, from=2-2, to=3-1]
%         \arrow[equals, from=2-4, to=3-1]
%         \arrow["f_0", from=2-4, to=3-5]
%         \arrow["f_1"'{pos=0.7}, from=2-2, to=3-5, crossing over]
%     \end{tikzcd}\]
%     Then firstly we see that $u = g$ and taking the $0$-th cohomology, shows $f_0 = f_1$. For surjectivity we assign to a morphism $\phi \in \Hom_{\dercat{\A}}(A, B)$, represented by a roof $A \to C^\bullet \leftarrow B$, 


%     To see $\A \to \dercat{\A}$ is fully faithful one uses the $0$-th cohomology functor $H^0$ on roof representatives
%     % The functor $\A \to \dercat{\A}$ can be written as the composition $\A \to \derplus{\A} \to \dercat{\A}$, so we only need to verify fully faithfulness of $\A \to \derplus{\A}$. 
%     % This is seen through the fact that any quasi-isomorphism between two complexes concentrated in degree $0$ is an isomorphism. 
%     % That the functor is fully faithful is a consequence of the fact that it can be written as the composition $\A \to \derplus{\A} \to \dercat{\A}$ and any quasi-isomorphism between two complexes concentrated in degree $0$ is an isomorphism. 
%     Essential surjectivity follows from a similar argument as in the proof of Proposition \ref{Subcategories of D(A) but bigger} (iii) by taking $n$ to be $0$, thus showing that in $\dercat{\A}$ the complex $A^\bullet$ with trivial cohomology in non-zero degrees is isomorphic to $H^0(A^\bullet)$ concentrated in degree $0$. \info{this proof is wack.} 
% \end{proof}

% ============================================================== %

\subsubsection{Derived categories of abelian categories with enough injectives}

After possibly working with proper classes, when constructing the derived category, this section will once again place us back on familiar grounds of set theory. Aside from these concerns the establishing result of this section will have a very important practical application -- construction of derived functors. Under an assumption on the abelian category $\A$, we will establish an equivalence of $\derplus{\A}$ with the homotopy category of a full additive subcategory of $\A$, spanned on \emph{injective objects}, which we define presently. 

\begin{definition}
    An object $I$ of an abelian category $\A$ is \emph{injective}\footnote{Dually, an object $P$ of $\A$ is \emph{projective}, if $P$ is injective in $\A^\op$, or more explicitly, the functor $\Hom_\A(P, -) \colon \A \to \mod{k}$ is exact. Category $\A$ is said to \emph{have enough projectives}, if every object $A$ of $\A$ is a quotient of some projective object \ie there is an epimorphism $P \twoheadrightarrow A$ for some projective object $P$.
}, if the functor 
\[
    \Hom_\A(-, I)\colon \A^\op \to \mod{k}
\]
is exact. Equivalently, $I$ is injective, whenever for any monomorphism ${A \hookrightarrow B}$ and morphism $A \to I$ there is a morphism $B \to I$, for which the following diagram commutes.
    \[\begin{tikzcd}[column sep = 2em]
        % https://q.uiver.app/#q=WzAsNCxbMSwxLCJBIl0sWzIsMSwiQiJdLFsxLDAsIkkiXSxbMCwxLCIwIl0sWzMsMF0sWzAsMSwiIiwwLHsic3R5bGUiOnsidGFpbCI6eyJuYW1lIjoiaG9vayIsInNpZGUiOiJ0b3AifX19XSxbMCwyXSxbMSwyLCIiLDEseyJzdHlsZSI6eyJib2R5Ijp7Im5hbWUiOiJkYXNoZWQifX19XV0=
        & I \\
        0 & A & B
        \arrow[from=2-1, to=2-2]
        \arrow[from=2-2, to=1-2]
        \arrow[hook, from=2-2, to=2-3]
        \arrow[dashed, from=2-3, to=1-2]
    \end{tikzcd}\]
\end{definition}
A category $\A$ is said to \emph{have enough injectives}, if every object $A$ of $\A$ embeds into an injective object \ie there is a monomorphism $A \hookrightarrow I$ for some injective object $I$. 

\begin{remark}
    A simple verification shows, that the full subcategory of an abelian or $k$-linear category $\A$, spanned on all injective objects of $\A$ is a $k$-linear category, as the zero object $0$ is injective and the biproduct of two injective objects is again injective. We denote this category by $\J$.
\end{remark}

Injective objects will be utilized as building blocks for representing complexes. They are the preferred class of objects to work with in this context because they enable a favourable interplay between cohomology and homotopy. 
For example, as a consequence of lemma \ref{acyclic to injective is nulhomotopic}, every acyclic bounded below complex of injectives $I^\bullet$ is actually nullhomotopic \ie isomorphic to the object $0$ in $\kplus{\A}$. 
In light of this, consider a complex $A^\bullet$ of $\kplus{\A}$. A complex of injectives $I^\bullet$ of $\kplus{\J}$ together with a quasi-isomorphism $f \colon A^\bullet \to I^\bullet$ is called an \emph{injective resolution} of $A^\bullet$.
Later we will also see that this injective resolution is unique up to homotopy equivalence.
% of their nice behaviour with respect to cohomology and homotopy

% We call a quasi-isomorphism $f \colon A^\bullet \to I^\bullet$ an \emph{injective resolution} of the complex $A^\bullet$. Later we will also see that this injective resolution is unique up to homotopy \ie any two injective resolutions of $A^\bullet$ are homotopically equivalent. 
% Throughout this section differentials of $I^\bullet$ will be denoted with $\delta^i\colon I^i \to I^{i+1}$.

\begin{proposition}
    \label{injective resolution}
    % Suppose $\A$ contains enough injectives. Then every $A^\bullet$ in $\kplus{\A}$ has an injective resolution.
    Suppose $\A$ contains enough injectives. Then for every $A^\bullet$ in $\kplus{\A}$ there is a quasi-isomorphism $f \colon A^\bullet \to I^\bullet$, where $I^\bullet \in \Ob \kplus{\A}$ is a complex of injectives.
\end{proposition}

\begin{proof}
    We will inductively construct a complex $I^\bullet$ in $\kplus{\A}$, built up from injectives, and a quasi-isomorphism $f \colon A^\bullet \to I^\bullet$. For simplicity assume $A^i = 0$ for $i < 0$. 
    
    The base case is trivial. Define $I^i = 0$ and $f^i = 0$ for $i < 0$ and take $f^0\colon A^0 \to I^0$ to be the morphism obtained from the assumption about $\A$ having enough injectives.
    \[\begin{tikzcd}[column sep = 3em, row sep = 2.7em]
        % https://q.uiver.app/#q=WzAsOCxbMSwwLCIwIl0sWzIsMCwiQV4wIl0sWzMsMCwiQV4xIl0sWzQsMCwiXFxjZG90cyJdLFswLDAsIlxcY2RvdHMiXSxbMiwxLCJJXjAiXSxbMSwxLCIwIl0sWzAsMSwiXFxjZG90cyJdLFs0LDBdLFswLDFdLFsxLDJdLFsyLDNdLFsxLDVdLFs3LDZdLFs2LDVdXQ==
        \cdots & 0 & {A^0} & {A^1} & \cdots \\
        \cdots & 0 & {I^0}
        \arrow[from=1-1, to=1-2]
        \arrow[from=1-2, to=1-3]
        \arrow[from=1-3, to=1-4]
        \arrow[from=1-3, to=2-3]
        \arrow[from=1-4, to=1-5]
        \arrow[from=2-1, to=2-2]
        \arrow[from=2-2, to=2-3]
        \arrow[from=1-2, to=2-2]
    \end{tikzcd}\]

    For the induction step suppose we have constructed injective objects $I^i$, with differentials ${\delta^i \colon I^i \to I^{i+1}}$, and morphisms $f^i$ for all $i$ up to $n$, such that the diagram below commutes. 
    \[\begin{tikzcd}[column sep = normal, row sep = 3em]
        % https://q.uiver.app/#q=WzAsOCxbMSwwLCJBXntuLTF9Il0sWzIsMCwiQV5uIl0sWzMsMCwiQV57bisxfSJdLFs0LDAsIlxcY2RvdHMiXSxbMCwwLCJcXGNkb3RzIl0sWzIsMSwiSV5uIl0sWzEsMSwiSV57bi0xfSJdLFswLDEsIlxcY2RvdHMiXSxbNCwwXSxbMCwxLCJkXntuLTF9Il0sWzEsMiwiZF5uIl0sWzIsM10sWzEsNSwiZl57bn0iLDJdLFs3LDZdLFs2LDUsIlxcZGVsdGFee24tMX0iXSxbMCw2LCJmXntuLTF9IiwyXV0=
        \cdots & {A^{n-1}} & {A^n} & {A^{n+1}} & \cdots \\
        \cdots & {I^{n-1}} & {I^n}
        \arrow[from=1-1, to=1-2]
        \arrow["{d^{n-1}}", from=1-2, to=1-3]
        \arrow["{f^{n-1}}"', from=1-2, to=2-2]
        \arrow["{d^n}", from=1-3, to=1-4]
        \arrow["{f^{n}}"', from=1-3, to=2-3]
        \arrow[from=1-4, to=1-5]
        \arrow[from=2-1, to=2-2]
        \arrow["{\delta^{n-1}}", from=2-2, to=2-3]
    \end{tikzcd}\]
    Consider the cokernel of the differential\footnote{Strictly speaking this is not a differential of a complex yet, because we have not specified the chain map $f$ entirely. But this is easily fixed by setting $I^i = 0$ and $f^i$ to be the zero morphism $A^i \to 0$, for $i > n$.} $d^{n-1}_{C(f)}$
    \[
        A^n \oplus I^{n-1} \xrightarrow{\ d^{n-1}_{C(f)} \ } A^{n+1} \oplus I^n \xrightarrow{\ e \ } \coker d^{n-1}_{C(f)} 
    \]
    and denote with $j \colon \coker d^{n-1}_{C(f)} \hookrightarrow I^{n+1}$ an embedding to an injective object $I^{n+1}$.
    %  we get from $\A$ having enough injectives. 
    After precomposing the morphism
    \[
        A^{n+1} \oplus I^n \xrightarrow{\ e \ } \coker d^{n-1}_{C(f)} \xrightarrow{ \ j \ } I^{n+1}
    \]
    with canonical embeddings of $A^{n + 1}$ and $I^{n}$ into their biproduct $A^{n+1} \oplus I^n$, we obtain morphisms $\delta^{n}\colon I^n \to I^{n+1}$ and $f^{n+1}\colon A^{n+1} \to I^{n+1}$. As $j \circ e$ is a morphism fitting into the following coproduct diagram,
    % \[\begin{tikzcd}[row sep = 2em]
    %     % https://q.uiver.app/#q=WzAsNCxbMSwxLCJBXntuKzF9IFxcb3BsdXMgSV5uIl0sWzAsMCwiQV57bisxfSJdLFsyLDAsIklee259Il0sWzEsMywiSV57bisxfSJdLFswLDMsImogXFxjaXJjIGUiLDJdLFsxLDMsImZee24rMX0iLDIseyJjdXJ2ZSI6Mn1dLFsyLDMsIlxcZGVsdGFebiIsMCx7ImN1cnZlIjotMn1dLFsxLDBdLFsyLDBdXQ==
    %     {A^{n+1}} && {I^{n}} \\
    %     & {A^{n+1} \oplus I^n} \\
    %     \\
    %     & {I^{n+1}}
    %     \arrow[from=1-1, to=2-2]
    %     \arrow["{f^{n+1}}"', curve={height=12pt}, from=1-1, to=4-2]
    %     \arrow[from=1-3, to=2-2]
    %     \arrow["{\delta^n}", curve={height=-12pt}, from=1-3, to=4-2]
    %     \arrow["{j \circ e}"', from=2-2, to=4-2]
    % \end{tikzcd}\]
    \[\begin{tikzcd}[column sep = small, row sep = 2em]
        % https://q.uiver.app/#q=WzAsNCxbMSwxLCJBXntuKzF9IFxcb3BsdXMgSV5uIl0sWzAsMSwiQV57bisxfSJdLFsxLDAsIklee259Il0sWzIsMiwiSV57bisxfSJdLFswLDMsImogXFxjaXJjIGUiLDIseyJsYWJlbF9wb3NpdGlvbiI6NDAsInN0eWxlIjp7ImJvZHkiOnsibmFtZSI6ImRhc2hlZCJ9fX1dLFsxLDMsImZee24rMX0iLDIseyJjdXJ2ZSI6Mn1dLFsyLDMsIlxcZGVsdGFebiIsMCx7ImN1cnZlIjotMn1dLFsxLDBdLFsyLDBdXQ==
        & {I^{n}} \\
        {A^{n+1}} & {A^{n+1} \oplus I^n} \\
        && {I^{n+1}}
        \arrow[from=1-2, to=2-2]
        \arrow["{\delta^n}", curve={height=-18pt}, from=1-2, to=3-3]
        \arrow[from=2-1, to=2-2]
        \arrow["{f^{n+1}}"', curve={height=18pt}, from=2-1, to=3-3]
        \arrow["{j \circ e}"'{pos=0.4}, dashed, from=2-2, to=3-3]
    \end{tikzcd}\]
    we see that $j \circ e = \langle f^{n+1} , \delta^n \rangle$. Next, we see that $\delta^n\delta^{n-1} = 0$ and $\delta^n f^n = f^{n+1}d^n$ from the computation
    \[
        0 = j \circ e \circ d^{n-1}_{C(f)} = \langle f^{n+1},\ \delta^n \rangle \begin{pmatrix}
            -d^n & 0 \\ f^n & \delta^{n-1}
        \end{pmatrix} = \langle \delta^n f^n - f^{n+1}d^n , \ \delta^n\delta^{n-1}\rangle\\
    \]
    Thus far we have constructed a chain complex $I^\bullet$ of injective objects and a chain map ${f \colon A^\bullet \to I^\bullet}$. It remanis to be shown that $f$ is a quasi-isomorphism. Recall, that $f$ is a quasi-isomorphism if and only if its cone $C(f)^\bullet$ is acyclic. By Remark \ref{versions of cohomology}, we known
    \[
        H^n(C(f)^\bullet) = \ker(\coker d^{n-1}_{C(f)} \xrightarrow{\ d \ } A^{n+2} \oplus I^{n+1}),
    \]
    where $d$ is obtained from the universal property of $\coker d^{n-1}_{C(f)}$, induced by $d^n_{C(f)}$. Thus it suffices to show, that $d$ is a monomorphism. Consider the following diagram, which we will show to commute. 
    \[\begin{tikzcd}[column sep = 2em]
        % https://q.uiver.app/#q=WzAsNSxbMCwwLCJBXm5cXG9wbHVzIElee24tMX0iXSxbMiwwLCJBXntuKzF9XFxvcGx1cyBJXntufSJdLFs0LDAsIkFee24rMn1cXG9wbHVzIElee24rMX0iXSxbMywxLCJjb2tlciBkXntuLTF9X3tDKGYpfSJdLFs1LDEsIklee24rMX0iXSxbMCwxLCJkXntuLTF9X3tDKGYpfSJdLFsxLDIsImRee259X3tDKGYpfSJdLFsxLDMsImUiXSxbMywyLCJkIiwwLHsic3R5bGUiOnsiYm9keSI6eyJuYW1lIjoiZGFzaGVkIn19fV0sWzIsNCwicCJdLFszLDQsImoiXV0=
        {A^n\oplus I^{n-1}} && {A^{n+1}\oplus I^{n}} && {A^{n+2}\oplus I^{n+1}} \\
        &&& {\coker d^{n-1}_{C(f)}} && {I^{n+1}}
        \arrow["{d^{n-1}_{C(f)}}", from=1-1, to=1-3]
        \arrow["{d^{n}_{C(f)}}", from=1-3, to=1-5]
        \arrow["e", two heads, from=1-3, to=2-4]
        \arrow["p", from=1-5, to=2-6]
        \arrow["d", dashed, from=2-4, to=1-5]
        \arrow["j", hook, from=2-4, to=2-6]
    \end{tikzcd}\]
    Once we show, that its right most triangle is commutative, we see that $d$ is monomorphic, because $j$ is monomorphic. To see that the triangle in question commutes, we compute
    \[
        p \circ d \circ e = p \circ d^n_{C(f)} = \langle f^{n+1}, \delta^n \rangle = j \circ e.
    \] 
    As $e$ is epic, we may cancel it on the right, to arrive at $p \circ d = j$, which ends the proof.
\end{proof}

\begin{remark}
    As is evident from the proof by close inspection we have only used two facts about the class of all injective objects of $\A$ -- that every object of $\A$ embeds into some injective object and that the class of injectives is closed under finite direct sums. This will later on be used in subsection \ref{Subsection on F-adapted classes} when constructing a right derived functor of a given functor $F$ in the presence of an $F$-adapted class.
\end{remark}

\begin{theorem}
    \label{D+ = K+ and injectives}
    Assume $\A$ contains enough injectives and let $\J \subseteq \A$ denote the full subcategory on injecitve objects of $\A$. Then the inclusion $\kplus{\J} \hookrightarrow \kplus{\A}$ induces an equivalence of triangulated categories
    \[
        \kplus{\J} \iso \derplus{\A}.
    \]
\end{theorem}

\begin{remark}
    Theorem \ref{D+ = K+ and injectives} in particular shows that $\derplus{\A}$ is a category in the usual sense, \ie all its hom-sets are in fact sets.  
    \info{maybe I replace "usual" with locally small everywhere this comes up.}
\end{remark}

\begin{lemma}
    \label{acyclic to injective is nulhomotopic}
    Let $A^\bullet$ be any acyclic complex in $\kplus{\A}$ and $I^\bullet$ a complex of injectives from $\kplus{\J}$. Then
    \[
        \hom_{\kplus{\A}}(A^\bullet, I^\bullet) = 0.
    \]
    In other words every morphism from an acyclic complex to an injective one is nulhomotopic.
\end{lemma}

\begin{proof}
    We will show that $f \htpy 0$ for any chain map $f \colon A^\bullet \to I^\bullet$ by inductively constructing an appropriate homotopy $h$. For simplicity assume $A^i = 0$ and $I^i = 0$ for $i < 0$ and define $h^i \colon A^i \to I^{i-1}$ to be $0$ for $i \leq 0$. 
    
    As $A^\bullet$ is acyclic, the first non-trivial differential $d^0\colon A^0 \to A^1$ is mono. From $I^0$ being injective, we obtain a morphism $h^1\colon A^1 \to I^0$, satisfying
    \[
        f^0 = h^1 \circ d^0.
    \]
    Note that the above can actually be rewritten to $f^0 = h^1d^0 + \delta^{-1}h^0$, as $h^0 = 0$.

    For the induction step assume that we have already constructed $h^i \colon A^i \to I^{i-1}$, with $f^{i-1} = h^id^{i-1} + \delta^{i-2}h^{i-1}$ for all $i \leq n$. We are aiming to construct $h^{n+1} \colon A^{n+1} \to I^n$, for which $f^n = h^{n+1} d^n + \delta^{n-1}h^n$ holds. First, expand the differential $d^n$ into a composition of the canonical epimorphism $e \colon A^n \to \coker d^{n-1}$, followed by $j \colon \coker d^{n-1} \to A^{n+1}$ induced by the universal property of $\coker d^{n-1}$ by the map $d^n \colon A^n \to A^{n+1}$. Morphism $j$ is actually a monomorphism by acyclicity of $A^\bullet$, since we know that
    \[
        0 = H^n(A^\bullet) \iso \ker(j \colon \coker d^{n-1} \to A^{n+1}).
    \]
    Next, we see that by the universal property of $\coker d^{n-1}$, morphism $f^n - \delta^{n-1}h^n \colon A^n \to I^n$ induces a morphism ${g \colon \coker d^{n-1} \to I^n}$, because
    \begin{align*}
        (f^n - \delta^{n-1}h^n)d^{n-1} &= f^nd^{n-1} - \delta^{n-1}h^nd^{n-1} \\
        &= f^nd^{n-1} + \delta^{n-1}\delta^{n-2}h^{n-1} - \delta^{n-1}f^{n-1} \\
        &= 0.
    \end{align*}
    The second equality follows from the inductive hypothesis and the third one from $f$ being a chain map and $\delta^{n-1}$, $\delta^{n-2}$ being a differentials. 

    Lastly, for $I^n$ is injecive and $j$ mono, there is a morphism $h^{n+1} \colon A^{n+1} \to I^n$, satisfying $h^{n+1} \circ j = g$, thus after precomposing both sides with $e$, we arrive at $h^{n+1}d^n = f^n - \delta^{n-1}h^n$, which can be rewritten as
    \[
        f^n = h^{n+1}d^n + \delta^{n-1}h^n.
        \qedhere  
    \]
\end{proof}

\begin{lemma}
    \label{D+ K+ aux lemma 1}
    % Let $A^\bullet$ and $B^\bullet$ be complexes belonging to $\kplus{\A}$ and $I^\bullet$ a complex of injectives from $\kplus{\J}$.
    Let $A^\bullet$ and $B^\bullet$ belong to $\kplus{\A}$ and $I^\bullet \in \kplus{\J}$. Let $f \colon B^\bullet \to A^\bullet$ be a quasi-isomorphism, then 
    \[
        \Hom_{\kplus{\A}}(A^\bullet, I^\bullet) \xrightarrow{ \ f^* \ } \Hom_{\kplus{\A}}(B^\bullet, I^\bullet) 
    \]
    is an isomorphism of $k$-modules.
\end{lemma}

\begin{proof}
    In $\kplus{\A}$ we have a distinguished triangle $B^\bullet \to A^\bullet \to C(f)^\bullet \to B[1]^\bullet$, which induces a long exact sequence (of $k$-modules)
    \begin{multline*}
        \cdots \longrightarrow \Hom_{\kplus{\A}}(C(f)^\bullet, I^\bullet) \longrightarrow \\
        \longrightarrow \Hom_{\kplus{\A}}(A^\bullet, I^\bullet) \xrightarrow{ \ f^* \ }
        \Hom_{\kplus{\A}}(B^\bullet, I^\bullet) \longrightarrow \\
        \longrightarrow \Hom_{\kplus{\A}}(C(f)[-1]^\bullet, I^\bullet) \longrightarrow \cdots,
    \end{multline*}
    since $\Hom_{\kplus{\A}}(-, I^\bullet)$ is a cohomological functor by example \ref{partial homs are cohomological functors}. As $f$ is a quasi-isomorphism, the cone $C(f)^\bullet$ is acyclic along with all its shifts, thus showing that
    \[
        \Hom_{\kplus{\A}}(C(f)[i]^\bullet, I^\bullet) = 0,
    \]
    for all $i \in \Z$ by Lemma \ref{acyclic to injective is nulhomotopic}, since $I^\bullet$ is a complex of injectives. From the long exact sequence it then clearly follows that $f^*$ is an isomorphism. 
\end{proof}

\begin{lemma}
    \label{D+ K+ aux lemma 2}
    Let $A^\bullet$ belong to $\kplus{\A}$ and $I^\bullet$ to $\kplus{\J}$. Then the morphism action 
    \begin{equation}
        \label{eq: morphism action of Q}
        \Hom_{\kplus{\A}}(A^\bullet, I^\bullet) \to \Hom_{\derplus{\A}}(A^\bullet, I^\bullet)
    \end{equation}
    of the localization functor $Q$ is an isomorphism of $k$-modules. 
\end{lemma}

\begin{proof}
    We already know this is a $k$-module homomorphism, thus it is enough to show that it is bijective. This is accomplished by constructing its inverse. Let $\phi \colon A^\bullet \to I^\bullet$ be a morphism in $\derplus{\A}$ and let $A^\bullet \xleftarrow{ \ s \ } B^\bullet \xrightarrow{ \ f \ } I^\bullet$ be its left roof representative. The morphism $s$ being a quasi-isomorphism implies that $s^* \colon \Hom_{\kplus{\A}}(A^\bullet, I^\bullet) \to \Hom_{\kplus{\A}}(B^\bullet, I^\bullet)$ is bijective by lemma \ref{D+ K+ aux lemma 1}, so there is a unique morphism $g\colon A^\bullet \to I^\bullet$, such that $f = g \circ s$ in $\kplus{\A}$. The inverse to \eqref{eq: morphism action of Q} is then defined by sending $\phi$ to $g$. 
    
    First let's argue why this map is well defined \ie independent of the choice of left roof representative for $\phi$. We pick two left roof representatives $A^\bullet \xleftarrow{ \ s_0 \ } B_0^\bullet \xrightarrow{ \ f_0 \ } I^\bullet$ and $A^\bullet \xleftarrow{ \ s_1 \ } B_1^\bullet \xrightarrow{ \ f_1 \ } I^\bullet$ for $\phi$ and let $g_0$ and $g_1$ be such that $f_i = g_i \circ s_i$ for both $i$. As the roofs are equivalent there are quasi-isomorphisms $t_0 \colon C^\bullet \to B^\bullet_0$ and $t_1 \colon C^\bullet \to B^\bullet_1$, which fit into the following commutative diagram.
    \[\begin{tikzcd}[row sep = small, column sep = small]
        % https://q.uiver.app/#q=WzAsNSxbMCwyLCJBIl0sWzEsMSwiQl8wIl0sWzQsMiwiSSJdLFszLDEsIkJfMSJdLFsyLDAsIkMiXSxbMSwyLCJmXzAiLDIseyJsYWJlbF9wb3NpdGlvbiI6NjB9XSxbMywwLCJzXzEiLDAseyJsYWJlbF9wb3NpdGlvbiI6NjB9XSxbMywyLCJmXzEiXSxbNCwzLCJ0XzEiXSxbMSwwLCJzXzAiLDJdLFs0LDEsInRfMCIsMl1d
        && C^\bullet \\
        & {B^\bullet_0} && {B^\bullet_1} \\
        A^\bullet &&&& I^\bullet
        \arrow["{t_0}"', from=1-3, to=2-2]
        \arrow["{t_1}", from=1-3, to=2-4]
        \arrow["{s_0}"', from=2-2, to=3-1]
        \arrow["{s_1}"{pos=0.6}, from=2-4, to=3-1]
        \arrow["{f_1}", from=2-4, to=3-5]
        \arrow["{f_0}"'{pos=0.6}, from=2-2, to=3-5, crossing over]
    \end{tikzcd}\]
    Therefore $g_0s_0t_0 = f_0t_0 = f_1t_1 = g_1s_1t_1 = g_1s_0t_0$. As $s_0t_0$ is a quasi-isomorphism, we see that $g_0 = g_1$ by applying lemma \ref{D+ K+ aux lemma 1}.
    %  follows by applying lemma \ref{D+ K+ aux lemma 1} as the composition $s_0t_0$ is a quasi-isomorphism.
    
    % The following diagram shows why the constructed map is a left inverse to \eqref{eq: morphism action of Q}
    % \[
    %     \left[ A^\bullet \xleftarrow{ \ s \ } B^\bullet \xrightarrow{ \ f \ } I^\bullet \right] \longmapsto 
    %     \left( g\colon A^\bullet \to I^\bullet\right) \longmapsto 
    %     \left[A^\bullet \xleftarrow{ \ \id{A^\bullet} } A^\bullet \xrightarrow{ \ g \ } I^\bullet\right] = 
    %     \left[ A^\bullet \xleftarrow{ \ s \ } B^\bullet \xrightarrow{ \ f \ } I^\bullet \right]
    % \]

    % \begin{align*}
    %     \Hom_{\derplus{\A}}(A^\bullet, I^\bullet) \to \Hom_{\kplus{\J}}(A^\bullet, I^\bullet) \to \Hom_{\derplus{\A}}(A^\bullet, I^\bullet) \\
    %     \left[ A^\bullet \xleftarrow{ \ s \ } B^\bullet \xrightarrow{ \ f \ } I^\bullet \right] \longmapsto 
    %     \left( g\colon A^\bullet \to I^\bullet\right) \longmapsto 
    %     \left[A^\bullet \xleftarrow{ \ \id{A^\bullet} } A^\bullet \xrightarrow{ \ g \ } I^\bullet\right] = 
    %     \left[ A^\bullet \xleftarrow{ \ s \ } B^\bullet \xrightarrow{ \ f \ } I^\bullet \right]
    % \end{align*}

%     \[\begin{tikzcd}[column sep = small , row sep = small]
%         % https://q.uiver.app/#q=WzAsNixbMCwwLCJcXEhvbV97RF4rKFxcQSl9KEFeXFxidWxsZXQsIEleXFxidWxsZXQpIl0sWzQsMCwiXFxIb21fe0ReKyhcXEEpfShBXlxcYnVsbGV0LCBJXlxcYnVsbGV0KSJdLFsyLDAsIlxcSG9tX3tLXisoXFxKKX0oQV5cXGJ1bGxldCwgSV5cXGJ1bGxldCkiXSxbMCwxLCJcXGxlZnRbIEFeXFxidWxsZXQgXFx4bGVmdGFycm93eyBcXCBzIFxcIH0gQl5cXGJ1bGxldCBcXHhyaWdodGFycm93eyBcXCBmIFxcIH0gSV5cXGJ1bGxldCBcXHJpZ2h0XSJdLFsyLDEsIlxcbGVmdCggZ1xcY29sb24gQV5cXGJ1bGxldCBcXHRvIEleXFxidWxsZXRcXHJpZ2h0KSJdLFs0LDEsIlxcbGVmdFtBXlxcYnVsbGV0IFxceGxlZnRhcnJvd3sgXFwgXFxpZHtBXlxcYnVsbGV0fSB9IEFeXFxidWxsZXQgXFx4cmlnaHRhcnJvd3sgXFwgZyBcXCB9IEleXFxidWxsZXRcXHJpZ2h0XSA9ICAgICAgICAgIFxcbGVmdFsgQV5cXGJ1bGxldCBcXHhsZWZ0YXJyb3d7IFxcIHMgXFwgfSBCXlxcYnVsbGV0IFxceHJpZ2h0YXJyb3d7IFxcIGYgXFwgfSBJXlxcYnVsbGV0IFxccmlnaHRdIl0sWzAsMl0sWzIsMV0sWzMsNCwiIiwyLHsic3R5bGUiOnsidGFpbCI6eyJuYW1lIjoibWFwcyB0byJ9fX1dLFs0LDUsIiIsMix7InN0eWxlIjp7InRhaWwiOnsibmFtZSI6Im1hcHMgdG8ifX19XV0=
% 	{\Hom_{\derplus{\A}}(A^\bullet, I^\bullet)} && {\Hom_{\kplus{\J}}(A^\bullet, I^\bullet)} && {\Hom_{\derplus{\A}}(A^\bullet, I^\bullet)} \\
% 	{\left[ A^\bullet \xleftarrow{ \ s \ } B^\bullet \xrightarrow{ \ f \ } I^\bullet \right]} && {\left( g\colon A^\bullet \to I^\bullet\right)} && {\left[A^\bullet \xleftarrow{ \ \id{A^\bullet} } A^\bullet \xrightarrow{ \ g \ } I^\bullet\right] =          \left[ A^\bullet \xleftarrow{ \ s \ } B^\bullet \xrightarrow{ \ f \ } I^\bullet \right]}
% 	\arrow[from=1-1, to=1-3]
% 	\arrow[from=1-3, to=1-5]
% 	\arrow[maps to, from=2-1, to=2-3]
% 	\arrow[maps to, from=2-3, to=2-5]
% \end{tikzcd}\]

    % and why it is also a right inverse to \eqref{eq: morphism action of Q}
    % \[
    %     \left(  g\colon A^\bullet \to I^\bullet \right) \longmapsto
    %     \left[A^\bullet \xleftarrow{ \ \id{A^\bullet} } A^\bullet \xrightarrow{ \ g \ } I^\bullet\right] \longmapsto
    %     \left(  g\colon A^\bullet \to I^\bullet \right).
    %     \qedhere
    % \]


    % https://q.uiver.app/#q=WzAsNyxbMCwwLCJcXEhvbV97RF4rKFxcQSl9KEFeXFxidWxsZXQsIEleXFxidWxsZXQpIl0sWzIsMCwiXFxIb21fe0ReKyhcXEEpfShBXlxcYnVsbGV0LCBJXlxcYnVsbGV0KSJdLFsxLDAsIlxcSG9tX3tLXisoXFxKKX0oQV5cXGJ1bGxldCwgSV5cXGJ1bGxldCkiXSxbMCwxLCJcXGxlZnRbIEFeXFxidWxsZXQgXFx4bGVmdGFycm93eyBcXCBzIFxcIH0gQl5cXGJ1bGxldCBcXHhyaWdodGFycm93eyBcXCBmIFxcIH0gSV5cXGJ1bGxldCBcXHJpZ2h0XSJdLFsxLDEsIlxcbGVmdCggZ1xcY29sb24gQV5cXGJ1bGxldCBcXHRvIEleXFxidWxsZXRcXHJpZ2h0KSJdLFsyLDEsIlxcbGVmdFtBXlxcYnVsbGV0IFxceGxlZnRhcnJvd3sgXFwgXFxpZHtBXlxcYnVsbGV0fSB9IEFeXFxidWxsZXQgXFx4cmlnaHRhcnJvd3sgXFwgZyBcXCB9IEleXFxidWxsZXRcXHJpZ2h0XSJdLFszLDEsIlxcbGVmdFsgQV5cXGJ1bGxldCBcXHhsZWZ0YXJyb3d7IFxcIHMgXFwgfSBCXlxcYnVsbGV0IFxceHJpZ2h0YXJyb3d7IFxcIGYgXFwgfSBJXlxcYnVsbGV0IFxccmlnaHRdIl0sWzAsMl0sWzIsMV0sWzMsNCwiIiwyLHsic3R5bGUiOnsidGFpbCI6eyJuYW1lIjoibWFwcyB0byJ9fX1dLFs0LDUsIiIsMix7InN0eWxlIjp7InRhaWwiOnsibmFtZSI6Im1hcHMgdG8ifX19XSxbNSw2LCIiLDIseyJsZXZlbCI6Miwic3R5bGUiOnsiaGVhZCI6eyJuYW1lIjoibm9uZSJ9fX1dXQ==
% \[\begin{tikzcd}[row sep = 0.5em, column sep = 0.75em]
% 	{\Hom_{\derplus{\A}}(A^\bullet, I^\bullet)} & {\Hom_{\kplus{\J}}(A^\bullet, I^\bullet)} & {\Hom_{\derplus{\A}}(A^\bullet, I^\bullet)} \\
% 	{\left[ A^\bullet \xleftarrow{ \ s \ } B^\bullet \xrightarrow{ \ f \ } I^\bullet \right]} & {\left( g\colon A^\bullet \to I^\bullet\right)} & {\left[A^\bullet \xleftarrow{ \ \id{A^\bullet} } A^\bullet \xrightarrow{ \ g \ } I^\bullet\right]} & {\left[ A^\bullet \xleftarrow{ \ s \ } B^\bullet \xrightarrow{ \ f \ } I^\bullet \right]}
% 	\arrow[from=1-1, to=1-2]
% 	\arrow[from=1-2, to=1-3]
% 	\arrow[maps to, from=2-1, to=2-2]
% 	\arrow[maps to, from=2-2, to=2-3]
% 	\arrow[equals, from=2-3, to=2-4]
% \end{tikzcd}\]

% \[\begin{tikzcd}[row sep = 0.5em, column sep = 0.75em]
%     % https://q.uiver.app/#q=WzAsNixbMCwwLCJcXEhvbV97RF4rKFxcQSl9KEFeXFxidWxsZXQsIEleXFxidWxsZXQpIl0sWzIsMCwiXFxIb21fe0ReKyhcXEEpfShBXlxcYnVsbGV0LCBJXlxcYnVsbGV0KSJdLFsxLDAsIlxcSG9tX3tLXisoXFxKKX0oQV5cXGJ1bGxldCwgSV5cXGJ1bGxldCkiXSxbMCwxLCJcXGxlZnRbIEFeXFxidWxsZXQgXFx4bGVmdGFycm93eyBcXCBzIFxcIH0gQl5cXGJ1bGxldCBcXHhyaWdodGFycm93eyBcXCBmIFxcIH0gSV5cXGJ1bGxldCBcXHJpZ2h0XSJdLFsxLDEsIlxcbGVmdCggZ1xcY29sb24gQV5cXGJ1bGxldCBcXHRvIEleXFxidWxsZXRcXHJpZ2h0KSJdLFsyLDEsIlxcbGVmdFtBXlxcYnVsbGV0IFxceGxlZnRhcnJvd3sgXFwgXFxpZHtBXlxcYnVsbGV0fSB9IEFeXFxidWxsZXQgXFx4cmlnaHRhcnJvd3sgXFwgZyBcXCB9IEleXFxidWxsZXRcXHJpZ2h0XSJdLFswLDJdLFsyLDFdLFszLDQsIiIsMix7InN0eWxlIjp7InRhaWwiOnsibmFtZSI6Im1hcHMgdG8ifX19XSxbNCw1LCIiLDIseyJzdHlsZSI6eyJ0YWlsIjp7Im5hbWUiOiJtYXBzIHRvIn19fV1d
% 	{\Hom_{\derplus{\A}}(A^\bullet, I^\bullet)} & {\Hom_{\kplus{\J}}(A^\bullet, I^\bullet)} & {\Hom_{\derplus{\A}}(A^\bullet, I^\bullet)} \\
% 	{\left[ A^\bullet \xleftarrow{ \ s \ } B^\bullet \xrightarrow{ \ f \ } I^\bullet \right]} & {\left( g\colon A^\bullet \to I^\bullet\right)} & {\left[A^\bullet \xleftarrow{ \ \id{A^\bullet} } A^\bullet \xrightarrow{ \ g \ } I^\bullet\right]}
% 	\arrow[from=1-1, to=1-2]
% 	\arrow[from=1-2, to=1-3]
% 	\arrow[maps to, from=2-1, to=2-2]
% 	\arrow[maps to, from=2-2, to=2-3]
% \end{tikzcd}\]


Following the diagram below, it is clear why the constructed map is a right inverse to \eqref{eq: morphism action of Q} 
\[\begin{tikzcd}[row sep = 0.5em, column sep = 0.75em]
    % https://q.uiver.app/#q=WzAsNixbMSwwLCJcXEhvbV97RF4rKFxcQSl9KEFeXFxidWxsZXQsIEleXFxidWxsZXQpIl0sWzIsMCwiXFxIb21fe0teKyhcXEopfShBXlxcYnVsbGV0LCBJXlxcYnVsbGV0KSJdLFsyLDEsIlxcbGVmdCggZ1xcY29sb24gQV5cXGJ1bGxldCBcXHRvIEleXFxidWxsZXRcXHJpZ2h0KSJdLFswLDAsIlxcSG9tX3tLXisoXFxKKX0oQV5cXGJ1bGxldCwgSV5cXGJ1bGxldCkiXSxbMCwxLCJcXGxlZnQoIGdcXGNvbG9uIEFeXFxidWxsZXQgXFx0byBJXlxcYnVsbGV0XFxyaWdodCkiXSxbMSwxLCJcXGxlZnRbQV5cXGJ1bGxldCBcXHhsZWZ0YXJyb3d7IFxcIFxcaWR7QV5cXGJ1bGxldH0gfSBBXlxcYnVsbGV0IFxceHJpZ2h0YXJyb3d7IFxcIGcgXFwgfSBJXlxcYnVsbGV0XFxyaWdodF0iXSxbMCwxXSxbNCw1LCIiLDIseyJzdHlsZSI6eyJ0YWlsIjp7Im5hbWUiOiJtYXBzIHRvIn19fV0sWzUsMiwiIiwyLHsic3R5bGUiOnsidGFpbCI6eyJuYW1lIjoibWFwcyB0byJ9fX1dLFszLDBdXQ==
	{\Hom_{\kplus{\A}}(A^\bullet, I^\bullet)} & {\Hom_{\derplus{\A}}(A^\bullet, I^\bullet)} & {\Hom_{\kplus{\A}}(A^\bullet, I^\bullet)} \\
	{\left( g\colon A^\bullet \to I^\bullet\right)} & {\left[A^\bullet \xleftarrow{ \ \id{A^\bullet} } A^\bullet \xrightarrow{ \ g \ } I^\bullet\right]} & {\left( g\colon A^\bullet \to I^\bullet\right).}
	\arrow["Q", from=1-1, to=1-2]
	\arrow[from=1-2, to=1-3]
	\arrow[maps to, from=2-1, to=2-2]
	\arrow[maps to, from=2-2, to=2-3]
\end{tikzcd}\]
Lastly we see that it is also a left inverse by the following diagram \[\begin{tikzcd}[row sep = 0.5em, column sep = 0.75em]
    % https://q.uiver.app/#q=WzAsNixbMCwwLCJcXEhvbV97RF4rKFxcQSl9KEFeXFxidWxsZXQsIEleXFxidWxsZXQpIl0sWzIsMCwiXFxIb21fe0ReKyhcXEEpfShBXlxcYnVsbGV0LCBJXlxcYnVsbGV0KSJdLFsxLDAsIlxcSG9tX3tLXisoXFxKKX0oQV5cXGJ1bGxldCwgSV5cXGJ1bGxldCkiXSxbMCwxLCJcXGxlZnRbIEFeXFxidWxsZXQgXFx4bGVmdGFycm93eyBcXCBzIFxcIH0gQl5cXGJ1bGxldCBcXHhyaWdodGFycm93eyBcXCBmIFxcIH0gSV5cXGJ1bGxldCBcXHJpZ2h0XSJdLFsxLDEsIlxcbGVmdCggZ1xcY29sb24gQV5cXGJ1bGxldCBcXHRvIEleXFxidWxsZXRcXHJpZ2h0KSJdLFsyLDEsIlxcbGVmdFtBXlxcYnVsbGV0IFxceGxlZnRhcnJvd3sgXFwgXFxpZHtBXlxcYnVsbGV0fSB9IEFeXFxidWxsZXQgXFx4cmlnaHRhcnJvd3sgXFwgZyBcXCB9IEleXFxidWxsZXRcXHJpZ2h0XSJdLFswLDJdLFsyLDFdLFszLDQsIiIsMix7InN0eWxlIjp7InRhaWwiOnsibmFtZSI6Im1hcHMgdG8ifX19XSxbNCw1LCIiLDIseyJzdHlsZSI6eyJ0YWlsIjp7Im5hbWUiOiJtYXBzIHRvIn19fV1d
	{\Hom_{\derplus{\A}}(A^\bullet, I^\bullet)} & {\Hom_{\kplus{\A}}(A^\bullet, I^\bullet)} & {\Hom_{\derplus{\A}}(A^\bullet, I^\bullet)} \\
	{\left[ A^\bullet \xleftarrow{ \ s \ } B^\bullet \xrightarrow{ \ f \ } I^\bullet \right]} & {\left( g\colon A^\bullet \to I^\bullet\right)} & {\left[A^\bullet \xleftarrow{ \ \id{A^\bullet} } A^\bullet \xrightarrow{ \ g \ } I^\bullet\right]}
	\arrow[from=1-1, to=1-2]
	\arrow["Q", from=1-2, to=1-3]
	\arrow[maps to, from=2-1, to=2-2]
	\arrow[maps to, from=2-2, to=2-3]
\end{tikzcd}\]
along with observing that $A^\bullet \xleftarrow{ \ \id{A^\bullet} } A^\bullet \xrightarrow{ \ g \ } I^\bullet$ and $A^\bullet \xleftarrow{ \ s \ } B^\bullet \xrightarrow{ \ f \ } I^\bullet$ are equivalent roofs.
\end{proof}

\begin{proof}[Proof of theorem \ref{D+ = K+ and injectives}]
    % We start of with a

    % \noindent
    % % \textbf{Preliminary claim.} Let $A^\bullet$ and $B^\bullet$ be complexes belonging to $\kplus{\A}$ and $I^\bullet$ a complex of injectives from $\kplus{\J}$.
    % \textbf{Preliminary claim.} Let $A^\bullet$ and $B^\bullet$ belong to $\kplus{\A}$ and $I^\bullet \in \kplus{\J}$. Let $f \colon B^\bullet \to A^\bullet$ be a quasi-isomorphism, then 
    % \[
    %     \Hom_{\kplus{\A}}(A^\bullet, I^\bullet) \xrightarrow{ \ - \circ f \ } \Hom_{\kplus{\A}}(B^\bullet, I^\bullet) 
    % \]
    % is a bijection.
    As $\kplus{\J} \to \derplus{\A}$ is a triangulated functor, we only need to show that it is fully faithful and essentially surjective. Let $I^\bullet$ and $J^\bullet$ be objects of $\kplus{\J}$. Then
    \[
        \Hom_{\kplus{\J}}(I^\bullet, J^\bullet) \xrightarrow{\ = \ } \Hom_{\kplus{\A}}(I^\bullet, J^\bullet) \xrightarrow{\ \eqref{D+ K+ aux lemma 2} \ } \Hom_{\derplus{\A}}(I^\bullet, J^\bullet)
    \]
    is a bijection, showing fully faithfulness (the last map is a bijection by Lemma \ref{D+ K+ aux lemma 2}). Essential surjectivity is clear from the existence of injective resolutions (\cf Proposition \ref{injective resolution}) because quasi-isomorphisms now play the role of isomorphisms in $\derplus{\A}$.
\end{proof}

Within the scope of an abelian category $\A$ with enough injectives, theorem \ref{D+ = K+ and injectives} now offers some well known and classical results of homological algebra. 

\begin{corollary}
    Let $\A$ be an abelian category with enough injectives.
    \begin{enumerate}[label = (\roman*)]
        \item{Any two injective resolutions of a complex $A^\bullet$ of $\kplus{\A}$ are homotopically equivalent \ie isomorphic in $\kplus{\A}$}
        \item{For any morphism of complexes $f \colon A^\bullet \to B^\bullet$ in $\kplus{\A}$ and any injective resolutions $A^\bullet \to I^\bullet_A$ and $B^\bullet \to I^\bullet_B$ of $A^\bullet$ and $B^\bullet$ respectively, there is a unique up to homotopy chain map $I^\bullet_A \to I^\bullet_B$ for which the square below commutes up to homotopy.
        \[\begin{tikzcd}[row sep = normal]
            % https://q.uiver.app/#q=WzAsNCxbMiwwLCJBXlxcYnVsbGV0Il0sWzIsMiwiQl5cXGJ1bGxldCJdLFswLDAsIkleXFxidWxsZXRfQSJdLFswLDIsIkleXFxidWxsZXRfQiJdLFswLDEsImYiXSxbMCwyXSxbMSwzXSxbMiwzLCIiLDAseyJzdHlsZSI6eyJib2R5Ijp7Im5hbWUiOiJkYXNoZWQifX19XV0=
            {I^\bullet_A} && {A^\bullet} \\
            \\
            {I^\bullet_B} && {B^\bullet}
            \arrow[dashed, from=1-1, to=3-1]
            \arrow[from=1-3, to=1-1]
            \arrow["f", from=1-3, to=3-3]
            \arrow[from=3-3, to=3-1]
        \end{tikzcd}\]
        }
    \end{enumerate}
\end{corollary}

% \info{add a little argument for why injective res. are unique up to htpy.}

% We leave out the proof for why this operation is associative, but refer the intersted reader to \cite{milicic-dercat} or \cite[\S 7]{kashiwara2006categories} for a more detailed and general approach astablishing the theory of localization of categories.
% \begin{remark}
%     This lemma reminds us of the Ore condition in the context of localization in non-commutative algebra as so analogously there is also a dual version of the lemma. 
% \end{remark}



% ============================================================== %

\subsection{Derived functors}

The main goal of this section is to assign to a sensible additive functor $F \colon \A \to \B$ between two abelian categories an appropriate \emph{triangulated} functor on the level of derived categories, called a \emph{derived functor}. We will first deal with the easiest case, when $F$ is exact and then move on to the case, where $F$ will only be assumed to be left or right exact and instead require the domain category $\A$ to posses some nice properties (\eg contain enough injectives or contain an $F$-adapted class of objects). In practise the additional conditions $\A$ is required to satisfy are not very restrictive. 

\subsubsection{Derived functors of exact functors}
\label{subsection: Derived functors of exact functors}

We start with some establishing lemmas.

\begin{lemma}
    \label{qis iff cone acyclic}
    Let $f\colon A^\bullet \to B^\bullet$ be a chain map. Then $f$ is a quasi-isomorphism if and only if its associated cone $C(f)^\bullet$ is acyclic. 
\end{lemma}

\info{proof can reference a proposition from section on triangulated cats...}

\begin{proof}
    One only needs to consider the distinguished triangle $A^\bullet \xrightarrow{f} B^\bullet \to C(f)^\bullet \to A[1]^\bullet$ in $\kcat{\A}$ and its associated long exact sequence in cohomology.
\end{proof}

\begin{lemma}
    \label{F preserves qis iff preserves acyclic}
    Let $F\colon \kcat{\A} \to \kcat{\B}$ be a triangulated functor, then the following two conditions are equivalent. 
    % then $F$ preserves quasi-isomorphisms if and only if it preserves acyclic complexes.
    % \begin{enumerate}
    %     \item[\emph{(i)}] $F$ preserves quasi-isomorphisms
    %     \item[\emph{(ii)}] $F$ preserves acyclic complexes.
    % \end{enumerate}
    \begin{enumerate}[label = (\roman*)]
        \item For every quasi-isomorphism $s$ in $\kcat{\A}$, $F(s)$ is a quasi-isomorphism in $\kcat{\B}$.
        \item For every acyclic complex $A^\bullet$ of $\kcat{\A}$, $F(A^\bullet)$ is acyclic.
    \end{enumerate}
\end{lemma}

\begin{proof}
    $(\Rightarrow)$ Let $A^\bullet$ be an acyclic complex. Then the zero morphism $A^\bullet \to 0$ is a quasi-isomorphism and is by assumption mapped to a quasi-isomorphism $F(A^\bullet) \to 0$ by $F$. This means $F(A^\bullet)$ is acyclic.
    
    $(\Leftarrow)$ Let $f\colon A^\bullet \to B^\bullet$ be a quasi-isomorphism. The image of a distinguished triangle $A^\bullet \to B^\bullet \to C(f)^\bullet \to A^\bullet[1]$ in $\kcat{\A}$ by the triangulated functor $F$ is a distinguished triangle $FA^\bullet \to FB^\bullet \to F(C(f)^\bullet) \to F(A^\bullet)[1]$ in $\kcat{\B}$. By Lemma \ref{qis iff cone acyclic} $C(f)^\bullet$ is acyclic, implying $F(C(f)^\bullet)$ is acyclic by the assumption (ii). Hence again by Lemma \ref{qis iff cone acyclic} $F(f)$ is a quasi-isomorphism.
    % The cone $F(C(f)^\bullet)$ is then acyclic, because $C(f)^\bullet$ was acyclic by lemma \ref{qis iff cone acyclic}, thus showing $Ff$ is a quasi-isomorphism.  
\end{proof}

\begin{proposition}
    \label{Exact functor induces a derived functor}
    Let $F\colon \kcat{\A} \to \kcat{\B}$ be a triangulated functor and assume it satisfies one of the equivalent conditions of Lemma \ref{F preserves qis iff preserves acyclic}. Then there exists a triangulated functor $\dercat{\A} \to \dercat{\B}$ on the level of derived categories, for which the following diagram of triangulated functors commutes.
    \begin{equation}
        \label{eq: exact functor derives}
        \begin{tikzcd}[column sep = 1.5em]
        % https://q.uiver.app/#q=WzAsNCxbMCwwLCJLKFxcQSkiXSxbMiwwLCJLKFxcQikiXSxbMCwyLCJEKFxcQSkiXSxbMiwyLCJEKFxcQikiXSxbMCwxLCJGIl0sWzAsMl0sWzIsM10sWzEsM11d
        {\kcat{\A}} && {\kcat{\B}} \\
        \\
        {\dercat{\A}} && {\dercat{\B}}
        \arrow["F", from=1-1, to=1-3]
        \arrow["{Q_\A}"', from=1-1, to=3-1]
        \arrow["{Q_\B}", from=1-3, to=3-3]
        \arrow[from=3-1, to=3-3]
        \end{tikzcd}
    \end{equation}
\end{proposition}

\begin{proof}
    The claim follows from the universal property of $\dercat{\A}$ (\cf proposition \ref{universal property for D(A) proposition}), as $Q_\B \circ F$ sends quasi-isomorphisms of $\kcat{\A}$ to isomorphisms of $\dercat{\B}$.
\end{proof}

% The induced functor $\dercat{\A} \to \dercat{\B}$ will be called the \emph{derived functor of $F$}, often being denoted with $F$ as well.

Every $k$-linear functor $F\colon \A \to \B$ induces a triangulated functor on the level of homotopy categories $\kcat{\A} \to \kcat{\B}$, defined by the assignment.
\begin{center}
    \begin{tabular}{r l}
        \textsl{Objects:} & $A^\bullet \longmapsto \left(\cdots \to F(A^i) \xrightarrow{F(d^i)} F(A^{i+1}) \to \cdots \right)$. \\
        \textsl{Morphisms:} & $\left(f\colon A^\bullet \to B^\bullet\right) \longmapsto \left(F(f^i) \colon F(A^i) \to F(B^i)\right)_{i \in \Z}$.
    \end{tabular}
    \end{center}
 If moreover $F$ is assumed to be exact, then $\kcat{\A} \to \kcat{\B}$ satisfies the assumptions of lemma \ref{Exact functor induces a derived functor} and thus induces a triangulated functor $\dercat{\A} \to \dercat{\B}$ on the level of derived categories. We call this the \emph{derived functor of $F$} and often denote it with $F$ as well. We emphasise again that this convention will be used only in the case when $F\colon \A \to \B$ is an \emph{exact} functor between abelian categories.


% ============================================================== %

\subsubsection{Derived functors of left (right) exact functors}
% \subsubsection*{$F$-adapted classes}

Assume $F \colon \A \to \B$ is a l\emph{eft} exact functor between abelian categories $\A$ and $\B$. The theory for right exact functors is obtained in parallel by dualizing everything. A crucial downside of functors, which are no longer exact, is that their induced functors on the category of complexes no longer satisfy the two equivalent conditions of Lemma \ref{F preserves qis iff preserves acyclic}, so the same idea with the universal property of Definition \ref{universal property of D(A)}, which worked in the case of exact functors, cannot be applied in this case. Instead we will present two other more restrictive methods of constructing right derived functors, which are the following:
\begin{enumerate}[label = $\triangleright$]
    \item Assuming $\A$ has enough injectives, construct a \emph{right derived functor} 
    \[
        {\rderived{}{F} \colon \derplus{\A} \to \derplus{\B}} 
    \]
    for any left exact functor $F \colon \A \to \B$.
    \item Fix a left exact functor $F \colon \A \to \B$ and assume the category $\A$ contains a so-called \emph{$F$-adapted class} then construct a \emph{right derived functor} $\rderived{}{F} \colon \derplus{\A} \to \derplus{\B}$.
\end{enumerate}
These methods are however no less restrictive than the case of exact functors with our applications in mind.
These methods are however no less restrictive than the case of exact functors, when viewed through the perspective of our applications later on. 
Lastly we note, that the first method is in some sense a special instance of the second one, which will also be explained in this subsection.
% considered in the context of our applications later on.

\vspace{0.3 cm}

\noindent
\textsc{Method 1.} 
\label{method 1}
Let $F\colon \A \to \B$ be a left exact functor between abelian categories with $\A$ having enough injectives. By Theorem \ref{D+ = K+ and injectives} we know that the composition of $\kplus{\J} \to \kplus{\A}$ followed by the localization functor $Q_\A \colon \kplus{\A} \to \derplus{\A}$ is an equivalence of triangulated categories. The \emph{right derived functor of $F$} is then defined to be the composition
\begin{equation}
    \label{eq: construction of RF with injectives}
    \rderived{}{F}\colon \quad \derplus{\A} \xrightarrow{ \ \natiso \ } \kplus{\J} \hookrightarrow \kplus{\A} \xrightarrow{ \ F \ } \kplus{\B} \xrightarrow{\ Q_\B \ } \derplus{\B}
\end{equation}
The functor $\rderived{}{F}$ is clearly triangulated for it is a composition of triangulated functors. Note that in this case the functor $\rderived{}{F}$ does not necessarily make the square \eqref{eq: exact functor derives} commutative.

We will use this construction many times through the course of the thesis, so we reiterate the above definition by providing a simple recipe for computing the image of $\rderived{}{F}$ on complexes $A^\bullet$ of $\derplus{\A}$.   
% a complex $A^\bullet$ via the functor $\rderived{}{F}$.
% and conceptually computing the image of a right derived functor $\rderived{}{F}$ on a complex $A^\bullet$ from $\kplus{\A}$ will be achieved using the next two steps. 
\begin{enumerate}
    \item Pick an injective resolution $A^\bullet \to I_A^\bullet$ of $A^\bullet$.
    \item Apply the functor $F$ to the complex $I^\bullet_A$ term-wise to obtain $F(I^\bullet_A)$. As a complex
    \[
        \rderived{}{F}(A^\bullet) = F(I^\bullet_A).
    \]
\end{enumerate}

\begin{remark}
    \label{Deriving homotopy functors}
    In the slightly more general situation, if one is already given a triangulated functor $F \colon \kplus{\A} \to \kplus{\B}$ at the level of homotopy categories and assumes that $\A$ contains enough injectives, procedure \eqref{eq: construction of RF with injectives} again yields a right derived functor $\rderived{}{F} \colon \derplus{\A} \to \derplus{\B}$. This remark will be esspecially useful when considering the \emph{differentialy graded inner} $\Hom^\bullet$-functor of subsection \ref{subsection: Ext functors}, which does not appear as a functor induced by some left exact functor $\A \to \B$ on the level of abelian categories.
\end{remark}

\begin{remark}
    When considering a right exact functor $F \colon \A \to \B$, we instead assume the category $\A$ contains enough projectives. In this case we use the dualized version of Theorem \ref{D+ = K+ and injectives}, stating that $\kminus{\mathcal P} \natiso \derminus{\A}$, where $\mathcal P$ denotes the full subcategory of $\A$, spanned on projective objects of $\A$. The left derived functor is then defined as the composition
    % \[
    %     \lderived{}{F} \colon \derminus{\A} \to \derminus{\B}.
    % \]
    \[
        \lderived{}{F}\colon \quad \derminus{\A} \xrightarrow{ \ \natiso \ } \kminus{\mathcal P} \hookrightarrow \kminus{\A} \xrightarrow{ \ F \ } \kminus{\B} \xrightarrow{\ Q_\B \ } \derminus{\B}.
    \]
    Again, as a composition of triangulated functors, we see that $\lderived{}{F}$ is triangulated as well.
\end{remark}

\noindent
\textsc{Method 2.}
Unfortunately some of our categories will \emph{not} contain enough injectives, as can already be seen with the category of coherent sheaves $\coherent{X}$ of a scheme $X$ 
% differing from a point.
, which is not a point. 
In this case \textsc{Method 1} of constructing right derived functors will not work. Luckily however, there exists another way of obtaining a right derived functor $\rderived{}{F} \colon \derplus{\A} \to \derplus{\B}$, assigned to a left exact functor $F: \A \to \B$, when $\A$ does not contain enough injectives, but instead possesses a broader and less restrictive class of objects. To this end we first introduce \emph{$F$-adapted classes}.

\begin{definition}
    \label{F-adapted}
    A class of objects $\I_F \subseteq \A$ is \emph{adapted} to a \textit{left exact}\footnote{Dually, a class of objects $\mathcal P_F$ of $\A$ is \emph{adapted} to a \textit{right exact} functor $F \colon \A \to \B$ if it is closed under finite direct sums, every object of $\A$ is a quotient of some object of $\mathcal P_F$ and for every acyclic complex $P^\bullet$ in $\kminus{\A}$, with $P^i \in \mathcal P_F$, its image $F(P^\bullet)$ under $F$ is also acyclic.} functor $F \colon \A \to \B$, if the following three conditions are satisfied.
    \begin{enumerate}[label = (\roman*)]
        \item{$\I_F$ is stable under finite direct sums. \info{I started calling biproducts direct sums now...}} \label{F-adapted i} 
        \item{Every object $A$ of $\A$ \emph{embeds} into some object of $\I_F$, \ie there exists an object $I \in \I_F$ and a monomorphism $A \hookrightarrow I$.} \label{F-adapted ii}
        \item{For every acyclic complex $I^\bullet$ in $\kplus{\A}$, with $I^i \in \I_F$ for all $i \in \Z$, its image $F(I^\bullet)$ under $F$ is also acyclic.} \label{F-adapted iii}
    \end{enumerate}
\end{definition}

\begin{example}
    We observe, that whenever $\A$ contains enough injectives, the class of all injective objects forms an $F$-adapted class for \emph{every} left exact functor ${F\colon \A \to \B}$. Points (i) and (ii) are clearly true for the class of injective objects and point (iii) follows from Lemma \ref{acyclic to injective is nulhomotopic}. Indeed, it tells us that an acyclic complex of injectives $I^\bullet$ is nullhomotopic, which implies $F(I^\bullet)$ is nullhomotopic as well, therefore, in particular, also acyclic.
\end{example}

In the presence of an $F$-adapted class $\I$ we will now construct the right derived functor of $F$. Firstly we upgrade the class of objects $\I$ to a full additive subcategory of $\A$, also denoted by $\I$, having its class of objects be precisely the $F$-adapted class $\I$. 
% This is done by declaring $\hom_\J(X, Y) := \hom_\A(X, Y)$ for all objects $X$, $Y$ of the class $\I$. 
Then extending $F$ term-wise to a functor on the level of homotopy categories, also denoted by $F \colon \kplus{\I} \to \kplus{\A}$, utilizing the procedure outlined at the end of subsection \ref{subsection: Derived functors of exact functors}, we obtain a functor, which maps acyclic complexes to acyclic complexes because by definition the $F$-adapted class $\I$ satisfies condition \ref{F-adapted iii} of Definition \ref{F-adapted}. By Proposition \ref{Exact functor induces a derived functor} the functor $F$ then descends to a well defined functor on the level of derived categories
\begin{equation}
    \label{eq: D+(I) -> D+(B)}
    \derplus{\I} \to \derplus{\B}.
\end{equation}
Since we want to define the derived functor $\rderived{}{F}$, whose domain is $\derplus{\A}$, it remains to construct a functor $\derplus{\A} \to \derplus{\I}$, which we can then post-compose with $\derplus{\I} \to \derplus{\B}$ to obtain $\rderived{}{F}$. What we will show instead is the following.

\begin{proposition}
    \label{D+(I) = D+(A) F-adapted}
    The inclusion of categories $\I \hookrightarrow \A$ induces an equivalence of triangulated categories 
    \[
        \derplus{\I} \iso \derplus{\A}.
    \]
\end{proposition}

The proof of Proposition \ref{D+(I) = D+(A) F-adapted} hinges on the following lemma, closely resembling Lemma \ref{injective resolution}. In the same sprit, we call the complex $I^\bullet$ belonging to $\kplus{\I}$ together with a quasi-isomorphism $A^\bullet \to I^\bullet$ an $\I$-resolution of the complex $A^\bullet$.

\begin{lemma}
    \label{F-acyclic resolutions}
    Every complex $A^\bullet$ of $\kplus{\A}$ has an $\I$-resolution.
    % For every complex $A^\bullet$ in $\kplus{\A}$ there is a quasi-isomorphism $A^\bullet \to I^\bullet$, where $I^\bullet$ is a complex in $\kplus{\I}$.
\end{lemma}

\begin{proof}
    A close inspection of the proof of Proposition \ref{injective resolution} reveals, that formally only properties \ref{F-adapted i} and \ref{F-adapted ii} of Definition \ref{F-adapted}, which the class of all injective objects clearly satisfies, were used. Thus the same proof can be copied for this lemma to be true.
    % By inspecting the proof of Proposition \ref{injective resolution}, we see, that only conditions \ref{F-adapted i} and \ref{F-adapted ii} of Definition \ref{F-adapted} were actually used, so the proof of this lemma is just a copy 
\end{proof}

\begin{proof}[Proof of Proposition \ref{D+(I) = D+(A) F-adapted}]
    First, there exists a triangulated functor $\derplus{\I} \to \derplus{\A}$ by Proposition \ref{Exact functor induces a derived functor}, since the inclusion $\I \hookrightarrow \A$ induces an inclusion $\kplus{\I} \hookrightarrow \kplus{\A}$ sending acyclic complexes to acyclic ones. We claim, that functor $\derplus{\I} \to \derplus{\A}$ is an equivalence, so we will show that it is full, faithful and essentially surjective. 
    
    % Next we show that it is faithful \ie that the morphism action
    % \begin{equation}
    %     \label{eq: morphism action of D(J) - D(A)}
    %     \Hom_{\derplus{\I}}(I^\bullet, J^\bullet) \to \Hom_{\derplus{\A}}(I^\bullet, J^\bullet)
    % \end{equation}

    First, we consider the morphism action 
    \begin{equation}
        \label{eq: morphism action of D(J) - D(A)}
        \Hom_{\derplus{\I}}(I^\bullet, J^\bullet) \to \Hom_{\derplus{\A}}(I^\bullet, J^\bullet)
    \end{equation}
    and show it to be injective for all objects $I^\bullet$, $J^\bullet$ of $\derplus{\I}$.
    % is injective for all objects $I^\bullet$, $J^\bullet$ of $\derplus{\I}$. 
    Suppose two morphisms $\phi_0$, $\phi_1 \colon I^\bullet \to J^\bullet$, represented by \emph{right} roofs $I^\bullet \to I_0^\bullet \xleftarrow{\sim} J^\bullet$ and $I^\bullet \to I_1^\bullet \xleftarrow{\sim} J^\bullet$, are sent to the same morphism in $\Hom_{\derplus{\A}}(I^\bullet, J^\bullet)$, meaning that there is the following commutative diagram in $\kplus{\A}$.
    \[\begin{tikzcd}[row sep = small, column sep = small]
        % https://q.uiver.app/#q=WzAsNSxbMSwxLCJJXzAiXSxbMywxLCJBIl0sWzAsMiwiSSJdLFs0LDIsIkoiXSxbMiwwLCJCIl0sWzIsMF0sWzMsMCwiXFxzaW0iXSxbMiwxXSxbMywxLCJcXHNpbSIsMl0sWzAsNF0sWzEsNCwiXFxzaW0iLDJdXQ==
        && A^\bullet \\
        & {I_0^\bullet} && I_1^\bullet \\
        I^\bullet &&&& J^\bullet
        \arrow[from=2-2, to=1-3]
        \arrow["\sim"', from=2-4, to=1-3]
        \arrow[from=3-1, to=2-2]
        \arrow[from=3-1, to=2-4]
        \arrow["\sim"{pos=0.3}, from=3-5, to=2-2, crossing over]
        \arrow["\sim"', from=3-5, to=2-4]
    \end{tikzcd}\]
    Then simply extending the diagram by a quasi-isomorphism $A^\bullet \to I_2^\bullet$, whose existence is ensured by Lemma \ref{F-acyclic resolutions}, proves that $\phi_0$ and $\phi_1$ were in fact equal and that the action \eqref{eq: morphism action of D(J) - D(A)} is faithful.

    Next, we show that the action on morphisms is also surjective, \ie the homomorphism \eqref{eq: morphism action of D(J) - D(A)} is surjective. Pick any $\psi\colon I^\bullet \to J^\bullet$ in $\derplus{\A}$, which is represented by a right roof $I^\bullet \to A^\bullet \xleftarrow{\sim} J^\bullet$. As before, extending the roof by a quasi-isomorphism $A^\bullet \to I^\bullet_0$ from Lemma \ref{F-acyclic resolutions}, leaves us with a representative $I^\bullet \to I_0^\bullet \xleftarrow{\sim} J^\bullet$ of some morphism $\phi\colon I^\bullet \to J^\bullet$ in $\derplus{\I}$.
    \[\begin{tikzcd}[column sep = small, row sep = small]
        % https://q.uiver.app/#q=WzAsNSxbMSwxLCJJXzAiXSxbMywxLCJBIl0sWzAsMiwiSSJdLFs0LDIsIkoiXSxbMiwwLCJJXzAiXSxbMiwwXSxbMywwLCJcXHNpbSIsMCx7ImxhYmVsX3Bvc2l0aW9uIjo0MH1dLFsyLDFdLFszLDEsIlxcc2ltIiwyXSxbMCw0LCIiLDIseyJsZXZlbCI6Miwic3R5bGUiOnsiaGVhZCI6eyJuYW1lIjoibm9uZSJ9fX1dLFsxLDQsIlxcc2ltIiwyXV0=
        && {I_0^\bullet} \\
        & {I_0^\bullet} && A^\bullet \\
        I^\bullet &&&& J^\bullet
        \arrow[equals, from=2-2, to=1-3]
        \arrow["\sim"', from=2-4, to=1-3]
        \arrow[from=3-1, to=2-2]
        \arrow[from=3-1, to=2-4]
        \arrow["\sim"{pos=0.3}, from=3-5, to=2-2, crossing over]
        \arrow["\sim"', from=3-5, to=2-4]
    \end{tikzcd}\]
    % \[\begin{tikzcd}[column sep = small, row sep = 1em]
    %     % https://q.uiver.app/#q=WzAsNSxbMSwyLCJJXzAiXSxbMywyLCJBIl0sWzAsNCwiSSJdLFs0LDQsIkoiXSxbMiwwLCJJXzAiXSxbMiwwXSxbMywwLCJcXHNpbSIsMCx7ImxhYmVsX3Bvc2l0aW9uIjo0MH1dLFsyLDFdLFszLDEsIlxcc2ltIiwyXSxbMCw0LCIiLDIseyJsZXZlbCI6Miwic3R5bGUiOnsiaGVhZCI6eyJuYW1lIjoibm9uZSJ9fX1dLFsxLDQsIlxcc2ltIiwyXV0=
    %     && {I_0^\bullet} \\
    %     \\
    %     & {I_0^\bullet} && A^\bullet \\
    %     \\
    %     I^\bullet &&&& J^\bullet
    %     \arrow[equals, from=3-2, to=1-3]
    %     \arrow["\sim"', from=3-4, to=1-3]
    %     \arrow[from=5-1, to=3-2]
    %     \arrow[from=5-1, to=3-4]
    %     \arrow["\sim"{pos=0.4}, from=5-5, to=3-2]
    %     \arrow["\sim"', from=5-5, to=3-4]
    % \end{tikzcd}\]
    The above diagram then shows, that $\phi$ is sent to $\psi$ and thus the action on morphisms is surjective.
    
    Lastly, Lemma \ref{F-acyclic resolutions} asserts that our functor is essentially surjective.
\end{proof}

\begin{remark}
    Upon close inspection, we recognise Theorem \ref{D+ = K+ and injectives} as a special case of Proposition \ref{D+(I) = D+(A) F-adapted}, namely take the $F$-adapted class $\I$ to be the class of all injectives of $\A$, provided there is enough of them. Indeed, it can be shown that any quasi-isomorphism $f\colon I^\bullet \to J^\bullet$ between two bounded below complexes of injectives is an isomorphism in $\kplus{\I}$. This is done by showing both $f^*\colon \Hom_{\kplus{\I}}(J^\bullet, I^\bullet) \to \Hom_{\kplus{\I}}(I^\bullet, I^\bullet)$ and $f_* \colon \Hom_{\kplus{\I}}(J^\bullet, I^\bullet) \to \Hom_{\kplus{\I}}(J^\bullet, J^\bullet)$ are isomorphisms. Lemma \ref{D+ K+ aux lemma 1} already shows us that $f^*$ is an isomorphism and by a very similar argument as in the proof of the same lemma $f_*$ is as well.
\end{remark}

We can now define the \emph{right derived functor} $\rderived{}{F}$ as the composition
\[
    \rderived{}{F} \colon \quad \derplus{\A} \xrightarrow{\ \iso \ } \derplus{\I} \longrightarrow D^+(B),
\] 
where $\derplus{\A} \xrightarrow{\ \iso \ } \derplus{\I}$ denotes a quasi-inverse to the equivalence $\derplus{\I} \iso \derplus{\A}$ established in Proposition \ref{D+(I) = D+(A) F-adapted} and $\derplus{\I} \to \derplus{\B}$ is the functor \eqref{eq: D+(I) -> D+(B)}.
The procedure for computing the right derived functor $\rderived{}{F}(A^\bullet)$ of a complex $A^\bullet$ in $\derplus{\A}$ is then very similar to \textsc{Method 1}. One takes an $\I$-resolution $A^\bullet \to I^\bullet_A$ of $A^\bullet$ and then applies the functor $F$ to $I^\bullet_A$, \ie
\[
    \rderived{}{F}(A^\bullet) = F(I^\bullet_A).
\]

We mention that both methods give rise to right derived functors satisfying the following defining universal property. We only state this here, but refer the reader to \cite[\S III.6]{gelfand2002methods} for the claim and its proof or \cite[\S 13.3, \S 10.3, \S 7.3]{kashiwara2006categories} for a more comprehensive and high level treatment using the formalism of localization. 

\begin{definition}\cite[\S III.6, Definition 6]{gelfand2002methods}
    Let $F \colon \A \to \B$ be a left exact functor between abelian categories. The \emph{right derived functor of $F$} is a triangulated functor $\rderived{}{F} \colon \derplus{\A} \to \derplus{\B}$ together with a natural transformation $\varepsilon_F \colon Q_\B \circ F \Longrightarrow \rderived{}{F} \circ Q_\A$, such that for any triangulated functor $G \colon \derplus{\A} \to \derplus{\B}$ and any natural transformation $\varepsilon \colon Q_\B \circ F \Longrightarrow G \circ Q_\A$, there exists a unique natural transformation $\eta \colon \rderived{}{F} \Longrightarrow G$, for which
    \[\begin{tikzcd}[column sep = small]
        % https://q.uiver.app/#q=WzAsMyxbMCwxLCJhIl0sWzEsMCwiYiJdLFsyLDEsImMiXSxbMSwwLCJ4IiwyLHsibGV2ZWwiOjJ9XSxbMSwyLCJ5IiwwLHsibGV2ZWwiOjJ9XSxbMCwyLCJ6IiwwLHsibGV2ZWwiOjJ9XV0=
        & Q_\B \circ F \\
        \rderived{}{F} \circ Q_\A && G \circ Q_\A
        \arrow["\varepsilon_F"', Rightarrow, from=1-2, to=2-1]
        \arrow["\varepsilon", Rightarrow, from=1-2, to=2-3]
        \arrow["\eta Q_\A", Rightarrow, from=2-1, to=2-3]
    \end{tikzcd}\]
    is commutative in the functor category $\operatorname{Fct}(\derplus{\A}, \derplus{\B})$.  
\end{definition}

% \noindent
% \textsc{Method 2$^*$.} \info{I think I will delete this method 2$^*$.} There is one instance in this thesis in which the two described methods will not suffice, namely in deriving the \emph{differentialy graded} $\Hom^\bullet$-functor. In this case our starting point will be a triangulated functor already on the level of homotopy categories $F \colon \kplus{\A} \to \kplus{\B}$, which does not originate from any functor at the level of abelian categories $\A \to \B$. Analogous to Definition \ref{F-adapted}, we introduce the following.
% \begin{definition}
%     \label{F-adapted for homotopy category}
%     Let $F \colon \kplus{\A} \to \kplus{\B}$ be a triangulated functor. A full triangulated subcategory $\mathcal K_F \subseteq \kplus{\A}$ is siad to be \emph{$F$-adapted} if the following two conditions hold.
%     \begin{enumerate}[label = (\roman*)]
%         \item{For every complex $A^\bullet$ in $\kplus{\A}$, there is a } \label{F-adapted for homotopy category i}
%         \item{For every acyclic complex $A^\bullet$ of $\mathcal K_F$, the complex $F(A^\bullet)$ is acyclic as well.} \label{F-adapted for homotopy category ii}
%     \end{enumerate}
% \end{definition}

% In order to construct the right derived functor $\rderived{}{F} \colon \derplus{\A} \to \derplus{\B}$, we assume the existence of an $F$-adapted category $\mathcal K_F \subseteq \kplus{\A}$ and introduce the full additive subcategory $\D_F \subseteq \derplus{\A}$ spanned on objects of $\mathcal K_F$. Proposition \ref{triangulated subcategory proposition} guarantees that $\D_F$ is a full triangulated subcategory of $\derplus{\A}$ and is moreover equivalent to the latter by \ref{F-adapted for homotopy category i} of Definition \ref{F-adapted for homotopy category}. Then we can extend the funtor $F$ to a functor $\D_F \to \derplus{\B}$ given by the assignment
% \begin{center}
%     \begin{tabular}{r r c l}
%         \textsl{Objects:} & $A^\bullet$ & $\longmapsto$ & $F(A^\bullet)$. \\
%         \textsl{Morphisms:} & $[A^\bullet \xleftarrow{s} C^\bullet \xrightarrow{f} B^\bullet]$ & $\longmapsto$ & $Q_\B F(f) \circ Q_\B F(s)\inv$.
%     \end{tabular}
% \end{center}
% This is a well-defined functor because condition \ref{F-adapted for homotopy category ii} of Definition \ref{F-adapted for homotopy category} ensures $F$ sends quasi-isomorphisms to quasi-isomorphisms by Proposition \ref{F preserves qis iff preserves acyclic}.
% Lastly the right derived functor is then given as the composition
% \[
%     \rderived{}{F} \colon \quad \derplus{\A} \xrightarrow{\ \iso \ } \D_F \longrightarrow \derplus{\B}.
% \]
% \begin{example}
%     When $\A$ contains enough injectives, the category $\kplus{\J} \subseteq \kplus{\A}$ is $F$-adapted for any triangulated functor $F \colon \kplus{\A} \to \kplus{\B}$.
% \end{example}

% \vspace{3cm}

For two composable the following proposition tells us that first deriving and then composing or first composing and then deriving results in the same functor.

\begin{proposition}
    
\end{proposition}

\begin{proof}
    
\end{proof}

\subsubsection*{Higher derived functors}

Classically derived functors of a left exact functor $F \colon \A \to \B$ first appeared as a sequence of additive functors $\A \to \B$. We introduce them in the following definition, using our established (modern) viewpoint. 

\begin{definition}
    For a right derived functor $\rderived{}{F} \colon \derplus{\A} \to \derplus{\B}$ of a left exact functor $F \colon \A \to \B$, we define the \emph{higher derived functors of} $F$ as compositions
    \[
        \rderived{i}{F} := H^i \circ \rderived{}{F} \colon \quad \derplus{\A} \to \B
    \]
    for all $i \in \Z$. 
\end{definition}

Frequently we will be interested in the image of a higher derived functor $\rderived{i}{F}$ of just a single object $A$ belonging to category $\A$. In that case we will always interpret\footnote{More precisely, there is always a fully faithful functor $\A \to \kcat{\A}$ sending an object $A$ to a complex $\cdots \to 0 \to A \to 0 \to \cdots$ concentrated in degree $0$ and a morphism $f\colon A \to B$ to the homotopy class of the chain map defined by $f^0 = f$ and $f^i = 0$, for $i \neq 0$ \cite[Chapter 3, Lemma 1.3.5]{milicic-dercat}.} it as a bounded complex concentrated in degree $0$. 

\begin{example}
    For an object $A$ of $\A$, equipped with an $F$-adapted class $\I$, we can compute that $\rderived{i}{F}(A) \iso 0$, for $i < 0$, and $\rderived{0}{F}(A) \iso F(A)$. 
    % Indeed, considering $A$ as a complex concentrated in degree $0$, let $I^\bullet$ be its $\I$-resolution. 
    Indeed, consider an $\I$-resolution $I^\bullet$ of $A$.
    Then, as $I^i = 0$ for $i < 0$, we have that $\rderived{i}{F}(A) = H^i(F(I^\bullet)) \iso 0$ for $i < 0$. For the second claim, because $I^\bullet$ is quasi-isomorphic to $A$, we have $\ker(I^0 \to I^1) \iso A$, and since $F$ is left exact, we conclude that $F(A) \iso \ker(F(I^0) \to F(I^1)) \iso H^0(F(I^\bullet)) = \rderived{0}{F}(A)$.
\end{example}

\begin{definition}
    \label{F-acyclic}
    Suppose the right derived functor $\rderived{}{F}$ exists. We say an object $A$ of $\A$ is \emph{$F$-acyclic}, if $\rderived{i}{F}(A) \iso 0$ for all $i \neq 0$.
\end{definition}

A useful tool for gathering information about the action of the higher derived functors of a left exact functor $F$ on objects will be the long exact sequence associated to $F$ introduced in the following proposition. In view of this proposition we also interpret higher derived functors of $F$ as measuring the extent to which $F$ fails to be exact. 

\begin{proposition}
    Let $F \colon \A \to \B$ be a left exact functor and let $0 \to A \to B \to C \to 0$ be a short exact sequence in $\A$. Then there exists a long exact sequence
    \begin{multline}
        \label{eq: LES for left exact functor F}
        0 \to F(A) \to F(B) \to F(C) \to \rderived{1}{F}(A) \to \rderived{1}{F}(B) \to \rderived{1}{F}(C) \to \cdots \\
        \cdots \to \rderived{i}{F}(A) \to \rderived{i}{F}(B) \to \rderived{i}{F}(C) \to \rderived{i+1}{F}(A) \to \cdots.
    \end{multline}
\end{proposition}

\begin{proof}
    First, we describe how the short exact sequence gives rise to a distinguished triangle $A \to B \to C \to A[1]$ in $\derplus{\A}$, where $A$, $B$ and $C$ are considered as complexes concentrated in degree $0$.
    \info{finish this} 

    After first applying the triangulated functor $\rderived{}{F}$ and then the cohomological functor $H^0$ to the triangle $A \to B \to C \to A[1]$, we obtain a long exact sequence 
    \begin{multline*}
        0 \to H^0(\rderived{}{F}(A)) \to H^0(\rderived{}{F}(B)) \to H^0(\rderived{}{F}(C)) \to H^0(\rderived{}{F}(A)[1]) \to \cdots \\
        \cdots H^0(\rderived{}{F}(A)[i]) \to H^0(\rderived{}{F}(B)[i]) \to H^0(\rderived{}{F}(C)[i]) \to H^0(\rderived{}{F}(A)[i+1]) \to \cdots
    \end{multline*}
    by Proposition \ref{LES in cohomology}, whose terms are readily seen to be isomorphic to the terms of \eqref{eq: LES for left exact functor F}. \qedhere
\end{proof}

The rest of this section is devoted to proving some useful results relating $F$-acyclic and $F$-adapted objects. 

\begin{lemma}
    \label{F-adapted is F-acyclic}
    Every object of an $F$-adapted class is also $F$-acyclic.
\end{lemma}

\begin{proof}
    We only have to recall definition \ref{F-acyclic}. An object $I$ belonging to an $F$-adapted class $\I$ trivially forms its own $\I$-resolution $I^\bullet = (\cdots \to 0 \to I \to 0 \to \cdots)$, which allows us to compute $\rderived{i}{F}(I) = H^i(\rderived{}{F}(I^\bullet)) = H^i(F(I^\bullet)) \iso 0$, whenever $i \neq 0$ (and also $F(I)$ for $i = 0$).
\end{proof}

\begin{proposition}
    \label{F-acyclic is F-adapted}
    Let $F : \A \to \B$ be a left exact functor between two abelian categories. Let $\I \subset \A$ be an $F$-adapted class of objects in $\A$. Then the class of all $F$-acyclic objects of $\A$, denoted by $\I_F$, is also $F$-adapted. 
\end{proposition}

\begin{proof}
    We verify conditions (i)--(iii) of definition \ref{F-adapted}. (i) Since the higher derived functors $\rderived{i}{F}$ are additive, $\I_F$ is stable under finite sums. (ii) Holds by lemma \ref{F-adapted is F-acyclic}. (iii) Suppose $A^\bullet$ is an acyclic complex in $K^+(\A)$ with $A^i \in \I_F$ for all $i$. Then using acyclicity of $A^\bullet$, we may break this complex up into a series of short exact sequences 
    % as follows.
    % \begin{align*}
    %     0 \to A^0 \to & A^1 \to \ker d^2 \to 0 \\
    %     0 \to \ker d^2 \to & A^2 \to \ker d^3 \to 0 \\
    %     & \vdots
    % \end{align*} 
    \begin{equation}
        \label{eq: SES from acyclic cx}
        0 \to \ker d^i \to A^i \to \ker d^{i+1} \to 0 \quad \text{ for $i \in \Z$.}
    \end{equation}
    As our complex $A^\bullet$ is supported on $\N$, the first interesting short exact sequence happens at index $i = 1$, where $\ker d^1 = \im d^0 \iso A^0$
    It yields the following long exact sequence associated to $F$
    \begin{multline*}
        0 \to F(A^0) \to F(A^1) \to F(\ker d^2) \to \rderived{1}{F}(A^0) \to \cdots \\ \cdots \to \rderived{i}{F}(A^0) \to \rderived{i}{F}(A^1) \to \rderived{i}{F}(\ker d^2) \to \rderived{i+1}{F}(A^0) \to \cdots.
    \end{multline*}
    Exactness of this sequence together with $F$-acyclicity of $A^0$ and $A^1$ shows that $\ker d^2$ is also $F$-acyclic and by only considering the beginning few terms of the sequence, we obtain the short exact sequence 
    % From the fact that $A^0$ and $A^1$ are $F$-acyclic we see that $\ker d^2$ is $F$-acyclic and we obtain a short exact sequence 
    \[
        0 \to F(A^0) \to F(A^1) \to F(\ker d^2) \to 0.
    \]
    After inductively applying the same reasoning for each of the original short exact sequences of \eqref{eq: SES from acyclic cx} for $i > 1$, we are left with short exact sequences\footnote{To be more precise, we have used that $F$ is left exact in order to swap the order of $F$ and $\ker$ to reach \[\ker Fd^i \iso F(\ker d^i)\].}
    \[
        0 \to \ker Fd^i \to F(A^i) \to \ker Fd^{i+1} \to 0 \quad \text{ for $i \geq 1$.}
    \]
    Collecting them all back together we conclude that the complex $F(A^\bullet)$ is acyclic.
    % \info{There are inconsistencies with $F(A^\bullet) = K^+(F)(A^\bullet)$ and so on. $\to$ I am going to abuse notation and use $F(A^\bullet)$, bc this is also done in practise \eg $f_*\F^\bullet$, not $K^+(f_*)(\F^\bullet)$} \qedhere
\end{proof}

% \[\begin{tikzcd}[column sep = small, row sep = 1.8em]
%     % https://q.uiver.app/#q=WzAsNSxbMCwwLCJcXGtwbHVze1xcSn0iXSxbMiwwLCJcXGtwbHVze1xcQX0iXSxbMiwyLCJcXGRlcnBsdXN7XFxBfSJdLFs0LDAsIlxca3BsdXN7XFxCfSJdLFs0LDIsIlxcZGVycGx1c3tcXEJ9Il0sWzEsMiwiUV9cXEEiXSxbMiwwLCJcXG5hdGlzbyIsMl0sWzAsMSwiXFxpb3RhIiwwLHsic3R5bGUiOnsidGFpbCI6eyJuYW1lIjoiaG9vayIsInNpZGUiOiJ0b3AifX19XSxbMSwzLCJGIl0sWzMsNCwiUV9cXEIiXV0=
% 	{\kplus{\J}} && {\kplus{\A}} && {\kplus{\B}} \\
% 	\\
% 	&& {\derplus{\A}} && {\derplus{\B}}
% 	\arrow["\iota", hook, from=1-1, to=1-3]
% 	\arrow["F", from=1-3, to=1-5]
% 	\arrow["{Q_\A}", from=1-3, to=3-3]
% 	\arrow["{Q_\B}", from=1-5, to=3-5]
% 	\arrow["\natiso"', from=3-3, to=1-1]
% \end{tikzcd}\]

% In th subsection we will deal with the case, when $F$ is only left exact. The theory dually runs in parallel, when $F$ is right exact. 

\subsubsection{Ext functors}
\label{subsection: Ext functors}

As a first application of the established theory of derived functors we consider the $\Hom$-functor
\[
    \Hom_\A(A, -) \colon \A \to \mod{k},
\]
where $A$ is an object of an abelian category $\A$. Assume that category $\A$ contains enough injectives. By Example \ref{Homs are left exact functors}, we know that $\Hom_\A(A, -)$ is a left exact functor, so we can right derive it, to obtain the functor
% so there exists its right derived functor
\[
    \mathbf{R}\Hom_\A(A, -) \colon \derplus{\A} \to \derplus{\mod{k}}.
\] 
Taking the $i$-th cohomology, we obtain higher derived functors of $\Hom_\A(A, -)$, called the \emph{Ext-functors}
\[
    \Ext^i_\A(A, -) := \rderived{i}{\Hom_\A(A, -)} = H^i(\rderived{}{\Hom_\A(A, -)}).
\]
There exists a very beautiful connection relating the Ext-functors of category $\A$ and the hom-functors of the bounded below derived category $\derplus{\A}$ captured in the next proposion, given without proof, for in a moment we will present its generalized verison.

\begin{proposition}
    Let $\A$ be an abelian category with enough injectives and let $A$ and $B$ belong to $\A$. Then for all $i \in \Z$ there exist isomorphisms
    \[
        \Ext^i_\A(A, B) \iso \Hom_{\derplus{\A}}(A, B[i]).
    \]
\end{proposition}

For any two complexes $A^\bullet$ and $B^\bullet$ we can define a complex $\Hom^\bullet(A^\bullet, B^\bullet)$ of $k$-modules given by terms
\[
    \Hom^i(A^\bullet, B^\bullet) = \prod_{j \in \Z}\Hom_\A(A^j, B^{j + i}),
\]
\ie the set of all degree $i$ maps from $A^\bullet$ to $B^\bullet$, considered as graded objects of $\A^\Z$, and differentials 
\[
    d^i \colon \Hom^i(A^\bullet, B^\bullet) \to \Hom^{i+1}(A^\bullet, B^\bullet), \qquad 
    d^i(f) = \left(d^{i+j}_B \circ f^j + (-1)^{i+1} f^{j+1} \circ d^j_A\right)_{j \in \Z}.
\]
This differential will also be denoted by $\delta^i_B$.

By setting $\Hom^\bullet(A^\bullet, -)(B^\bullet) = \Hom^\bullet(A^\bullet, B^\bullet)$ and for a chain map $u\colon B^\bullet \to C^\bullet$ in $\com(\A)$ defining $u_* = \Hom^\bullet(A^\bullet, -)(u) \colon \Hom^\bullet(A^\bullet, B^\bullet) \to \Hom^\bullet(A^\bullet, C^\bullet)$ to be the chain map given by the collection $(u_*^i)_{i \in \Z}$, where $u_*^i$ sends $f = (f^j)_{j \in \Z} \in \Hom^i(A^\bullet, B^\bullet)$ to $u \circ f = (u^{i+j} \circ f^j)_{j \in \Z}$, we obtain a functor
\[
    \Hom^\bullet(A^\bullet, -) \colon \com(\A) \to \com(\mod{k}).
\]
Indeed, clearly $u_*$ is a graded morphism of degree $0$, which moreover commutes with the differentials, according to the following computation
\begin{align*}
    u^{i+1}_*\left(d^i_{\Hom^\bullet(A, B)}(f)\right) &= u^{i+1}_*\left(\left(d^{i+j}_B f^j + (-1)^{i+1} f^{j+1} d^j_A\right)_j\right) \\
    &= \left(u^{i+j+1} d^{i+j}_B f^j + (-1)^{i+1} u^{i+j+1} f^{j+1} d^j_A\right)_j \\
    &= \left(d^{i+j}_C u^{i+j} f^j + (-1)^{i+1} u^{i+j+1} f^{j+1} d^j_A\right)_j \\
    &= d^i_{\Hom^\bullet(A, C)}\left(\left(u^{i + j}f^j\right)_j\right) \\
    &= d^i_{\Hom^\bullet(A, C)}\left(u^i_*(f)\right),
\end{align*}
for all $f \in \Hom^i(A^\bullet, B^\bullet)$. While not difficult to check, the verification of functoriality is omitted. 

Next, we would like $\Hom^\bullet(A^\bullet, -)$ to descend to a homotopy functor $\kcat{\A} \to \kcat{\mod{k}}$. This is in fact so, as a null-homotopic chain map $u\colon B^\bullet \to C^\bullet$ is sent to a null-homotopic chain map $u_* \htpy 0$. Indeed, suppose homotopy $h \in \Hom^{-1}(B^\bullet, C^\bullet)$ witnesses $u \htpy 0$, then for all $i \in \Z$, we have
\[
    u^i = d^{i-1}_C \circ h^i + h^{i+1} \circ d^i_B.
\]
We let $h^i_*$ denote the morphism $\Hom^i(A^\bullet, B^\bullet) \to \Hom^{i-1}(A^\bullet, C^\bullet)$, given by sending $f$ to $(h^{i+j} \circ f^j)_{j \in \Z}$, and then compute
\begin{align*}
    (\delta^{i-1}_C \circ h^{i}_* + h^{i+1}_* \circ \delta^i_B)(f) &= 
    \delta^{i-1}_C\left((h^{i+j}f^j)_j\right) + h^{i+1}_*\left((d_B^{i+j} f^j + (-1)^{i+1}f^{j+1}d^j_A)_j\right) \\
    &= \left(d^{i+j-1}_C h^{i+j} f^j + (-1)^{i} h^{i+j+1} f^{j+1} d^j_A \right)_j  + \\ 
    & \qquad \qquad + \left(h^{i+j+1}d_B^{i+j}f^j + (-1)^{i+1} h^{i+j+1} f^{j+1} d^j_A \right)_j \\
    &= \left(d^{i+j-1}_C h^{i+j} f^j + h^{i+j+1}d_B^{i+j}f^j\right)_j \\
    &= \left((d^{i+j-1}_C h^{i+j} + h^{i+j+1}d_B^{i+j})\circ f^j\right)_j \\
    &= \left(u^{i+j} \circ f^j\right)_j \\
    &= u_*^{i}(f).
\end{align*}
Thus the homotopy $h_* = (h^i_*)_{i \in \Z}$ whitnesses $u_* \htpy 0$, allowing us to conclude that 
\[
    \Hom^\bullet(A^\bullet, -) \colon \kcat{\A} \to \kcat{\mod{k}}
\] 
is well defined. The preceding functor is also triangulated, because it commutes with the translation functors and it sends distinguished triangles of $\kcat{\A}$ to distinguished triangles of $\kcat{\mod{k}}$ This is because of the following isomorphism of chain complexes of $k$-modules,
\[
    C(u_*)^\bullet = C(\Hom^\bullet(A^\bullet, -)(u))^\bullet \cong \Hom^\bullet(A^\bullet, -)(C(u)^\bullet),
\]
which holds for any chain map $u \colon B^\bullet \to C^\bullet$. Indeed, we have
\begin{align*}
    C(u_*)^i 
    &= \Hom^{i+1}(A^\bullet, B^\bullet) \oplus \Hom^i(A^\bullet, C^\bullet) \\
    &= \prod_{j \in \Z}\Hom_\A(A^j, B^{i+j+1}) \oplus \prod_{j \in \Z}\Hom_\A(A^j, C^{i+j}) \\
    &\cong \prod_{j \in \Z} \left(\Hom_\A(A^j, B^{i+j+1}) \oplus \Hom_\A(A^j, C^{i+j})\right) \\
    &\cong \prod_{j \in \Z}\Hom_\A(A^j, B^{i+j+1} \oplus C^{i+j}) \\
    &= \prod_{j \in \Z}\Hom_\A(A^j, C(u)^{i+j}) \\
    &= \Hom^i(A^\bullet, C(u)^\bullet),
\end{align*}
where these isomorphisms also commute with the differentials. Pick $f \in \Hom^i(A^\bullet, C(u)^\bullet)$ which is given by a pair $f$


\vspace{3cm}



At last, for $A^\bullet$ belonging to $\kminus{\A}$ \ie bounded from \emph{above}\footnote{This assumption cannot be skipped, for otherwise we cannot claim, that the functor $\Hom^\bullet(A^\bullet, -)$ maps into the bounded \emph{below} homotopy category $\kplus{\A}$.}, we have a triangulated functor $\Hom^\bullet(A^\bullet, -) \colon \kplus{\A} \to \kplus{\mod{k}}$, which we may derive, in accordance with Remark \ref{Deriving homotopy functors}, of course assuming $\A$ contains enough injectives, to obtain
\[
    \rderived{}{\Hom^\bullet(A^\bullet, -)} \colon \derplus{\A} \longrightarrow \derplus{\mod{k}}.
\]
Its higher derived counterpart will be denoted by $\Ext^i_\A(A^\bullet, - ) := \rderived{i}{\Hom^\bullet(A^\bullet, -)}$.

% \begin{remark}
%     There is also a generalization of this result providing a way to identify the hom-functors in $\derplus{\A}$ applied not only to objects of $\A$, but to bounded below complexes. In that case, on the Ext-side, one considers the \emph{differentialy graded hom-functors} of $\kplus{\A}$, namely $\Hom^\bullet(-,-) \colon \kminus{\A}^\op \times \kplus{\A} \to \kcat{\mod{k}}$, for which $\Hom^\bullet(A^\bullet, -)$ turns out to be right derivable inducing its higher derived functors $\Ext^i(A^\bullet, -) = \rderived{i}{\Hom^\bullet(A^\bullet, -)}$. Then for all $i \in \Z$ and all $A^\bullet$ and $B^\bullet$ of $\derplus{\A}$ there are isomorphisms
%     \[
%         \Ext^i(A^\bullet, B^\bullet) \iso \Hom_{\derplus{\A}}(A^\bullet, B[i]^\bullet).
%     \]
%     \info{this has to be expaned, because I need it for Serre functors, naturality in $A^\bullet$ and $B^\bullet$ is important.}
% \end{remark}

\begin{remark}
    Behind the scenes of this whole process one actually consideres an additive bifunctor $F \colon \A \times \A' \to \A''$, with $\A$, $\A'$ and $\A''$ abelian, where the latter is assumed to contain countable products. From this bifunctor we induce another additive bifunctor on the level of chain complexes $F_{\pi} \colon \com(\A) \times \com(\A') \to \com(\A'')$ via the total complex of a double complex as described in \cite[\S 11.6]{kashiwara2006categories}.
    %  with terms $F^\bullet_\pi(X^\bullet, Y^\bullet) = \operatorname{tot_\pi}(F(X^\bullet, Y^\bullet))$, where $\operatorname{tot_\pi}(Z^{\bullet, \bullet})^n = \prod_{p+q = n} Z^{p,q}$. 
    It then turns out that $F_\pi$ induces a well defined triangulated bifunctor on the level of the homotopy category 
    \[
        \kcat{\A} \times \kcat{\A'} \to \kcat{\A''}.
    \]
    % $\kplus{\A} \times \kplus{\A'} \to \kplus{\A''}$, $\kminus{\A} \times \kminus{\A'} \to \kminus{\A''}$, $\kb{\A} \times \kcat{\A'} \to \kcat{\A''}$ and $\kcat{\A} \times \kb{\A'} \to \kcat{\A''}$.
\end{remark}

\begin{proposition}
    Let $\A$ be an abelian category with enough injectives. Then for every $A^\bullet$ of $\kminus{\A}$ and $B^\bullet$ belonging to $\kplus{\A}$ there natural isomorphisms
    \[
        \alpha_{A^\bullet, B^\bullet} : \Ext^i_\A(A^\bullet, B^\bullet) \iso \Hom_{\dercat{\A}}(A^\bullet, B[i]^\bullet).
    \]
\end{proposition}
\info{I think this is true, but I won't be able to show it as I need to use $\derplus{\A} \natiso \kplus{\J}$, since $A^\bullet$ is not bounded from below.}

\begin{proposition}
    \label{Homs in derived category}
    Let $\A$ be an abelian category with enough injectives. Then for every $A^\bullet$ and $B^\bullet$ belonging to $\derb{\A}$ there natural isomorphisms
    \[
        \alpha_{A^\bullet, B^\bullet} : \Ext^i_\A(A^\bullet, B^\bullet) \iso \Hom_{\derb{\A}}(A^\bullet, B[i]^\bullet).
    \]
\end{proposition}

\begin{lemma}
    \label{Homotopy hom = cohomology of inner hom}
    Let $A^\bullet$ and $B^\bullet$ be chain complexes in $\kcat{\A}$ and let $i \in \Z$. Then
    \[
        H^i(\Hom^\bullet(A^\bullet, B^\bullet)) = \Hom_{\kcat{\A}}(A^\bullet, B[i]^\bullet).
    \] 
\end{lemma}

\begin{proof}
    This is practically the definition of $\Hom_{\kcat{\A}}(A^\bullet, B[i]^\bullet)$. The cohomology $k$-module $H^i(\Hom^\bullet(A^\bullet, B^\bullet))$ is defined to be the quotient $\ker \delta^i / \im \delta^{i-1}$, thus let us identify $\ker \delta^i$ and $\im \delta^{i-1}$. Notice, that for $f \in \Hom^i(A^\bullet, B^\bullet)$ we have
    \[
        \delta^i(f) = \left(d^{i+j}_B \circ f^j + (-1)^{i+1} f^{j+1} \circ d^j_A\right)_{j \in \Z} = (-1)^i \left(d^j_{B[i]} \circ f^j - f^{j+1} \circ d^j_A\right)_{j \in \Z},
    \]
    thus $\ker \delta^i$ is the set of all chain maps $A^\bullet \to B[i]^\bullet$, and $\im \delta^{i-1}$ the set of all null-homotopic chain maps.
\end{proof}

\begin{proof}[Proof of Proposition \ref{Homs in derived category}]
    Let $B^\bullet \to I^\bullet$ be an injective resolution for $B^\bullet$. We follow the chain of isomorphisms
    \begin{align*}
        \Hom_{\derb{\A}}(A^\bullet, B[i]^\bullet) &= \Hom_{\derplus{\A}}(A^\bullet, B[i]^\bullet)  \\
        &\iso \Hom_{\derplus{\A}}(A^\bullet, I[i]^\bullet)  \\
        &\iso \Hom_{\kplus{\J}}(A^\bullet, I[i]^\bullet) &\text{(Theorem \ref{D+ = K+ and injectives})}\\
        &= H^i(\Hom^\bullet(A^\bullet, I^\bullet)) &\text{(Lemma \ref{Homotopy hom = cohomology of inner hom})} \\
        &= \Ext_\A^i(A^\bullet, B^\bullet). 
    \end{align*}\qedhere
\end{proof}
\newpage

\section{Derived categories in geometry}
\label{Chapter: Derived categories in geometry}

This chapter will be devoted to applying results of Chapter \ref{Chapter: Derived categories} to the abelian category $\coherent{X}$ of coherent sheaves on $X$. 
We will first define the bounded derived category of coherent sheaves $\derb{X}$ of a scheme $X$ over $k$ and discuss how this category fits into the broader context of other derived categories. We prove that under some assumptions $\derb{X}$ is indecomposable and admits a spanning class, which will play an important role later on. Generalizing Serre duality we also equip $\derb{X}$ with a Serre functor. 

The second section will cover the construction of various different derived functors having their origin in geometry. These will include the global sections functor, the push-forward along a morphism, different Hom-functors and Ext-functors, the tensor product and Tor-functors, and the pull-back along a morphism. 
We will conclude this chapter by providing a collection of results showcasing relationships that hold between the derived functors of the previous section. These will serve as convenient computational tools especially in Chapter \ref{Chapter: Fourier-Mukai transforms} on Fourier--Mukai transforms.

\subsection{Derived category of coherent sheaves}
\label{Subsection: Derived category of coherent sheaves}

One of the main invariants of a scheme $X$ over $k$ is its category of coherent sheaves $\coherent{X}$. We can think of this category as a slight extension of the category of locally free $\OO_X$-modules of finite rank also known as $k$-vector bundles on $X$ in the sense that $\coherent{X}$ is abelian, whereas the former generally is not. In this chapter we restrict ourselves to noetherian schemes and later to smooth projective varieties over an algebraically closed field $k$. Our primary object of study will be the bounded derived category of coherent sheaves
\[
    \derb{X} := \derb{\coherent{X}}.
\]

In this section some of the results will have to be given without proof,
but we will provide precise references when required. Certain results will serve as the groundwork on which the theory is then further developed, for example statements like certain functors map coherent sheaves into coherent sheaves, while others, like Grothendieck-Verdier duality, lie well outside the scope of this thesis, but are relevant for establishing certain relations among the derived functors. We also mainly focus on showcasing the methods and interactions between the derived functors and not that much on being as general as one can be. 

% \vspace{0.3cm}
\noindent
\textsl{Notation.}
Copying Huybrechts \cite{huybrechts2006fouriermukai}, we will denote by $\mathcal H^i(\F^\bullet)$ the $i$-th cohomology of a complex $\F^\bullet$, not to mistake it for sheaf cohomology $\cohom{i}{X}{\F}$ of $X$ with coefficients in a sheaf $\F$.

% \vspace{0.3cm}

As observed in Chapter \ref{Chapter: Derived categories} the construction of derived functors relied heavily on the existence of some special class of objects in the domain category. The most special of them all being the class of injective objects. As the category of coherent sheaves on $X$ often does not contain enough injectives\footnote{Intuitively this statement may be mirrored with the analogous statement that injective $A$-modules almost never appear to be finitely generated.}, we are forced to expand our scope. Luckily, the contrary is true if we restrict ourselves to noetherian schemes and replace coherent sheaves with quasi-coherent ones. 

\begin{proposition}
    \label{qcoh of noetherian scheme contains enough injectives}
    \emph{\cite[\S II, Theorem 7.18]{Hartshorne1966}}
    The category of quasi-coherent sheaves $\quasicoh{X}$ of a noetherian scheme $X$ contains enough injectives.
    % For a noetherian scheme $X$, its category of quasi-coherent shaves $\quasicoh{X}$ contains enough injectives.
\end{proposition}

\begin{proof}
    This is a consequence of \cite[\S II, Theorem 7.18]{Hartshorne1966},  which states that every quasi-coherent sheaf on a noetherian scheme $X$ can be embedded into a quasi-coherent sheaf, which is an injective $\struct{X}$-module. 
\end{proof}

\begin{proposition}
    \label{Db(X) is bounded derived of quasi-coherent with coherent cohomology}
    \emph{\cite[\S 3, Proposition 3.5]{huybrechts2006fouriermukai}.}
    For a noetherian scheme $X$ the inclusion $\derb{X} \hookrightarrow \derb{\quasicoh{X}}$ induces an equivalence of triangulated categories
    \[
        \derb{X} \natiso \mathsf{D}^b_{\mathsf{coh}}(\quasicoh{X}),
    \]
    where $\mathsf{D}^b_{\mathsf{coh}}(\quasicoh{X})$ is the full triangulated subcategory of $\derb{\quasicoh{X}}$, consisting of bounded complexes of quasi-coherent sheaves on $X$ with coherent cohomology.
\end{proposition}

% \begin{proof}
%     For a proof see \cite[\S 3, Proposition 3.5]{huybrechts2006fouriermukai}.
% \end{proof}

\begin{proposition}
    \label{In Db(X) everything is a complex of vector bundles}
    \emph{\cite[\S 3, Proposition 3.26]{huybrechts2006fouriermukai}}
    Let $X$ be a smooth projective variety over $k$. Then any complex $\F^\bullet$ of $\derb{X}$ is isomorphic in $\derb{X}$ to a bounded complex $\E^\bullet$ of locally free $\struct{X}$-modules of finite rank also known as \emph{vector bundles}.
\end{proposition}

% By a slight abuse of terminology locally free $\struct{X}$-modules of finite rank will be called \emph{vector bundles}.

\begin{proof}
    To prove this proposition we will need the following two facts concerning locally free sheaves on smooth projective varieties.
    \begin{itemize}[label = $\vartriangleright$]
        \item{Any coherent sheaf $\F$ on $X$ is a quotient of a locally free $\struct{X}$-module of finite rank $\E$, \ie there is an epimorphism $\E \twoheadrightarrow \F$ \cite[\S II, Corollary 5.18]{Hartshorne1977}}.
        \item{Let $n$ denote the dimension of $X$. For any coherent sheaf $\F$ on $X$ admitting an exact sequence of coherent sheaves
        \[
            0 \to \E^{-n} \to \E^{-n+1} \to \cdots \to \E^{-1} \to \E^0 \to \F \to 0.
        \]
        If $\E^0$, $\E^{-1}$,..., $\E^{-n+1}$ are locally free, the sheaf $\E^{-n}$ is locally free as well. See \cite[\S B.8.3]{Fulton1998} and note that this is essentially a local statement since a coherent sheaf $\E$ is a vector bundle if and only if all its stalks $\E_x$ are free $\struct{X, x}$-modules \cite[\S1, Lemma 5.1.3]{Weibel2013}.
        % \info{needs reference}
        }
    \end{itemize}
    Pick a bounded complex of coherent sheaves $\F^\bullet$. Without loss of generality assume $\F^i \iso 0$ for $i \leq 1$,
    % and $i > m$ for some $m \in \Z$, 
    otherwise replace $\F^\bullet$ by an appropriate shift of itself. By the first fact and a dual argument to the proof of Proposition \ref{injective resolution} we see that there exists a complex $\E^\bullet$ of vector bundles bounded from above and a quasi-isomorphism $\E^\bullet \to \F^\bullet$.
    % \[
    %     \cdots \to \E^{-n} \to \E^{-n+1} \to \cdots \to \E^{-1} \to \E^0 \to \E^1 \to \cdots \to \E^m \to 0 \to \cdots
    % \]
    \[\begin{tikzcd}[column sep = 1em]
        % https://q.uiver.app/#q=WzAsMjIsWzEsMCwiXFxFXnstbn0iXSxbMiwwLCJcXEVeey1uKzF9Il0sWzMsMCwiXFxjZG90cyJdLFs0LDAsIlxcRV57LTF9Il0sWzUsMCwiXFxFXjAiXSxbNywwLCJcXGNkb3RzIl0sWzgsMCwiXFxFXm0iXSxbOSwwLCIwIl0sWzEwLDAsIlxcY2RvdHMiXSxbMCwwLCJcXGNkb3RzIl0sWzYsMCwiXFxFXjEiXSxbMSwxLCIwIl0sWzIsMSwiMCJdLFszLDEsIlxcY2RvdHMiXSxbNCwxLCIwIl0sWzUsMSwiXFxGXjAiXSxbNywxLCJcXGNkb3RzIl0sWzgsMSwiXFxGXm0iXSxbOSwxLCIwIl0sWzEwLDEsIlxcY2RvdHMiXSxbMCwxLCJcXGNkb3RzIl0sWzYsMSwiXFxGXjEiXSxbMCwxXSxbMSwyXSxbMiwzXSxbMyw0XSxbNSw2XSxbNiw3XSxbNyw4XSxbOSwwXSxbNCwxMF0sWzEwLDVdLFsxMSwxMl0sWzEyLDEzXSxbMTMsMTRdLFsxNCwxNV0sWzE2LDE3XSxbMTcsMThdLFsxOCwxOV0sWzIwLDExXSxbMTUsMjFdLFsyMSwxNl0sWzAsMTFdLFsxLDEyXSxbMywxNF0sWzQsMTVdLFsxMCwyMV0sWzYsMTddLFs3LDE4XV0=
        \cdots & {\E^{-n}} & {\E^{-n+1}} & \cdots & {\E^{0}} & {\E^1} & {\E^2} & \cdots & {\E^m} & 0 & \cdots \\
        \cdots & 0 & 0 & \cdots & 0 & {0} & {\F^2} & \cdots & {\F^m} & 0 & \cdots
        \arrow[from=1-1, to=1-2]
        \arrow[from=1-2, to=1-3]
        \arrow[from=1-2, to=2-2]
        \arrow[from=1-3, to=1-4]
        \arrow[from=1-3, to=2-3]
        \arrow[from=1-4, to=1-5]
        \arrow[from=1-5, to=1-6]
        \arrow[from=1-5, to=2-5]
        \arrow[from=1-6, to=1-7]
        \arrow[from=1-6, to=2-6]
        \arrow[from=1-7, to=1-8]
        \arrow[from=1-7, to=2-7]
        \arrow[from=1-8, to=1-9]
        \arrow[from=1-9, to=1-10]
        \arrow[from=1-9, to=2-9]
        \arrow[from=1-10, to=1-11]
        \arrow[from=1-10, to=2-10]
        \arrow[from=2-1, to=2-2]
        \arrow[from=2-2, to=2-3]
        \arrow[from=2-3, to=2-4]
        \arrow[from=2-4, to=2-5]
        \arrow[from=2-5, to=2-6]
        \arrow[from=2-6, to=2-7]
        \arrow[from=2-7, to=2-8]
        \arrow[from=2-8, to=2-9]
        \arrow[from=2-9, to=2-10]
        \arrow[from=2-10, to=2-11]
    \end{tikzcd}\]
    Hence $\mathcal H^i(\E^\bullet) \iso 0$ for $i \leq 1$ and the right truncation $\rtruncation{1}(\E^\bullet)$ is acyclic. Truncating once more on the \emph{left} at degree $-n$, where $n$ is the dimension of $X$, we obtain an exact sequence $\ltruncation{-n}(\rtruncation{1}(\E^\bullet))$, depicted below
    \[
        0 \to \coker d_\E^{-n-1} \to \E^{-n+1} \to \cdots \to \E^{0} \to \ker d^1_\E \to 0.
    \]
    This is now a resolution of a coherent sheaf $\ker d_\E^1$ by coherent sheaves, where $\E^0$, $\E^{-1}$,..., $\E^{-n+1}$ are vector bundles. By the second fact we see that $\coker d_\E^{-n-1}$ is then a vector bundle, therefore $\ltruncation{-n}(\E^\bullet)$ is a complex of vector bundles quasi-isomorphic to $\E^\bullet$ and consequently also quasi-isomorphic to $\F^\bullet$.
\end{proof}

\begin{definition}
    \label{Definition of support}
    Let $X$ be a scheme. The \emph{support} of an $\struct{X}$-module $\F$ is defined to be the set
    \[
        \supp{\F} = \{x \in X \mid \F_x \not\iso 0 \}.
    \] 
    The \emph{support} of a complex $\F^\bullet$ of $\derb{X}$ is defined to be the union
    \[
        \supp{\F^\bullet} = \bigcup_{i \in \Z} \supp{\mathcal H^i(\F^\bullet)}.
    \]
\end{definition}

\begin{remark}
    \label{supports closed}
    The support $\supp{\F}$ of a coherent sheaf $\F \in \coherent{X}$ is a closed subset of $X$, therefore the support of any complex $\F^\bullet$ of $\derb{X}$, being a finite union of closed sets, is closed as well.
\end{remark}

\begin{theorem}[\text{\cite[\S 3, Proposition 3.10]{huybrechts2006fouriermukai}}]
    \label{Db(X) indecomposable for X connected}
    The bounded derived category $\derb{X}$ of a connected scheme $X$ over $k$ is indecomposable.
\end{theorem}

\begin{proof}
    For the sake of contradiction, assume $\derb{X}$ decomposes into triangulated subcategories $\D_0$ and $\D_1$. We will first prove that the structure sheaf $\struct{X}$ belongs to either $\D_0$ or $\D_1$.
    By Proposition \ref{In a decomposable cat every object splits}, the  structure sheaf $\struct{X}$ is isomorphic to a direct sum $\F_0^\bullet \oplus \F_1^\bullet$, with complexes $\F_i^\bullet$ belonging to $\D_i$.
    % Then the structure sheaf $\struct{X}$ is isomorphic to a direct sum $\F_0^\bullet \oplus \F_1^\bullet$ with complexes $\F_i^\bullet$ belonging to $\D_i$ by Proposition \ref{In a decomposable cat every object splits}. 
    Since $\struct{X}$ clearly has cohomology supported only in degree $0$, the same is true for $\F_0^\bullet$ and $\F_1^\bullet$, thus, by Proposition \ref{A -> D(A) fully faithful}, we may replace them with coherent sheaves $\F_0$ and $\F_1$, respectively. By the same proposition this means that there is an isomorphism $\struct{X} \iso \F_0 \oplus \F_1$ in $\coherent{X}$. As the functor computing stalks of a sheaf at any given point $x \in X$ is exact, we obtain a direct sum decomposition $\OO_{X,x} \iso \F_{0,x} \oplus \F_{1,x}$, enabling us to write
    \[
        X = \supp{\struct{X}} = \supp{\F_0} \cup \supp{\F_1}.
    \]
    As $\F_i$ are coherent sheaves, their supports are closed in $X$, according to Remark \ref{supports closed}. Since $\OO_{X,x}$ is a local ring, and therefore \emph{indecomposable}\footnote{
        A ring $R$ is \emph{indecomposable} if it cannot be written in the form $R \iso I \oplus J$ for two non-zero $R$-modules $I$ and $J$. If such a decomposition exists, $I$ and $J$ can be identified with \emph{proper} ideals of $R$. Now assuming $R$ is local with a maximal ideal $\mathfrak{m}$, we end up with a contradiction $R \subseteq I + J \subseteq \mathfrak{m} \subsetneq R$.
        % which are necessarily cyclic. Letting $e$ denote the generator of $I \trianglelefteq R$ and $f$ the generator of $J \trianglelefteq R$, we see that from $I \cap J = 0 $, $e$ and $f$ are idempotents. For the decomposition is non-trivial $e$ and $f$ are non-zero. If we now assume $R$ is local, we arrive at a contradiction as  
        % generated respectively by a pair of orthogonal idempotents. If $R$ is assumed to be a local ring
        % is not isomorphic to a direct sum of two non-trivial $R$-modules. 
    }, for every $x \in X$, either $\F_{0,x} = 0$ or $\F_{1,x} = 0$, meaning $\supp{\F_0} \cap \supp{\F_1} = \varnothing$. By connectedness of $X$, it follows that either $\F_0 = 0$ or $\F_1 = 0$, thus, without loss of generality, we may assume that $\struct{X}$ belongs to $\D_0$. 

    Secondly picking any closed point $x \in X$, we see that $k(x)$ belongs to $\D_0$ as well. Indeed, as with $\struct{X}$, we may decompose $k(x)$ as $k(x) \iso \F_0 \oplus \F_1$ in $\coherent{X}$ for some coherent sheaves $\F_i$ belonging to $\D_i$. Then firstly $\supp{\F_i} \subset \{x\}$ and computing the stalk at $x$ shows either $\F_0$ or $\F_1$ is trivial. If $\F_1 = 0$, $k(x)$ belongs to $\D_0$, but if $\F_0 = 0$, then the isomorphism $k(x) \iso \F_1$ allows us to construct a non-zero homomorphism of sheaves $\struct{X} \to k(x) \to \F_1$,
    % \footnote{The morphism $\struct{X} \to k(x)$ comes from the morphism $\spec{k} \to X$, which includes the closed point $x$ into $X$.}
    contradicting orthogonality of $\D_0$ and $\D_1$. In the latter composition the morphism $\struct{X} \to k(x)$ comes from the morphism $\spec{k} \to X$, which includes the closed point $x$ into $X$.

    To finish off the proof, we now show that $\D_1$ contains only trivial objects. Suppose $\F^\bullet$ is an object of $\D_1$. If $\F^\bullet$ is not trivial, let $m \in \Z$ be maximal with the property $\mathcal H^m(\F^\bullet) \neq 0$. Then $\supp{\mathcal H^m(\F^\bullet)} \neq \varnothing$ and there is a \emph{closed} point $x \in X$, which is also contained in $\supp{\mathcal H^m(\F^\bullet)}$\footnote{
        If $x_0 \in \supp{\mathcal H}$ is not a closed point, consider it inside an affine open chart $U \iso \spec{A}$. Then $\supp{\mathcal H} \cap U$ corresponds in $\spec{A}$ to a non-empty closed subset of the form $V(I)$ for some ideal $I \trianglelefteq A$. Since every ideal $I$ is contained in some maximal ideal, we obtain a closed point of $V(I)$ in this way.
    }. As before this means there is a non-zero map $\mathcal H^m(\F^\bullet) \to k(x)$. Using this we will construct a non-zero morphism $\F^\bullet \to k(x)[m]$ in $\derb{X}$, once again contradicting orthogonality of $\D_0$ and $\D_1$. By \eqref{right truncation isos on cohomology}, we know the inclusion $\rtruncation{m}(\F^\bullet) \to \F^\bullet$ is a quasi-isomorphism. If we view $\mathcal H^m(\F^\bullet)$ 
    % in one of its equivalent ways 
    as $\coker(\F^{m-1} \to \ker d^m)$, we obtain a commutative diagram
    \[\begin{tikzcd}[row sep = small, column sep = small]
        % https://q.uiver.app/#q=WzAsMTIsWzQsMCwiXFxrZXIgZF5tIl0sWzYsMCwiMCJdLFs0LDIsIlxcbWF0aGNhbCBIXm0oXFxtYXRoY2FsIEZeXFxidWxsZXQpIl0sWzYsMiwiMCJdLFsyLDAsIlxcbWF0aGNhbCBGXnttLTF9Il0sWzIsMiwiMCJdLFsxLDAsIlxcY2RvdHMiXSxbMCwyLCJcXG1hdGhjYWwgSF5tKFxcbWF0aGNhbCBGXlxcYnVsbGV0KVttXToiXSxbMSwyLCJcXGNkb3RzIl0sWzcsMCwiXFxjZG90cyJdLFs3LDIsIlxcY2RvdHMiXSxbMCwwLCJcXHJ0cnVuY2F0aW9ue219XFxtYXRoY2FsIEZeXFxidWxsZXQ6Il0sWzIsM10sWzAsMV0sWzEsM10sWzAsMl0sWzQsMF0sWzQsNV0sWzUsMl0sWzYsNF0sWzgsNV0sWzEsOV0sWzMsMTBdXQ==
        {\rtruncation{m}(\mathcal F^\bullet)} & \cdots & {\mathcal F^{m-1}} && {\ker d^m} && 0 & \cdots \\
        \\
        {\mathcal H^m(\mathcal F^\bullet)[m]} & \cdots & 0 && {\mathcal H^m(\mathcal F^\bullet)} && 0 & \cdots,
        \arrow[from=1-2, to=1-3]
        \arrow[from=1-3, to=1-5]
        \arrow[from=1-3, to=3-3]
        \arrow[from=1-5, to=1-7]
        \arrow[from=1-5, to=3-5]
        \arrow[from=1-7, to=1-8]
        \arrow[from=1-7, to=3-7]
        \arrow[from=3-2, to=3-3]
        \arrow[from=3-3, to=3-5]
        \arrow[from=3-5, to=3-7]
        \arrow[from=3-7, to=3-8]
    \end{tikzcd}\]
    which gives rise to a chain map $\rtruncation{m}(\F^\bullet) \to \mathcal H^m(\F^\bullet)[m]$. The left roof \eqref{eq: contrived roof} below then represents a non-zero morphism $\F^\bullet \to k(x)[m]$ in $\derb{X}$, as it is clearly non-zero once the $m$-th cohomology functor is applied.
    \begin{equation}
        \label{eq: contrived roof}
        \F^\bullet \xleftarrow{\ \ \sim \ \ } \rtruncation{m}(\F^\bullet) \to \mathcal H^m(\F^\bullet) \to k(x)[m]
    \end{equation}
    As this cannot be so, $\F^\bullet$ has trivial cohomology making it and the category $\D_1$ trivial. This contradicts our initial assumption of $\D_0$ and $\D_1$ forming a decomposition of $\derb{X}$ and proves that $\derb{X}$ is indecomposable. 
\end{proof}

\subsubsection*{Serre functors}

Let $X$ be a smooth projective scheme over $k$. In order to define a Serre functor on $\derb{X}$ we first establish finite-dimensionality of its Hom-sets. 

\begin{proposition}
    \label{Db(X) has finite dimensional homs}
    Let $X$ be a projective variety over $k$.
    For any two bounded complexes $\E^\bullet$ and $\F^\bullet$ of $\derb{X}$ the Hom-set $\Hom_{\derb{X}}(\E^\bullet, \F^\bullet)$ is a finite dimensional $k$-vector space. 
\end{proposition}

The above proposition relies on a finiteness result of Serre found in \cite[\S III, Theorem 5.2]{Hartshorne1977}. As we will actually only need a very special case of this result, we also include a separate reference to the special case \cite[\S II, Theorem 5.19]{Hartshorne1977}.

\begin{theorem}
    \label{Serre finitness}
    \emph{\cite[\S III, Theorem 5.2]{Hartshorne1977}}
    Let $X$ be a projective variety over $k$ and $\F$ a coherent sheaf on $X$. Then $\cohom{i}{X}{\F}$ are finite dimensional $k$-vector spaces for all $i \in \Z$.
\end{theorem}

% \info{I actually need a reference for $Ext^i(\F, \G)$ are finite dimensional or $\localext{i}{\F}{\G}{}$ are coherent for $\F$, $\G$ coherent.}

\begin{proof}[Proof of Proposition \ref{Db(X) has finite dimensional homs}]
    % We will show that $\Ext^i_{\struct{X}}(\E^\bullet, \F^\bullet)$ is finite dimensional using spectral sequences \eqref{eq: Spectral sequence Ext 2} and \eqref{eq: Spectral sequence Ext 1}.
    % For this proof we will momentarily borrow 
    Since $\derb{X} \to \derb{\quasicoh{X}}$ is fully faithful and $\quasicoh{X}$ contains enough injectives, Proposition \ref{Homs in derived category}, tells us that
    \[
        \Hom_{\derb{X}}(\E^\bullet, \F^\bullet) \iso \Ext^0_{\struct{X}}(\E^\bullet, \F^\bullet) = H^0(\Hom^\bullet_{\struct{X}}(\E^\bullet, \F^\bullet)).
    \]
    Recall that $\Hom^i_{\struct{X}}(\E^\bullet, \F^\bullet) = \bigoplus_{j \in \Z} \Hom_{\struct{X}}(\E^j, \F^{i+j})$, where the direct sum is actually finite due to $\E^\bullet$ and $\F^\bullet$ being bounded complexes. Momentarily borrowing the internal Hom of two sheaves from Section \ref{Subsubsection: Internal Hom}, \cite[II, \S 9, Proposition 9.1.1]{EGA} lets us state that the internal Hom of two coherent sheaves is again coherent.
    % From \cite[\href{https://stacks.math.columbia.edu/tag/01CQ}{Tag 01CQ}]{stacks-project}, 
    % we see that the internal Hom of two coherent sheaves is again coherent, 
    Thus for any two coherent sheaves $\F$ and $\G$ the module of global sections $\Gamma(X, \localhom_{\struct{X}}(\F, \G)) = \Hom_{\struct{X}}(\F, \G)$ is finite dimensional by Theorem \ref{Serre finitness}. In turn this implies $\Hom^i_{\struct{X}}(\E^\bullet, \F^\bullet)$ are finitely dimensional $k$-vector spaces for all $i \in \Z$, meaning $H^0(\Hom^\bullet_{\struct{X}}(\E^\bullet, \F^\bullet))$ is finite dimensional as well. 
\end{proof}

Recall that for a smooth and projective scheme $X$ over $k$ of dimension $n$, Serre duality \cite[\S III, Theorem 7.6]{Hartshorne1977} states that for any coherent sheaf $\F$ on $X$ there is an isomorphism
\begin{equation}
    \label{eq: Classic Serre duality}
    \Ext_{\struct{X}}^{n-i}(\F, \omega_X) \iso \cohom{i}{X}{\F}^*,
\end{equation}
where $\omega_X$ denotes the canonical bundle of $X$, \ie the $n$-th exterior power of the sheaf of Kähler differentials, $\omega_X = \bigwedge^n \Omega_X$. Generalizing Serre's duality theorem for smooth projective schemes over $k$ to the context of complexes leads us to define Serre functors. This also clarifies why we called such functors \emph{Serre functors} already in Section \ref{Subsection: Triangulated categories}, where no geometry was present. 

\begin{definition}
    \label{Serre functor on Db(X)}
    For a smooth and projective variety $X$ over $k$ define the \emph{Serre functor} $S_X \colon \derb{X} \to \derb{X}$ to be the composition
    \[
        S_X = (-) \otimes_\struct{X} \omega_X[\dim X].
    \]
\end{definition}

\begin{remark}
    This remark serves as a quick justification, why $S_X$ may be defined in this way. The canonical bundle $\omega_X$ is a line bundle, thus in particular flat, so the additive functor $(-) \otimes_\struct{X} \omega_X \colon \coherent{X} \to \coherent{X}$ is exact. The induced homotopy functor $\kb{\coherent{X}} \to \kb{\coherent{X}}$ then satisfies the conditions of Lemma \ref{F preserves qis iff preserves acyclic} and therefore induces a triangulated functor
    \[
        (-) \otimes_\struct{X} \omega_X \colon \derb{X} \to \derb{X}
    \]
    by Proposition \ref{Exact functor induces a derived functor}.
    The Serre functor $S_X$ is then obtained by composing the latter functor with the translation functor $\dim X$ times. 
    
    % formally induced by the triangulated functor on the homotopy category $\kb{X}$ given by tensoring with the canonical bundle $\omega_X$. 
\end{remark}

\begin{theorem}[Serre duality for derived categories, \text{\cite[\S 3.4, Remark 3.37]{huybrechts2006fouriermukai}}]
    \label{Serre duality for derived cats}
    Let $X$ be a smooth projective variety over a field $k$ and let $\E^\bullet$ and $\F^\bullet$ belong to $\derb{X}$. The functor $S_X$ is a Serre functor in the sense of Definition \ref{Definiton of Serre functor}, \ie there is an isomorphism
    \[
        \Hom_{\derb{X}}(\F^\bullet, \E^\bullet) \iso \Hom_{\derb{X}}(\E^\bullet, S_X(\F^\bullet))^*,
    \]
    which is natural in $\E^\bullet$ and $\F^\bullet$.
\end{theorem}

\begin{proof}
    See \cite[\S 3.4, Remark 3.37]{huybrechts2006fouriermukai} for an elegant proof using Grothendieck--Verdier duality. One is able to derive this version of Serre duality also from the classical one \eqref{eq: Classic Serre duality}. 
    % This will follow as a corollary of Grothendieck-Verdier duality \ref{}.
\end{proof}

% We will use Serre duality for example to help us prove 

\subsubsection*{A spanning class}

Suppose $X$ is a variety over an algebraically closed field $k$ and let $x \in X$ be a closed point. Then we know it is a $k$-point and there is a corresponding morphism of schemes over $k$, namely $x \colon \spec{k} \to X$. We define the skyscraper sheaf $k(x)$ to be the push-forward of the structure sheaf $\struct{\spec{k}}$ along the map $x$, 
\[
    k(x) := x_* \struct{\spec{k}}.
\]
The set of all such skyscrapers will turn out to form a spanning class for the triangulated category $\derb{X}$, when $X$ is smooth and projective.
The following proposition can be found in \cite[\S 3, Proposition 3.17]{huybrechts2006fouriermukai} or in Bridgeland's paper \cite[Example 2.2]{bridgeland2019equivalencestriangulatedcategoriesfouriermukai}. As we shall see, it is proven by a very neat application of the spectral sequence \eqref{eq: Spectral sequence derived hom 1}.

\begin{proposition}
    \label{k(x) spanning class}
    \emph{\cite[Example 2.2]{bridgeland2019equivalencestriangulatedcategoriesfouriermukai}}
    Suppose $X$ is a smooth projective variety over $k$. The set of all skyscraper sheaves $k(x)$, for closed points $x \in X$, forms a spanning class of the bounded derived category $\derb{X}$.
\end{proposition}

\begin{proof}
    By virtue of contradiction we will show that for any non-trivial complex $\F^\bullet$ of $\derb{X}$ there are closed points $x_0$, $x_1 \in X$ and integers $i_0$, $i_1 \in \Z$, for which
    \[
        \Hom_{\derb{X}}(\F^\bullet, k(x_0)[i_0]) \neq 0 \quad \text{and} \quad \Hom_{\derb{X}}(k(x_1), \F^\bullet[i_1]) \neq 0.
    \] 
    By Serre duality \ref{Serre duality for derived cats}, we see it suffices to find only $x_0$ and $i_0$, since
    \begin{align*}
        \Hom_{\derb{X}}(k(x_1), \F^\bullet[i_1]) 
        &\iso \Hom_{\derb{X}}(\F^\bullet[i_1], k(x) \otimes_{\struct{X}} \omega_X[\dim X])^* \\
        &\iso \Hom_{\derb{X}}(\F^\bullet, k(x)[\dim X - i_1])^*. 
    \end{align*}
    As $\F^\bullet$ is non-trivial, let $m \in \Z$ be maximal with the property that $\mathcal{H}^m(\F^\bullet) \neq 0$. As in the proof of Theorem \ref{Db(X) indecomposable for X connected}, pick a closed point in the non-empty support of $\mathcal H^m(\F^\bullet)$, for which we then know there is a non-zero morphism $\mathcal H^m(\F^\bullet) \to k(x)$ in $\derb{X}$. Consider now the spectral sequence \eqref{eq: Spectral sequence derived hom 1}
    \[
        E^{p,q}_2 = \Hom_{\derb{X}}(\mathcal H^{-q}(\F^\bullet), k(x)[p]) \Rightarrow \Hom_{\derb{X}}(\F^\bullet, k(x)[p+q]) = E^{p+q}.  
    \] 
    This is a spectral sequence lying in the right half plane, since $E^{p,q}_2 = 0$ for $p < 0$ as negative $\Ext$-modules are always trivial. A portion of the second page of the spectral sequence is depicted below.
    \vspace{0.2cm}
    \[\begin{tikzcd}[column sep = 1em, row sep = 0.5em]
        % https://q.uiver.app/#q=WzAsMjAsWzIsMiwiKiJdLFsxLDIsIkVeezAsIC1tfV8yIl0sWzMsMiwiKiJdLFsxLDEsIioiXSxbMiwxLCIqIl0sWzMsMSwiKiJdLFsxLDAsIioiXSxbMiwwLCIqIl0sWzMsMCwiKiJdLFswLDAsIjAiXSxbMCwxLCIwIl0sWzAsMiwiMCJdLFswLDMsIjAiXSxbMSwzLCIwIl0sWzIsMywiMCJdLFszLDMsIjAiXSxbMiw0LCIwIl0sWzMsNCwiMCJdLFswLDQsIjAiXSxbMSw0LCIwIl0sWzksMV0sWzEsMTZdXQ==
        0 & {*} & {*} & {*} \\
        0 & {*} & {*} & {*} \\
        0 & {E^{0, -m}_2} & {*} & {*} \\
        0 & 0 & 0 & 0 \\
        0 & 0 & 0 & 0
        \arrow[from=1-1, to=3-2]
        \arrow[from=3-2, to=5-3]
    \end{tikzcd}\]
    % \vspace{0.2cm}
    
    The highlighted arrows are the two trivial differentials, mapping into and out of $E^{0,-m}_2$, meaning that $E^{0,-m}_3$ is isomorphic to $E^{0,-m}_2$, thus also non-trivial. The same may be deduced for the $(0,-m)$-th term of each page, allowing us to conclude that $E^{0,-m}_\infty$ is non-trivial. The filtration for $E^{-m} = \Hom_{\derb{X}}(\F^\bullet, k(x)[-m])$ therefore contains a non-trivial intermediary quotient, making it itself non-trivial and proving our claim. 
\end{proof}

% \info{Grothendieck's vanishing results and Serre's finitness}

\subsection{Derived functors in algebraic geometry}
\label{Subsection: Derived functors in geometry}

In this section we will derive some functors occurring in algebraic geometry 
% and later state some important facts relating them with each other. 
To derive these functors we will pursue one of the two following options. Either deal with injective sheaves and access the realm of quasi-coherent sheaves, following the first part of Subsection \ref{Subsubsection: Derived functors of left exact functors}, or introduce certain special classes of coherent sheaves depending on the functor we wish to derive, which are going to take the role of adapted classes, mirroring what was done in the second part of that subsection.

\subsubsection{Global sections functor}
Arguably the most common functor in algebraic geometry is the \emph{global sections functor}. Associated to a noetherian scheme $X$, the functor of global sections
\[
    \Gamma \colon \mod{\OO_X} \to \mod{k}
\]
assigns to each $\OO_X$-module $\F$ its module of global sections $\F(X) = \Gamma(X, \F) = H^0(X, \F)$ and to a morphism of $\OO_X$-modules $\alpha\colon \F \to \G$ a homomorphism $\alpha_X \colon \F(X) \to \G(X)$.
By abuse of notation we use $\Gamma$ to also denote the restrictions of the global sections functor to subcategories $\quasicoh{X}$ and $\coherent{X}$ of $\mod{\OO_X}$. When we want to emphasise that $\Gamma$ is associated to a scheme $X$, we write $\Gamma(X, -)$ to denote the global sections functor on $X$.

As is commonly known, $\Gamma$ is a left exact functor, so our aim is to construct its right derived counterpart. Due to $\coherent{X}$ not having enough injectives, we resort to $\quasicoh{X}$\footnote{The scheme $X$ is noetherian, so $\quasicoh{X}$ contains enough injectives by Proposition \ref{qcoh of noetherian scheme contains enough injectives}}, on which we may define 
\[
    \rderived{}{\Gamma}\colon \derplus{\quasicoh{X}} \to \derplus{\mod{k}},
\]
according to method 1 of Subsection \ref{method 1}. Precomposing the latter functor with the inclusion $\derplus{\coherent{X}} \to \derplus{\quasicoh{X}}$ leaves us with the derived functor
\[
    \rderived{}{\Gamma} \colon \derplus{\coherent{X}} \to \derplus{\mod{k}}.
\]
Utilizing the cohomology functors, we obtain higher derived functors of $\Gamma$, denoted by
\[
    \rderived{i}{\Gamma} = \cohom{i}{X}{-},
\]
for $i \in \Z$. Applied to a complex of sheaves $\F^\bullet$ the resulting modules $\rderived{i}{\Gamma}(\F^\bullet) = H^i(X, \F^\bullet)$ are sometimes called the \emph{hypercohomology modules} of $\F^\bullet$.
% Higher derived functors of $\rderived{}{\Gamma}$ are denoted by $\rderived{i}{\Gamma}$ and sometimes also called the \emph{hypercohomology}. Applied to a 
% \[
%     \rderived{i}{\Gamma} = \cohom{i}{X}{-}.
% \]
Plugging in just a single coherent sheaf $\F$, we get the familiar \emph{sheaf cohomology} of $X$ with respect to a sheaf $\F$, denoted with $\cohom{i}{X}{\F}$. 

\begin{example}
    \label{LES in cohomology of SES of coherent sheaves}
    As a nice application of the abstract machinery of Chapter \ref{Chapter: Derived categories} we obtain the famous long exact sequence in cohomology 
    \begin{multline}
        \label{eq: LES in cohomology of SES of coherent sheaves}
        0 \to \cohom{0}{X}{\E} \to \cohom{0}{X}{\F} \to \cohom{0}{X}{\G} \to \cohom{1}{X}{\E} \to \cohom{1}{X}{\F} \to \cdots \\
        \cdots \to \cohom{i}{X}{\E} \to \cohom{i}{X}{\F} \to \cohom{i}{X}{\G} \to \cohom{i+1}{X}{\E} \to \cdots 
    \end{multline}
    associated to a short exact sequence of coherent sheaves $0 \to \E \to \F \to \G \to 0$. This is an immediate corollary of Proposition \ref{LES for right derived functor}.
\end{example}

The above definition of $\rderived{}{\Gamma}$ will suffice for most cases, but we can go slightly further, utilizing a vanishing result of Grothendieck, to be found in \cite[\S III, Theorem 2.7]{Hartshorne1977}.

\begin{theorem}
    \label{Grothendieck vanishig}
    \emph{\cite[\S III, Theorem 2.7]{Hartshorne1977}}
    Let $\F$ be a quasi-coherent sheaf on a noetherian scheme $X$ of dimension $n$. Then $H^i(X, \F) = 0$ for all $i > n$.
\end{theorem}
\noindent
If $X$ is a noetherian scheme of dimension $n$, the modules $\rderived{i}{\Gamma}(\F)$ are non-trivial only for finitely many $i \in \Z$, for any coherent sheaf $\F$. By Proposition \ref{Spectral sequences for derived cats} \ref{Bounded on objects implies bounded on complexes} we thus obtain a functor
\[
    \rderived{}{\Gamma} \colon \derb{X} \to \derb{\mod{k}}.
\]


\subsubsection{Push-forward $f_*$}
A generalization of the global sections functor is the \emph{push-forward} or the \emph{direct image} functor along a morphism of schemes $f \colon X \to Y$. The \emph{push-forward} functor
\[
    f_* \colon \mod{\struct{X}} \to \mod{\struct{Y}}
\]
is defined on objects by assigning an $\struct{X}$-module $\F$ the $\struct{Y}$-module $f_*\F$ defined on open subsets $V \subseteq Y$ to be $(f_*\F)(V) = \F(f\inv(V))$ with the obvious restriction homomorphisms of $\F$. On a morphism $\alpha \colon \F \to \G$ of $\struct{X}$-modules, $f_*$ is defined to act as $(f_*\alpha)_V = \alpha_{f\inv(V)}$ for all open $V \subseteq Y$. 

We are going to strengthen our initial assumption that our schemes are noetherian to them being of finite type over $k$ 
% Within the class of varieties over $k$ this assumption is satisfied by definition. 
and consider only morphisms $f \colon X \to Y$ between schemes of this kind. 
To induce a functor between the quasi-coherent and coherent categories of $X$ and $Y$, we need to recall the following proposition first.

\begin{proposition}
    \label{push-forward of coherent and quasi coherent shaves}
    \emph{\cite{Hartshorne1977,EGA}}
    Let $f \colon X \to Y$ be a morphism of schemes of finite type over $k$. Then for every quasi-coherent sheaf $\F$ on $X$, $f_*\F$ is quasi-coherent on $Y$. If $f$ is assumed to be projective, or more generally, proper and $\F$ is coherent on $X$, then $f_*\F$ is coherent on $Y$.
\end{proposition}

\begin{proof}
    The quasi-coherent part is \cite[\S II, Proposition 5.8]{Hartshorne1977}, because schemes of finite type over $k$ are noetherian. For the coherent part see \cite[\S II, Proposition 5.20]{Hartshorne1977} in the case that $f$ is projective and \cite[Part III, \S 3.2, Theorem 3.2.1]{EGA}, when $f$ is proper.
\end{proof}

It is a well known fact that $f_*$ is left exact, because it is a right adjoint and right adjoints preserve limits. As $\quasicoh{X}$ contains enough injectives, we may right derive $f_*$ to obtain 
\[
    \rderived{}{f_*} \colon \derplus{\quasicoh{X}} \to \derplus{\quasicoh{Y}}.
\]
This allows us to introduce the higher derived direct image functors
\[
    \rderived{i}{f_*} = \mathcal H^i \circ \rderived{}{f_*}
\] 
for all $i \in \Z$.

Next, we would like to introduce the derived push-forward also between the \emph{bounded} derived categories of quasi-coherent sheaves. The next proposition will enable us to do so. 

\begin{proposition}
    Let $f \colon X \to Y$ be a morphism of schemes of finite type over $k$ and let $\F$ be a coherent sheaf on $X$. Then $\rderived{i}{f_*}\F \iso 0$ for all $i > n$, where $n$ is the dimension of $X$.
\end{proposition}

\begin{proof}
    Let $\F \to \I^\bullet$ be an injective resolution of $\F$ and let $\delta$ denote the differential of $\I^\bullet$. Then 
    \[
        \rderived{i}{f_*}\F = \mathcal H^i(\rderived{}{f_*}\F) = \mathcal H^i (f_*\I^\bullet) = \coker(\im(f_*\delta^{i-1}) \to \ker(f_*\delta^i)).
    \]
    We recognise the cokernel at the end as the sheafification of the presheaf described on open subsets of $X$ by the assignment
    \begin{equation}
        \label{eq: presheaf of higher derived direct image functor}
        V \mapsto \coker(\im(\delta^{i-1}_{f\inv(V)}) \to \ker(\delta^i_{f\inv(V)})).
    \end{equation}
    For every open $V \subseteq Y$, the open subset $f\inv(V)$ determines a noetherian subscheme of $Y$ whose dimension is at most $n$. Therefore, by Grothendieck's vanishing Theorem \ref{Grothendieck vanishig}, the presheaf given by \eqref{eq: presheaf of higher derived direct image functor} is trivial for $i > n$, because 
    \begin{align*}
        \coker(\im(\delta^{i-1}_{f\inv(V)}) \to \ker(\delta^i_{f\inv(V)})) &= H^i(\Gamma(f\inv(V), \I^\bullet)) = \\
        &= \rderived{i}{\Gamma(f\inv(V), \I^\bullet)} \iso \cohom{i}{f\inv(V)}{\F}.
        \qedhere
    \end{align*} 
\end{proof}

\noindent
Applying Proposition \ref{Spectral sequences for derived cats} \ref{Bounded on objects implies bounded on complexes}, we may induce the derived direct image functor
\begin{equation}
    \label{eq: derived direct image on bounded quasicoherent derived cats}
    \rderived{}{f_*} \colon \derb{\quasicoh{X}} \to \derb{\quasicoh{Y}}.
\end{equation}
Finally, we obtain the derived direct image functor between the bounded derived categories of coherent sheaves formulated in the following proposition.
\begin{proposition}
    \label{push-forward on Db(X)}
    Let $f \colon X \to Y$ be a proper morphism between schemes $X$ and $Y$ of finite type over $k$. Then the derived direct image functor \eqref{eq: derived direct image on bounded quasicoherent derived cats} induces
    \[
        \rderived{}{f_*} \colon \derb{X} \to \derb{Y}.
    \]
\end{proposition} 

\begin{proof}
    By \cite[Part III, \S 3.2, Theorem 3.2.1]{EGA} all the higher direct images $\rderived{i}{f_*}\F$ of a coherent sheaf $\F$ are again coherent. As the category of coherent sheaves $\coherent{X}$ is thick in $\quasicoh{X}$, according to \cite[\S II, Proposition 5.7]{Hartshorne1977}, Proposition \ref{Spectral sequences for derived cats} \ref{Cohomology in a thick subcategory} tells us that the derived direct image $\rderived{}{f_*}\F^\bullet$ has coherent cohomology for any complex $\F^\bullet$ of $\derb{X}$. Functor $\rderived{}{f_*}$ thus maps complexes of $\derb{X}$ to complexes of $\mathsf{D}^b_{\mathsf{coh}}(\quasicoh{Y})$, inducing
    \[
        \rderived{}{f_*} \colon \derb{X} \to \derb{Y},
    \]
    by Proposition \ref{Db(X) is bounded derived of quasi-coherent with coherent cohomology}.
\end{proof}

\begin{remark}
    Whenever $f \colon X \to Y$ is an isomorphism of schemes of finite type over $k$, the push-forward functor $f_* \colon \coherent{X} \to \coherent{Y}$ is an equivalence of categories. Therefore it also induces an equivalence on the level of derived categories $\derb{X} \to \derb{Y}$.
    This nearly obvious but nonetheless important observation shows that $\derb{-}$ is an invariant of the isomorphism class of a scheme of finite type over $k$. 
\end{remark}

\begin{example}
    An example of a direct image functor we will encounter very often is pushing-forward along a projection $p \colon X \times Y \to X$ for smooth and projective varieties $X$ and $Y$ over a field $k$. In this case the projection $p$ is proper because the properness condition is stable under base change \cite[\href{https://stacks.math.columbia.edu/tag/01W4}{Tag 01W4}]{stacks-project} and the structure map $Y \to \spec{k}$ of a projective variety $Y$ is by definition proper. 
\end{example}

\subsubsection{Hom functors}
\label{Subsubsection: Internal Hom}

Let $X$ be a noetherian scheme over a field $k$. For any two quasi-coherent sheaves $\F$ and $\G$ on $X$, we denote with $\Hom_{\struct{X}}(\F, \G)$ the $k$-vector space of morphisms in the abelian category $\quasicoh{X}$. As with any category it gives rise to a functor
\[
    \Hom_{\struct{X}}(\F, -) \colon \quasicoh{X} \to \mod{k}
\]
for any quasi-coherent sheaf $\F$ on $X$. Since $\quasicoh{X}$ contains enough injectives by Proposition \ref{qcoh of noetherian scheme contains enough injectives}, we can derive this Hom functor according to Section \ref{subsection: Ext functors} to obtain
\[
    \rderived{}{\Hom_{\struct{X}}(\F, -)} \colon \derplus{\quasicoh{X}} \to \derplus{\mod{k}}.
\]
In this way we also obtain all its higher derived functors, namely the \emph{Ext-functors}
\[
    \Ext_{\struct{X}}^i(\F, -) = \rderived{i}{\Hom_{\struct{X}}(\F, -)}
\]
for all $i \in \Z$.


\subsubsection*{Internal Hom}

The \emph{internal Hom} or \emph{sheaf Hom} of two sheaves $\F$ and $\G$ on $X$ is defined to be a sheaf on $X$ given on open sets $U \subseteq X$ by
\[
    U \mapsto \Hom_{\struct{U}}(\F|_U, \G|_U),
\]
with the restriction homomorphisms being the usual restrictions of homomorphisms of sheaves. When $\F$ and $\G$ are taken to be $\struct{X}$-modules, the internal Hom may also be equipped with an $\struct{X}$-action making it an $\struct{X}$-module. We denote it by $\localhom_{\struct{X}}(\F, \G)$ when we want to clarify that we are working over a base space $X$ and $\localhom(\F, \G)$ otherwise. Applying the global sections functor $\Gamma(X, -)$ to $\localhom(\F, \G)$ it is clear that
\[
    \Gamma(X, \localhom(\F, \G)) = \Hom_{\struct{X}}(\F, \G).
\]
The internal Hom also defines a bifunctor
\begin{equation}
    \label{eq: internal hom on O_X-modules}
    \localhom(-, -) \colon \mod{\struct{X}}^\op \times \mod{\struct{X}} \to \mod{\struct{X}}.
\end{equation}
For a fixed $\struct{X}$-module $\G$ the $\localhom(-, \G)$-action on a homomorphism of $\struct{X}$-modules ${\alpha \colon \F_1 \to \F_0}$ is defined as the $\struct{X}$-module homomorphism
\[
    \alpha^* \colon \localhom(\F_0, \G) \to \localhom(\F_1, \G),
\]
given on open subsets $U \subseteq X$ by $\Gamma(U, \struct{X})$-linear maps
\[
    (\alpha^*)_U = \left(\Hom_{\struct{U}}(\F_0|_U, \G|_U) \xrightarrow{(\alpha_U)^*} \Hom_{\struct{U}}(\F_1|_U, \G|_U)\right).
\]
This action is clearly functorial and the same may be said for the similarly defined $\localhom(\F, -)$-action, for a fixed $\struct{X}$-module $\F$. By \cite[II, \S 9, Proposition 9.1.1]{EGA} the sheaf $\localhom_{\struct{X}}(\F, \G)$ is coherent for any two coherent sheaves $\F$ and $\G$ on $X$. Therefore \eqref{eq: internal hom on O_X-modules} may be restricted to a bifunctor
\[
    \localhom(-, -) \colon \coherent{X}^\op \times \coherent{X} \to \coherent{X}.
\]
% By Remark \ref{} 


% \[
%     \rderived{}{\localhom^\bullet(-, -)} \colon \derb{X}^\op \times \derb{X} \to \derb{X}
% \]

Analogous to the Hom complex of Section \ref{subsection: Ext functors} (Definition \ref{definition of Hom complex}) we can also define the following complex. Let $\F^\bullet$ and $\G^\bullet$ be bounded complexes of coherent sheaves on $X$. The \emph{internal Hom complex} $\localhom^\bullet(\F^\bullet, \G^\bullet)$ is then defined by its terms
\[
    \localhom^i(\F^\bullet, \G^\bullet) = \bigoplus_{j \in \Z} \localhom(\F^{j}, \G^{i+j})
\]
and differentials
\[
    \delta^i \colon \localhom^i(\F^\bullet, \G^\bullet) \to \localhom^{i+1}(\F^\bullet, \G^\bullet),
\]
which one can formally described as follows. Let $\pi_j \colon \localhom^i(\F^\bullet, \G^\bullet) \to \localhom(\F^{j}, \G^{i+j})$ denote the canonical projection onto the $j$-th component. Then $\delta^i$ is the unique map obtained by the universal property of the product $\localhom^{i+1}(\F^\bullet, \G^\bullet)$, induced by the family of morphisms
\[
    (d_{\G}^{i+j})_*\circ \pi_j + (-1)^i(d_\F^{j})^*\circ \pi_{j+1} \colon \quad  \localhom^i(\F^\bullet, \G^\bullet) \to \localhom(\F^{j}, \G^{i+j+1}).
\]
Not very different from how we showed in Section \ref{subsection: Ext functors} that the Hom complex induces a functor or even a bifunctor according to Remark \ref{bifunctors on the homotopy category}, we obtain a bifunctor 
\[
    \localhom^\bullet \colon \kminus{\coherent{X}}^\op \times \kplus{\coherent{X}} \to \kplus{\coherent{X}}.
\] 
When $X$ is smooth and projective and Proposition \ref{In Db(X) everything is a complex of vector bundles} is at ones disposal, it is possible to derive the internal Hom complex functor above to obtain a bifunctor
\[
    \rderived{}{\localhom^\bullet}(-,-) \colon \derb{X}^\op \times \derb{X} \to \derb{X}.
\]  
The main ingredients for how one goes about proving the above are described in \cite[75--77]{huybrechts2006fouriermukai}. On this note we also mention that Proposition \ref{In Db(X) everything is a complex of vector bundles} greatly simplifies how we compute
%  values of the functor 
$\rderived{}{\localhom^\bullet}(\F^\bullet,-)$ for some complex $\F^\bullet$ from $\derb{X}$. This is because $\F^\bullet$ can be replaced by an isomorphic (in $\derb{X}$) complex $\E^\bullet$ of vector bundles, for which $\rderived{}{\localhom^\bullet(\E^\bullet, -)}$ and $\localhom^\bullet(\E^\bullet, -)$ coincide. The \emph{dual} of a complex $\F^\bullet$ of $\derb{X}$ will be taken to be 
\[
    {\F^\bullet}^\vee := \rderived{}{\localhom}(\F^\bullet, \struct{X}).
\]
In general, even if $\F^\bullet$ is just a coherent sheaf concentrated in degree $0$, its dual may be an honest complex.

% We also establish notation ${\F^\bullet}^\vee$ to 

% For any complex $\F^\bullet$ belonging to $\derminus{X}$ one is then able to obtain by similar reasoning as in Section \ref{subsection: Ext functors} the following derived functor
% \[
%     \rderived{}{\localhom(\F^\bullet, -)} \colon \derb{X} \to \derb{X}
% \]



% In practice we will always be dealing with smooth projective varieties, so Proposition \ref{In Db(X) everything is a complex of vector bundles} will be at our disposal. 

\subsubsection{Tensor product}

For two $\struct{X}$-modules $\F$ and $\G$ their \emph{tensor product} $\F \otimes_{\struct{X}} \G$ is defined as the sheafification of the presheaf given on open subsets $U \subseteq X$ by
\[
    U \mapsto \Gamma(U, \F) \otimes_{\Gamma(U, \struct{X})} \Gamma(U, \G).
\] 
Using this description $\F \otimes_{\struct{X}} \G$ is equipped with the $\struct{X}$-action in the obvious way making it into an $\struct{X}$-module. Moreover this definition is functorial in both arguments $\F$ and $\G$ in the sense that a morphism of $\struct{X}$-modules $\alpha \colon \F_0 \to \F_1$ induces a morphism ${\alpha \otimes \id{\G} \colon} \F_0 \otimes_{\struct{X}} \G \to \F_1 \otimes_{\struct{X}} \G $ in a functorial way. The morphism $\alpha \otimes \id{\G}$ comes from the universal property of sheafification, such that for every open subset $U \subseteq X$ we have a commutative diagram
\[
    \begin{tikzcd}[column sep = 1.5em]
        % https://q.uiver.app/#q=WzAsNCxbMCwwLCJLKFxcQSkiXSxbMiwwLCJLKFxcQikiXSxbMCwyLCJEKFxcQSkiXSxbMiwyLCJEKFxcQikiXSxbMCwxLCJGIl0sWzAsMl0sWzIsM10sWzEsM11d
        {\Gamma(U, \F_0) \otimes_{\Gamma(U, \struct{X})} \Gamma(U, \G)} && {\Gamma(U, \F_0 \otimes_{\struct{X}} \G)} \\
        \\
        {\Gamma(U, \F_1) \otimes_{\Gamma(U, \struct{X})} \Gamma(U, \G)} && {\Gamma(U, \F_1 \otimes_{\struct{X}} \G)}
        \arrow["", from=1-1, to=1-3]
        \arrow["{\alpha_U \otimes (\id{\G})_U}"', from=1-1, to=3-1]
        \arrow["{(\alpha \otimes \id{\G})_U}", dashed, from=1-3, to=3-3]
        \arrow["", from=3-1, to=3-3]
    \end{tikzcd}
\]
In exactly the same way one also sees that the tensor product is functorial in the other variable. Hence the tensor product assembles a bifunctor
\[
    (-) \otimes_{\struct{X}} (-) \colon \mod{\struct{X}} \times \mod{\struct{X}} \to \mod{\struct{X}}.
\]
According to \cite[\href{https://stacks.math.columbia.edu/tag/01CE}{Tag 01CE}]{stacks-project} tensor product of two coherent sheaves is again coherent, so we also have a bifunctor 
\[
    (-) \otimes_{\struct{X}} (-) \colon \coherent{X} \times \coherent{X} \to \coherent{X}.
\]
Partially evaluating this functor at a coherent sheaf $\F$ on $X$ gives a functor ${\F \otimes_{\struct{X}} (-) \colon}$ $ \coherent{X} \to \coherent{X}$, which we know is left exact (for example due to the tensor-Hom adjunction \cite[\href{https://stacks.math.columbia.edu/tag/01CN}{Tag 01CN}]{stacks-project}). We also mention that tensor product commutes with the stalks functor \cite[\href{https://stacks.math.columbia.edu/tag/01CB}{Tag 01CB}]{stacks-project}, meaning that for any $x \in X$ there are natural isomorphisms
\[
    (\F \otimes_{\struct{X}} \G)_x \iso \F_x \otimes_{\struct{X,x}} \G_x.
\]
In order to derive the functor $\F \otimes_{\struct{X}} (-)$, we need an adapted class, which is provided by the following proposition.

\begin{proposition}
    \label{vector bundles adapted to tensor product}
    Let $X$ be a smooth projective variety and $\F$ a coherent sheaf on $X$. Then the class of vector bundles on $X$ forms a $\F \otimes_{\struct{X}} (-)$-adapted class.
\end{proposition}

\begin{proof}
    The class of vector bundles is obviously closed under finite direct sums and we already know that every coherent sheaf is a quotient of some vector bundle. Thus it remains to show that given an acyclic complex of vector bundles $\E^\bullet$, which is bounded from above, the tensor product $\F \otimes \E^\bullet$ is again acyclic. This is verified on stalks, where localizing at a point $x \in X$ gives
    \[
        \cdots \to \F_x \otimes_{\struct{X,x}} \E^{i-1}_x \to \F_x \otimes_{\struct{X,x}} \E^{i}_x \to \F_x \otimes_{\struct{X,x}} \E^{i+1}_x \to \cdots.
    \]
    This complex is acyclic, because the complex $\E^{\bullet}_x$ is acyclic and consists of free $\struct{X,x}$-modules.
\end{proof}

On a smooth projective variety $X$ Proposition \ref{vector bundles adapted to tensor product} allows us to derive $\F \otimes_{\struct{X}} (-)$ and obtain the left derived tensor product functor
\[
    \derivedtensor{\F}{(-)}{\struct{X}} \colon \derminus{X} \to \derminus{X}.
\]
Along with it also come its higher derived companions, called the \emph{internal Tor sheaves}
\[
    \localtor{i}{\F}{\G}{{\struct{X}}} = \mathcal{H}^{-i}(\derivedtensor{\F}{\G}{\struct{X}}).
\] 

Generalizing the tensor product of sheaves, we can define also a tensor product of complexes of sheaves in an analogous way to what was done with the internal Hom complex. For bounded complexes $\F^\bullet$ and $\G^\bullet$ of coherent sheaves on $X$ we introduce the \emph{internal tensor complex} $\F^\bullet \otimes_{\struct{X}} \G^\bullet$ to consist of terms
\[
    (\F^\bullet \otimes_\struct{X} \G^\bullet)^i = \bigoplus_{p + q = i} \F^p \otimes_{\struct{X}} \G^q
\]
and differentials $\delta^i \colon (\F^\bullet \otimes_\struct{X} \G^\bullet)^i \to (\F^\bullet \otimes_\struct{X} \G^\bullet)^{i+1}$ formally described as the unique map induced by the family of morphisms, indexed by $p$, $q$, satisfying $p + q = i+1$
\[
    (d_\F^{p-1} \otimes \id{\G^q})\circ \pi_{p-1,q} + (-1)^i(\id{\F^p} \otimes d_{\G}^{q-1})\circ \pi_{p,q-1} \colon \quad (\F^\bullet \otimes_\struct{X} \G^\bullet)^i \to \F^p \otimes_{\struct{X}} \G^q
\]
by universal property of the product $(\F^\bullet \otimes_\struct{X} \G^\bullet)^{i+1}$. Here $\pi_{p,q}$ denote the canonical projections $(\F^\bullet \otimes_\struct{X} \G^\bullet)^i \to \F^p \otimes_{\struct{X}} \G^q$ for $p + q = i$.

The contents of \cite[79--80]{Huybrechts2016} allow us to first form a bifunctor out of the internal tensor complex and provided we are working over a smooth projective variety $X$ derive it into a bifunctor
\[
    \derivedtensor{(-)}{(-)}{\struct{X}} \colon \derb{X} \times \derb{X} \to \derb{X}.
\]
We remark that due to Proposition \ref{In Db(X) everything is a complex of vector bundles} one can always replace bounded complexes of coherent sheaves with bounded complexes of vector bundles and compute the derived tensor product as the ordinary complex tensor product of the corresponding complexes of vector bundles. 

\subsubsection{Pull-back $f^*$}

Let $f \colon X \to Y$ be a morphism of smooth projective varieties. The \emph{pull-back} or the \emph{inverse image} functor along $f$ is defined to be 
\[
    f^* : \mod{\struct{Y}} \to \mod{\struct{X}} \qquad f^* = \struct{X} \otimes_{f\inv \struct{Y}} f\inv(-).
\]
As with the tensor product, the pull-back $f^*$ is also \emph{left exact} because it is a composition of an exact functor $f\inv \colon \mod{\struct{Y}} \to \mod{f\inv\struct{Y}}$ and a left exact functor ${\struct{X} \otimes_{f\inv\struct{Y}} (-)} \colon$ $\mod{f\inv\struct{Y}} \to \mod{\struct{X}}$. 

\begin{proposition}
    \label{pull-back of coherent sheaves}
    \emph{\cite[\S II, Proposition 5.8]{Hartshorne1977}}
    Let $f \colon X \to Y$ be a morphism of schemes. 
    \begin{enumerate}[label = (\roman*)]
        \item{If $\G$ is a quasi-coherent sheaf on $Y$, then $f^*\G$ is a quasi-coherent sheaf on $X$}
        \item{If $X$ and $Y$ are noetherian and $\G$ is a coherent sheaf on $Y$, then $f^*\G$ is a coherent sheaf on $X$.} 
    \end{enumerate}
\end{proposition}

% \begin{proof}
%     See \cite[\S II, Proposition 5.8]{Hartshorne1977}.
% \end{proof}

Clearly we may thus restrict the pull-back functor to the categories of coherent sheaves 
\[
    f^* \colon \coherent{Y} \to \coherent{X}.
\]
To derive it we will utilize the class of vector bundles on $X$.

\begin{proposition}
    \label{vector bundles f^* adapted}
    Let $f \colon X \to Y$ be a morphism of smooth projective varieties. Then the class of vector bundles on $Y$ forms an $f^*$-adapted class. 
\end{proposition}

\begin{proof}
    % Just as with The class of vector bundles is obviously closed under finite direct sums and we already know that every coherent sheaf is a quotient of some vector bundle. 
    Similarly as in the proof of Proposition \ref{vector bundles adapted to tensor product}, it remains to show that given a bounded above acyclic complex of vector bundles $\E^\bullet$ the pull-back $f^*\E^\bullet$ is still acyclic. This is verified on stalks, where one first sees that upon localizing at any $y \in Y$, the complex $\E^\bullet$ becomes an acyclic complex of free $\struct{Y,y}$-modules
    \[
        \cdots \to \E^{i-1}_y \to \E^i_y \to \E^{i+1}_y \to \cdots.
    \]
    Since localization and $f^*$ naturally commute, we see that the complex $f^*\E^\bullet$ localized at a point $x \in X$ becomes
    \[
        \cdots \to \struct{X,x} \otimes_{\struct{Y,f(x)}} \E^{i-1}_{f(x)} \to \struct{X,x} \otimes_{\struct{Y,f(x)}} \E^i_{f(x)} \to \struct{X,x} \otimes_{\struct{Y,f(x)}} \E^{i+1}_{f(x)} \to \cdots.
    \] 
    As $\E_{f(x)}^i$ are free $\struct{Y,f(x)}$-modules the sequence stays acyclic. This holds true for any $x \in X$, therefore $f^*\E^\bullet$ is acyclic.
    \qedhere
    
    % Since $f\inv$ is exact we only need to show that This follows from the fact that $f\inv\E^i$ are locally free $f\inv\struct{Y}$-modules and therefore $\struct{X} \otimes_{f\inv\struct{Y}}(-)$-acyclic and a very similar argument 

    % We show this inductively using the long exact sequence arising from the derived tensor product. Without loss of generality assume $\E^{i} \iso 0$ for $i > 0$. Let $\KK^{-1}$ denote the kernel of $d^{-1} \colon \E^{-1} \to \E^0$.
\end{proof}

\begin{corollary}
    \label{vector bundles f^* acyclic}
    Let $f \colon X \to Y$ be a morphism of smooth projective varieties. Then any vector bundle on $Y$ is $f^*$-acyclic. 
\end{corollary}

\begin{proof}
    This is a consequence of Proposition \ref{vector bundles f^* adapted} and Lemma \ref{F-adapted is F-acyclic}.
\end{proof}

Under the assumption that $f \colon X \to Y$ is a morphism of smooth projective varieties the presence of an $f^*$-adapted class ensured by Proposition \ref{vector bundles f^* adapted} allows us to derive the pull-back functor to obtain
\[
    f^* \colon \derminus{Y} \to \derminus{X}.
\]
Since any bounded complex in $\derb{Y}$ is isomorphic to a \emph{bounded} complex of vector bundles by Proposition \ref{In Db(X) everything is a complex of vector bundles} it is also clear that the functor above may be restricted to the bounded derived categories
\[
    f^* \colon \derb{Y} \to \derb{X}.
\]

\begin{remark}
    In many cases the morphism $f \colon X \to Y$ will be \emph{flat}. By definition this means that for any $x \in X$ the local ring $\struct{X, x}$ is flat as a $\struct{Y, f(x)}$-module, which in turn translates to the functor $\struct{X,x} \otimes_{\struct{Y,f(x)}} (-)$ being exact. For any such flat morphism $f$ the pull-back functor $f^* \colon \coherent{Y} \to \coherent{X}$ is exact and allows us to induce the derived functor $f^* \colon \derb{Y} \to \derb{X}$ simply by Proposition \ref{Exact functor induces a derived functor}. 
\end{remark}


\subsection{Interactions between derived functors}
\label{Subsection: Interactions between derived functors}

In this section we will present two interactions that appear between the derived functors of geometric origin. We will need them in the next chapter on Fourier--Mukai transforms, in particular when dealing with compositions of two such transforms. 

% Some of them will arise as derived incarnations of known formulas, which hold true already at the level of coherent sheaves. 

% While others are deep theorems, which are far from from being simple consequences of known facts. We will start with the former.

% \info{projection formula, base change, GV-duality}

% Define the functor $f^! \colon \derb{Y} \to \derb{X}$ to be
% \[
%     f^! = \lderived{}{f^*}(-) \otimes_{\struct{X}} \omega_f[\dim(f)],
% \]
% where $\omega_f = \omega_X \otimes f^*\omega_Y^*$ and $\dim(f) = \dim(Y) - \dim(X)$. As $\omega_f$ is just a line bundle on $X$, the tensor product need not be derived. 

% \begin{theorem}[Grothendieck-Verdier duality]
%     \label{Grothendieck-Verdier duality}
%     Let $X$ and $Y$ be smooth schemes over $k$. Let $f \colon X \to Y$ be a proper morphism. Then for every $\F^\bullet$ in $\derb{X}$ and $\E^\bullet$ of $\derb{Y}$, there is an isomorphism in $\derb{Y}$
%     \[
%         \rderived{}{f_*}\rderived{}{\localhom^\bullet_X}(\F^\bullet, f^! \E^\bullet) \iso \rderived{}{\localhom^\bullet_Y}(\rderived{}{f_*}(\F^\bullet), \E^\bullet),
%     \]
%     which is natural in $\F^\bullet$ and $\E^\bullet$.
% \end{theorem}

% \begin{proof}
%     See \cite[Chapter VII, \S 4, Corollary 4.3]{Hartshorne1966}.
% \end{proof}

% \begin{remark}
%     Serre duality in the form of Theorem \ref{Serre duality for derived cats} is seen to follow from Grothendieck-Verdier duality by the next argument.
% \end{remark}

\begin{proposition}[Flat base change formula, \text{\cite[\S 3.3]{huybrechts2006fouriermukai}}]
    \label{flat base change formula}
    Let the following be a pull-back square of smooth projective varieties
    \[\begin{tikzcd}[row sep = 2.8em]
        % https://q.uiver.app/#q=WzAsNCxbMCwwLCJYJyJdLFswLDEsIlgiXSxbMSwxLCJZIl0sWzEsMCwiWSciXSxbMSwyLCJmIl0sWzMsMiwiZyJdLFswLDEsInYiLDJdLFswLDMsInUiXSxbMCwyLCIiLDEseyJzdHlsZSI6eyJuYW1lIjoiY29ybmVyIn19XV0=
        {X'} & {Y'} \\
        X & Y
        \arrow["u", from=1-1, to=1-2]
        \arrow["v"', from=1-1, to=2-1]
        \arrow["\lrcorner"{anchor=center, pos=0.125}, draw=none, from=1-1, to=2-2]
        \arrow["g", from=1-2, to=2-2]
        \arrow["f", from=2-1, to=2-2]
    \end{tikzcd}\]
    Assuming $f\colon X \to Y$ is flat and $g \colon Y' \to Y$ is proper, then $u \colon X' \to Y'$ is flat and there is a natural isomorphism of functors $\derb{Y'} \to \derb{X}$
    \begin{equation}
        \label{eq: flat base change formula}
        f^*\circ \rderived{}{g_*} \natiso \rderived{}{v_*} \circ u^*.
    \end{equation}
\end{proposition}

\begin{proof}
    % See \cite[\href{https://stacks.math.columbia.edu/tag/02KH}{Tag 02KH}]{stacks-project}.
    See \cite[\S II.5, Proposition 5.12]{Hartshorne1966}.
\end{proof}

The following projection formula we will see is just a derived incarnation of the ordinary projection formula, which is known to hold for sheaves.

\begin{proposition}[Projection formula, \text{\cite[\S II.5, Proposition 5.6]{Hartshorne1966}}]
    \label{projection formula}
    Let $f \colon X \to Y$ be a proper map between projective schemes over $k$. Let $\F^\bullet$ and $\E^\bullet$ denote complexes belonging to $\derb{X}$ and $\derb{Y}$ respectively. Then there is an isomorphism of complexes in $\derb{Y}$
    \begin{equation}
        \label{eq: projection formula}
        \rderived{}{f_*}(\derivedtensor{\F^\bullet}{\lderived{}{f^*\E^\bullet}}{\struct{X}}) \iso \derivedtensor{\rderived{}{f_*}\F^\bullet}{\E^\bullet}{\struct{Y}}.
    \end{equation}
\end{proposition}

\begin{proof}
    We first recall the underived version of the projection formula, from which we will derive \eqref{eq: projection formula}. It states that for any morphism of $f \colon X \to Y$, coherent sheaf $\F$ on $Y$ and locally free sheaf $\E$ on $X$
    \begin{equation}
        \label{eq: underived projection formula}
        f_*(\F \otimes_{\struct{Y}} f^*\E) \iso f_*\F \otimes_{\struct{X}} \E.
    \end{equation}
    By Proposition \ref{In Db(X) everything is a complex of vector bundles} we may assume $\E^\bullet$ is a complex of locally free sheaves on $Y$, otherwise we just replace it by an isomorphic complex of this sort. Then $f^*\E^\bullet$ is a complex of locally free sheaves on $X$ (we can check local freeness on stalks because of the useful criterion found in \cite[\S 1, Lemma 5.1.3]{Weibel2013}). This already implies that it is enough to only show
    \[
        \rderived{}{f_*}({\F^\bullet}\otimes_{\struct{X}}{{f^*\E^\bullet}}) \iso {\rderived{}{f_*}\F^\bullet}\otimes_{\struct{Y}}{\E^\bullet},
    \]
    where the pull-back and the tensor product are underived. Let $\I^\bullet \to \F^\bullet$ be an injective resolution of $\F^\bullet$, momentarily accessing the category of quasi-coherent sheaves on $X$. Since $f^*\E^\bullet$ is locally free, we see that 
    \[
        \I^\bullet \otimes_{\struct{X}} f^*\E^\bullet \to \F^\bullet \otimes_{\struct{X}} f^*\E^\bullet
    \] 
    is an injective resolution of $\F^\bullet \otimes_{\struct{X}} f^*\E^\bullet$, because $(-) \otimes_\struct{X} f^*\E^\bullet$ is exact and tensoring injectives with locally free sheaves again results in an injective sheaf \cite[\href{https://stacks.math.columbia.edu/tag/01E7}{Tag 01E7}]{stacks-project}. This means that we can compute $\rderived{}{f_*}$ using this resolution. Utilizing \eqref{eq: underived projection formula} on each term of the complex now lets us see that 
    \[
        \rderived{}{f_*}({\F^\bullet}\otimes_{\struct{X}}{{f^*\E^\bullet}}) \iso
        f_*(\I^\bullet \otimes_{\struct{X}} f^*\E^\bullet) \iso f_*\I^\bullet \otimes_{\struct{Y}} \E^\bullet \iso
        {\rderived{}{f_*}\F^\bullet}\otimes_{\struct{Y}}{\E^\bullet}. \qedhere
    \]
\end{proof}
\newpage

\section{Fourier-Mukai transforms}

\begin{example}
    \label{Identifying fm transforms}
    % In this example we notice and identity some common functors from algebraic geometry appear as Fourier-Mukai transforms. 
\begin{enumerate}[label = (\roman*)]
    \item{
    $\id{\derb{X}} \iso \fm{\struct{\Delta_X}}$. Recall that $\struct{\Delta_X}$ is defined to be push-forward $\Delta_*\struct{X}$, of the structure sheaf of $X$ along the diagonal embedding $\Delta \colon X \to X \times X$. Then for any object $E$ of $\derb{X}$ we compute 
    % \begin{align*}
    %     \fm{\struct{\Delta_X}} &= \rderived{}{p_*}(\derivedtensor{\struct{\Delta_X}}{\lderived{}{p^*}(E)}{\struct{X \times X}}) \\
    %     &\iso \rderived{}{p_*}(\derivedtensor{\Delta_*\struct{X}}{\lderived{}{p^*}(E)}{\struct{X \times X}}) \\
    %     &\iso \rderived{}{p_*}(\derivedtensor{\Delta_*\struct{X}}{\lderived{}{\Delta^*}(\lderived{}{p^*}(E))}{\struct{X \times X}}) \\
    % \end{align*}
    \begin{align*}
        \fm{\struct{\Delta_X}} &= p_*(\struct{\Delta_X} \otimes_{\struct{X \times X}} p^*(E)) \\
        &= p_*(\Delta_*\struct{X} \otimes_{\struct{X \times X}} p^*(E)) \\
        &\iso p_*(\Delta_*(\struct{X} \otimes_{\struct{X}} \Delta^*p^*(E))) \\
        % &\iso \struct{X} \otimes E \\
        &\iso \id{\derb{X}}(E).
    \end{align*}
    All isomorphisms are natural in $E$, thus proving $\id{\derb{X}} \iso \fm{\struct{\Delta_X}}$.
    }
    \item{$(-)[1] \iso \fm{\struct{\Delta_X}[1]}$. A similar computation as above also identifies the translation functor as a Fourier-Mukai transform}
\end{enumerate}
\end{example}

\begin{theorem}[\text{\cite[Theorem 2.2]{Orlov-K3}}]
    \label{Orlov's theorem}
    Let $X$ and $Y$ be smooth projective varieties over a field $k$. Suppose $F \colon \derb{X} \to \derb{Y}$ is a fully faithful functor admitting a right (and, consequently, left) adjoint functor. Then there exists up to an isomorphism a unique object $E$ of $\derb{X \times Y}$ such that $F$ and $\fm{E}$ are naturally isomorphic.
\end{theorem}

\begin{proposition}
    \label{Composition of fm is fm}
    Let $\fm{\E^\bullet} \colon \derb{X} \to \derb{Y}$ and $\fm{\F^\bullet} \colon \derb{Y} \to \derb{Z}$ be Fourier-Mukai transforms with kernels $\E^\bullet$ and $\F^\bullet$ belonging to $\derb{X \times Y}$ and $\derb{Y \times Z}$ respectively. Then the composition $\fm{\F^\bullet} \circ \fm{\E^\bullet} \colon \derb{X} \to \derb{Z}$ is also a Fourier-Mukai transform with kernel $\mathcal R^\bullet = ...$, which is sometimes also called the \emph{convolution} of $\E^\bullet$ and $\F^\bullet$.
\end{proposition}

\subsection{on $K$-groups}


\subsection{on rational cohomology}
\label{Subsection: FM transform on cohomology}

\begin{proposition}
    \label{Composition of cohomological fm is fm}
    Let $\fmcoh{\E^\bullet} \colon H^\bullet(X, \Q) \to H^\bullet(Y, \Q)$ and $\fmcoh{\F^\bullet} \colon H^\bullet(Y, \Q) \to H^\bullet(Z, \Q)$ be cohomological Fourier-Mukai transforms for $\E^\bullet$ and $\F^\bullet$ belonging to $\derb{X \times Y}$ and $\derb{Y \times Z}$ respectively. Then the composition $\fmcoh{\F^\bullet} \circ \fmcoh{\E^\bullet}$ is also a cohomological Fourier-Mukai transform with kernel $\mathcal R^\bullet$ of Proposition \ref{}
\end{proposition}

\begin{theorem}[Grothendieck-Riemann-Roch]
    \label{Grothendieck-Riemann-Roch}
    Let
\end{theorem}

\begin{remark}
    What's the deal with Chow groups $A_\Q(X)$    
\end{remark}

\begin{corollary}[Hirzebruch-Riemann-Roch]
    \label{Hirzebruch-Riemann-Roch}
    \[
        \chi(X, \E) = \int_X \ch{\E}\tdclass{X}.
    \]
\end{corollary}

\begin{remark}
    Meaning of integral $\int_X \ch{\E}\tdclass{X} = \pairing{\ch{\E}\tdclass{X}}{[X]}$
\end{remark}

\begin{proof}
    from GRR.
\end{proof}

\begin{proposition}
    \label{cohomological projection formula}
    Let $f \colon X \to Y$ be a continuous map between closed orientable manifolds $X$ and $Y$. Then for all $\alpha \in \cohom{*}{X}{\Z}$ and $\beta \in \cohom{*}{Y}{\Z}$
    \[
        f_*\alpha \smallsmile \beta = f_*(\alpha \smallsmile f^*\beta).
    \]
\end{proposition}

\begin{proof}
    
\end{proof}

\begin{proposition}
    \label{Hodge lattice, fm transform interaction}
    Then for every pair of integers $(p,q)$
    \begin{equation}
        \label{eq: Hodge lattice, fm transform}
        \fmcoh{E}_\C\left(H^{p,q}(X)\right) \subseteq \bigoplus_{s - t = p - q} H^{s,t}(Y).
    \end{equation}
\end{proposition}
\newpage

\section{K3 surfaces}
\label{Chapter: K3}

This chapter serves as a gentle introduction and overview of the main properties of K3 surfaces. We will start off by defining algebraic K3 surfaces and their complex counterparts. Next, using a combination of both viewpoints we will derive the most important invariants of K3 surfaces, beginning with the computation of the Chern and Todd classes, which will enable us to compute the so-called Hodge diamond of a K3 surface. Using some topological methods we will identify all the integral cohomology groups. Along the way we will introduce the Picard and Néron--Severi group and see that they are isomorphic. We will conclude the chapter by examining the intersection pairing and finally stating the global Torelli theorem at the end characterizing K3 surfaces through their Hodge lattices. 
% staring with computing their integral cohomology groups using the exponential sequence. We will shed some light on the Hodge decomposition and compute the so-called Hodge diamond of a K3 surface. We will conclude the chapter with examining the intersection pairing and introducing the Picard and Neron-Severi groups. 

% \vspace{0.15cm}
% \noindent
% Throughout this chapter we will be accessing the language of lattice theory. For its main themes we refer the reader to Appendix \ref{appendix B}.

\subsection{Algebraic and complex}

K3 surfaces come in two flavours, algebraic and complex, essentially equivalent, but still formally different to allow us to exploit both the methods of algebraic geometry as well as topology in combination with complex geometry. We start with the algebraic ones.
% but formally allowing us to effortlessly pass between the realms of algebraic and complex geometry. 

\begin{definition}
    A smooth projective surface $X$ over a field $k$ is a \emph{K3 surface} if 
    \[
        \omega_X \iso \struct{X} \quad \text{and} \quad \cohom{1}{X}{\struct{X}} = 0.
    \]
\end{definition}

\begin{example}
    Let $X$ be a smooth quartic in $\PP^3$, meaning a smooth projective variety given by some homogeneous polynomial equation of degree four.
    We will show that any such variety is a K3 surface. Let $f \in \Gamma(\PP^3, \struct{\PP^3}(4))$ be the defining homogeneous polynomial for $X$ of degree four.
    % We will now show that any such smooth quartic $X$ in $\PP^3$, defined by some homogeneous degree $4$ polynomial $f \in \Gamma(X, \struct{\PP^3}(4))$, is a K3 surface. 
    At this point we recall that $\struct{\PP^3}(n)$ for all $n \in \Z$ denote \emph{Serre's twisting sheaves} on the projective space $\PP^3$ and refer the reader to \cite[\S II.5]{Hartshorne1977} for its construction.
    Since $X$ is a hypersurface in $\PP^3$ it is of dimension two, \ie a surface. Let $i \colon X \hookrightarrow \PP^3$ denote the closed embedding of $X$ into $\PP^3$ and consider the following short exact sequence of coherent sheaves on $\PP^3$
    \[
        0 \to \struct{\PP^3}(-4) \to \struct{\PP^3} \to i_*\struct{X} \to 0.
    \]
    The first non-trivial morphism is given by multiplication by $f$.
    %  and is clearly injective. 
    The second non-trivial morphism arises as the unit of the adjunction $i\inv \dashv i_*$, since by definition we set $\struct{X}$ to be $i\inv\struct{\PP^3}$.
    % It is then also clear that this map is surjective by performing the verification on stalks. The exactness in the middle is again verified on stalks and follows from the facts that $f_x \in \struct{\PP^3,x}$ is invertible and $(i_*\struct{X})_x = 0$ if $x \notin X$ and that 
    % $f_x \in \struct{\PP^3,x}$ is either invertible if $x \notin X$ or $0$ if $x \in X$ and the fact that the unit $\struct{\PP^3} \to i_*\struct{X}$ is an isomorphism if $x \in X$ and $0$ otherwise.
    First consider a portion of the long exact sequence in cohomology \eqref{eq: LES in cohomology of SES of coherent sheaves}
    \[
        \cdots \to \cohom{1}{\PP^3}{\struct{\PP^3}} \to \cohom{1}{\PP^3}{i_*\struct{X}} \to \cohom{2}{\PP^3}{\struct{\PP^3}(-4)} \to \cdots.
    \]
    Since the groups on the edge are known to vanish by \cite[\S III, Theorem 5.1]{Hartshorne1977}, the middle one does as well. Noting that the push-forward $i_*$ is exact, since $i$ is an embedding, we see that Proposition \ref{composition of derived functors} yields
    \begin{align*}
        \cohom{1}{\PP^3}{i_*\struct{X}} &=
        H^1(\rderived{}{\Gamma_{\PP^3}}(i_*\struct{X})) \\
        &\iso H^1((\rderived{}{\Gamma_{\PP^3} \circ i_*})(\struct{X})) \\
        &\iso H^1(\rderived{}{\Gamma_X}(\struct{X})) \\
        & = \cohom{1}{X}{\struct{X}}.
    \end{align*}
    Therefore $\cohom{1}{X}{\struct{X}} = 0$. Triviality of the canonical bundle $\omega_X$ is seen from the \emph{conormal sequence}
    \begin{equation}
        \label{eq: conormal SES}
        0 \to \struct{X}(-4) \to i^*\Omega_{\PP^3} \to \Omega_X \to 0.
    \end{equation}
    The first non-trivial morphism in this short exact sequence can roughly be described by sending a section $s$ of $\struct{X}(-4)$ to $d(f \cdot s)$. The second non-trivial morphism can be interpreted as sending a section of the form $d\varphi$ of $i^*\Omega_{\PP^3}$ to $d(\varphi|_X)$. 

    Recall that for a locally free sheaf $\E$ of rank $n$ its determinant line bundle is defined as $\det(\E) = \bigwedge^n \E$. Moreover, we know that for a short exact sequence of locally free sheaves $0 \to \E_0 \to \E_1 \to \E_2 \to 0$, we have $\det(\E_1) \iso \det(\E_0) \otimes \det(\E_2)$. Thus taking determinants of \eqref{eq: conormal SES}, we obtain
    \[
        i^*\omega_{\PP^3} \iso \struct{X}(-4) \otimes \omega_X.
    \]
    Upon tensoring with the inverse of the pull-back of $\omega_{\PP^3} \iso \struct{\PP^3}(-4)$, we conclude with $\omega_X \iso \struct{X}$, which shows that $X$ is a K3 surface.
\end{example}

\begin{example}        
    A very concrete example of a smooth quartic in $\PP^3$ is the \emph{Fermat quartic}\footnote{It is easily seen to be smooth by the Jacobian criterion for smoothness.}, which is a surface cut out by the homogeneous polynomial equation 
    \[
        x_0^4 + x_1^4 + x_2^4 + x_3^4 = 0.
    \]
\end{example}

Complex analytic K3 surfaces come with essentially the same definition as the algebraic ones. Of course we are now interpreting the structure sheaf $\struct{X}$ as the sheaf of \emph{holomorphic} functions on $X$ and $\omega_X$ as the holomorphic canonical bundle.  

\begin{definition}
    \label{Definition of complex K3}
    A connected complex projective manifold $X$ of dimension two is a \emph{K3 surface} if  
     \[
        \omega_X \iso \struct{X} \quad \text{and} \quad \cohom{1}{X}{\struct{X}} = 0.
    \]
\end{definition}

\begin{remark}
    To give a little intuition about the condition $\cohom{1}{X}{\struct{X}} = 0$ in the definition, we will shortly see that also $\cohom{1}{X}{\Z} = 0$. The latter is a necessary condition for $X$ to be simply connected, however not sufficient. But in the case of complex analytic K3 surfaces all of them turn out to be simply connected. This is a deep fact we will not go into, but mention that it is proven using another profound result, namely that all analytic K3 surfaces are diffeomorphic and then showing that the Fermat quartic is simply connected. More on this can be found in \cite[\S 7.1]{Huybrechts2016}.
\end{remark}

As already mentioned in the introduction to Section \ref{Subsection: FM transform on cohomology} one can pass between the algebraic and the analytic categories using Serre's GAGA principle \cite{Serre1956}. We also mentioned the existence of an equivalence between categories of coherent sheaves on the algebraic side and analytic coherent sheaves on the complex analytic side preserving cohomology. This passage is also well behaved in the sense that the associated analytic coherent sheaf to the structure sheaf $\struct{X}$ is the sheaf of holomorphic functions $\struct{X\an}$ on $X\an$ and the sheaf associated to the sheaf of Kähler differentials $\Omega_X$ on $X$ is the sheaf of holomorphic 1-forms $\Omega^1_{X\an}$ on $X\an$. From this we see that the complex analytic space associated to an algebraic K3 surface is a complex K3 surface in the sense of Definition \ref{Definition of complex K3}.

\begin{remark}
    The \emph{projective} assumption is usually not present in the definition. In fact it is replaced by a more general assumption of compactness. We made this choice, because we will only be dealing with algebraic K3 surfaces and thus every complex analytic K3 surface will have its algebraic counterpart. It is also worth noting that there do exist non-projective complex K3 surfaces and that they show up in important places. One of which is in a proof of the global Torelli theorem \ref{Classical Torelli theorem}, which will make an appearance at the end of this chapter, but whose proof we omit, instead referring to \cite[\S 7]{Huybrechts2016}. 
\end{remark}









\noindent
\textsl{Notation.}
K3 surfaces are complex manifolds therefore orientable, implying $\cohom{4}{X}{\Z} \iso \Z$. Until the rest of the chapter we fix a generator $\mu_X \in \cohom{4}{X}{\Z}$ for which $\pairing{\mu_X}{[X]} = 1$, where $[X] \in H_4(X, \Z)$ is a chosen fundamental class of $X$. Throughout the remainder of this chapter we will be accessing the language of lattice theory. For its main themes we refer the reader to Appendix \ref{appendix B}.

\subsection{Main invariants}

The first vital tool for our examination is Serre duality \eqref{eq: Classic Serre duality}. Given a vector bundle $\E$ on $X$ it gives us isomorphisms $\Ext_{\struct{X}}^{2-i}(\E, \struct{X}) \iso \cohom{i}{X}{\E}^*$ for all $i \in \Z$. Here we have already used a defining property of a K3 surface, namely that $\omega_X \iso \struct{X}$. The left side of the duality can be slightly modified to obtain 
\begin{equation}
    \label{eq: Serre duality on K3}
    \cohom{2-i}{X}{\E^\vee} \iso \cohom{i}{X}{\E}^*.
\end{equation}
% This already allows us to compute nearly all the Hodge numbers. We have 
Next we recall the general fact that the pairing
\[
    \Omega_X \otimes_\struct{X} \Omega_X \to \bigwedge^2\Omega_X
\]
is \emph{perfect}. By definition this means that the morphism $\Omega_X \to \localhom[X](\Omega_X, \bigwedge^2 \Omega_X)$, associated to the pairing above, is an isomorphism. Since the canonical bundle $\omega_X = \bigwedge^2 \Omega_X$ of a K3 surface $X$ is trivial, we see that 
\[
    \Omega_X \iso \Omega_X^\vee = \tangent{X}.
\]
This useful observation will be key in the computation of Chern and Todd classes of K3 surfaces. 

\begin{lemma}
    \label{characteristic classes of K3}
    \emph{\cite[\S 4, Lemma 4.7]{vanBree2020}}
    Let $X$ be a complex K3 surface, then
    % and let $\mu_X \in \cohom{4}{X}{\Z}$ denote the generator for which $\pairing{\mu_X}{[X]} = 1$. Then
    \[
        \cclass[0](X) = 2, \qquad \cclass[1](X) = 0, \qquad \cclass[2](X) = 24\mu_X,  \qquad \tdclass{X} = 1 + 2\mu_X.
    \]
\end{lemma}

\begin{proof}
    Tangent sheaf $\tangent{X}$ is a vector bundle of rank $2$, so $\cclass[0](X) = 2$. Since the sheaf of differentials $\Omega_X$ is self-dual, the tangent sheaf $\tangent{X}$ is as well. First, we see that $\ch[1](\tangent{X}) = \cclass[1](X)$. Then property \ref{chern character and dual} of Definition \ref{Definition of Chern character} applied to $\tangent{X} \iso \tangent{X}^\vee$ shows that $\cclass[1](X) = -\cclass[1](X)$. In Proposition \ref{cohomology of K3} we will see that $\cohom{2}{X}{\Z}$ is torsion-free, thus we reach $\cclass[1](X) = 0$. The second Chern class is computed using the Hirzebruch--Riemann--Roch formula applied to the trivial line bundle $\struct{X}$. Utilizing \eqref{eq: Serre duality on K3} we compute its Euler characteristic to be 
    \[
        \chi(X, \struct{X}) = h^0(X, \struct{X}) - h^1(X, \struct{X}) + h^2(X, \struct{X}) = 1 - 0 + 1 = 2.
    \]
    Noting that $\ch[]{\struct{X}} = 1$ and simplifying \eqref{eq: todd class} to $\td_X = 1 + \frac{1}{12}\cclass[2]{X}$, Theorem \ref{Hirzebruch-Riemann-Roch} implies
    \[
        2 = \chi(X, \struct{X}) = \int_X \ch[]{\struct{X}}\tdclass{X} = \pairing{\tfrac{1}{12}\cclass[2]{X}}{[X]}.
    \]
    Thus $\cclass[2](X) = 24 \mu_X$ and $\tdclass{X} = 1 + 2\mu_X$.
\end{proof}

For later we record the following K3 version of the Hirzebruch--Riemann--Roch formula. Let $\E$ be a coherent sheaf on $X$ then 
\begin{equation}
    \label{eq: HRR for K3}
    \chi(X, \E) = \pairing{\ch[2](\E)}{[X]} + 2\rk{\E} = \pairing{\tfrac{1}{2}(\cclass[1]{\E}^2 - 2\cclass[2]{\E})}{[X]} + 2 \rk{\E}.
\end{equation}
This already allows us to compute the Hodge numbers of a K3 surface $X$.

\begin{proposition}
    \emph{\cite[\S 1.2.4]{Huybrechts2016}}
    Let $X$ be a K3 surface. Arranged in a so-called \emph{Hodge diamond} the Hodge numbers of $X$ are the following:
    \begin{center}
        \begin{tabular}{c c c}
            \begin{tabular}{ccccc}
                &  & $h^{2,2}$ &  &  \\
                & $h^{2,1}$ &  & $h^{1,2}$ &  \\
                $h^{2,0}$ &  & $h^{1,1}$ &  & $h^{0,2}$ \\
                & $h^{1,0}$ &  & $h^{0,1}$ &  \\
                &  & $h^{0,0}$ &  &  \\
            \end{tabular} & \qquad &
            \begin{tabular}{ccccc}
                &  & $\ 1 \ $ &  &  \\
                & $\ 0 \ $ &  & $\ 0 \ $ &  \\
                $\ 1 \ $ &  & $20$ &  & $\ 1 \ $ \\
                & $0$ &  & $0$ &  \\
                &  & $1$ &  &  \\
            \end{tabular}
        \end{tabular}
    \end{center}
\end{proposition}

\begin{proof}
    Due to the symmetry $h^{p,q} = h^{q,p}$ coming from $\olsi{H^{p,q}(X)} = H^{q,p}(X)$, we will only compute $h^{p,q}$ for $p \geq q$. Starting from the bottom $h^{0,0}(X) = h^0(X, \struct{X}) = 1$. By definition we have $h^{1,0}(X) = h^1(X, \struct{X}) = 0$. By Serre duality \eqref{eq: Serre duality on K3} we get $h^{2,0}(X) = h^2(X, \struct{X}) = h^{0}(X, \struct{X}) = 1$ and $h^{2,2}(X) = h^{2}(X, \Omega_X^2) = h^2(X, \struct{X}) = h^0(X, \struct{X}) = 1$. Next $h^{2,1}(X) = h^{1,2}(X) = h^1(X, \Omega_X^2) = h^1(X, \struct{X}) = 0$. Finally, for the center of the diamond, we use the Hirzebruch--Riemann--Roch formula \eqref{eq: HRR for K3} applied to the tangent sheaf $\tangent{X} \iso \Omega_X^\vee \iso \Omega_X$, to obtain
    \begin{align*}
        h^{0,1}(X) - h^{1,1}(X) + h^{2,1}(X) 
        % &= \chi(X, \Omega_X) = \\ 
        =\pairing{\tfrac{1}{2}(\cclass[1]{\Omega_X}^2 - 2\cclass[2]{\Omega_X})}{[X]} + 2 \rk{\Omega_X} = -20.
    \end{align*}
    Thus, since $h^{0,1}(X) = h^{2,1}(X) = 0$, we get $h^{1,1}(X) = 20$.
\end{proof}

Notice that Proposition \ref{Hodge lattice, fm transform interaction}, says in particular that a Fourier--Mukai transform, which arises from an equivalence of derived categories, preserves sums of all the columns in the Hodge diamond. As a simple corollary of the Hodge numbers computed above, we prove that the bounded derived category $\derb{-}$ can ``detect'' K3 surfaces.
% As a consequence of the Hodge numbers just computed and Proposition \ref{Hodge lattice, fm transform interaction}, saying in particular that a Fourier--Mukai transform, which is an isomorphism, preserves columns of the Hodge diamond, we prove that $\derb{-}$ ``detects K3 surfaces''.

\begin{theorem}
    \label{Db(-) detects K3}
    \emph{\cite[\S 16, Proposition 2.1]{Huybrechts2016}}
    Suppose $X$ and $Y$ are smooth projective varieties over $\C$. Assume $X$ is a K3 surface and
    \[
        \derb{X} \natiso \derb{Y}.
    \]
    Then $Y$ is a K3 surface as well. 
\end{theorem}

\begin{proof}
    By Theorem \ref{Db detects dimension and triviality of canonical bundle}, we already know $Y$ is a surface with a trivial canonical bundle. It is left to show that $\cohom{1}{Y}{\struct{Y}} = 0$. By Proposition \ref{Hodge lattice, fm transform interaction}, we see that 
    \[
        h^{0,1}(X) + h^{1,2}(X) = h^{0,1}(Y) + h^{1,2}(Y). 
    \]
    However $h^{0,1}(X) = 0$ and $h^{1,2}(X) = h^{2,1}(X) = 0$.
    %  \dim \cohom{1}{X}{\omega_X} = \dim \cohom{1}{X}{\struct{X}} = 0$. 
    Thus $h^{0,1}(Y) = 0$ proves that $Y$ is a K3 surface.
\end{proof}

\subsubsection*{Cohomology and intersection pairing}

Now we turn our attention to cohomology of K3 surfaces. The passage to complex K3 surfaces will unlock techniques of singular and Čech cohomology, which will turn out to be really useful here. Fluidly transitioning to the complex analytic side of things, let $X$ be a complex K3 surface, let $\struct{X}$ denote the sheaf of holomorphic functions on $X$ and define $\OO_X^*$ to be the sheaf of nowhere vanishing holomorphic functions on $X$. Consider the \emph{exponential sequence}
\[
    0 \to \Z \to \struct{X} \to \OO_X^* \to 0.
\]
The first term $\Z$ is seen as the sheaf of locally constant functions taking values in the ring of integers. The first non-trivial morphism is just the inclusion and the second one is given locally by $f \mapsto e^{2\pi i f}$. The associated long exact sequence in
%  Čech
cohomology then reads
\begin{equation}
    \label{eq: LES of exponential SES}
    \cdots \to \cohom{i}{X}{\Z} \to \cohom{i}{X}{\struct{X}} \to \cohom{i}{X}{\OO^*_X} \to \cohom{i+1}{X}{\Z} \to \cdots.
\end{equation}
% We will return to it momentarily, but presently focus on the connecting morphism
% \[
%     \cdots \to \cohom{1}{X}{\OO_X^*} \xrightarrow{\delta} \cohom{2}{X}{\Z} \to \cdots.
% \]

\begin{definition}
    The \emph{Picard group} of a complex K3 surface $X$ is defined to be the group of isomorphism classes of line bundles on $X$ with multiplication given by the tensor product of sheaves. We denote it with $\pic{X}$.
\end{definition}

\begin{remark}
    % Using \emph{Čech's} viewpoint on cohomology 
    Based on the cocycle representation of line bundles, one is able to construct a natural isomorphism from $\pic{X}$ to the first \emph{Čech cohomology} group $H^1(X,\OO_X^*)$ \cite[\S 4, Theorem 4.49]{Voisin2002}.
    %  with coefficients in the sheaf $\OO_X^*$ of nowhere vanishing holomorphic functions. 
    The Čech cohomology groups are also known to be isomorphic to the usual sheaf cohomology groups we have been using all along \cite[\S 4, Theorem 4.44]{Voisin2002}.
\end{remark}

Based on the previous remark we are able to bring new meaning to the second connecting homomorphism $\cohom{1}{X}{\OO_X^*} \to \cohom{2}{X}{\Z}$ of \eqref{eq: LES of exponential SES}. 
% Under the composition isomorphism $\pic{X} \iso \cohom{1}{X}{\OO_X^*}$ it turns out to be precisely the first Chern class $\cclass[1]$
The value of a line bundle $\mathcal L$ on $X$ under the composition
\[
    \pic{X} \xrightarrow{\ \sim \ } \cohom{1}{X}{\OO_X^*} \longrightarrow \cohom{2}{X}{\Z}
\]
% turns out to result 
can actually be taken to be the definition of its \emph{first Chern class} $\cclass[1]{\mathcal L}$.
% is actually precisely the definition of its first Chern class $\cclass[1]{\mathcal L}$.
\begin{definition}
    The image of the connecting homomorphism $\cohom{1}{X}{\OO_X^*} \to \cohom{2}{X}{\Z}$ is defined to be the \emph{Néron--Severi group} of $X$, denoted by $\NS{X}$. 
\end{definition}

\begin{remark}
    By an already listed property of characteristic classes, when discussing Hodge decompositions in Section \ref{Subsection: FM transform on cohomology}, we note that the Néron--Severi group $\NS{X}$ can be seen as a subgroup of $H^{1,1}(X) \cap \cohom{2}{X}{\Z}$. The converse inclusion is also true and follows from the Lefschetz Theorem on $(1,1)$-classes \cite[\S 1.2]{GriffithsHarris1994}. Thus we have an equality
    \begin{equation}
        \label{eq: NS = 1,1 cap 2nd coh}
        \NS{X} = H^{1,1}(X) \cap \cohom{2}{X}{\Z}.
    \end{equation}
\end{remark}

\begin{proposition}
    \label{picard group torsion free}
    \emph{\cite[\S 1, Proposition 2.4]{Huybrechts2016}}
    Let $X$ be a complex K3 surface. The Picard group $\pic{X}$ and the Néron--Severi group $\NS{X}$ are isomorphic. They are torsion-free and carry a symmetric non-degenerate integral bilinear form called the intersection form. 
    % called the \emph{intersection pairing}. 
\end{proposition}

\begin{proof}[Sketch of proof]
    Observe that $\cohom{1}{X}{\struct{X}} = 0$ in \eqref{eq: LES of exponential SES} shows that the connecting homomorphism is injective, therefore $\pic{X} \iso \NS{X}$. For the definition of the bilinear form on $\pic{X}$ and the rest see \cite[\S 1, Proposition 2.4]{Huybrechts2016}.
    % For a proof see \cite[\S 1, Proposition 2.4]{Huybrechts2016}.
    % See \cite[\S 1, Remark]{Huybrechts2016} for why $\pic{X}$ is torsion free and \cite[\S 1, Proposition 2.4]{Huybrechts2016} for the rest. 
\end{proof}

\begin{proposition}
    \label{cohomology of K3}
    \emph{\cite[\S 1.3.2]{Huybrechts2016}}
    Let $X$ be a complex K3 surface. The integral cohomology of $X$ is
    \begin{equation}
        \label{eq: cohomology of K3}
        \cohom{\bullet}{X}{\Z}: \qquad \Z \ \quad 0\ \quad \Z^{22} \quad 0\ \quad \Z.
    \end{equation}
\end{proposition}

\begin{proof}
    We already know $\cohom{0}{X}{\Z} \iso \Z$ and $\cohom{4}{X}{\Z} \iso \Z$ for $X$ is connected and orientable. Looking at the Hodge numbers
    we can already conclude that the cohomology groups are as in \eqref{eq: cohomology of K3}, but only up to torsion. 
    % we know what the integral cohomology groups are up to torsion. 
    We now show why they are in fact all torsion-free. For $\cohom{1}{X}{\Z}$ consider the following part of \eqref{eq: LES of exponential SES} 
    \[
        \cohom{0}{X}{\struct{X}} \to \cohom{0}{X}{\OO_X^*} \to \cohom{1}{X}{\Z} \to \cohom{1}{X}{\struct{X}}.
    \]
    The two groups on the left are $\C$ and $\C^*$, respectively, since $X$ is compact. The morphism relating them is the surjective exponential function, thus the connecting homomorphism going into $\cohom{1}{X}{\Z}$ is trivial. This makes $\cohom{1}{X}{\Z}$ into a subgroup of the trivial group $\cohom{1}{X}{\struct{X}} = 0$.
    % , therefore $\cohom{1}{X}{\Z}$ is trivial.
    Identifying $\pic{X} \iso \cohom{1}{X}{\OO_X^*}$ we have an exact sequence
    \[
        0 \to \pic{X} \to \cohom{2}{X}{\Z} \to \cohom{2}{X}{\struct{X}}.
    \]
    By Serre duality \eqref{eq: Serre duality on K3} we see that $\cohom{2}{X}{\struct{X}} \iso \C$. 
    % By Proposition \ref{picard group torsion free} $\pic{X}$ is torsion free. 
    Since $\pic{X}$ is torsion free by Proposition \ref{picard group torsion free}, two torsion free groups surround $\cohom{2}{X}{\Z}$ in an exact sequence, preventing it from having torsion either.
    By the universal coefficients theorem the torsion subgroup of $\cohom{3}{X}{\Z}$ is isomorphic to the torsion subgroup of the homology $H_2(X,\Z)$. As the latter is torsion free, being isomorphic to $\cohom{2}{X}{\Z}$ by Poincaré duality, the former must be torsion free as well. 
\end{proof}

The middle cohomology group $\cohom{2}{X}{\Z}$ now being free, enables us to introduce a lattice structure on it. The pairing it comes equipped with is called the \emph{intersection pairing} and is naturally induced by the cohomological $\smallsmile$-product. 
% It comes naturally equipped with the so-called \emph{intersection pairing} induced by the $\smallsmile$-product. 
\begin{definition}
    The \emph{intersection pairing} on $\cohom{2}{X}{\Z}$ is defined to be 
    \[
        \intersect{\alpha}{\beta} = \pairing{\alpha \smallsmile \beta}{[X]}.
    \] 
\end{definition}

\begin{proposition}
    \label{intersection pairing on K3 is even}
    \emph{\cite[\S 1, Proposition 3.5]{Huybrechts2016}}
    Let $X$ be a K3 surface. Then the intersection pairing $\intersect{\cdot}{\cdot}$ makes $\cohom{2}{X}{\Z}$ into an even unimodular lattice of rank $22$ and signature $(3,19)$, making it isomorphic to the following orthogonal direct sum
    \begin{equation}
        \label{eq: intersection pairing on K3}
        \cohom{2}{X}{\Z} \iso E_8(-1)^{\oplus 2} \oplus \hyperbolic^{\oplus 3}.
    \end{equation}
    % given by the $\smallsmile$-product is an even lattice. 
\end{proposition}

\begin{proof}[Sketch of proof]
    % To see that the intersection pairing is unimodular we
    Since $\cohom{2}{X}{\Z}$ is free, the Kronecker pairing from the universal coefficients theorem together with Poincaré duality show that the intersection pairing is unimodular. For a display of these methods one can consult \cite[\S VI.9]{Bredon1993}. To see that the intersection pairing is even one considers the mod $2$ reduction in the coefficients $\cohom{2}{X}{\Z} \to \cohom{2}{X}{\Z/2}$ and shows that the pairing on $\cohom{2}{X}{\Z/2}$ is always $0$. This is indeed sufficient as the mod $2$ reduction on coefficients is a ring homomorphism $\cohom{*}{X}{\Z} \to \cohom{*}{X}{\Z/2}$. By Proposition \ref{characteristic classes of K3} we know that the first Chern class $\cclass[1](X)$ is trivial and we also know that it reduces to the second Stiefel--Whitney class $w_2(X)$ $\mathrm{(mod \ 2)}$, which is therefore also trivial. Using the results of \cite[\S VI.17]{Bredon1993}, one is able to show that
    \[
        x \smallsmile x = w_2(X) \smallsmile x
    \]
    holds for all classes $x \in \cohom{2}{X}{\Z/2}$ from which the desired result follows. Proposition \ref{cohomology of K3} shows that $\cohom{2}{X}{\Z}$ is free of rank $22$. The claim about the signature of this lattice is proven in \cite[\S 1, Proposition 3.5]{Huybrechts2016}. Lastly employing Milnor's classification result for even indefinite unimodular lattices (Theorem \ref{Classification of indefinite even unimodular lattices}), we obtain \eqref{eq: intersection pairing on K3}.
\end{proof}

\begin{remark}
    When $\NS{X}$ is equipped with the lattice structure of Proposition \ref{picard group torsion free}, the inclusion $\cclass[1] \colon \NS{X} \hookrightarrow \cohom{2}{X}{\Z}$ turns out to be an isometry. Roughly this due to $\cclass[1]$ being closely related to a cycle map and as such being compatible with these kinds of intersections \cite[\S 19]{Fulton1998}. But more importantly this fact now makes it possible to interpret the intersection pairing
    of two (isomorphism classes of) line bundles $\mathcal L$ and $\mathcal L'$  on $X$ cohomologically as 
    \[
        \intersect{\mathcal L}{\mathcal L'} = \intersect{\cclass[1](\mathcal L)}{\cclass[1](\mathcal{L'})}.
    \]
\end{remark}

\begin{corollary}
    Let $X$ be a K3 surface. The Chern character $\ch[](\E)$ of any coherent sheaf $\E$ on $X$ is integral, meaning that it lies in $\cohom{\bullet}{X}{\Z} \subseteq \cohom{\bullet}{X}{\Q}$.
\end{corollary}

\begin{proof}
    We see that the Chern character on a K3 surface reduces to the following
    \[
        \ch[]{\E} = \rk{\E} + \cclass[1]{\E} + \frac{1}{2}\Bigl(\cclass[1]{\E}^2 - 2\cclass[2]{\E}\Bigr).
    \]
    Since the intersection pairing is even, $\cclass[1](\E)^2$ is twice a multiple of some cohomology class from $\cohom{4}{X}{\Z}$.
\end{proof}

\begin{remark}
    \label{chern character of K3 integral}
The Chern character of a coherent sheaf $\E$ being integral also implies that its Mukai vector $v(\E)$ is integral. Indeed, using the power series expansion \eqref{eq: sqrt power series} (or just guessing), we see that 
\[
    \sqrt{\tdclass{X}} = 1 + \mu_X.
\]
% (the other possible choice being $1 - \mu_X)$
Thus, one easily sees that the Mukai vector on a K3 surface is integral because it has the following form
\begin{align}
    v(\E) &= \ch[0](\E) + \ch[1]{\E} + \bigl(\ch[2]{\E} + \ch[0]{\E}\mu_X\bigr)
    \label{eq: Mukai vector on K3 chern character} \\
    &= \rk{\E} + \cclass[1]{\E} + \frac{1}{2}\Bigl(\cclass[1]{\E}^2 - 2\cclass[2]{\E}\Bigr) + \rk{\E}\mu_X. \notag
\end{align}
We also remark that on a K3 surface the inverse of the square root of the Todd class $(\sqrt{\tdclass{X}})\inv$ equals $1 - \mu_X$ and is therefore integral as well.
\end{remark}

With the Néron--Severi group seen as a sublattice of $\cohom{2}{X}{\Z}$ one can consider also its orthogonal complement. The resulting group will play a very important role in the last chapter in characterizing derived equivalent K3 surfaces. 

\begin{definition}
    The \emph{transcendental lattice} of a complex K3 surface $X$ is defined to be the orthogonal complement
    \[
        \transc{X} = \NS{X}^\perp = \{ x \in \cohom{2}{X}{\Z} \mid \intersect{x}{y} = 0 \text{ for all $y \in \NS{X}$}\}.
    \]
\end{definition}

\begin{remark}
    By Lemma \ref{orthogonal complemet is a lattice} the transcendental lattice $\transc{X}$ is indeed a lattice with the bilinear form inherited from $\cohom{2}{X}{\Z}$. It is also a primitive sublattice of $\cohom{2}{X}{\Z}$ by Example \ref{orthogonal complement primitive}.
\end{remark}

\subsubsection*{Global Torelli theorem}

Lastly we mention the global Torelli theorem. It will be utilized in the last chapter to prove that a K3 surface $Y$ is isomorphic to a moduli space of certain sheaves on its Fourier--Mukai partner. We have already seen that $\cohom{2}{X}{\Z}$ comes equipped with a lattice structure given by the intersection pairing. Moreover due to $X$ being a complex projective manifold it also possesses a Hodge structure. We will call such lattices \emph{Hodge lattices} and two Hodge lattices will be considered isomorphic if there is a \emph{Hodge isometry} between them. 

\begin{definition}

    A \emph{Hodge isometry} between two Hodge lattices $\Lambda$ and $\Lambda'$ is a Hodge structure preserving isomorphism of lattices $f \colon \Lambda \xrightarrow{\sim} \Lambda'$.
    % , which also preserves the Hodge structures. 
\end{definition}

\begin{theorem}[Global Torelli theorem, \text{\cite{PjateckiiShafarevich1971}}]
    \label{Classical Torelli theorem}
    Let $X$ and $Y$ be K3 surfaces over $\C$. Then $X$ and $Y$ are isomorphic if and only if there exists a Hodge isometry
    \[
        f\colon \cohom{2}{X}{\Z} \xrightarrow{\sim} \cohom{2}{Y}{\Z}.
    \]
\end{theorem}

\begin{proof}
    For a proof see \cite[\S 7, Theorem 5.3]{Huybrechts2016}.
\end{proof}

\begin{remark}
    The global Torelli theorem emphasises that the Hodge lattice $\cohom{2}{X}{\Z}$ of a K3 surface $X$ is a complete invariant of its isomorphism type. 
    % In particular this means that the set of all isomorphism classes of K3 surfaces is parametrized by the set of all Hodge structures on the abstract K3 lattice $E_8(-1)^{\oplus 2} \oplus \hyperbolic^{\oplus 3}$
\end{remark}

% By noting where the isomorphism sends the 

% ========= %


% \subsection{Two important theorems}

% When discussing K3 surfaces one can not pass two of the arguably most important theorems on K3 surfaces -- the Global Toreli theorem and the surjectivity of the period map. The first theorem will be utilized in the last chapter to prove that a K3 surface $Y$ is isomorphic to a moduli space of certain sheaves on its Fourier--Mukai partner. 
% % another, derived equivalent, K3 surface $X$.

% We have already seen that $\cohom{2}{X}{\Z}$ comes equipped with a lattice structure given by the intersection pairing. Moreover due to $X$ being a complex projective manifold it also posseses a Hodge structure. We will call such lattices Hodge lattices and two Hodge lattices will be considered isomorphic, if there is a \emph{Hodge isometry} between them. 

% \begin{definition}

%     A \emph{Hodge isometry} between two Hodge lattices $\Lambda$ and $\Lambda'$ is a Hodge structure preserving isomorphism of lattices $f \colon \Lambda \to \Lambda'$.
%     % , which also preserves the Hodge structures. 
% \end{definition}

% \begin{theorem}[Global Torelli theorem]
%     \label{Classical Torelli theorem}
%     Let $X$ and $Y$ be K3 surfaces over $\C$. Then $X$ and $Y$ are isomorphic if and only if there exists a Hodge isometry
%     \[
%         f\colon \cohom{2}{X}{\Z} \to \cohom{2}{Y}{\Z}.
%     \]
% \end{theorem}

% \begin{proof}
%     For a proof see \cite[\S 7, Theorem 5.3]{Huybrechts2016}.
% \end{proof}

% By Proposition \ref{intersection pairing on K3 is even} we know that the lattice $\cohom{2}{X}{\Z}$ of every complex K3 surface $X$ is abstractly isomorphic to $E_8(-1)^{\oplus 2} \oplus \hyperbolic^{\oplus 3}$, which we denote with $\LK$. 
\newpage

\section{Derived Torelli theorem}
\label{Chapter: Derived Torelli theorem}

\info{in this chapter sheaves will always be coherent.}

In this last chapter we will again be dealing with K3 surfaces over the field of complex numbers $\C$. In particular we will characterize when two K3 surfaces $X$ and $Y$ have equivalent bounded derived categories, through their internal structure, namely their \emph{transcendental lattices}. Recall that the \emph{transcendental lattice} $\transc{X}$ of a K3 surface $X$ is defined to be the orthogonal complement of the Néron-Severi group $\NS{X} \subseteq \cohom{2}{X}{\Z}$ with respect to the intersection pairing $\pairing{-}{-}$ on $\cohom{2}{X}{\Z}$.
Naturally, $\transc{X}$ also comes equipped with a weight $2$ Hodge structure induced from that of $\cohom{2}{X}{\Z}$, by declaring $\transc{X}^{p,q} = H^{p,q}(X) \cap (\transc{X} \otimes_\Z \C)$, making it a Hodge lattice. The transcendental lattice of a K3 surface plays a prominent role in the following theorem. 

% \info{todd class of K3}

\begin{theorem}
    \label{Derived Torelli}
    Let $X$ and $Y$ be K3 surfaces over the field of complex numbers $\C$. Then $\derb{X} \natiso \derb{Y}$ if and only if there exists a Hodge isometry $f\colon T_X \to T_Y$ between their transcendental lattices.
\end{theorem}

Along with the transcendental lattice, Fourier-Mukai transforms and related concepts of chapter \ref{} will become very involved once we start proving the above theorem. We provide a quick summary for the incarnation of these concepts, when $X$ is a K3 surface. 



We will be using Fourier-Mukai transforms in an essential way. When applied to the case of K3 surfaces adjoints of Fourier-Mukai transforms take up an interesting form.

Thus we see that in the case of K3 surfaces every left and right adjoints of Fourier-Mukai transforms agree.
Here we have of course used one of the defining properties of K3 surfaces -- triviality of their canonical bundle.

% \subsection{Mukai lattice}

In order to consider all the cohomology groups of a K3 surface $X$ at once Mukai introduced the following pairing on $\cohom{\bullet}{X}{\Z}$. For 

For this lattice to nicely fit into our context of Hodge lattices, we also equip it with a compatible weight $2$ Hodge structure. This is achieved through the introduction of the \emph{Mukai lattice}. 

\begin{definition}
    The \emph{Mukai lattice} of a K3 surface $X$ over $\C$, denoted $\mukai{X}$, consists of the free abelian group $\cohom{\bullet}{X}{\Z}$ together with the bilinear pairing introduced above in \eqref{} equipped with a weight $2$ Hodge structure described by 
\end{definition}

\begin{center}
    \begin{tabular}{r c c}
        $\widetilde{H}^{2,0}(X)$ & $=$ & $H^{2,0}(X)$ \\
        $\widetilde{H}^{1,1}(X)$ & $=$ & $\cohom{0}{X}{\C} \oplus H^{1,1}(X) \oplus \cohom{4}{X}{\C}$ \\
        $\widetilde{H}^{0,2}(X)$ & $=$ & $H^{0,2}(X)$.
    \end{tabular}
\end{center}
\info{describe which parts are orthogonal and why}

% \subsection{Moduli space of sheaves on a K3 surface}
\subsection{Mukai lattice and one side of the proof}

\info{recall what the projections $p$ and $q$ are.}

\noindent
\textsc{Implication $\derb{X} \iso \derb{Y} \implies \mukai{X} \iso \mukai{Y}$.} After first establishing a few preliminary results, we will prove Proposition \ref{D equivalence implies Hodge isometry} by borrowing the kernel of a Fourier-Mukai transform appearing as the equivalence $\derb{X} \iso \derb{Y}$, which Orlov's result \ref{Orlov's theorem} enables us to do, and then show that its corresponding cohomological version gives a Hodge isometry $\mukai{X} \iso \mukai{Y}$.

\begin{proposition}
    \label{D equivalence implies Hodge isometry}
    Suppose there exists an equivalence of triangulated categories $\derb{X} \natiso \derb{Y}$, then there exists a Hodge isometry $\mukai{X} \iso \mukai{Y}$.
\end{proposition}

\begin{lemma}
    \label{Mukai vector is integral}
    For any object $\E^\bullet$ of $\derb{X \times Y}$, its Chern character $\ch{\E^\bullet}$ is integral, \ie belongs to $H^\bullet(X \times Y, \Z) \subseteq H^\bullet(X \times Y, \Q)$. Consequently the Mukai vector $v(\E^\bullet)$ is integral as well.
\end{lemma}

\begin{proof}
    We will use the Künneth formula in an essential way so we recall it here. Since $X$ and $Y$, as K3 surfaces, have torsion-free cohomology the $n$-th cohomology of the product $X \times Y$ decomposes as
    \[
        H^n(X \times Y, \Z) \iso \bigoplus_{p+q = n} H^p(X, \Z) \otimes_\Z H^q(Y, \Z),
    \] 
    for all $n \in \N$.
    
    First, rewrite the Chern character with respect to the ordinary grading on cohomology $H^\bullet(X \times Y, \Q) = \bigoplus_{i=0}^4 H^{2i}(X \times Y, \Q)$ as the vector
    \[
        \ch{\E^\bullet} = \left(\rk{\E^\bullet}, \cclass[1](\E^\bullet), \tfrac{1}{2}(\cclass[1]{\E^\bullet}^2 - 2\cclass[2]{\E^\bullet}), \ch[2]{\E^\bullet}, \ch[3]{\E^\bullet} \right).
    \]
    Components $\rk{\E^\bullet}$ and $\cclass[1]{\E^\bullet}$ are integral by definition. The first Chern class $\cclass[1]{\E^\bullet} \in H^2(X\times Y, \Z)$ may be rewritten, in accordance with the Künneth decomposition
    \[
        H^2(X\times Y, \Z) \iso H^2(X, \Z) \oplus H^2(Y, \Z),
    \]
    in the form $\cclass[1]{\E^\bullet} = p^*\alpha + q^*\beta$, for some $\alpha \in H^2(X, \Z)$ and $\beta \in H^2(Y, \Z)$. Therefore, as K3 surfaces $X$ and $Y$ have \emph{even} intersection pairings by Proposition \ref{intersection pairing on K3 is even}, we see that $\cclass[1]{\E^\bullet}^2 = p^*\alpha^2 + 2 p^*\alpha.q^*\beta + q^*\beta^2$ is an even multiple of some class from $\cohom{4}{X \times Y}{\Z}$. Along with $\cclass[2]{\E^\bullet}$ being integral this shows the $H^4(X \times Y, \Q)$-component of $\ch[](\E^\bullet)$ is integral.

\end{proof}

\begin{lemma}
    \label{Intersection pairing through push-forward of structure map}
    Let $X$ be a K3 surface over $\C$ and $\nu \colon X \to \spec{\C}$ its structure map. Then for all $\alpha$, $\alpha' \in \mukai{X}$
    \[
        \pairing{\alpha}{\alpha'} = \nu_*(\alpha \smallsmile \alpha')
    \] 
    and 
    \[
        \nu_*(\alpha^\vee) = \nu_*(\alpha). 
    \]
\end{lemma}

\begin{proof}

\end{proof}

\begin{proof}[Proof of Proposition \ref{D equivalence implies Hodge isometry}]
    By Orlov's Theorem \ref{Orlov's theorem}, the functor witnessing the equivalence $\derb{X} \natiso \derb{Y}$ is naturally isomorphic to a Fourier-Mukai transform $\fm{\E^\bullet}$ for some kernel $\E^\bullet$ of the category $\derb{X \times Y}$. We will show that the cohomological Fourier-Mukai transform $\fmcoh{\E^\bullet} \colon H^\bullet(X, \Q) \to H^\bullet(Y, \Q)$ of section \ref{Subsection: FM transform on cohomology} induces a Hodge isometry between the Mukai lattices
    \[
        \fmcoh{\E^\bullet} \colon \mukai{X} \to \mukai{Y}.
    \]

    First recall that by Definition \ref{}, the cohomologiacl Fourier-Mukai transform $\fmcoh{\E^\bullet}$ is defined to be
    \[
        \fmcoh{\E^\bullet} \colon H^\bullet(X, \Q) \to H^\bullet(Y, \Q) \qquad \alpha \mapsto p_*(v(\E^\bullet) \smallsmile q^*(\alpha)).
    \]
    By lemma \ref{Mukai vector is integral} the Mukai vector $v(\E^\bullet)$ lies in $H^\bullet(X \times Y, \Z) \subseteq H^\bullet(X \times Y, \Q)$, thus $\fmcoh{\E^\bullet}$ may be restricted to the integral parts, thus forming a homomorphism of abelian groups
    \[
        \fmcoh{\E^\bullet} \colon H^\bullet(X, \Z) \to H^\bullet(Y, \Z).
    \]

    \noindent
    \textsl{Bijection.}
    Next we verify that $\fmcoh{\E^\bullet}$ is bijective. We do so by constructing its right inverse. For $\fm{\E^\bullet}$ is an equivalence, the Fourier-Mukai transform associated to the the kernel $\E^\bullet_L$, of its left adjoint, is actually its quasi-inverse. Thus from $\fm{\E^\bullet_L} \circ \fm{\E^\bullet} \natiso \id{\derb{X}}$ along with uniqueness of kernels and Example \ref{Identifying fm transforms}, we see that $\struct{\Delta_X} \iso \E^\bullet_L * \E^\bullet$. As cohomological Fourier-Mukai transforms compose just like the categorical ones do, according to Proposition \ref{Composition of cohomological fm is fm}, we see that $\fmcoh{\E^\bullet_L} \circ \fmcoh{\E^\bullet} = \fmcoh{\struct{\Delta_X}}$, thus it suffices to show that $\fmcoh{\struct{\Delta_X}} = \id{H^\bullet(X, \Z)}$.
    
    We start by computing the Mukai vector of $\struct{\Delta_X}$. Consulting the help of Grothendieck-Riemann-Roch \ref{Grothendieck-Riemann-Roch}, we see that
    \begin{align*}
        \ch(\struct{\Delta_X}) \tdclass{X \times X} &= \ch(\Delta_*\struct{X})\tdclass{X \times X} = \\ &= \ch(\Delta_!\struct{X})\tdclass{X \times X} = \Delta_*(\ch(\struct{X}) \tdclass{X}) = \Delta_*\tdclass{X},
    \end{align*}
    holds true in $\cohom{*}{X \times X}{\Q}$. In the second equality, we have used the relation $\Delta_*\struct{X} = \Delta_!\struct{X}$ in $K(X \times X)$, because for any closed embedding, such as $\Delta \colon X \to X \times X$, its push-forward $\Delta_*$ is exact, and separately in the last equality $\ch(\struct{X}) = \exp(\cclass[1](\struct{X})) = 1$, because $\struct{X}$ is a line bundle. Next, we observe, that from $\tdclass{X \times X} = p^*\tdclass{X} \smallsmile p^*\tdclass{X}$, equation $\Delta^*\sqrt{\tdclass{X\times X}} = \tdclass{X}$ follows. By the cohomological projection formula (\cf Proposition \ref{cohomological projection formula}) for the map $\Delta$, we further compute
    \[
        \Delta_*\tdclass{X} = \Delta_*\Delta^*\sqrt{\tdclass{X \times X}} = \Delta_*(1)\sqrt{\tdclass{X \times X}},
    \] 
    implying that $v(\struct{\Delta_X}) = \Delta_*(1)$. 
    Lastly, for any $\alpha \in \cohom{\bullet}{X}{\Z}$, using the projection formula again, we arrive at
    \[
        \fmcoh{\struct{\Delta_X}}(\alpha) = p_*(v(\struct{\Delta_X})\smallsmile p^*\alpha) = p_*(\Delta_*(1)\smallsmile p^*\alpha) = p_*(\Delta_*(1\smallsmile \Delta^*p^*(\alpha))) = \alpha.
    \]
    % By letting $\mathcal R^\bullet$ denote the convolution of $\E^\bullet$ and $\E^\bullet_L$, we see that by uniqueness of kernels of Fourier-Mukai transforms it follows that 
    The map $\fmcoh{\E^\bullet}$ is now a homomorphism of free abelian groups admitting a right inverse, therefore it is an isomorphism of abelian groups. We are left to show that $\fmcoh{\E^\bullet}$ is an isometry and that it preserves the Hodge structure. 
    
    \vspace{0.5cm}
    \noindent
    \textsl{Isometry.}
    To see that $\fmcoh{\E^\bullet}$ is an isometry, we do two preliminary computations. Let $\alpha \in \mukai{X}$ and $\beta \in \mukai{Y}$ and let $\nu_X \colon X \to \spec{\C}$, $\nu_Y \colon Y \to \spec{\C}$ and $\nu_{X \times Y} \colon X \times Y \to \spec{\C}$ denote the structure maps, then by Lemma \ref{Intersection pairing through push-forward of structure map}
    \begin{align*}
        \pairing{\fmcoh{\E^\bullet}(\alpha)}{\beta}_Y &= \nu_{Y, *}(\beta^\vee \smallsmile \fmcoh{\E^\bullet}(\alpha))
        \\
        &= \nu_{Y, *}\left(
            \beta^\vee \smallsmile p_*\left(v(\E^\bullet) \smallsmile q^*\alpha\right)\right) 
        \\
        &= \nu_{Y, *}\left(
            \beta^\vee \smallsmile p_*\left(\ch{\E^\bullet}
            {\sqrt{\tdclass{X \times Y}}}
            \smallsmile q^*\alpha\right)
        \right) 
        \\
        &= \nu_{Y, *}\left( 
            p_*\left(
                p^*\beta^\vee \smallsmile \ch{\E^\bullet}
                {\sqrt{\tdclass{X \times Y}}}
                \smallsmile q^*\alpha
            \right)
        \right) \\
        &= \nu_{X \times Y, *}\left(
            p^*\beta^\vee \smallsmile \ch{\E^\bullet}
                {\sqrt{\tdclass{X \times Y}}}
                \smallsmile q^*\alpha
        \right)
    \end{align*}
    and similarly one computes
    \info{$\ch(E^\vee) = \ch(E)^\vee$?}
    \[
        \pairing{\alpha}{\fmcoh{\E^\bullet_L}(\beta)}_X = \nu_{X \times Y, *}\left(
            q^*\alpha^\vee \smallsmile \ch{\E^\bullet}^\vee
                {\sqrt{\tdclass{X \times Y}}}
                \smallsmile p^*\beta
        \right)
    \]

    Then clearly for all $\alpha$, $\alpha' \in \mukai{X}$
    \[
        \pairing{\fmcoh{\E^\bullet}(\alpha)}{\fmcoh{\E^\bullet}(\alpha')}_Y = 
        \pairing{\alpha}{\fmcoh{\E^\bullet_L}(\fmcoh{\E^\bullet}(\alpha'))}_X = 
        \pairing{\alpha}{\alpha'}_X,
    \]       
    proving that $\fmcoh{\E^\bullet}$ is an isometry. 

    \vspace{0.5cm}

    \noindent
    \textsl{Hodge structures.}
    Lastly, we show that $\fmcoh{\E^\bullet}$ preserves the Hodge structure \ie
    \[
        \fmcoh{\E^\bullet}_\C(\widetilde{H}^{p,q}(X)) \subseteq \widetilde{H}^{p,q}(Y) 
    \]
    for all $p$, $q$, with $p + q = 2$. 
    By Proposition \ref{Hodge lattice, fm transform interaction}, the condition \eqref{eq: Hodge lattice, fm transform} specializes to 
    \begin{equation}
        \label{eq: H20 is mapped to H20}
        f_\C(\widetilde{H}^{2,0}(X)) \subseteq \widetilde{H}^{2,0}(Y)
    \end{equation}
    % since the term on the right is the only non-trivial one of the direct sum decomposition.
    and in fact this turns out to be sufficient for $f$ to preserve the Hodge structures. This is a very special coincidence, which happens as a consequence of the Hodge structure on the Mukai lattice of a K3 surface taking a particular from.
    % In fact, we will show this inclusion holds only for $(p,q) = (2,0)$, as this turns out to be sufficient. 
    Indeed, if $f_\C(\widetilde{H}^{2,0}(X)) \subseteq \widetilde{H}^{2,0}(Y)$, then $f_\C(\widetilde{H}^{0,2}(X)) \subseteq \widetilde{H}^{0,2}(Y)$ is obtained through conjugation
    \[
        f_\C\left(\widetilde{H}^{0,2}(X)\right) = f_\C\left(\ols{\widetilde{H}^{2,0}(X)}\right) = \ols{f_\C\left(\widetilde{H}^{2,0}(X)\right)} \subseteq \ols{\widetilde{H}^{2,0}(Y)} = \widetilde{H}^{0,2}(Y).
    \]
    And, to see $f_\C(\widetilde{H}^{1,1}(X)) \subseteq \widetilde{H}^{1,1}(Y)$, we pick a non-zero class $x \in \widetilde{H}^{1,1}(X)$ and map it over to $y = f_\C(x) \in \mukai{X} \otimes \C$. Then for any $y^{2,0} \in \widetilde{H}^{2,0}(Y)$, we have $\pairing{y}{y^{2,0}} = 0$. Indeed, first as $f_\C$ is an isomorphism of $\C$-vector spaces and $\widetilde{H}^{2,0}(X)$ and $\widetilde{H}^{2,0}(Y)$ are one dimensional, the inclusion \eqref{eq: H20 is mapped to H20} is actually an equality, thus we may represent $y^{2,0}$ as $f_\C(x^{2,0})$ for some $x^{2,0} \in \widetilde{H}^{2,0}(X)$. As $\widetilde{H}^{2,0}(X) \perp \widetilde{H}^{1,1}(X)$ and $f_\C$ is an isometry, we obtain $0 = \pairing{x}{x^{2,0}} = \pairing{y}{y^{2,0}}$ to conclude $y \in \widetilde{H}^{2,0}(Y)^\perp$. Symmetrically one obtains $y \in \widetilde{H}^{0,2}(Y)^\perp$. 

    Finally, once we verify that $\widetilde{H}^{2,0}(Y)^\perp \cap \widetilde{H}^{0,2}(Y)^\perp \subseteq \widetilde{H}^{1,1}(Y)$, we may conclude that $y \in \widetilde{H}^{1,1}(Y)$ to prove our last claim.
    This is indeed the case for once we write a class $z \in \widetilde{H}^{2,0}(Y)^\perp \cap \widetilde{H}^{0,2}(Y)^\perp$ according to the Hodge decomposition as $z = z^{2,0} + z^{1,1} + z^{0,2}$, we see that $z^{2,0} = z^{0,2} = 0$, by non-degeneracy of the pairing. For example for any $w = w^{2,0} + w^{1,1} + w^{0,2} \in \mukai{Y}$ we have
    \[
        \pairing{z^{2,0}}{w} = \pairing{z^{2,0}}{w^{2,0}} + \pairing{z^{2,0}}{w^{1,1}} + \pairing{z^{2,0}}{w^{0,2}} = 0,
    \]
    where $\pairing{z^{2,0}}{w^{2,0}} = 0$, because $z^{2,0} = z - z^{1,1} - z^{0,2}$ shows $z^{2,0} \in \widetilde{H}^{0,2}(Y)^\perp$ and $\pairing{z^{2,0}}{w^{1,1}} = 0$ together with $\pairing{z^{2,0}}{w^{0,2}} = 0$ follow from orthogonality relations \eqref{}. \qedhere
    
    % arising from the fact that $y^{2,0}$ is of the form $f_\C(x^{2,0})$ for some $x^{2,0} \in \widetilde{H}^{2,0}(X)$
    
    
    % As $f_\C$ is an isomorphism of $\C$-vector spaces, the class $y$ is non-zero. 
    
    % $\widetilde{H}^{2,0}(Y) \oplus \widetilde{H}^{1,1}(Y) \oplus \widetilde{H}^{0,2}(Y)$
\end{proof}

\subsection{Moduli spaces of sheaves and the other side of the proof}

The last subsection is devoted to proving the right to left implication of Theorem \ref{Derived Torelli}, captured in the ensuing proposition.

\begin{proposition}
    \label{Hodge isometry implies D equivalence}
    Let $X$ and $Y$ be K3 surfaces over $\C$ and suppose there exists a Hodge isometry 
    \[
        f \colon \mukai{X} \to \mukai{Y}
    \]
    of their Mukai lattices. Then there exists an equivalence of their bounded derived categories of coherent sheaves $\derb{X} \iso \derb{Y}$.
    % If there is a Hodge isometry of Mukai lattices $\mukai{X} \iso \mukai{Y}$, there is an equivalence of triangulated categories $\derb{X} \iso \derb{Y}$.
\end{proposition}
The method of our proof will lead us to introduce another K3 surface $M$, which will turn out to be isomorphic to $Y$ through an application of the classical Torelli theorem \ref{Classical Torelli theorem}, and whose derived category we will show to be equivalent to $\derb{X}$. Not only will the K3 surface $M$ itself be important, but also how it is constructed or what it \emph{represents}. We will expand on this viewpoint to some extent, but will have to refer the reader to \cite{HuybrechtsLehn2010} or \cite{vanBree2020} for details and proofs, as this part is unfortunately outside the scope of our thesis. We also mention that this line of thinking was guided by the influential paper of Mukai \cite{Mukai1987}. This approach will serve as an invaluable tool for obtaining a kernel for a Fourier-Mukai transform of the from $\derb{M} \to \derb{X}$, which we will later prove to be an equivalence. 
% In proving the latter claim we will take a bit of an alternative route, than what is presented in \cite{Orlov-K3}, and use the abstract machinery of triangulated categories -- indecomposability and spanning classes, established towards the latter parts of Section \ref{Subsection: Triangulated categories}, together with results of Section \ref{Subsection: Derived category of coherent sheaves}. 
In proving the latter claim we will diverge slightly from the original proof, presented in \cite{Orlov-K3}, and use the abstract machinery of triangulated categories. Utilizing indecomposability of $\derb{X}$ and the spanning class of skyscraper sheaves $k(x)$, for closed points $x \in X$, we will show fully faithfulness through the use of Proposition \ref{Fully faithful functor on a spanning class} and essential surjectivity via the use of Corollary \ref{condition for functor to be an equivalence}, which we have established towards the latter parts of Section \ref{Subsection: Triangulated categories}.


% using the abstract machinery of triangulated categories and spanning classes, established towards the latter parts of subsection \ref{}. Through an application of the classical Torelli theorem \ref{Classical Torelli}, the K3 surface $M$ will turn out to be isomorphic to $Y$.

% A few more words are required to introduce how the K3 surface $M$ comes about. Not only is the K3 surface $M$ important, but also how it is constructed or what it \emph{represents}. This viewpoint will serve as an invaluable tool for obtaining a kernel for a Fourier-Mukai transform of the from $\derb{M} \to \derb{X}$, which we will later prove to be an equivalence. 

Recall that any object $X$ of a category $\mathcal C$ gives rise to a functor 
\[
    h_X = \Hom_{\mathcal C}(-, X) \colon  \quad \mathcal C^\op \to \Set,
\]
% which is given by $h_X = \Hom_{\mathcal C}(-, X)$. 
A functor $F \colon \mathcal C^\op \to \Set$ is said to be \emph{representable}, if there is some object $X$ of $\mathcal C$, for which $F$ and $h_X$ are naturally isomorphic. Moreover the Yoneda lemma \cite[text]{keylist} then asserts that the map  
\begin{equation}
    \label{eq: nat iso of Yoneda}
    \Hom_{\operatorname{Fct(\mathcal C^\op, \Set)}}(h_X, F) \to F(X), \qquad \alpha \mapsto \alpha_X(\id{X}),
\end{equation}
is an isomorphism for every object $X$ of $\mathcal C$ and any functor $F \colon \mathcal C ^\op \to \Set$, and is natural both in $X$ and $F$.

From now on we focus on a fixed K3 surface $X$ over the field of complex numbers $\C$ and consider the category $\sch{\C}$ of schemes over $\C$. The main idea for conceiving a new space out of $X$, we will employ, is an attempt to see the collection of all coherent sheaves on $X$ as a space itself. What was said above is not precise nor fully correct, as the collection of all coherent sheaves on $X$ is way to large to consider all at once, so we will instead restrict ourself to a smaller class of sheaves by imposing some additional constraints, which we present presently. 

Firstly, in the spirit of deformation theory, we might consider our sheaves arising as members of certain families of sheaves on $X$ ``parametrized'' by points of some other scheme $S$ over $\C$. The notion of such a family is encoded in a single coherent sheaf $\F \in \coherent{X \times S}$, defined on the product $X \times S$. Indeed, consider a closed point $s \in S$, to which there is a corresponding morphism of schemes over $\C$, also denoted $s \colon \spec{\C} \to S$. Then we define the morphism $i_s \colon X \to X \times S$ to be the composition
\begin{equation}
    \label{eq: inclusion of X to X x S}
    i_s \colon \quad X \longrightarrow X \times \spec{\C} \xrightarrow{\id{X} \times s} X \times S,
\end{equation}
where the first morphism into the product is induced by $\id{X}$ and the structure map $X \to \spec{\C}$ and introduce $\F|_s$ to mean the pull-back sheaf $i_s^*\F$ on $X$. In particular, this will allow us to focus only on those sheaves on $X$ parametrized by $S$, with a fixed Mukai vector.
% Let $s \in S$ be a closed point, to which we know there is a corresponding morphism of schemes over $\C$, also denoted $s \colon \spec{\C} \to S$. Then we define $i_s \colon X \to X \times S$ to be the composition
% \[
%     i_s \colon \quad X \longrightarrow X \times \spec{\C} \xrightarrow{\id{X} \times s} X \times S,
% \]
% where the first morphism into the product is induced by $\id{X}$ and the structure map $X \to \spec{\C}$ and introduce $\F|_s$ to mean the pull-back sheaf $i_s^*\F$ on $X$. This will allow us to focus only on those sheaves on $X$ parametrized by $S$, which have a fixed Mukai vector.
Since we would like the geometry of our scheme $S$ to have a greater impact on the parametrization of sheaves on $X$, we would like our families $\F$ to vary somewhat continuously with respect to the parameter space $S$. To this end we impose the following \emph{flatness} condition on $\F$. 

% play a part in parametrizing sheaves on $X$, we would like our families $\F$ to vary somewhat continuously with respect to the parameter $s \in S$. To this end we will impose the following \emph{flatness} condition on $\F$. 


% Since we want the parametrizing scheme $S$ to have a greater impact on the whole family 
% In order to capture that the members of such a family vary somewhat continuously with respect to the parameter $s \in S$, we impose the following \emph{flatness} condition on $\F$. 


% Already one way of obtaining a bunch of coherent sheaves on $X$ all at once is through families indexed  


% The main idea we are trying to is to consider sheaves on $X$. These sheaves will

% appear in families ``indexed'' by some other scheme $S$ over $\C$. Such families may be encoded as single sheaves on the product $\F \in \coherent{X \times S}$. 


% As this would be an absolute overkill for there are too many coherent sheaves, we restrict ourselves by imposing some additional constraints on our sheaves.

% \vspace{1cm}

\begin{definition}
    \label{F is flat over S}
    For any scheme $S$ over $\C$, a sheaf $\F \in \coherent{X \times S}$ is said to be \emph{flat over $S$}, if $\F$ is a flat $\pi_S\inv \struct{S}$-module, where $\pi_S \colon X \times S \to S$ is the canonical projection. 
\end{definition}
 

A condition, which restricts the scope of available sheaves even further is \emph{stability}. We quickly glance over the main components, which enable us to define what a stable sheaf is in the first place and give a property analogous to Schur's lemma, but for stable sheaves, which will become relevant in the proof of Proposition \ref{Hodge isometry implies D equivalence}.
% reason about two properties of stable sheaves relevant for us.
\begin{itemize}[label = $\vartriangleright$]
    \item{
        As $X$ is projective, let $\struct{X}(1)$ denote the pull-back of $\struct{\mathbb P^n}(1)$ along an inclusion $X \hookrightarrow \mathbb{P}^n$ and define $\F(m) := \F \otimes_{\struct{X}} \struct{X}(1)^{\otimes m}$ for $\F \in \coherent{X}$ and $m \in \N$.
        }
    \item{
        The \emph{Hilbert polynomial} of a coherent sheaf $\F \in \coherent{X}$ is defined as
        \[
            P(\F, -) \colon \N \to \Z \qquad m \mapsto \chi(X, \F(m)),
        \]
        which is know to be a polynomial function uniquely expressible in the form 
        \[
            P(\F, m) = \sum_{i = 0}^d \alpha_i(\F)\frac{m^i}{i!},
        \]
        for some rational coefficients $\alpha_i(\F)$ and $d \in \N$ (\cf \cite[Part I, \S 1.2, Lemma 1.2.1]{HuybrechtsLehn2010}). Let $P(\F)$ denote the corresponding polynomial. The \emph{reduced Hilbert polynomial} of $\F$ is defined to be $p(\F) = P(\F)/\alpha_d(\F)$.
    }
    \item{
        Equip the set of all polynomials $\Q[t]$ with the lexicographical ordering of their coefficients $<$, meaning that for polynomials $f$ and $g$ we set $f < g$, if the leading coefficient of $g - f$ is positive. 
    }
    \item{
        We declare the \emph{dimension} of a sheaf to be the dimension of its support.
        A sheaf is said to be \emph{pure} if all its non-trivial subsheaves have the same dimension. 
        % A sheaf $\F \in \coherent{X}$ is said to be \emph{pure} of dimension $d$, if $\supp{\E} \subseteq X$ has dimension $d$ for every non-trivial subsheaf $\E \subseteq \F$ (\cf \cite[Part I, \S 1.1, Definition 1.1.2]{HuybrechtsLehn2010}).
    }
    \item{
        A pure sheaf $\F \in \coherent{X}$ is called \emph{stable} if for each proper non-trivial subsheaf $\G \subseteq \F$, $p(\G) < p(\F)$ holds. 
    }
    \item{
        Any homomorphism $\F \to \G$ between two stable sheaves with $P(\F) = P(\G)$ is either $0$ or an isomorphism. Consequently, for two stable sheaves $\F$ and $\G$, we have
        \begin{equation}
            \label{eq: schur lemma for stable sheaves}
            \Hom_{\struct{X}}(\F, \G) \iso \begin{cases}
                \C, &\text{if $\F \iso \G$,} \\
                0, &\text{if $\F \not\iso \G$.}
            \end{cases}
        \end{equation}
        For the proof one may consult Proposition 1.2.7 and Corollary 1.2.8 of \cite[Part I, \S 1.2]{HuybrechtsLehn2010}.
    }
    \item{
        For a coherent sheaf $\F$, it is possible to compute its Hilbert polynomial using the Hirzebruch-Riemann-Roch theorem \ref{Hirzebruch-Riemann-Roch}. In the case that $X$ is a K3 surface the result of the computation is
        \[
            P(\F, m) = (2\ch[0](\F) + \ch[2](\F)) + \pairing{h}{\ch[1](\F)}m + \tfrac{1}{2}\pairing{h}{h}\ch[0]{\F}m^2,
        \]
        where $h = \cclass[1](\struct{X}(1))$. We therefore see that the Mukai vector $v(\F)$ determines the Hilbert polynomial $P(\F)$. 
    }
\end{itemize}

% As it is not really illuminating, we will not give a definition of when a sheaf is stable 
\noindent
With all the necessary terminology now established, we can introduce the following functor. 

\begin{definition}
    \label{Definition of moduli functor}
    Let $X$ be a K3 surface over $\C$ with a prescribed Mukai vector $v \in \mukai{X}$. The \emph{moduli functor}
    \[
        \M_v \colon \sch{\C}^\op \longrightarrow \Set
    \]
    is defined on objects as 
    \[
        S \longmapsto \left\{\F \in \coherent{X \times S} \ \middle| 
        \begin{array}{l}
            \text{$\F$ is flat over $S$, and for each closed point $s \in S$,} \\
            \text{$\F|_s$ is a stable sheaf on $X$, with $v(\F|_s) = v$.}
        \end{array}
        \right\}_{/\sim},
    \]
    where $\F \sim \F'$ is defined to hold precisely when there exist a line bundle $\mathcal L$ on $S$, for which $\F \iso \F' \otimes_{\struct{X \times S}} \pi_S^*\mathcal L$, and on morphisms\footnote{
                    The assignment is well defined because
                    \[
                        (\id{X} \times f)^*(\F \otimes \pi_S^*\mathcal L) \iso (\id{X} \times f)^*\F \otimes (\id{X} \times f)^*\pi_S^*\mathcal L \iso (\id{X} \times f)^*\F \otimes \pi_{S'}^*f^*\mathcal L.
                    \]
                } as
    \[
        (f\colon S' \to S) \longmapsto \left( (\id{X} \times f)^* \colon
            \begin{array}{r l}
                & \M_v(S) \to \M_v(S') \\
                & [\F] \mapsto [(\id{X} \times f)^*\F]
            \end{array}
            \right).
    \]
\end{definition}

\begin{remark}
    Our moduli functor is constrained by a prescribed Mukai vector as opposed to \cite[\S 4.2, Proposition 4.15]{vanBree2020}, where they prescribe the Chern character. As Mukai vectors and Chern characters determine each other on a K3 surface, the definitions are in fact equivalent. 
\end{remark}

\begin{theorem}
    \label{Representability of moduli functor}
    Let $v \in \mukai{X}$ be an isotropic vector with $\gcd_{w \in N(X)}\pairing{v}{w} = 1$. Then the moduli functor $\M_v \colon \sch{\C}^\op \to \Set$ is representable and is represented by a K3 surface $M_v$ over $\C$, also called the \emph{fine moduli space}.
\end{theorem}
\info{Not sure if these conditions are actually sufficient for $M$ to be K3, but I actually only need $M$ to be smooth and projective, b/c I show $M$ and $X$ are derived equivalent.}

% \begin{example}
%     The vector $(0,0,1)$ is an example of such a $v$ and so is any isometric image of it. 
% \end{example}

\begin{remark}
    The moduli functor $\M_v$ is actually representable already even under no additional assumptions on the Mukai vector $v$. The conditions are there only to ensure $M_v$ is a K3 surface. For example in general $M_v$ is smooth and has dimension $\pairing{v}{v} + 2$ (\cf \cite[\S 4.3, Propositon 4.20]{vanBree2020}).
\end{remark}

Circling back to the original idea of constructing a space, which parametrizes isomorphism classes of stable coherent shaves on $X$ with a prescribed Mukai vector, we now argue as to why $M_v$ achieves this. Since the set of all closed points of $M_v$ is in bijection with the set of morphisms $\spec{\C} \to M_v$ in $\sch{\C}$, we obtain the following chain of natural bijections
\begin{equation}
    \label{eq: natural iso applied to SpecC}
    \Hom_{\sch{\C}}(\spec{\C}, M_v) \xrightarrow{\ = \ } h_{M_v}(\spec{\C}) \xrightarrow{\ \iso \ } \M_v(\spec{\C}).
\end{equation}
Thus every closed point of $M_v$ represents and is represented by a unique class $[\F]_\sim \in \M_v(\spec{\C})$ of coherent sheaves on $X \times \spec{\C}$. As $\spec{\C}$ carries only the trivial line bundle $\struct{\spec{\C}}$, the equivalence relation $\sim$ of Definition \ref{Definition of moduli functor} is enhanced to distinguishing only non-isomorphic sheaves on $X \times \spec{\C}$. Since clearly $X \iso X \times \spec{\C}$, we arrive at the bijective correspondence
% \begin{equation}
%     \{ p \in M \mid \text{$p$ closed point}\} \leftrightarrows \{\F \in \coherent{X} \mid \text{$\F$ is stable with $v(\F) = v$.}\}
% \end{equation}
\begin{equation}
    \label{eq: closed points of M}
    \{ \text{closed points of $M_v$}\} 
    \leftrightarrow \left\{
        \begin{array}{c}
            \text{isomorphism classes $[\F]_\iso$} \\
            \text{of stable sheaves on $X$ with $v(\F) = v$}
        \end{array}
        \right\},
\end{equation}
tying in perfectly into our initial idea of $M_v$ parametrizing certain sheaves on $X$.  

Another key ingredient, playing a principal part in the proof of Proposition \ref{Hodge isometry implies D equivalence}, is the \emph{universal bundle} defined by the subsequent category theoretic magic.

% \begin{definition}
%     The \emph{universal bundle (on $X$)} is the unique sheaf $\E \in \coherent{X \times M}$, to which the natural isomorphism \eqref{eq: nat iso of Yoneda} of the Yoneda lemma composed with the natural isomorphism of Theorem \ref{Representability of moduli functor}, witnessing representability of $\M_v$, assigns the identity natural transformation $\id{h_M}$ of the functor $h_M$
%     \[
%         \Hom_{\operatorname{Fct(\sch{\C}^\op, \Set)}}(h_M, h_M) \xrightarrow{\ \iso\ } h_M(M) \xrightarrow{\ = \ } \Hom_{\sch{\C}}(M, M) \xrightarrow{\ \iso\ } \M_v(M).
%     \]
%     \info{Why is Yoneda even here, }
% \end{definition}

\begin{definition}
    The \emph{universal bundle (on $X$)} is any representative $\E \in \coherent{X \times M}$ of the unique equivalence class of $\M_v(M)$, to which the isomorphism 
    \[
        \Hom_{\operatorname{Fct(\sch{\C}^\op, \Set)}}(h_M, \M_v) \to \M_v(M)
    \]
    of Yoneda lemma sends the natural isomorphism $\eta \colon h_M \Longrightarrow \M_v$, detecting representability of the moduli functor $\M_v$. In other words 
    \[
        [\E]_\sim = \eta_M(\id{M}).
    \]
\end{definition}

As a consequence of a more general fact \cite[text]{keylist}, we obtain the following. 

\begin{theorem}
    The universal bundle $\E$ is a \emph{bundle} \ie a locally free $\struct{X \times M}$-module of finite rank. 
\end{theorem}

\begin{remark}
    \label{Points of M are sheaves}
Utilizing the universal bundle $\E$ we can provide another way of interpreting the isomorphism classes of stable sheaves on $X$ with Mukai vector $v$ \ie elements of $\M_v(\spec{\C})$. Under the natural bijection $\eta_{\spec{\C}}$ of \eqref{eq: natural iso applied to SpecC} every isomorphism class $[\F]_\iso \in \M_v(\spec{\C})$ corresponds to a unique morphism $t \colon \spec{\C} \to M$. By naturality of $\eta$, we obtain the following commutative square 
\[\begin{tikzcd}[column sep = normal, row sep = 1.8em]
    % https://q.uiver.app/#q=WzAsNCxbMCwwLCIxIl0sWzIsMCwiMiJdLFswLDIsIjMiXSxbMiwyLCI0Il0sWzAsMSwiXFxldGFfTSJdLFswLDIsInBeKiIsMl0sWzEsMywiXFxNX3YocCkiXSxbMiwzLCJcXGV0YSIsMl1d
	\Hom_{\sch{\C}}(M, M) && \M_v(M) \\
	\\
	\Hom_{\sch{\C}}(\spec{\C}, M) && \M_v(\spec{\C}),
	\arrow["{\eta_M}", from=1-1, to=1-3]
	\arrow["{- \circ t}"', from=1-1, to=3-1]
	\arrow["{\M_v(t)}", from=1-3, to=3-3]
	\arrow["\eta_{\spec{\C}}", from=3-1, to=3-3]
\end{tikzcd}\]
which spells out that 
\begin{align*}
    [\F]_\iso = \eta_{\spec{\C}}(t) &= \eta_{\spec{\C}}(((-)\circ t)(\id{M})) = \\
    &= \M_v(t)(\eta_M(\id{M})) = \M_v(t)([\E]_\sim) = [(\id{X} \times t)^*\E]_\iso.
    % = [\E|_{X \times \{p\}}]_\iso.
\end{align*}
Taking the liberty of identifying $t \colon \spec{\C} \to M$ with its image -- a closed point corresponding to a stable sheaf $\F \in \coherent{X}$, with $v(\F) = v$, and denoting it with $[\F]$, we see that the above calculation reads
\[
    \E|_{[\F]} \iso \F.
\]
\end{remark}

% of viewing the closed points of the moduli space $M_v$. 
% Since we know closed points of $M_v$ are in bijective correspondence with isomorphism classes of stable sheaves $\F$ on $X$, with $v(\F) = v$ 


Lastly we need the following theorem of Bondal and Orlov characterizing fully faithfulness of a Fourier-Mukai transform.

\begin{theorem}[Bondal, Orlov]
    Let $X$ and $Y$ be smooth projective varieties over $\C$ and let $\fm{E} \colon \derb{X} \to \derb{Y}$ be a Fourier-Mukai transform associated to a kernel $E$ of $\derb{X \times Y}$. Then $\fm{E}$ is fully faithful if and only if for any two closed points $x$ and $y$ of $X$ the following condition is satisfied
    \begin{equation}
            \Hom_{\derb{Y}}(\fm{E}(k(x)), \fm{E}(k(y))[i]) \iso \begin{cases}
                \C, &\text{if $x = y$ and $i = 0$,} \\
                0, &\text{if $x \neq y$ or $i \notin [0, \dim X]$.}
            \end{cases}
    \end{equation}
\end{theorem}

\begin{proof}
    The proof may be found in \cite[\S 7, Proposition 7.1]{huybrechts2006fouriermukai} or \cite[Theorem 5.1]{bridgeland2019equivalencestriangulatedcategoriesfouriermukai}.
\end{proof}

\begin{remark}
    Let us give some evidence for why one might expect such a result to hold and why is it so important.

    We touch upon this approach after the proof. 
\end{remark}

\begin{remark}        
    The assumption of picking $\C$ to be our ground field is not very far from the full generality, where the underlying field is required to be algebraically closed and of characteristic zero.
\end{remark}

\noindent
With all the preparations now out of the way, we may begin with the proof.
% where we will exploit brilliant results of Mukai 
% In the definition of the moduli functor, we will use the term \emph{stable}, which we have not defined. 
\begin{proof}[Proof of Proposition \ref{Hodge isometry implies D equivalence}]

    We will split the proof in a couple of cases, depending on where the Hodge isometry $f \colon \mukai{X} \to \mukai{Y}$ sends the vector $(0,0,1) \in \mukai{X}$. Let $v = (v_0, v_1, v_2) \in \mukai{Y}$ be the image of $(0,0,1)$ under $f$.



    \vspace{0.5 cm}
    \noindent
    \textsl{Identifying $M$ and $Y$.}

    \vspace{0.5 cm}
    \noindent
    \textsl{Fully faithfulness.} To show $\fm{\E}$ is fully faithful we will use Proposition \ref{Fully faithful functor on a spanning class}. By Proposition \ref{k(x) spanning class}, we know the skyscrapers $k(t)$, for closed points $t \in M$ form a spanning class of $\derb{M}$, so we will first compute $\Hom_{\derb{X}}(k(s), k(t)[i])$ for all closed points $s$ and $t$ in $M$ and $i \in \Z$. We do this in a few cases.

    \vspace{0.3cm}
    \noindent
    \textsl{Case $s \neq t$.} For $i < 0$ or $i > 2$, we know by Proposition \ref{Homs in derived category} that $\Hom_{\derb{M}}(k(s), k(t)[i]) \iso \Ext^i_{\struct{M}}(k(s), k(t))$, which are trivial for $i < 0$. They are also trivial for $i > 2$, by Serre duality \ref{Serre duality for derived cats}, as
    \[
        \Hom_{\derb{M}}(k(s), k(t)[i]) \iso \Hom_{\derb{M}}(k(t)[i], k(s)[2])^* \iso \Hom_{\derb{M}}(k(t), k(s)[2 - i])^*
    \]   
    For $i = 0$, using fully faithfulness of $\coherent{X} \to \derb{X}$ and noticing that the supports of $k(s)$ and $k(t)[i]$ are disjoint we obtain 
    \[
    \Hom_{\derb{X}}(k(s), k(t)[i]) = \Hom_{\struct{X}}(k(s), k(t)) = 0.
    \]
    For $i = 2$, Serre duality allows us to copy the result of $i = 0$, to see $\Hom_{\derb{X}}(k(s), k(t)[2])$ is also trivial.
    And, when $i = 1$ 

    \vspace{0.3cm}
    \noindent
    \textsl{Case $s = t$.}

    % For $i = 2$, by Serre duality \ref{}, we obtain $\Hom_{\derb{X}}(k(s), k(t)[2]) \iso \Hom_{\derb{X}}(k(t), k(s)) = 0$.

    % And for $i = 0$, we obtain by
    % Then the supports of $k(s)$ and $k(t)[i]$ are disjoint for all $i \in \Z$
    %     $s = t$ 
    

    In conclusion we have computed that
    \begin{equation}
            \Hom_{\derb{X}}(k(s), k(t)[i]) \iso \begin{cases}
                \C, &\text{if $s = t$ and $i \in \{0, 2\}$,} \\
                \C^2, &\text{if $s = t$ and $i = 1$,} \\
                0, &\text{else.}
            \end{cases}
    \end{equation}
    
    Next, we compute $\Hom_{\derb{X}}(\fm{\E}(k(s)), \fm{\E}(k(t))[i])$ for all closed points $s$, $t \in M$ and all $i \in \Z$. The first step is to obtain a more practical description of $\fm{\E}(k(t))$. As usual, let $t \colon \spec{\C} \to M$ denote the morphism corresponding to a closed point $t$ and let $i_t \colon X \to X \times M$ be defined like \eqref{eq: inclusion of X to X x S}. Then the following is a pull-back square, as $X$ may be seen as the fibre of the projection $p \colon X \times M \to M$ above the closed point $t$.
    \[\begin{tikzcd}[column sep = small, row sep = 1.8em]
        % https://q.uiver.app/#q=WzAsNCxbMCwwLCIxIl0sWzIsMCwiMiJdLFswLDIsIjMiXSxbMiwyLCI0Il0sWzAsMSwiXFxldGFfTSJdLFswLDIsInBeKiIsMl0sWzEsMywiXFxNX3YocCkiXSxbMiwzLCJcXGV0YSIsMl1d
        X && X \times M \\
        \\
        \spec{\C} && M
        \arrow["{i_t}", from=1-1, to=1-3]
        \arrow["{p|_t}"', from=1-1, to=3-1]
        \arrow["\lrcorner"{anchor=center, pos=0.125}, draw=none, from=1-1, to=3-3]
        \arrow["{p}", from=1-3, to=3-3]
        \arrow["{t}", from=3-1, to=3-3]
    \end{tikzcd}\]



    
    % our first task is to compute what the objects $\fm{\E}(k(t))$ are. As usual, let $t \colon \spec{\C} \to M$ denote the corresponding morphism and let $i_t \colon X \to X \times M$ be defined like \eqref{eq: inclusion of X to X x S}. Then the following is a pull-back square, as $X$ may be seen as the fibre above $t$ of the projection $p \colon X \times M \to X$.
    % \[\begin{tikzcd}[column sep = small, row sep = 1.8em]
    %     % https://q.uiver.app/#q=WzAsNCxbMCwwLCIxIl0sWzIsMCwiMiJdLFswLDIsIjMiXSxbMiwyLCI0Il0sWzAsMSwiXFxldGFfTSJdLFswLDIsInBeKiIsMl0sWzEsMywiXFxNX3YocCkiXSxbMiwzLCJcXGV0YSIsMl1d
    %     X && X \times M \\
    %     \\
    %     \spec{\C} && M
    %     \arrow["{i_t}", from=1-1, to=1-3]
    %     \arrow["{p|_t}"', from=1-1, to=3-1]
    %     \arrow["\lrcorner"{anchor=center, pos=0.125}, draw=none, from=1-1, to=3-3]
    %     \arrow["{p}", from=1-3, to=3-3]
    %     \arrow["{t}", from=3-1, to=3-3]
    % \end{tikzcd}\]
    As projections are always flat, by the flat base change formula \ref{} and the projection formula \ref{}, we compute
    \begin{align*}
        \fm{\E}(k(t)) &= ...\\
        & \iso i_t^*\E \\
        & = \E|_t.
    \end{align*}
    Again, we consider the following cases. 

    \vspace{0.3cm}
    \noindent
    \textsl{Case $s \neq t$.}  
    For $i < 0$ or $i > 2$, we know by Proposition \ref{Homs in derived category} that $\Hom_{\derb{X}}(\E|_s, \E|_t[i]) \iso \Ext^i_{\struct{X}}(\E|_s, \E|_t[i])$, which are trivial for $i < 0$. They are also trivial for $i > 2$, by Serre duality \ref{Serre duality for derived cats}. When $i = 0$, we claim that $\Hom_{\derb{X}}(\E|_s, \E|_t)$ is trivial. First, fully faithfulness of $\coherent{X} \to \derb{X}$ (\cf Proposition \ref{A -> D(A) fully faithful}), allows us to transport back into $\coherent{X}$ and show triviality of $\Hom_{\struct{X}}(\E|_s, \E|_t)$. Next, following the discussion of Remark \ref{Points of M are sheaves}, points $s$ and $t$ of $M$ naturally correspond to isomorphism classes of stable sheaves $\F$ and $\G$ on $X$, respectively. In this case we have
    \[
        \F \iso \E|_{[\F]} \iso \E|_s \quad \text{and} \quad \G \iso \E|_{[\G]} \iso \E|_t.
    \]
    As $s$ and $t$ are distinct, sheaves $\F$ and $\G$ are non-isomorphic, but share the same Mukai vector. As the Mukai vector of a sheaf determines its Hilbert polynomial, $P(\F) = P(\G)$. Applying \eqref{eq: schur lemma for stable sheaves} yields 
    \[
        \Hom_{\derb{X}}(\E|_s, \E|_t) = \Hom_{\struct{X}}(\F, \G)= 0. 
    \]
    From Serre duality \ref{Serre duality for derived cats} and freshly established case of $i = 0$, we see 
    \[
        \Hom_{\derb{X}}(\E|_s, \E|_t[2]) = 0.
    \]
    Lastly, for $i = 1$ it follows that $\Hom_{\derb{X}}(\E|_s, \E|_t[1]) \iso \Ext^1_\struct{X}(\E|_s, \E|_t)$ is trivial, because $v$ is isotropic and 
    \[
        \pairing{v}{v} = \chi(\E|_s, \E|_t) = \dim\Ext^0_\struct{X}(\E|_s, \E|_t) - \dim\Ext^1_\struct{X}(\E|_s, \E|_t) + \dim\Ext^2_\struct{X}(\E|_s, \E|_t),
    \]
    for $\Ext^0_\struct{X}(\E|_s, \E|_t)$ and $\Ext^2_\struct{X}(\E|_s, \E|_t)$ are trivial.

    \vspace{0.3cm}
    \noindent
    \textsl{Case $s = t$.}
    If $i < 0$ or $i > 2$, $\Hom_{\derb{X}}(\E|_s, \E|_s[i])$ is trivial, following the same argument as in the case of $s \neq t$. When $i = 0$, stability and \eqref{eq: schur lemma for stable sheaves} show that $\Hom_{\derb{X}}(\E|_s, \E|_s) \iso \C$. In the case $i = 2$, Serre duality \ref{Serre duality for derived cats} in conjunction with the previous isomorphism imply $\Hom_{\derb{X}}(\E|_s, \E|_s[2]) \iso \C$. Finally, for $v$ is isotropic 
    % Since by Proposition \ref{A -> D(A) fully faithful} the inclusion of $\coherent{X}$ into $\derb{X}$ is fully faithful and we have thus computed 

    In conclusion
    \begin{equation}
            \Hom_{\derb{X}}(\E|_s, \E|_t[i]) \iso \begin{cases}
                \C, &\text{if $s = t$ and $i \in \{0, 2\}$,} \\
                0, &\text{else.}
            \end{cases}
    \end{equation}

    Notice that \eqref{} now perfectly reflects \eqref{}, thus whenever $s \neq t$ or $s = t$ and $i \neq 0,2$, the functor $\fm{\E}$ induces isomorphisms between two trivial vector spaces
    \[
        \Hom_{\derb{M}}(k(s), k(t)[i]) \to \Hom_{\derb{X}}(\fm{\E}(k(s)), \fm{\E}(k(t))[i]).
    \]
    If $s = t$ and $i = 0$, $\fm{\E}$ clearly sends the identity morphism $\id{k(s)}$ to the identity $\id{\fm{\E}(k(s))}$, thus the linear map
    \[
        \Hom_{\derb{M}}(k(s), k(s)) \to \Hom_{\derb{X}}(\fm{\E}(k(s)), \fm{\E}(k(s)))
    \] 
    is non-zero. Because both spaces are 1-dimensional vector spaces, it is an isomorphism.
    If $s = t$ and $i = 2$, we use that Serre functors on $\derb{X}$ and $\derb{M}$ clearly commute with $\fm{\E}$, for they are just double shifts, to obtain a commutative square.
    \[\begin{tikzcd}[column sep = normal, row sep = 1.8em]
        % https://q.uiver.app/#q=WzAsNCxbMCwwLCIxIl0sWzIsMCwiMiJdLFswLDIsIjMiXSxbMiwyLCI0Il0sWzAsMSwiXFxldGFfTSJdLFswLDIsInBeKiIsMl0sWzEsMywiXFxNX3YocCkiXSxbMiwzLCJcXGV0YSIsMl1d
        \Hom_{\derb{M}}(k(s), k(s)[2]) && \Hom_{\derb{X}}(\fm{\E}(k(s)), \fm{\E}(k(s))[2]) \\
        \\
        \Hom_{\derb{M}}(k(s), k(s))^* && \Hom_{\derb{X}}(\fm{\E}(k(s)), \fm{\E}(k(s)))^*
        \arrow["{}", from=1-1, to=1-3]
        \arrow["{\sim}"', from=1-1, to=3-1]
        \arrow["{\sim}", from=1-3, to=3-3]
        \arrow["\sim"', from=3-3, to=3-1]
    \end{tikzcd}\]
    As three out of the four arrows are isomorphisms, so is the fourth. The assumption of Proposition \ref{Fully faithful functor on a spanning class} are now verified, allowing us to conclude that $\fm{\E}$ is fully faithful.

    \vspace{0.5 cm}
    \noindent
    \textsl{Essential surjectivity.} 
    
\end{proof}

\info{I mention Serre duality a million times in this proof, do I need to reference the theorem every time?}

\begin{remark}
    
\end{remark}
\newpage

\appendix
\section{Spectral sequences and how to use them}

In this chapter we will first define cohomological spectral sequences and discuss their meaning through applications connected with derived categories.

At first, when encountering spectral sequences, one might think of them as just  bookkeeping devices encoding a tremendous amount of data, but we will soon see how elegantly one can infer certain properties related to derived categorical claims exploiting the fact that they can naturally be encoded with spectral sequences.
% from a spectral sequence associated to a given problem  that we can encode a problem in terms of such a sequence. 
% just by considering their form for example.
One of the most important spectral sequences, which is also very general within our scope of inspection, will be the Grothendieck spectral sequence relating the higher derived functors of two composable functors with the higher derived functors of their composition. Later on we will see that many useful and well-known spectral sequences occur as special cases of the Grothendieck spectral sequence. What follows was gathered mostly from \cite{}

\info{
• Huybrechts FMinAG : chapter on ss \\
• Gelfand, Manin : III.7 \\
• McCleary  
}

For the time being we fix an abelian category $\mathcal A$ and start off with a definition.

\begin{definition}
    A \emph{cohomological spectral sequence} in an abelian category $\mathcal A$ consists of the following data on which we further impose two convergence conditions.
\begin{itemize}
    \item \emph{Sequence of pages.} 
    % A finite page at step $r \in \N$ is a bi-graded object $E^{\bullet, \bullet}_r$ together with differentials of bi-degree $(r, 1-r)$ \ie morphisms
    % \[
    %     d^{p,q}_r : E^{p,q}_r \to E^{p + r, q - r + 1}_r
    % \] 
    % satisfying
    % \[
    %     d^{p + r, q - r + 1}_r \circ d^{p,q}_r = 0,
    % \]
    % for each $p, q \in \Z$.
    A sequence of bi-graded objects $(E^{\bullet, \bullet}_r)_{r \in \N}$ equipped with differentials of bi-degree $(r, 1-r)$. The $r$-th term of this sequence is called the $r$-th \emph{page} and it consists of a lattice of objects $E^{p,q}_r$ of $\mathcal A$, for $p, q \in \Z$, and differentials
    \[
        d^{p,q}_r : E^{p,q}_r \to E^{p + r, q - r + 1}_r,
    \] 
    satisfying
    \[
        d^{p + r, q - r + 1}_r \circ d^{p,q}_r = 0,
    \]
    for each $p, q \in \Z$.
    \item \emph{Isomorphisms.} A collection of isomorphisms
    \[
        \alpha^{p,q}_r : H^{p,q}(E_r) \xrightarrow{\sim} E^{p,q}_{r + 1},
    \]
    for all $p, q \in \Z$ and $r \in \N$, where 
    \[
        H^{p,q}(E_r) := \ker(d^{p,q}_r)/\im(d^{p-r, q + r - 1}_r),
    \]
    which allow us to turn the pages.
    \item \emph{Transfinite page.} A bi-graded object $E^{\bullet, \bullet}_\infty$. 
    % Without any differentials.
    \item \emph{Goal of computation.} A sequence of objects $(E^n)_{n \in \Z}$ of the category $\mathcal A$.
\end{itemize}
The above collection of data also has to satisfy the following two convergence conditions.
\begin{enumerate}[(a)]
    \item For each pair $(p,q)$, there exists $r_0 \geq 0$, such that for all $r \geq r_0$ we have
    \[
        d^{p,q}_r = 0 \quad \text{and} \quad d^{p + r, q - r + 1}_r = 0
    \]
    and the isomorphism $\alpha^{p,q}_r$ can be taken to be the identity. We then say that the $(p,q)$-term \emph{stabilizes} after page $r_0$ and we denote $E^{p,q}_{r_0}$ (along with all the subsequent $E^{p,q}_{r}$ for $r \geq r_0$) by $E^{p,q}_\infty$.
    \item For each $n \in \Z$ there is a decreasing \emph{regular}\footnote{In our case the filtration $(F^p E^n)_{p \in \Z}$ is \emph{regular}, whenever $\bigcap_p F^p E^n = \lim_p F^p E^n = 0$ and $\bigcup_p F^p E^n = \colim_p F^p E^n = E^n$.} filtration of $E^n$
    \[
        E^n \supseteq \dots \supseteq F^p E^n \supseteq F^{p + 1}E^n \supseteq \dots \supseteq 0
    \]
    and isomorphisms
    \[
        \beta^{p,q} : E^{p,q}_\infty \xrightarrow{\sim} F^p E^{p + q}/F^{p + 1}E^{p + q}
    \]
    for all $p, q \in \Z$.
\end{enumerate}

In this case we also denote the existence of such a spectral sequence by 
\[
    E^{p,q}_r \implies E^n.
\]
\end{definition}

\begin{remark}
    A few words are in order to justify us naming the sequence $(E^n)_{n \in \Z}$ our \emph{goal of computation}. Usually one is given a starting page or a small number of them and the first goal is to identify the transfinite page -- we are refering to convergence condition (a). Often one is able to infer the differentials degenerate after a number of turns of the pages from context or by observing the shape of the spectral sequence. For example \emph{first quadrant spectral sequences}, \ie the ones with non-trivial $E^{p,q}_r$ only for $(p,q)$ lying in the first quadrant, always satisfy condition (a).
    
    The second part of the computation is concerned with relating objects from the transfinte page $E^{\bullet, \bullet}_\infty$ with objects $E^n$. This is captured in the convergence condition (b), from which we can clearly observe that the intermediate quotients of the filtration $(F^p E^n)_{p \in \Z}$ for a fixed term $E^n$ lie on the anti-diagonal of the transfinite page passing through \eg $E^{n, 0}_\infty$.
    
    In condition (b) the existence of isomorphisms $\beta^{p,q}$ can also be restated by saying that $E^{p,q}_\infty$ fits into a short exact sequence
    \[
        0 \to F^{p + 1} E^n \to F^p E^n \to E^{p,q}_\infty \to 0.
    \]
    This observation becomes very useful when considering properties of objects of the category $\mathcal A$ which are closed under extensions, especially when the filtration of $E^n$ is finite.

    % In that case one can inductively reason about such a property of $E^n$ provided the filtration of $E^n$ is finite.
 \end{remark}

% Next we mention two ways of naturally obtaining a spectral sequence, the first one arises from a filtered chain complex. A filtration of a chain complex $K^\bullet \in \Ob(\com(\mathcal A))$ is a decreasing sequence of chain complexes 
% \[
%  K^\bullet \supseteq \dots \supseteq F^p K^\bullet \supseteq F^{p+1}K^\bullet \supseteq \dots \supseteq 0
% \]
% respecting the differential of $K^\bullet$ \ie $d(F^p K^n) \subseteq F^p K^{n+1}$. Using only this piece of information one can 

\info{mention Tohoku paper}

\begin{theorem}[Grothendieck]
    Let $F\colon \A \to \B$ and $G\colon \B \to \mathcal C$ be left exact functors between abelian categories. Let $\I_F$ be an $F$-adapted class, $\I_G$ a $G$-adapted class of objects and suppose every object of $\I_F$ is sent to a $G$-acyclic object by the functor $F$. Then for every $A^\bullet$ in $K^+(\A)$ there exists a spectral sequence
    \[
        E^{p,q}_2 = \rderived{p}{G}(\rderived{q}{F}(A^\bullet)) \implies \rderived{p+q}{(G \circ F)}(A^\bullet)
    \] 
\end{theorem}
\section{Lattice theory}
\label{appendix B}

This short chapter's purpose is mainly to establish the language of lattice theory. It contains two highly non-trivial results as well. One on extending certain isometries due to Nikulin \cite{Nikulin1980}, which we are generously utilizing in the proof of Proposition \ref{transcendental iso iff Mukai iso}. And another due to Milnor, one which the first one builds upon, classifying indefinite even unimodular lattices. The second one is not strictly necessary anywhere in this thesis, but serves as a pleasant way to make the intersection pairing on $\cohom{2}{X}{\Z}$ and later the Mukai pairing on $\mukai{X}$ of a K3 surface $X$ a bit more concrete. Our main sources were \cite[\S 14]{Huybrechts2016} and \cite{Nikulin1980}.

\begin{definition}
    A free finitely generated $\Z$-module $\Lambda$ together with a bilinear form $\intersect{\cdot}{\cdot} \colon \Lambda \times \Lambda \to \Z$, which is
    \begin{itemize}
        \item \emph{symmetric:} $\intersect{x}{y} = \intersect{y}{x}$, for all $x$, $y \in \Lambda$,
        \item \emph{non-degenerate:} if $\intersect{x}{y} = 0$ for all $y \in \Lambda$, then $x = 0$,
    \end{itemize}
    is called a \emph{lattice}. Lattice $\Lambda$ is said to be \emph{even} if $\intersect{x}{x}$ is an even integer for all $x \in \Lambda$. An element $x \in \Lambda$ is \emph{isotropic} if $\intersect{x}{x} = 0$.
    % a non-degenerate symmetric bilinear form $\intersect{\cdot}{\cdot} \colon \Lambda \times \Lambda \to \Z$ is called a \emph{lattice}. The lattice is said to be \emph{even} if $\intersect{x}{x}$ is an even integer for all $x \in \Lambda$.
\end{definition}

By picking an integral basis of $\Lambda$ one can represent a lattice by a symmetric integral matrix, called the \emph{intersection matrix}. If $x_1, \dots, x_n$ form an integral basis of $\Lambda$ the intersection matrix is
\[
    \bigl( (x_i.x_j)\bigr)_{1 \leq i, j \leq n}.
\]
The determinant of the intersection matrix is called the \emph{discriminant} of the lattice $\Lambda$, denoted $\operatorname{disc}\Lambda$, and is independent of the choice of the integral basis for $\Lambda$. Upon tensoring $\Lambda$ with $\R$ and $\R$-linearly extending the bilinear form $\intersect{\cdot}{\cdot}$, we obtain a non-degenerate symmetric bilinear form on $\Lambda \otimes_\Z \R$. These are known to be diagonalizable with only $1$ or $-1$ on the diagonal and we let $n_\pm$ denote the number of $\pm 1$ on the diagonal. The quantities $n_\pm$ are readily seen to be invariants of $\Lambda$. We call the pair $(n_+, n_-)$ the \emph{signature} of $\Lambda$ and the $n_+ + n_- = \rk\Lambda$ the \emph{rank} of $\Lambda$. The lattice $\Lambda$ is called \emph{definite} if either $n_+ = 0$ or $n_- = 0$ and \emph{indefinite} otherwise.



% Since the intersection matrix is symmetric, we may diagonalize it over $\R$. Because the bilinear form is non-degenerate the intersection matrix is non-singular thus the diagonalization process results in a matrix with only non-zero values on the diagonal. Denote with $n_+$ the number of positive values on the diagonal and with $n_-$ 

% Upon tensoring a lattice $\Lambda$ with $\R$ and $\R$-linearly extending the bilinear form $\intersect{\cdot}{\cdot}$,
% we obtain a finitely generated $\R$-vector space $\Lambda \otimes_\Z \R$ and $\R$-linearly extending the bilinear form $\intersect{\cdot}{\cdot}$, 
% we obtain a non-degenerate symmetric bilinear form on $\Lambda \otimes_\Z \R$. 
% In a given basis this form is represented by a symmetric matrix 
% We call the pair $(n_+, n_-)$ the \emph{signature} of $\Lambda$ and the quantity $n_+ + n_- = \rk{\Lambda}$ the \emph{rank} of $\Lambda$. The lattice $\Lambda$ is called \emph{definite}, if either $n_+ = 0$ or $n_- = 0$ and \emph{indefinite} otherwise.

\begin{definition}
    A lattice $\Lambda$ is called \emph{unimodular} if the morphism
    \[
        i_\Lambda \colon \Lambda \to \Lambda^* = \Hom_\Z(\Lambda, \Z), \qquad x \mapsto \intersect{x}{-}
    \]
    is an isomorphism of $\Z$-modules or equivalently\footnote{
        $\Lambda^*$ is also a free $\Z$-module and by non-degeneracy of $\intersect{\cdot}{\cdot}$, $i_\Lambda$ is an embedding. Then \eg using the Smith normal form allows one to see that $\abs{\operatorname{disc}\Lambda}$ equals the index $[\Lambda^* : \Lambda]$.
    } $\operatorname{disc}\Lambda = \pm 1$.
\end{definition}

\begin{remark}
    The morphism $i_\Lambda \colon \Lambda \to \Lambda^*$ is always injective, because the bilinear form $\intersect{\cdot}{\cdot}$ on $\Lambda$ is non-degenerate.
\end{remark}

\begin{example}
    \begin{enumerate}[label = (\roman*)]
        \item{The \emph{hyperbolic} lattice is defined by the intersection matrix
        \[
            \begin{pmatrix}
                0 & 1\\
                1 & 0
            \end{pmatrix}.
        \]
        We denote this indefinite unimodular lattice of rank two with $\hyperbolic$.}
        \item{The \emph{$E_8$-lattice} is given by the intersection matrix
        \[
            \begin{pmatrix}
                2 & -1 & & & & & & \\
                -1 & 2 & -1 & & & & & \\
                & -1 & 2 & -1 & -1 & & & \\
                & & -1 & 2 & 0 & & & \\
                & & -1 & 0 & 2 & -1 & & \\
                & & & &  -1 & 2 & -1 & \\
                & & & & &  -1 & 2 & -1 & \\
                & & & & & & -1 & 2
            \end{pmatrix}.
        \]
        The $E_8$-lattice is the unique even unimodular positive definite lattice of rank $8$ and we denote it by $E_8$.}
    \end{enumerate}
\end{example}

\begin{definition}
    The orthogonal direct sum of lattices $\Lambda$ and $\Lambda'$ is the free $\Z$-module $\Lambda \oplus \Lambda'$ equipped with the following bilinear form
    \[
        \intersect{(x,x')}{(y,y')} := \intersect{x}{y}_{\Lambda} + \intersect{x'}{y'}_{\Lambda'}.
    \]
    A \emph{twist} of a lattice $\Lambda$ by an integer $m \in \Z$ is the lattice $\Lambda(m)$ given by the same underlying $\Z$-module $\Lambda$, but equipped with the \emph{twisted} bilinear form
    \[
        \intersect{\cdot}{\cdot}_{\Lambda(m)} = m \cdot \intersect{\cdot}{\cdot}_{\Lambda}.
    \]
\end{definition}

\begin{definition}
    Let $\Lambda$ and $\Lambda'$ be two lattices. A morphism of lattices or an \emph{isometry} is a $\Z$-linear map $\phi \colon \Lambda \to \Lambda'$ satisfying
    \[
        \intersect{\phi(x)}{\phi(y)}_{\Lambda'} = \intersect{x}{y}_{\Lambda},
    \]
    for all $x$, $y \in \Lambda$. A lattice $L$ is called a \emph{sublattice} of $\Lambda$ if $L \subseteq \Lambda$ as underlying $\Z$-modules and the embedding $L\hookrightarrow \Lambda$ is an isometry. The \emph{orthogonal complement} of a sublattice $L$ in $\Lambda$ is defined to be
    \[
        L^\perp = \{x \in \Lambda \mid \intersect{x}{y} = 0, \text{ for all $y \in L$}\}.
    \]
\end{definition}

\begin{lemma}
    \label{orthogonal complemet is a lattice}
    Let $L$ be a sublattice of a lattice $\Lambda$. Then the orthogonal complement $L^\perp$ with the inherited bilinear form on $\Lambda$ is also a sublattice.
\end{lemma}

\begin{proof}
    First, $L^\perp$ is a subgroup of a free finite rank abelian group $\Lambda$, thus it is itself also free of finite rank. It is left to show that the inherited form on $L^\perp$ is also non-degenerate. For this we consider the rational extensions $L_\Q \subseteq \Lambda_\Q$ and show instead that the extended form on $L_\Q^\perp$ is non-degenerate. We denote with $W$ and $V$ the $\Q$-vector spaces $L_\Q$ and $\Lambda_\Q$, respectively. The adjoint map $V \to V^*$, given by $v \mapsto \intersect{v}{-}$, is injective and thus an isomorphism, therefore the linear map
    \[  
        V \longrightarrow W^*, \qquad v \mapsto \intersect{v}{-}|_W
    \]  
    is surjective. Its kernel is $W^\perp$, so $\dim W^\perp + \dim W = \dim V$. Thus $W^\perp \oplus W = V$ is an orthogonal direct sum decomposition. Assume now that $w \in W^\perp$ is such that $\intersect{w}{w'} = 0$ for all $w \in W^\perp$. Then $\intersect{w}{v} = 0$ for all $v \in V$ and by non-degeneracy of the form on $V$ this implies $w = 0$.
\end{proof}

\begin{definition}
    An embedding of lattices $L \hookrightarrow \Lambda$ is said to be \emph{primitive} if its cokernel $\coker(L \hookrightarrow \Lambda) = \Lambda/L$ is free. In that case we also call $L$ a \emph{primitive sublattice} of $\Lambda$.
\end{definition}

\begin{example}
    \label{orthogonal complement primitive}
    Let $L$ be any sublattice of a lattice $\Lambda$. Then the orthogonal complement $L^\perp$ is a primitive sublattice of $\Lambda$. Indeed, suppose $x \in \Lambda$ represents a torsion element in $\Lambda/L^\perp$. Then $n x \in L^\perp$ for some positive integer $n \in \Z$. But this implies that $x \in L^\perp$, so $x$ represents the zero element in $\Lambda/L^\perp$. Hence $\Lambda/L^\perp$ is free.
\end{example}

% \begin{proof}
%     First $L^\perp$ is a subgroup of a free finite rank abelian group $\Lambda$, thus it is itself also free of finite rank. 

    % To see that $L^\perp \hookrightarrow \Lambda$ is primitive, suppose $x \in \Lambda$ represents a torsion element in $\Lambda/L^\perp$. Then $n x \in L^\perp$ for some positive integer $n \in \Z$. But this implies that also $x \in L^\perp$ so $x$ represents the zero element in $\Lambda/L^\perp$.
% \end{proof}
    
    
    % Indeed, if $x \in \Lambda$ represented a torsion class in $\Lambda/L^\perp$, then $n x \in L^\perp$ for some positive integer $n \in \Z$. But this implies that $x \in L^\perp$, so its class in $\Lambda/L^\perp$ is trivial.

\begin{lemma}
    \label{composition of primitive embeddings}
    Let $i \colon L \hookrightarrow \Lambda$ and $i' \colon \Lambda \hookrightarrow \Lambda'$ be a pair of primitive embeddings. Then their composition $i' \circ i \colon L \hookrightarrow \Lambda'$ is also a primitive embedding.
\end{lemma}

\begin{proof}
    Since $\Lambda/L$ and $\Lambda'/\Lambda$ are free abelian, the short exact sequences $0 \to L \to \Lambda \to \Lambda/L \to 0$ and $0 \to \Lambda \to \Lambda' \to \Lambda'/\Lambda \to 0$ are split. This means that there are retracts $r \colon \Lambda \to L$ and $r' \colon \Lambda' \to \Lambda$, thus the short exact sequence $0 \to L \to \Lambda' \to \Lambda'/L \to 0$ also splits, which makes $\Lambda'/L$ free as a subgroup of $\Lambda'$.
\end{proof}

\begin{proposition}
    % [Nikulin, \cite{Nikulin1980}]
    \label{extending an isometry 1}
    \emph{\cite{Nikulin1980}}
    Let $L$ and $\Lambda$ be even lattices and assume $\Lambda$ is unimodular. Let $i_0$ and $i_1$ denote two primitive embeddings $L \hookrightarrow \Lambda$. Suppose the orthogonal complement $i_0(L)^\perp$ in $\Lambda$ contains a copy of the hyperbolic lattice $\hyperbolic$. Then there exists an isometry $\phi \colon \Lambda \to \Lambda$ such that $i_1 = \phi \circ i_0$.
    \[\begin{tikzcd}[row sep = 1em, column sep = 2em]
        % https://q.uiver.app/#q=WzAsMyxbMSwwLCJMIl0sWzAsMiwiXFxMYW1iZGEiXSxbMiwyLCJcXExhbWJkYSJdLFswLDEsImlfMCIsMix7InN0eWxlIjp7InRhaWwiOnsibmFtZSI6Imhvb2siLCJzaWRlIjoiYm90dG9tIn19fV0sWzAsMiwiaV8xIiwwLHsic3R5bGUiOnsidGFpbCI6eyJuYW1lIjoiaG9vayIsInNpZGUiOiJ0b3AifX19XSxbMSwyLCJcXHBoaSJdXQ==
        & L \\
        \\
        \Lambda && \Lambda
        \arrow["{i_0}"', hook', from=1-2, to=3-1]
        \arrow["{i_1}", hook, from=1-2, to=3-3]
        \arrow["\phi", from=3-1, to=3-3]
    \end{tikzcd}\]
\end{proposition}

\begin{proof}
    See \cite[\S 14, Theorem 1.12]{Huybrechts2016} and in particular point (iii) in the subsequent remark. For a more general statement, from which this one follows, consult \cite[Theorem 1.14.4]{Nikulin1980}.
\end{proof}

\begin{corollary}
    \label{extending an isometry 2}
    Let $i_0 \colon L_0 \hookrightarrow \Lambda_0$ and $i_1 \colon L_1 \hookrightarrow \Lambda_1$ denote two primitive embeddings of even lattices $L_0$ and $L_1$ into isomorphic even unimodular lattices $\Lambda_0$ and $\Lambda_1$, respectively. Suppose that there is an isomorphism of lattices $\psi \colon L_0 \to L_1$ and assume the orthogonal complement $i_0(L_0)^\perp$ in $\Lambda_0$ contains a copy of the hyperbolic lattice $\hyperbolic$. Then there is an isometry $\phi \colon \Lambda_0 \to \Lambda_1$, for which $\phi \circ i_0 = i_1 \circ \psi$.
    \[\begin{tikzcd}[column sep = 2em]
        % https://q.uiver.app/#q=WzAsNCxbMCwwLCJMXzAiXSxbMCwyLCJcXExhbWJkYSJdLFsyLDIsIlxcTGFtYmRhIl0sWzIsMCwiTF8xIl0sWzAsMSwiaV8wIiwyLHsic3R5bGUiOnsidGFpbCI6eyJuYW1lIjoiaG9vayIsInNpZGUiOiJib3R0b20ifX19XSxbMSwyLCJcXHBoaSJdLFszLDIsImlfMSIsMCx7InN0eWxlIjp7InRhaWwiOnsibmFtZSI6Imhvb2siLCJzaWRlIjoidG9wIn19fV0sWzAsMywiXFxwc2kiXV0=
        {L_0} && {L_1} \\
        \\
        \Lambda_0 && \Lambda_1
        \arrow["\psi", from=1-1, to=1-3]
        \arrow["{i_0}"', hook', from=1-1, to=3-1]
        \arrow["{i_1}", hook, from=1-3, to=3-3]
        \arrow["\phi", from=3-1, to=3-3]
    \end{tikzcd}\]
\end{corollary}

\begin{proof}
    Let $\theta \colon \Lambda_1 \to \Lambda_0$ be an isomorphism. Then $\theta \circ i_1 \circ \psi \colon L_0 \to \Lambda_0$ is a primitive embedding and we can use Proposition \ref{extending an isometry 1}.
\end{proof}

\begin{theorem}[Milnor, \text{\cite[Theorem 1.1.1]{Nikulin1980}}]
    \label{Milnor}
    Let $(n_+,n_-)$ be a pair of non-negative integers. Then there exists an even unimodular lattice of signature $(n_+, n_-)$ if and only if $n_+ - n_- \equiv 0 \ (\text{\emph{mod} }8)$. If $n_+$, $n_- > 0$, then the (indefinite) lattice is unique up to isomorphism.
\end{theorem}

% \begin{corollary}
%     Let $\Lambda_0$ and $\Lambda_1$ be positive definite even unimodular lattices with $\rk{\Lambda_0} = \rk{\Lambda_1}$, then
%     \[
%         \Lambda_0 \oplus \hyperbolic \iso \Lambda_1 \oplus \hyperbolic.
%     \]
% \end{corollary}

% \begin{proof}
%     Signature of $\Lambda_0$ and $\Lambda_1$ is $(\rk{\Lambda_0}, 0)$, thus the signature of $\Lambda_0 \oplus \hyperbolic$ and $\Lambda_1 \oplus \hyperbolic$ is $(\rk{\Lambda_0} + 1, 1)$, so the result follows by Milnor's Theorem \ref{Milnor}.
% \end{proof}

Its easy yet powerful corollary allow us to abstractly identify the lattice structures on $\cohom{2}{X}{\Z}$ and $\mukai{X}$.

\begin{theorem}
    \label{Classification of indefinite even unimodular lattices}
    Let $\Lambda$ be an indefinite even unimodular lattice of signature $(n_+, n_-)$. Setting $\tau = n_+ - n_-$ to be the index of $\Lambda$, then $\tau \equiv 0 \ (\text{\emph{mod }}8)$ and
    \begin{center}
        \begin{tabular}{r l}
            if $\tau \geq 0$, then & $\Lambda \iso E_8^{\oplus \frac{\tau}{8}} \oplus \hyperbolic^{\oplus n_-},$ \\
            if $\tau \leq 0$, then & $\Lambda \iso E_8(-1)^{\oplus \frac{-\tau}{8}} \oplus \hyperbolic^{\oplus n_+}.$
        \end{tabular}
    \end{center}
\end{theorem}

\begin{proof}
    The $E_8$ lattice has signature $(8,0)$, and the hyperbolic lattice $\hyperbolic$ has signature $(1,1)$, thus the signature of $E_8^{\oplus \frac{\tau}{8}} \oplus \hyperbolic^{\oplus n_-}$ equals $(8\cdot \tfrac{\tau}{8} + n_-,n_-) = (n_+, n_-)$. Now use uniqueness from Theorem \ref{Milnor}.
\end{proof}

\begin{remark}
    We know from Proposition \ref{intersection pairing on K3 is even} that for a K3 surface $X$ the Hodge lattice $\cohom{2}{X}{\Z}$ has signature $(3,19)$ and from Remark \ref{Mukai lattice structure} that $\mukai{X}$ has signature $(20,4)$, therefore Theorem \ref{Classification of indefinite even unimodular lattices} tells us that
    \[
        \cohom{2}{X}{\Z} \iso E_8(-1)^{\oplus 2} \oplus \hyperbolic^{\oplus 3} \quad \text{and} \quad
        \mukai{X} \iso E_8^{\oplus 2} \oplus \hyperbolic^{\oplus 4}.
    \]
\end{remark}
% \info{These three results are from Huybrects K3}

% \section{Test}
Text and math $a, b \in \Z$ Coh
\[
\int_{M} d\omega = \int_{\partial M } \omega
\]

\[
    D^b(\mathtt{Coh}(X))
\]

\end{document}