% \chapter{Derived categories}
\section{\textsf{redistribute: }Derived categories}

\subsection{Categories of complexes}
In order to define derived categories of an additive category $\A$ we first introduce the category of complexes and the homotopic category of complexes of $\A$. We will equip these with a triangulated structure and . Throughout this section $\A$ will be a fixed additeve or $k$-linear category and we also mention that we will be using the cohomological indexing convention.

As a preliminary we introduce graded objects in ??

\subsubsection*{{Category of chain complexes}}

By a \emph{chain complex} in $\A$ we mean a collection of objects and morphisms 
\[
    A^\bullet = \left((A^i)_{i \in \Z}, (d^i_A\colon A^i \to A^{i+1})_{i \in \Z}\right),
\] 
where $A^i$ are objects and $d^i$ are morphisms of $\A$, called \emph{differenitials}, subject to equations $d^{i+1}\circ d^i = 0$, for all $i \in \Z$. A complex is \emph{bounded from below} (\resp \emph{bounded from above}), if there exists $i_0 \in \Z$ for which $A^i \iso 0$ for all $i \leq i_0$ (\resp $i \geq i_0$) and is \emph{bounded}, if it is both bounded from below and bounded from above. A \emph{chain map} between two chain complexes $A^\bullet$ and $B^\bullet$ in $\A$ is a collection of morphisms in $\A$
\[
    f^\bullet = (f^i\colon A^i \to B^i)_{i \in \Z},
\]
for which $f^{i+1}\circ d^i_A = d^i_B \circ f^i$ holds for all $i \in \Z$. This may diagramatically be described by the following commutative ladder.
\[
\begin{tikzcd}
    % https://q.uiver.app/#q=WzAsMTAsWzEsMCwiQV57aS0xfSJdLFszLDAsIkFeaSJdLFs1LDAsIkFee2krMX0iXSxbMSwyLCJCXntpLTF9Il0sWzMsMiwiQl5pIl0sWzUsMiwiQl57aSsxfSJdLFs2LDIsIlxcY2RvdHMiXSxbNiwwLCJcXGNkb3RzIl0sWzAsMCwiXFxjZG90cyJdLFswLDIsIlxcY2RvdHMiXSxbMCwzLCJmXntpLTF9Il0sWzEsNCwiZl5pIl0sWzIsNSwiZl57aSsxfSJdLFswLDEsImRee2ktMX1fQSJdLFsxLDIsImReaV9BIl0sWzQsNSwiZF57aX1fQiJdLFszLDQsImRee2ktMX1fQiJdLFs1LDZdLFsyLDddLFs4LDBdLFs5LDNdXQ==
	\cdots & {A^{i-1}} && {A^i} && {A^{i+1}} & \cdots \\
	\\
	\cdots & {B^{i-1}} && {B^i} && {B^{i+1}} & \cdots
	\arrow[from=1-1, to=1-2]
	\arrow["{d^{i-1}_A}", from=1-2, to=1-4]
	\arrow["{f^{i-1}}", from=1-2, to=3-2]
	\arrow["{d^i_A}", from=1-4, to=1-6]
	\arrow["{f^i}", from=1-4, to=3-4]
	\arrow[from=1-6, to=1-7]
	\arrow["{f^{i+1}}", from=1-6, to=3-6]
	\arrow[from=3-1, to=3-2]
	\arrow["{d^{i-1}_B}", from=3-2, to=3-4]
	\arrow["{d^{i}_B}", from=3-4, to=3-6]
	\arrow[from=3-6, to=3-7]
\end{tikzcd}
\]
Sometimes we will denote a complex just by $A$, $B$,... insted of $A^\bullet$, $B^\bullet$,... to simplify notation. 

Next we define the \emph{category of chain complexes} in $\A$, denoted by $\com(\A)$, as the following additive category. 
\begin{center}
    \begin{tabular}{r l}
        \textsl{Objects:} & chain complexes in $\A$. \\
        \textsl{Morphims:} & $\Hom_{\com(\A)}(A, B)$ is the set of chain maps $A \to B$, \\ & equipped with a group structure inherited from $\A$ \\ & by applying operations componentwise.
    \end{tabular}
\end{center}
The composition law is defined componentwise and is clearly associative and bilinear. The identity morphisms $1_{A^\bullet}$ are defined to be $(1_{A^i})_{i \in \Z}$. The complex $\cdots \to 0 \to 0 \to \cdots$ plays the role of the zero object in $\com(\A)$ and the biproduct of complexes $A$ and $B$ exists and is witnessed by the chain complex
\[
    A \oplus B = \left( (A^i \oplus B^i)_{i \in \Z}, (d^i_A \oplus d^i_B)_{i \in \Z}\right),
    % \quad \quad \cdots \to A^i \oplus B^i \xrightarrow{d^i_A \oplus d^i_B} A^{i+1} \oplus B^{i+1} \to \cdots,
\]
together with the cannonical projection and injection morphisms arising from biproducts componentwise.

Additionally, we also define the following full additive subcategories of $\com(\A)$.
\begin{center}
    \begin{tabular}{c l}
        $\com^+(\A)$ & \emph{Category of complexes bounded below}, spanned on \\ &complexes in $\A$ bounded below. \\
        $\com^-(\A)$ & \emph{Category of complexes bounded above}, spanned on \\ &complexes in $\A$ bounded above. \\
        $\com^b(\A)$ & \emph{Category of bounded complexes}, spanned on \\ &bounded complexes in $\A$. \\    
    \end{tabular}
\end{center}
\begin{remark}
    Whenever $\A$ is $k$-linear, all the categories of complexes $\com^*(\A)$ become $k$-linear as well in the obvious way.
\end{remark}

On all the categories of complexes mentioned above, we can now define the translation functor
\[
    T\colon \com^*(\A) \to \com^*(\A)
\]  
given by its action on objects and morphisms as follows.

\begin{center}
    \begin{tabular}{r l}
        \textsl{Objects:} & $T(A^\bullet) = A^\bullet[1]$ is the chain complex with $(A^\bullet[1])^i := A^{i+1}$ \\ & and differentials $d^i_{A[1]} = - d^{i+1}_{A}$. \\
        \textsl{Morphims:} & For a chain map $f^\bullet\colon A^\bullet \to B^\bullet$ we define $f^\bullet[1]$ to have \\ & component maps $(f^\bullet[1])^i = f^{i+1}$.
    \end{tabular}
\end{center}
The translation functor $T$ thus acts on a complex $A^\bullet$ by twisting its differential by a sign and shifting it one step to the \emph{left}, which is graphically pictured below.

\[\begin{tikzcd}[row sep = 1.2em]
    % https://q.uiver.app/#q=WzAsMTcsWzAsMSwiQV5cXGJ1bGxldFxcY29sb24iXSxbMCwyLCJBXlxcYnVsbGV0WzFdXFxjb2xvbiJdLFsxLDAsIlxcY2RvdHMiXSxbMywxLCJBXjAiXSxbNCwxLCJBXjEiXSxbMiwxLCJBXnstMX0iXSxbMiwyLCJBXjAiXSxbMywyLCJBXjEiXSxbNCwyLCJBXjIiXSxbNSwxLCJcXGNkb3RzIl0sWzUsMiwiXFxjZG90cyJdLFsxLDEsIlxcY2RvdHMiXSxbMSwyLCJcXGNkb3RzIl0sWzMsMCwiXFx0ZXh0dHR7MH0iXSxbNCwwLCJcXHRleHR0dHsxfSJdLFsyLDAsIlxcdGV4dHR0ey0xfSJdLFs1LDAsIlxcY2RvdHMiXSxbNiw3XSxbNyw4XSxbNSwzXSxbMyw0XSxbNCw5XSxbMTEsNV0sWzEyLDZdXQ==
	& \color{linkcolor}\cdots & {\color{linkcolor}{\mathtt{-1}}} & {\color{linkcolor}\small{\mathtt{0}}} & {\color{linkcolor}\small{\mathtt{1}}} & {\color{linkcolor}\small{\mathtt{2}}} & \color{linkcolor}\cdots \\
	{A^\bullet} & \cdots & {A^{-1}} & {A^0} & {A^1} & {A^2} & \cdots \\
	{A^\bullet[1]} & \cdots & {A^0} & {A^1} & {A^2} & {A^3} & \cdots
	\arrow[from=2-2, to=2-3]
	\arrow[from=2-3, to=2-4]
	\arrow[from=2-4, to=2-5]
	\arrow[from=2-5, to=2-6]
	\arrow[from=2-6, to=2-7]
	\arrow[from=3-2, to=3-3]
	\arrow[from=3-3, to=3-4]
	\arrow[from=3-4, to=3-5]
    \arrow[from=3-5, to=3-6]
    \arrow[from=3-6, to=3-7]
\end{tikzcd}\]

\begin{remark}
    We remark that the translation functor $T$ is clearly also additive or $k$-linear, whenever $\A$ is additive or $k$-linear.
\end{remark}

Since $T$ is an auto-equivalence there exists a quasi-inverse $T\inv$ to $T$, which is defined and unique up to a natural isomorphism. We may then speak of $T^{k}$ for any $k \in \Z$, whose action on a complex $A^\bullet$ is described by $(A^\bullet[k])^i = A^{i + k}$ with differential $d^{i}_{A[k]} = (-1)^k d^{i+k}_A$.

\subsubsection*{Homotopy category of complexes}

In this subsection we construct the homotopy category of chain complexes associated to a given additive category $\A$ and equip it with a triangulated structure. The main motivation for its introduction in this work is the fact that we will later on use it to construct the derived category of $\A$. In particular the homotopy category of $\A$, as opposed to the category of complexes\footnote{It is still possible to construct the derived category of $\A$ without passing through the homotopy category of complexes, however equipping it with a triangulated structure in that case becomes less elegant.} $\com(\A)$, can be enhanced with a triangulated structure which will afterwards descend to the level of derived categories.

\begin{definition}
    Let $f^\bullet$ and $g^\bullet$ be two chain maps in $\Hom_{\com(\A)}(A^\bullet, B^\bullet)$. We define $f^\bullet$ and $g^\bullet$ to be \emph{homotopic}, if there exists a collection of morphisms $\left( h^i\colon A^i \to B^{i-1} \right)_{i \in \Z}$, satisfying
    \[
        f^i - g^i = h^{i+1} \circ d^i_A + d^{i-1}_B \circ h^i
    \] 
    for all $i \in \Z$. The collection of morphisms $(h^i)_{i \in \Z}$ is called a \emph{homotopy} and we denote $f^\bullet$ and $g^\bullet$ being homotopic by $f^\bullet \htpy g^\bullet$. 
    \[\begin{tikzcd}
        % https://q.uiver.app/#q=WzAsMTAsWzEsMCwiQV57aS0xfSJdLFszLDAsIkFeaSJdLFs1LDAsIkFee2krMX0iXSxbMSwyLCJCXntpLTF9Il0sWzMsMiwiQl5pIl0sWzUsMiwiQl57aSsxfSJdLFs2LDIsIlxcY2RvdHMiXSxbNiwwLCJcXGNkb3RzIl0sWzAsMCwiXFxjZG90cyJdLFswLDIsIlxcY2RvdHMiXSxbMCwzLCJmXntpLTF9Il0sWzEsNCwiZl5pIl0sWzIsNSwiZl57aSsxfSJdLFswLDEsImRee2ktMX1fQSJdLFsxLDIsImReaV9BIl0sWzQsNSwiZF57aX1fQiJdLFszLDQsImRee2ktMX1fQiJdLFs1LDZdLFsyLDddLFs4LDBdLFs5LDNdLFsxLDMsImheaSJdLFsyLDQsImhee2krMX0iXV0=
        \cdots & {A^{i-1}} && {A^i} && {A^{i+1}} & \cdots \\
        \\
        \cdots & {B^{i-1}} && {B^i} && {B^{i+1}} & \cdots
        \arrow[from=1-1, to=1-2]
        \arrow["{d^{i-1}_A}", from=1-2, to=1-4]
        \arrow["{f^{i-1}}"', from=1-2, to=3-2]
        \arrow["{d^i_A}", from=1-4, to=1-6]
        \arrow["{h^i}"', from=1-4, to=3-2]
        \arrow["{f^i}"', from=1-4, to=3-4]
        \arrow[from=1-6, to=1-7]
        \arrow["{h^{i+1}}"', from=1-6, to=3-4]
        \arrow["{f^{i+1}}"', from=1-6, to=3-6]
        \arrow[from=3-1, to=3-2]
        \arrow["{d^{i-1}_B}", from=3-2, to=3-4]
        \arrow["{d^{i}_B}", from=3-4, to=3-6]
        \arrow[from=3-6, to=3-7]
        \end{tikzcd}\] 
    We say $f^\bullet$ is \emph{nullhomotopic}, if $f^\bullet \htpy 0$.
\end{definition}

\begin{lemma}
    \label{nullhomotopics form a subgroup}
    Let $A^\bullet$, $B^\bullet$ and $C^\bullet$ be complexes in $\com(\A)$, and let $f, f' \in \Hom_{\com(\A)}(A^\bullet, B^\bullet)$ and $g, g' \in \Hom_{\com(\A)}(B^\bullet, C^\bullet)$ be chain maps.
    \begin{enumerate}[(i)]
        \item The subset of all nullhomotopic chain maps in $A^\bullet \to B^\bullet$ forms a submodule of $\Hom_{\com(\A)}(A^\bullet, B^\bullet)$.
        \item If $f \htpy f'$ and $g \htpy g'$, then $g \circ f \htpy g' \circ f'$.
        \info{Pick one notation convention, sometimes its $f$, $f'$, sometimes its $f_0$, $f_1$, ...}
    \end{enumerate}
\end{lemma}

\begin{proof}
    See \cite[]{}.
\end{proof}

The homotopy category of complexes in $\A$, denoted by $K(\A)$, is defined to be an additive category consisting of
\begin{center}
    \begin{tabular}{r l}
        \textsl{Objects:} & chain complexes in $\A$. \\
        \textsl{Morphims:} & $\Hom_{K(\A)}(X, Y) := \Hom_{\com(\A)}(X, Y)/_\htpy$
    \end{tabular}
\end{center}
The composition law descends to the quotient by lemma \ref{nullhomotopics form a subgroup} (ii), \ie $[g]\circ[f] := [g \circ f]$, for composable $[f]$ and $[g]$, and for any $A^\bullet$ the identity morphism is defined to be $[1_A^\bullet]$. All the hom-sets $\Hom_{K(\A)}(X, Y)$ are $k$-modules by lemma \ref{nullhomotopics form a subgroup} (i) and compositions are $k$-bilinear maps. The biproduct of two complexes is 

\info{define the biproduct as well}

As in the case of categories of complexes in $\A$, we can also define the following full additive subcategories of $K(\A)$.
\begin{center}
    \begin{tabular}{c l}
        $K^+(\A)$ & \emph{Homotopy category of complexes bounded below}, spanned on \\ &complexes in $\A$ bounded below. \\
        $K^-(\A)$ & \emph{Homotopy category of complexes bounded above}, spanned on \\ &complexes in $\A$ bounded above. \\
        $K^b(\A)$ & \emph{Homotopy category of bounded complexes}, spanned on \\ &bounded complexes in $\A$. \\    
    \end{tabular}
\end{center}
\info{in practise everywhere unless stated otherwise our complexes in $K^+(\A)$ will be supported in $\N$}

\subsubsection*{Cohomology}

A very important invariant of a chain complex in the homotopy category, which measures the extent to which it fails to be exact, is its cohomology. Here we are no longer assuming $\A$ is just $k$-linear, but abelian, since we will need kernels and cokernels to exists. 
For a chain complex $A^\bullet$ in $\com(\A)$ and $i \in \Z$ we define its \emph{$i$-th cohomology} to be 
\[
    H^i(A^\bullet) := \coker(\im d^{i-1} \to \ker d^{i}).
\]
\begin{remark}
    \label{versions of cohomology}
    A computation with universal properties inside an abelian category shows, that the following are all equivalent ways of defining the cohomology of a complex as well 
    \begin{align*}        
        H^i(A^\bullet) := & \coker(\im d^{i-1} \to \ker d^{i}) \iso \ker(\coker d^{i-1}\to \im d^i) \\ &\iso \coker(A^{i-1} \to \ker d^{i}) \iso \ker(\coker d^{i-1} \to A^{i}).
    \end{align*}
    See \cite[Def.~8.3.8.~(i)]{kashiwara2006categories}.
\end{remark}

For a morphism of complexes $f^\bullet\colon A^\bullet \to B^\bullet$ one can also define a morphism 
\[
    H^i(f^\bullet) \colon H^i(A^\bullet) \to H^i(B^\bullet)
\]
in $\A$, because $f^\bullet$ induces maps $\im d^{i-1}_A \to \im d^{i-1}_B$ and $\ker d^i_A \to \ker d^i_B$, which fit into the commutative diagram bellow.
\begin{equation}
    \label{eq: diagram for map on cohomology}
    \begin{tikzcd}[column sep = small, row sep = 1.5em]
    % https://q.uiver.app/#q=WzAsMTIsWzAsMSwiQV57aS0xfSJdLFsyLDEsIkFeaSJdLFs0LDEsIkFee2krMX0iXSxbMSwzLCJpbSBkXntpLTF9X0EiXSxbMSwyLCJpbWRee2ktMX1fQiJdLFswLDAsIkJee2ktMX0iXSxbMiwwLCJCXmkiXSxbNCwwLCJCXntpKzF9Il0sWzMsMiwiXFxrZXIgZF5pX0IiXSxbMywzLCJcXGtlciBkXmlfQSJdLFsyLDQsIkheaShCXlxcYnVsbGV0KSJdLFsyLDUsIkheaShBXlxcYnVsbGV0KSJdLFswLDNdLFs1LDRdLFs1LDYsImRee2ktMX1fQiJdLFswLDEsImRee2ktMX1fQSJdLFs2LDcsImReaV9CIl0sWzEsMiwiZF5pX0EiXSxbOCw2XSxbOSwxXSxbNCw2XSxbMywxXSxbNCw4XSxbMyw5XSxbOSwyLCIwIiwyXSxbOCw3LCIwIiwwLHsibGFiZWxfcG9zaXRpb24iOjcwfV0sWzExLDEwLCIiLDAseyJzdHlsZSI6eyJib2R5Ijp7Im5hbWUiOiJkYXNoZWQifX19XSxbOSw4XSxbMyw0XSxbOSwxMV0sWzgsMTBdLFs0LDEwLCIwIiwyLHsibGFiZWxfcG9zaXRpb24iOjcwfV0sWzMsMTEsIjAiLDJdLFswLDVdLFsxLDZdLFsyLDddXQ==
	{B^{i-1}} && {B^i} && {B^{i+1}} \\
	{A^{i-1}} && {A^i} && {A^{i+1}} \\
	& {\im d^{i-1}_B} && {\ker d^i_B} \\
	& {\im d^{i-1}_A} && {\ker d^i_A} \\
	&& {H^i(B^\bullet)} \\
	&& {H^i(A^\bullet)}
	\arrow["{d^{i-1}_B}", from=1-1, to=1-3]
	\arrow[from=1-1, to=3-2]
	\arrow["{d^i_B}", from=1-3, to=1-5]
	\arrow[from=2-1, to=1-1]
	\arrow["{d^{i-1}_A}", from=2-1, to=2-3]
	\arrow[from=2-1, to=4-2]
	\arrow[from=2-3, to=1-3]
	\arrow["{d^i_A}", from=2-3, to=2-5]
	\arrow[from=2-5, to=1-5]
	\arrow[from=3-2, to=1-3]
	\arrow[from=3-2, to=3-4]
	\arrow["0"'{pos=0.7}, from=3-2, to=5-3]
	\arrow[from=3-4, to=1-3]
	\arrow["0"{pos=0.7}, from=3-4, to=1-5]
	\arrow[from=3-4, to=5-3]
	\arrow[from=4-2, to=2-3]
	\arrow[from=4-2, to=3-2]
	\arrow[from=4-2, to=4-4]
	\arrow["0"', from=4-2, to=6-3]
	\arrow[from=4-4, to=2-3]
	\arrow["0"', from=4-4, to=2-5]
	\arrow[from=4-4, to=3-4]
	\arrow[from=4-4, to=6-3]
	\arrow[dashed, from=6-3, to=5-3]
    \end{tikzcd}
\end{equation}
All the induced morphisms come from universal properties and are as such unique for which the diagram commutes. Thus the assignment
% Since all the induced maps come from universal properties, they are unique for which the diagram commutes and thus the assignment 
\[
    f^\bullet \mapsto H^i(f^\bullet) \colon \Hom_{\com(\A)}(A, B) \to \Hom_\A(H^i(A), H^i(B))
\]
is functorial \ie respects composition and maps identity morphisms to identity morphisms. For the same reasons it is also a $k$-linear homomorphism, showing that
\[
    H^i \colon \com(\A) \to \A
\]
a $k$-linear functor. Due to the following proposition \ref{cohomology of nulhomotopic map}, the $i$-th cohomology functor $H^i$ descends to a well defiened functor 
\[
    H^i \colon \kcat{\A} \to \A
\]
on the homotopy category $\kcat{\A}$.

\begin{proposition}
    \label{cohomology of nulhomotopic map}
    Let $f \colon A^\bullet \to B^\bullet$ be a nulhomotopic chain map in $\com(\A)$. Then $f$ induces the zero map on cohomology, that is $H^i(f) = 0$ for all $i \in \Z$.
\end{proposition}

\begin{proof}
    As $f$ is nulhomotopic, there exists a homotopy $(h^i \colon A^i \to B^{i-1})_{i\in Z}$ such that 
    \[
        f^i =  h^{i+1} d^i_A + d^{i-1}_B  h^i \quad \text{ for all $i \in \Z$.}
    \]
    Let us name the following morphisms from diagram \eqref{eq: diagram for map on cohomology}.
    \begin{center}
        \begin{tabular}{c c}
            $i_A \colon \ker d^i_A \hookrightarrow A^{i}$ & $i_B \colon \ker d^i_B \hookrightarrow B^{i}$ \\
            $\psi_B \colon B^{i-1} \twoheadrightarrow \im d^{i-1}_B$ & $\xi_B \colon \im d^{i-1}_B \to \ker d^i_B$ \\
            $\pi_B \colon \ker d^i_B \twoheadrightarrow H^i(B^\bullet)$ & $\phi \colon \ker d^i_A \to \ker d^i_B$ \\
            % $j_B \colon \im d^{i-1}_B \hookrightarrow B^i$.
        \end{tabular}
    \end{center}
    We compute $i_B \phi = f^i i_A = (h^{i+1}d^i_A + d^{i-1}_Bh^i)i_A = d^{i-1}_B h^i i_A = 
    % j_B \psi_B h^i i_A = 
    i_B \xi_B \psi_B h^i i_A$. Since $i_B$ is a monomorphism, we may cancel it on the left, to express $\phi$ as $\xi_B \psi_B h^i i_A$. Then it is clear, that $\pi_B \phi = 0$ (as $\pi_B\xi_B = 0$), which means that $0$ is the unique map $H^i(A^\bullet) \to H^i(B^\bullet)$ fitting into the commutative diagram \eqref{eq: diagram for map on cohomology}.
    \info{ugly but uses only universal properties, no elements}
\end{proof}

It is very fruitful to consider all the cohomology functors $(H^i)_{i \in \Z}$ at once, as is witnessed by the next proposition. 

\begin{proposition} \cite[Theorem 12.3.3.]{kashiwara2006categories}
    Let $0 \to A^\bullet \to B^\bullet \to C^\bullet \to 0$ be a short exact sequence in $\com(\A)$. Then there exists a long exact sequence in $\A$
    \[
        \cdots \to H^i(A^\bullet) \to H^i(B^\bullet) \to H^i(C^\bullet) \to H^{i+1}(A^\bullet) \to \cdots.
    \]
\end{proposition}

\info{this is not done...}

% Functors $H^i$ actually descend to the homotopy category $K(\A)$ of $\A$ 

We now shift our focus to the construction of a triangulated structure on $K^*(\A)$. The translation functor $T\colon K^*(\A) \to K^*(\A)$ is defined on objects and morphisms in the following way.
\begin{center}
    \begin{tabular}{r l}
        \textsl{Objects:} & $A^\bullet \longmapsto A^\bullet[1]$. \\
        \textsl{Morphims:} & $[f^\bullet] \longmapsto [f^\bullet[1]]$.
    \end{tabular}
\end{center}
This assignment is clearly well defined on morphisms, as $f \htpy f'$ implies $f[1] \htpy f'[1]$.

\subsubsection*{Mapping cones}
The other piece of data required to obtain a triangulated category is a collection of distinguished triangles. To describe what distingushed triangles are in our case, we must first introduce the mapping cone of a morphism of complexes and to this end we will for a moment step outside the scope of the homotopy categroy of complexes back into the category of complexes of $\A$.

\begin{definition}
    Let $f\colon A^\bullet \to B^\bullet$ be a morphism of complexes in $\com(\A)$. The complex $C(f)^\bullet$ is specified by the collection of objects
    \[
        C(f)^i := A^{i+1} \oplus B^i
    \]
    and differentials
    \begin{equation}
        \label{eq: cone differential}
        d^i_{C(f)} := \begin{pmatrix}
            -d^{i+1}_A & 0 \\ f^{i+1} & d^i_B
        \end{pmatrix} = \begin{pmatrix}
            d^i_{A[1]} & 0 \\ f[1]^i & d^i_B
        \end{pmatrix}
    \end{equation}
    % \vspace{0.2cm}
    % \[
    %     d^i_{C(f)} \colon A^{i+1} \oplus B^i \to A^{i+2} \oplus B^{i+1}
    % \]
    for all $i \in \Z$.
\end{definition}

\begin{remark}
    Using matrix notiaion here might be a bit misleading at first. Formally, matrix \eqref{eq: cone differential} represents the morphism 
    \[
        \left(\langle -d^{i+1}_{A}, 0 \rangle, \langle f^{i+1}, d^i_B \rangle\right) = \left\langle (-d^{i+1}_{A}, f^{i+1}), (0, d^i_B) \right\rangle
    \]
    according to our convention for defining morphisms from and into biproducts. 
    A simple computation, using the fact that $f$ is a chain map, shows that $C(f)^\bullet$ is a chain complex.
\end{remark}

Along with the cone of a chain map $f$ we also define two chain maps
\[
    \tau_f \colon B^\bullet \to C(f)^\bullet,
\]
given by the collection $\left( \tau_f^i\colon B^i \to A^{i+1} \oplus B^i \right)_{i \in \Z}$, where $\tau_f^i = (0, \id{B^i})$, and 
\[
    \pi_f \colon C(f)^\bullet \to A^\bullet[1],
\]
given by the collection $\left( \pi_f^i\colon A^{i+1} \oplus B^i \to A^{i+1} \right)_{i \in \Z}$, where $\pi_f^i = \langle \id{A^{i+1}}, 0 \rangle$.

The naming convention of course comes from topology, where one can show that the singular chain complex associated to the topological mapping cone $M(f)$ of a continuous map $f\colon X \to Y$ is chain homotopically equivalent to the cone of the chain map induced by $f$ between singular chain complexes of $X$ and $Y$.

We define any triangle in $K(\A)$ isomorphic to a triangle of the form
\[
    A \xrightarrow{ \ f \ }  B \xrightarrow{\ \tau_f \ } C(f) \xrightarrow{\ \pi_f \ } A[1]
\]
to be distinguished.

\begin{proposition}
    The homotopy category $K(\A)$ together with the translation functor $T\colon K(\A) \to K(\A)$ and distinguished triangles defined above is a triangulated category.
\end{proposition}

\begin{proof}
    
\end{proof}

\begin{remark}
    The proposition still holds true, if we replace $K(\A)$ with any of the categories $K^+(\A)$, $K^-(\A)$, $K^b(\A)$.
\end{remark}


\newpage
\begin{lemma}
    Let $f^\bullet \colon A^\bullet \to B^\bullet$ be a quasi-isomorphism and let $g^\bullet\colon C^\bullet \to B^\bullet$ be a morphism. Then there exist a quasi-isomorphism $u^\bullet\colon C_0^\bullet \to C^\bullet$ and a morphism $v^\bullet\colon C_0^\bullet \to A^\bullet$, such that $f \circ u = g \circ v$ \ie the diagram below commutes.
    \[
    \begin{tikzcd}
        {C_0^\bullet} && C^\bullet \\
        \\
        A^\bullet && B^\bullet
        \arrow["u^\bullet", dashed, from=1-1, to=1-3]
        \arrow["\sim"', draw=none, from=1-1, to=1-3]
        \arrow["v^\bullet"', dashed, from=1-1, to=3-1]
        \arrow["g^\bullet", from=1-3, to=3-3]
        \arrow["f^\bullet", from=3-1, to=3-3]
        \arrow["\sim"', draw=none, from=3-1, to=3-3]
    \end{tikzcd}
    \]
\end{lemma}

The above also holds in categories $K^+(\A)$, $K^-(\A)$, $K^b(\A)$.

\begin{proposition}
    \label{injective resolution}
    Suppose $\A$ contains enough injectives. Then every $A^\bullet$ in $K^+(\A)$ has an injective resolution.
\end{proposition}

\begin{proof}
    
\end{proof}





\subsection{Derived functors}

The main goal of this section is to assign to a sensible additive functor $F \colon \A \to \B$ between two abelian categories an appropriate triangulated functor on the level of derived categories, called a \emph{derived functor} of $F$. We will first deal with the easiest case, when $F$ is exact and then move on to the case, where $F$ will only be assumed to be left or right exact and instead require the domain category $\A$ to posses some nice properties (\eg contain enough injectives or contain an $F$-adapted class of objects). In practise the additional conditions $\A$ is required to satify are not very restricitve. 

We start with some establishing lemmas.

\begin{lemma}
    \label{qis iff cone acyclic}
    Let $f\colon A^\bullet \to B^\bullet$ be a chain map. Then $f$ is a quasi-isomorphism if and only if its associated cone $C(f)^\bullet$ is acyclic. 
\end{lemma}

\info{proof can reference a proposition from section on triangulated cats...}

\begin{proof}
    One only needs to consider the distinguished triangle $A^\bullet \xrightarrow{f} B^\bullet \to C(f)^\bullet \to A[1]^\bullet$ in $K(\A)$ and its associated long exact sequence in cohomology.
\end{proof}

\begin{lemma}
    \label{F preserves qis iff preserves acyclic}
    Let $F\colon K(\A) \to K(\B)$ be a triangulated functor, then the following are equivalent.
    \begin{enumerate}
        \item[\emph{(i)}] $F$ preserves quasi-isomorphisms.
        \item[\emph{(ii)}] $F$ preserves acyclic complexes.
    \end{enumerate}
\end{lemma}

\begin{proof}
    $(\Rightarrow)$ Let $A^\bullet$ be an acyclic complex. Then the zero morphism $A^\bullet \to 0$ is a quasi-isomorphism and is by assumption mapped to a quasi-isomorphism $F(A^\bullet) \to 0$ by $F$. This means $F(A^\bullet)$ is acyclic.
    
    $(\Leftarrow)$ Let $f\colon A^\bullet \to B^\bullet$ be a quasi-isomorphism. The image of a distinguished triangle $A^\bullet \to B^\bullet \to C(f)^\bullet \to A^\bullet[1]$ in $K(\A)$ by the triangulated functor $F$ is a distinguished triangle $FA^\bullet \to FB^\bullet \to F(C(f)^\bullet) \to F(A^\bullet)[1]$ in $K(\B)$. By lemma \ref{qis iff cone acyclic} $C(f)^\bullet$ is acyclic, implying $F(C(f)^\bullet)$ is acyclic by the assumption (ii). Hence again by lemma \ref{qis iff cone acyclic} $F(f)$ is a quasi-isomorphism.
    % The cone $F(C(f)^\bullet)$ is then acyclic, because $C(f)^\bullet$ was acyclic by lemma \ref{qis iff cone acyclic}, thus showing $Ff$ is a quasi-isomorphism.  
\end{proof}

\begin{proposition}
    \label{Exact functor induces a derived functor}
    Let $F\colon K(\A) \to K(\B)$ be a triangulated functor and assume it satisfies one of the equivalent conditions of lemma \ref{F preserves qis iff preserves acyclic}. Then there exists a triangulated functor $D(\A) \to D(\B)$ on the level of derived categories, for which the following diagram of triangulated functors commutes.
    \begin{equation}
        \label{eq: exact functor derives}
        \begin{tikzcd}[column sep = 1.5em]
        % https://q.uiver.app/#q=WzAsNCxbMCwwLCJLKFxcQSkiXSxbMiwwLCJLKFxcQikiXSxbMCwyLCJEKFxcQSkiXSxbMiwyLCJEKFxcQikiXSxbMCwxLCJGIl0sWzAsMl0sWzIsM10sWzEsM11d
        {K(\A)} && {K(\B)} \\
        \\
        {D(\A)} && {D(\B)}
        \arrow["F", from=1-1, to=1-3]
        \arrow["{Q_\A}"', from=1-1, to=3-1]
        \arrow["{Q_\B}", from=1-3, to=3-3]
        \arrow[from=3-1, to=3-3]
        \end{tikzcd}
    \end{equation}
\end{proposition}

% skip this proof, there is a shorter one

\begin{proof}
    The functor $D(\A) \to D(\B)$ is defined on objects by the assignment $A^\bullet \mapsto F(A^\bullet)$ and on morphisms by the action
    \[
        \Hom_{D(\A)}(A^\bullet, B^\bullet) \to \Hom_{D(\B)}(F(A^\bullet), F(B^\bullet))
    \]
    sending a morphism $\phi \colon A^\bullet \to B^\bullet$, represented by a roof $A^\bullet \xleftarrow{\sim} C^\bullet \to B^\bullet$, to the equivalence class of the roof $F(A^\bullet) \xleftarrow{\sim} F(C^\bullet) \to F(B^\bullet)$. At this point it was crutial that $F$ preserves quasi-isomorphisms.
    % defined as follows. A morphism $\phi \colon A^\bullet \to B^\bullet$ represented by a roof $A^\bullet \xleftarrow{\sim} C^\bullet \to B^\bullet$ is sent to the equivalence class of the roof $F(A^\bullet) \xleftarrow{\sim} F(C^\bullet) \to F(B^\bullet)$, where it was crutial that $F$ preserves quasi-isomorphisms. 
    By functoriality of $F$ it is evident, that this map is well defined. The defined assignment is also functorial in the sense that it respects composition and identities. Clearly $D(\A) \to D(\B)$ fits into the (up to natural isomorphism) commutative square \eqref{eq: exact functor derives}, is also $k$-linear by the $k$-linearity assumption on $F$, commutes with the respective translation functors (up to natural isomorphism) and maps distinguished triangles to distingushed triangles, because of commutativity of the diagram \eqref{eq: exact functor derives}, making it a triangulated functor.
\end{proof}

Every $k$-linear functor $F\colon \A \to \B$ induces a triangulated functor on the level of homotopy categories $K(F)\colon K(\A) \to K(\B)$ and if moreover $F$ is assumed to be exact, then $K(F)$ satisfies the assumptions of lemma \ref{Exact functor induces a derived functor} and thus induces a triangulated functor on the level of derived categories. In the next subsection we will deal with the case, when $F$ is only left or right exact. 

\subsubsection{$F$-adapted classes}
\label{Subsection on F-adapted classes}

Unfortunately our categories at hand will sometimes not contain enough injectives, as can already be seen with the category of coherent sheaves $\coherent{X}$ on a scheme $X$.
% , which is not a point. 
In this case our previously defined method of constructing right derived functors will not work. Luckly however, there exists a method of obtaining a right derived functor $\rderived{}{F} \colon D^+(\A) \to D^+(\B)$, given a left exact functor $F: \A \to \B$, even when $\A$ does not contain enough injectives, but posseses a milder class of objects. To this end we first introduce $F$-adapted classes.

\begin{definition}
    \label{F-adapted}
    A class of objects $\I \subseteq \A$ is \emph{adapted} to a left exact functor $F \colon \A \to \B$, if the following three conditions are satisfied.
    \begin{enumerate}[(i)]
        \item $\I$ is stable under finite sums. 
        \item Every object $A$ of $\A$ \emph{embeds} into some object of $\I$, \ie there exists an object $I \in \I$ and a monomorphism $A \hookrightarrow I$.
        \item For every acyclic complex $I^\bullet$ in $K^+(\A)$, with $I^i \in \I$ for all $i \in \Z$, its image $F(I^\bullet)$ under $F$ is also acyclic.
    \end{enumerate}
\end{definition}

\begin{remark}
    We observe, that whenever $\A$ contains enough injectives, the class of all injective objects $\I_{\A}$ forms an $F$-adapted class for \emph{every} left exact functor ${F\colon \A \to \B}$. Points (i) and (ii) are clearly true for $\I_\A$ and point (iii) follows from lemma \ref{acyclic to injective is nulhomotopic}. Indeed it tells that an acyclic complex of injectives $I^\bullet$ is nulhomotopic, which implies $F(I^\bullet)$ is nulhomotopic and in particular acyclic.
\end{remark}

In the presence of an $F$-adapted class $\I$ we will now construct the right derived functor of $F$. Firstly one can upgrade the class of objects $\I$ to a full additive subcategory $\J$ of $\A$ having its class of objects be preciesly $\I$. This is done by declaring $\hom_\J(X, Y) := \hom_\A(X, Y)$ for all objects $X$, $Y$ of the class $\I$. Then we can define a functor $K^+(F) \colon K^+(\J) \to K^+(\B)$ between triangulated categories, which acts on objects of $K^+(\J)$ as 
\[
    \left(\cdots \to A^i \xrightarrow{d^i} A^{i+1} \to \cdots\right) \longmapsto \left(\cdots \to F(A^i) \xrightarrow{F(d^i)} F(A^{i+1}) \to \cdots\right)
\]
and on morphisms as
\[
    f^\bullet = (f^i \colon A^i \to B^i)_{i \in \Z} \longmapsto (F(f^i))_{i \in \Z}.
\]
It is then clear that $K^+(F)$ is triangulated and maps acyclic complexes to acyclic one, thus it descends to a well defined functor on the level of derived categories 
\[
    D^+(\J) \to D^+(\B)
\]
by proposition \ref{Exact functor induces a derived functor}.
Since we want to define the derived functor $\rderived{}{F}$, whose domain is $D^+(\A)$, it remains to construct a functor $D^+(\A) \to D^+(\J)$, which we can then post-compose with $K^+(F)$ to obtain $\rderived{}{F}$. What we will show instead is the following.

\begin{proposition}
    \label{Derived cat of F-adapted category}
    The inclusion of categories $\J \hookrightarrow \A$ induces an equivalence of triangulated categories $D^+(\J) \iso D^+(\A)$.
\end{proposition}

% that the inclusion of categories $\J \hookrightarrow \A$ induces an exact equivalence of triangulated categories $D^+(\J)$ and $D^+(\A)$.

% It remains to be shown that the inclusion of categories $\J \hookrightarrow \A$ induces an the equivalence of categories $D^+(\J)$ and $D^+(\A)$. Firstly we remark that this inclusion clearly descends to a well-defined functor on the level of derived categories

This hinges on the following lemma.

\begin{lemma}
    \label{F-acyclic resolutions}
    For every complex $A^\bullet$ in $K^+(\A)$ there is a quasi-isomorphism $A^\bullet \to I^\bullet$, where $I^\bullet$ is a complex in $K^+(\J)$.
\end{lemma}

\begin{proof}
    By inspecting the proof of proposition \ref{injective resolution} we see that only conditions (i) and (ii) of definition \ref{F-adapted} were actually used, so the proof of this lemma can be copied from the proof of said proposition.
\end{proof}

\begin{proof}[Proof of proposition \ref{Derived cat of F-adapted category}]
    First, there exists a triangulated functor $D^+(\J) \to D^+(\A)$ by proposition \ref{Exact functor induces a derived functor}, since the inclusion $\J \hookrightarrow \A$ sends acyclic complexes to acyclic ones. We claim, that this functor is an equivalence, thus we will show that it is full, faithful and essentially surjective. By lemma \ref{F-acyclic resolutions} we see that it is essentially surjective. Next we show that it is faithful \ie that the morphism action
    \begin{equation}
        \label{eq: morphism action of D(J) - D(A)}
        \Hom_{D^+(\J)}(I^\bullet, J^\bullet) \to \Hom_{D^+(\A)}(I^\bullet, J^\bullet)
    \end{equation}
    is injective for all objects $I^\bullet$, $J^\bullet$ of $D^+(\J)$. Suppose two morphisms $\phi_0$, $\phi_1 \colon I^\bullet \to J^\bullet$, repersented by right roofs $I^\bullet \to I_0^\bullet \xleftarrow{\sim} J^\bullet$ and $I^\bullet \to I_1^\bullet \xleftarrow{\sim} J^\bullet$, are sent to the same morphism in $\Hom_{D^+(\A)}(I^\bullet, J^\bullet)$, meaning that there is the following commutative diagram in $K^+(\A)$.
    \[\begin{tikzcd}[row sep = 1.5em, column sep = 2em]
        % https://q.uiver.app/#q=WzAsNSxbMSwxLCJJXzAiXSxbMywxLCJBIl0sWzAsMiwiSSJdLFs0LDIsIkoiXSxbMiwwLCJCIl0sWzIsMF0sWzMsMCwiXFxzaW0iXSxbMiwxXSxbMywxLCJcXHNpbSIsMl0sWzAsNF0sWzEsNCwiXFxzaW0iLDJdXQ==
        && A^\bullet \\
        & {I_0^\bullet} && I_1^\bullet \\
        I^\bullet &&&& J^\bullet
        \arrow[from=2-2, to=1-3]
        \arrow["\sim"', from=2-4, to=1-3]
        \arrow[from=3-1, to=2-2]
        \arrow[from=3-1, to=2-4]
        \arrow["\sim", from=3-5, to=2-2]
        \arrow["\sim"', from=3-5, to=2-4]
    \end{tikzcd}\]
    Then simply extending the diagram by a quasi-isomorphism $A^\bullet \to I_2^\bullet$, ensured by lemma \ref{F-acyclic resolutions}, proves that $\phi_0$ and $\phi_1$ were in fact equal and that the action is faithful.

    Lastly we show that the action on morphisms is also surjective, \ie the homomorphism of \eqref{eq: morphism action of D(J) - D(A)} is surjective. Pick any $\psi\colon I^\bullet \to J^\bullet$ in $D^+(\A)$, which is represented by a right roof $I^\bullet \to A^\bullet \xleftarrow{\sim} J^\bullet$. As before, extending the roof by a quasi-isomorphism $A^\bullet \to I^\bullet_0$ by lemma \ref{F-acyclic resolutions}, leaves us with a representative $I^\bullet \to I_0^\bullet \xleftarrow{\sim} J^\bullet$ of some morphism $\phi\colon I^\bullet \to J^\bullet$ in $D^+(\J)$.
    \[\begin{tikzcd}[column sep = small, row sep = 1em]
        % https://q.uiver.app/#q=WzAsNSxbMSwyLCJJXzAiXSxbMywyLCJBIl0sWzAsNCwiSSJdLFs0LDQsIkoiXSxbMiwwLCJJXzAiXSxbMiwwXSxbMywwLCJcXHNpbSIsMCx7ImxhYmVsX3Bvc2l0aW9uIjo0MH1dLFsyLDFdLFszLDEsIlxcc2ltIiwyXSxbMCw0LCIiLDIseyJsZXZlbCI6Miwic3R5bGUiOnsiaGVhZCI6eyJuYW1lIjoibm9uZSJ9fX1dLFsxLDQsIlxcc2ltIiwyXV0=
        && {I_0^\bullet} \\
        \\
        & {I_0^\bullet} && A^\bullet \\
        \\
        I^\bullet &&&& J^\bullet
        \arrow[equals, from=3-2, to=1-3]
        \arrow["\sim"', from=3-4, to=1-3]
        \arrow[from=5-1, to=3-2]
        \arrow[from=5-1, to=3-4]
        \arrow["\sim"{pos=0.4}, from=5-5, to=3-2]
        \arrow["\sim"', from=5-5, to=3-4]
    \end{tikzcd}\]
    The above diagram then shows, that $\phi$ is sent to $\psi$ and that the action on morphisms is surjective.
\end{proof}

We can now define the right dervied functor $\rderived{}{F}$ as the compositon
\[
    \rderived{}{F} \colon \quad D^+(\A) \xrightarrow{\ \iso \ } D^+(\J) \longrightarrow D^+(B),
\] 
where $D^+(\A) \xrightarrow{\ \iso \ } D^+(\J)$ denotes a quasi-inverse to the equivalence $D^+(\J) \iso D^+(\A)$ established in proposition \ref{Derived cat of F-adapted category}.
Informally to compute the right derived functor $\rderived{}{F}(A^\bullet)$ of a complex $A^\bullet$ in $D^+(\A)$, one takes its $F$-adapted resolution $A^\bullet \to I^\bullet_A$ and then applies the functor $K^+(F)$ on $I^\bullet_A$, \ie
\[
    \rderived{}{F}(A^\bullet) = K^+(F)(I^\bullet_A).
\]
% it is clear that the restriction of the functor $F$ to $\J$ descends to the right derived functor

% localization of K^+(\I) by qis equivalent to D^+(\A)

\begin{proposition}
    \label{F-acyclic is F-adapted}
    Let $F : \A \to \B$ be a left exact functor between two abelian categories. Let $\I \subset \A$ be an $F$-adapted class of objects in $\A$. Then the class of all $F$-acyclic objects of $\A$, denoted by $\I_F$, is also $F$-adapted. 
\end{proposition}

\begin{lemma}
    \label{F-adapted is F-acyclic}
    Every object of an $F$-adapted class is also $F$-acyclic.
\end{lemma}

\begin{proof}
    We only have to recall definition \ref{}. An $F$-adapted object $I$ of $\J$ trivially forms its own resolution $I^\bullet = (\cdots \to 0 \to I \to 0 \to \cdots)$, which allows us to compute $\rderived{i}{F}(I) = H^i(\rderived{}{F}(I^\bullet))$ to be $0$, whenever $i \neq 0$ and $I$ for $i = 0$.
\end{proof}

% \begin{proof}[Proof of proposition \ref{F-acyclic is F-adapted}]
%     We verify conditions (i)--(iii) of definition \ref{F-adapted}. (i) As the higher derived functors $R^iF$ are clearly additive, $\I_F$ is stable under finite sums. (ii) Holds by lemma \ref{F-adapted is F-acyclic}. (iii) Suppose $A^\bullet$ is an acyclic complex in $K^+(\A)$ with $A^i \in \I_F$ for all $i$. Then using acyclicity of $A^\bullet$, we may break this complex up into a series of short exact sequences as follows.
%     \begin{align*}
%         0 \to A^0 \to & A^1 \to \ker d^2 \to 0 \\
%         0 \to \ker d^2 \to & A^2 \to \ker d^3 \to 0 \\
%         & \vdots
%     \end{align*} 
%     The first short exact sequence yields the following long exact sequence associated to $F$.
%     \begin{multline*}
%         0 \to F(A^0) \to F(A^1) \to F(\ker d^2) \to \rderived{1}{F}(A^0) \to \cdots \\ \cdots \to \rderived{i}{F}(A^0) \to \rderived{i}{F}(A^1) \to \rderived{i}{F}(\ker d^2) \to \rderived{i+1}{F}(A^0) \to \cdots.
%     \end{multline*}
%     Exacness of the sequence together with $F$-acyclicity of $A^0$ and $A^1$ shows that $\ker d^2$ is also $F$-acyclic and by only considering the begining few terms of the sequence we obtain the  short exact sequence 
%     % From the fact that $A^0$ and $A^1$ are $F$-acyclic we see that $\ker d^2$ is $F$-acyclic and we obtain a short exact sequence 
%     \[
%         0 \to F(A^0) \to F(A^1) \to F(\ker d^2) \to 0.
%     \]
%     After inductively applying the same reasoning for each of the original short exact sequences of \eqref{}, we are left with the following set of short exact sequences\footnote{To be more precise we have used that $F$ is left exact in order to swap the order of $F$ and $\ker$ to reach \[\ker Fd^i \simeq F(\ker d^i)\].}.
%     \begin{align*}
%         0 \to F(A^0) \to & F(A^1) \to \ker Fd^2 \to 0 \\
%         0 \to \ker Fd^2 \to & F(A^2) \to \ker Fd^3 \to 0 \\
%         & \vdots
%     \end{align*}   
%     Collecting them all back together we can conclude that the complex $F(A^\bullet)$ is acyclic.\qedhere
% \end{proof}

\begin{proof}[Proof of proposition \ref{F-acyclic is F-adapted}]
    We verify conditions (i)--(iii) of definition \ref{F-adapted}. (i) As the higher derived functors $R^iF$ are clearly additive, $\I_F$ is stable under finite sums. (ii) Holds by lemma \ref{F-adapted is F-acyclic}. (iii) Suppose $A^\bullet$ is an acyclic complex in $K^+(\A)$ with $A^i \in \I_F$ for all $i$. Then using acyclicity of $A^\bullet$, we may break this complex up into a series of short exact sequences 
    % as follows.
    % \begin{align*}
    %     0 \to A^0 \to & A^1 \to \ker d^2 \to 0 \\
    %     0 \to \ker d^2 \to & A^2 \to \ker d^3 \to 0 \\
    %     & \vdots
    % \end{align*} 
    \begin{equation}
        \label{eq: SES from acyclic cx}
        0 \to \ker d^i \to A^i \to \ker d^{i+1} \to 0 \quad \text{ for $i \in \Z$.}
    \end{equation}
    As our complex $A^\bullet$ is supported on $\N$, the first true short exact sequence happens at index $i = 1$, where $\ker d^1 = \im d^0 \iso A^0$
    It yields the following long exact sequence associated to $F$
    \begin{multline*}
        0 \to F(A^0) \to F(A^1) \to F(\ker d^2) \to \rderived{1}{F}(A^0) \to \cdots \\ \cdots \to \rderived{p}{F}(A^0) \to \rderived{p}{F}(A^1) \to \rderived{p}{F}(\ker d^2) \to \rderived{p+1}{F}(A^0) \to \cdots.
    \end{multline*}
    Exacness of this sequence together with $F$-acyclicity of $A^0$ and $A^1$ shows that $\ker d^2$ is also $F$-acyclic and by only considering the begining few terms of the sequence, we obtain the short exact sequence 
    % From the fact that $A^0$ and $A^1$ are $F$-acyclic we see that $\ker d^2$ is $F$-acyclic and we obtain a short exact sequence 
    \[
        0 \to F(A^0) \to F(A^1) \to F(\ker d^2) \to 0.
    \]
    After inductively applying the same reasoning for each of the original short exact sequences of \eqref{eq: SES from acyclic cx} for $i > 1$, we are left with short exact sequences\footnote{To be more precise, we have used that $F$ is left exact in order to swap the order of $F$ and $\ker$ to reach \[\ker Fd^i \iso F(\ker d^i)\].}
    \[
        0 \to \ker Fd^i \to F(A^i) \to \ker Fd^{i+1} \to 0 \quad \text{ for $i \geq 1$.}
    \]
    Collecting them all back together we conclude that the complex $F(A^\bullet)$ is acyclic.
    \info{There are inconsistencies with $F(A^\bullet) = K^+(F)(A^\bullet)$ and so on. $\to$ I am going to abuse notation and use $F(A^\bullet)$, bc this is also done in practise \eg $f_*\F^\bullet$, not $K^+(f_*)(\F^\bullet)$} \qedhere
\end{proof}