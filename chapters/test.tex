\section{Test}
Text and math $a, b \in \Z$
\[
\int_{M} d\omega = \int_{\partial M } \omega
\]

\[
    \sum D^b(\mathtt{Coh}(X))
\]

\[
\localtor{i}{\F}{\G}{\mathcal O_X} \\
\localext{i}{\F}{\G}{\mathcal O_X}
\]

\[
    \Phi_\mathcal P : D^b(X) \to D^b(Y) \quad \mathcal E^\bullet \mapsto \rderived{}{p_*}(\derivedtensor{\lderived{}{q^*}\mathcal E^\bullet}{\mathcal P^\bullet}{\mathcal O_{X \times Y}})
\]

% % --------------------------------------------------------------

% \section*{Question 1}
% \addtocounter{section}{1}

% Components of natural transformations $\eta$ and $\mu$ are given by
% \[
%     \eta_X : X \to V(X), \quad \quad x \mapsto \delta_x,
% \]
% where $\delta_x : X \to S$ is the Kronecker delta $\delta_x(x) = 1$ and $\delta_x(y) = 0$ for $y \neq x$. When $X = \varnothing$, let $\eta_\varnothing$ be the empty map $\varnothing \to V(Y)$.
% The natural transformation $\mu$ is given by
% \[
%     \mu_X : V(V(X)) \to V(X), \quad \quad w \mapsto \sum_{v \in V(X)} w(v)   v
% \]
% for all $X \in \abs{\Set}$.

% % --------------------------------------------------------------

% \section*{Question 2}
% \addtocounter{section}{1}

% The Kleisli category $\Set_V$ for the monad $(V, \eta, \mu)$ is given by the following data:
% \begin{itemize}
%     \item objects $\abs{\Set_V} = \abs{\Set}$,
%     \item morphisms $\Set_V(X, Y) = \Set(X, V(Y))$ for all $X, Y \in \abs{\Set_V}$.
% \end{itemize} 
% For all $X,Y,Z \in \Set$ the identity morphism in $\Set_V(X, X)$ is $\id_X = \eta_X : X \to V(X)$ and the composition mapping
% \[
%     (-) \circ_V (-) : \Set_V(Y, Z) \times \Set_V(X, Y) \to \Set_V(X,Z), \\
% \]
% is given by 
% \[
%     g \circ_V f = \mu_Z \circ V(g) \circ f = \left(x \mapsto \sum_{w \in V(Z)} \sum_{y \in g\inv(w)} f(x)(y)g(y) \right).
% \]

% \subsection{Weierstrassova funkcija $\wp$}

% Osrednja tema tega poglavja, ki bo povezala eliptične krivulje s kompleksnimi torusi in za katero je bilo potrebno razvijati teorijo v prejšnjem razdelku, bo najprej definicija nato pa pregled lastnosti \emph{Weierstrassove funkcije} $\wp$. 

% Vseskozi naj bo $\Lambda$ mreža v $\C$ in naj velja oznaka $\Lambda' = \Lambda \setminus \{0\}$.

% \begin{definicija}
%     \label{eisensteinova vrsta}
%     Za naravno število $k > 2$ je \emph{Eisensteinova vrsta reda $k$} podana kot
%     \[
%         G_k(\Lambda) = \sum_{\omega \in \Lambda'} \frac{1}{\omega^k}.  
%     \]
% \end{definicija}

% \begin{opomba}
%     Opazimo, da za lihe $k$ velja $G_k(\Lambda) = 0$, saj se člena pri $\omega$ in $-\omega$ v vsoti odštejeta. 
% \end{opomba}

% \begin{lema}
%     \label{lema o konvergenci eisensteinove vrste}
%     Za vsako naravno število $k > 2$ je Eisensteinova vrsta reda $k$ absolutno konvergentna.
% \end{lema}

% \begin{proof}
%     Za vsak $n \in \N$ definirajmo množice
%     \[
%         C_n = \{k_1 \omega_1 + k_2 \omega_2\in \Lambda \mid {k_1} + {k_2} = n\}.  
%     \]
%     Induktivni sklep pokaže, da je moč posamezne od teh množic $C_n$ enaka $4n$.
%     % Preprosto se je prepričati, da je moč posamezne od teh množic $\#C_n = 4n$. 
%     Vsak element $\omega \in C_n$ pa lahko po absolutni vrednosti ocenimo $\left\lvert \omega \right\rvert \geq \rho n$, kjer je $\rho > 0$ razdalja od izhodišča $0$ do roba paralelograma z oglišči v točkah $\pm \omega_1, \pm \omega_2$. Tedaj velja ocena
%     \[
%         \sum_{\omega \in \Lambda'} \frac{1}{\left\lvert  \omega^k \right\rvert} = 
%         \sum_{n = 1}^\infty \sum_{\omega \in C_n} \frac{1}{\left\lvert  \omega \right\rvert^k} \leq
%         \sum_{n = 1}^\infty \sum_{\omega \in C_n} \frac{1}{(\rho n)^k} =
%         \sum_{n = 1}^\infty \frac{4n}{\rho^k n^k} = 
%         \frac{4}{\rho^k} \sum_{n = 1}^\infty \frac{1}{n^{k-1}}.
%     \]
%     Klasičen rezultat iz realne analize pove, da zadnja vrsta konvergira natanko tedaj, ko je $k - 1 > 1$ in tako po primerjalnem kriteriju dobimo absolutno kovergenco Eisensteinove vrste $\sum_{\omega \in \Lambda'} \omega^{-k}$.
% \end{proof}

