\section{Triangulated categories}

\subsection{Additive, $k$-linear and abelian categories}

We mainly follow \cite{kashiwara2006categories}.

A categorical \emph{biproduct} of objects $X$ and $Y$ of a category $\mathcal C$ is an object $X \oplus Y$ together with morphisms
\begin{center}
    \begin{tabular}{c c}
        $p_X\colon X \oplus Y \to X$ & $p_Y\colon X \oplus Y \to Y$ \\
        $i_X\colon X \to X \oplus Y$ & $i_Y\colon Y \to X \oplus Y$ \\
    \end{tabular}
\end{center}
for which the pair $p_X, p_Y$ is the categorical product of $X$ and $Y$ and the pair $i_X, i_Y$ is the categorical coproduct.

% $X \xleftarrow{p_X} X \oplus Y \xrightarrow{p_Y} Y$ is the categorical product of $X$ and $Y$ and $X \xrightarrow{i_X} X \oplus Y \xleftarrow{i_Y} Y$.

\begin{definition}
    A category $\A$ is additive (resp.\ $k$-linear) if all the hom-sets carry the structure of abelian groups (resp.\ $k$-modules) and the following axioms are satisfied
    \begin{enumerate}
        \item[A1] For all all objects $X$, $Y$ and $Z$ of $\A$ the composition
        \[
            \circ \colon \hom_{\A}(Y, Z) \times \hom_{\A}(X, Y) \to \hom_{\A}(X, Z)
        \] 
        is bilinear.
        \item[A2] There exists a \emph{zero object} $0$, for which $\Hom_\A(0,0) = 0$.
        \item[A3] For any two objects $X$ and $Y$ there exists a categorical biproduct of $X$ and $Y$. 
    \end{enumerate} 
\end{definition}

% Whenever it will be clear from context, we will sometimes omit the subindex $\A$ from $\Hom_\A(X, Y)$ for clarity and write just $\Hom(X, Y)$ instead.

\begin{definition}
    A functor $F \colon \A \to \A'$ between two additive (resp.\ $k$-linear) categories $\A$ and $\A'$ is \emph{additive} (\resp \emph{$k$-linear}), if its action on morphisms
    \[
        \hom_\A(X, Y) \to \hom_{\A'}(F(X), F(Y))
    \] 
    is a group homomorphism (resp.\ $k$-linear map).
\end{definition}

For a morphism $f\colon X \to Y$ in an additive category $\A$ recall, that the \emph{kernel} of $f$ is the equalizer of $f$ and $0$ in $\A$, if it exists, and, dually, the \emph{cokernel} of $f$ is the coequalizer of $f$ and $0$. It is well-known and easy to verify, that the structure maps $\ker f \hookrightarrow X$ and $Y \twoheadrightarrow \coker f$ are mono and epi respectively. We also define the \emph{image} and \emph{coimage} of $f$ to be
\begin{align*}
    (\im f \to Y) &:= \ker(Y \to \coker f) \\
    (X \to \coim f) &:= \coker(\ker f \to X).
\end{align*}

\begin{definition}
    An additive category $\A$ is \emph{abelian}, if it contains all kernels and cokernels and satisfies axiom A4.
    \begin{enumerate}
        \item[A4] For any morphism $f\colon X \to Y$ in $\A$ the canonical morphism $\coim f \xrightarrow{\ \sim \ } \im f$ is an isomorphism.
        \[\begin{tikzcd}[column sep = small]
            % https://q.uiver.app/#q=WzAsNixbMSwwLCJYIl0sWzIsMCwiWSJdLFswLDAsIlxca2VyIGYiXSxbMywwLCJjb2tlciBmIl0sWzEsMSwiY29pbWYiXSxbMiwxLCJpbWYiXSxbMiwwXSxbMCwxLCJmIl0sWzEsM10sWzAsNF0sWzEsNV0sWzQsNSwiaCJdXQ==
            {\ker f} & X & Y & {\coker f} \\
            & \coim f & \im f
            \arrow[hook', from=1-1, to=1-2]
            \arrow["f", from=1-2, to=1-3]
            \arrow[two heads, from=1-2, to=2-2]
            \arrow[two heads, from=1-3, to=1-4]
            \arrow[hook', from=2-3, to=1-3]
            \arrow["\sim", from=2-2, to=2-3]
        \end{tikzcd}\] 
    \end{enumerate}
\end{definition}

\begin{remark}
    We obtain the morphism mentioned in axiom A4 in the following way. Since $(\im f \to Y) = \ker(Y \to \coker f)$ and the composition $X \to Y \to \coker f$ is equal to $0$, there is a unique morphism $X \to \im f$ by the universal property of kernels. The composit $\ker f \to X \to \im f$ then equals $0$, by the fact that $\im f \hookrightarrow Y$ is mono and $\ker f \to X \to Y$ equals $0$. From the universal property of cokernels we obtain a unique morphism $\coim f \to \im f$, since $(X \to \coim f) = \coker(\ker f \to X)$.
\end{remark}

\begin{definition}
    Let
    \[
        X \xrightarrow{\ f \ } Y \xrightarrow{\ g\ } Z
    \]
    be a sequence of composable morphisms in an abelian category $\A$ satisfying $g \circ f = 0$ . We say it is \emph{exact}, if the canonical morphism $\im g \xrightarrow{\ \sim \ } \ker f$ is an isomorphism.
\end{definition}

\subsection{Triangulated categories}

In order to formulate the definition of a triangulated category more concisely, we introduce some preliminary notions. A \emph{category with translation} is a category $\D$ together with an auto-equivalence $T\colon \D \to \D$ called the \emph{translation functor}. If $\D$ is additive or $k$-linear, $T$ is moreover assumed to be additive or $k$-linear. We usually denote its action on objects $X$ with $X[1]$ and its action on morphisms $f$ with $f[1]$.

A \emph{triangle} in a category $\D$ with translation $T$ is a triplet of composable morphisms $(u, v, w)$ of category $\D$ having the form 
\[
    X\xrightarrow{\ u \ } Y \xrightarrow{\ v \ } Z \xrightarrow{\ w \ } X[1].
\]
A \emph{morphism} of triangles $X\xrightarrow{u} Y \xrightarrow{v} Z \xrightarrow{w} X[1]$ and $X'\xrightarrow{u'} Y' \xrightarrow{v'} Z' \xrightarrow{w'} X'[1]$ is given by a triple of morphisms $(\alpha, \beta, \gamma)$ for which the diagram below commutes.
\[\begin{tikzcd}
    % https://q.uiver.app/#q=WzAsOCxbMCwwLCJYIl0sWzEsMCwiWSJdLFsyLDAsIloiXSxbMywwLCJYWzFdIl0sWzAsMSwiWCciXSxbMSwxLCJZJyJdLFsyLDEsIlonIl0sWzMsMSwiWCdbMV0iXSxbMCwxXSxbMSwyXSxbMiwzXSxbNCw1XSxbNSw2XSxbNiw3XSxbMCw0LCJcXGFscGhhIiwyXSxbMSw1LCJcXGJldGEiLDJdLFsyLDYsIlxcZ2FtbWEiLDJdLFszLDcsIlxcYWxwaGFbMV0iLDJdXQ==
	X & Y & Z & {X[1]} \\
	{X'} & {Y'} & {Z'} & {X'[1]}
	\arrow[from=1-1, to=1-2]
	\arrow["\alpha"', from=1-1, to=2-1]
	\arrow[from=1-2, to=1-3]
	\arrow["\beta"', from=1-2, to=2-2]
	\arrow[from=1-3, to=1-4]
	\arrow["\gamma"', from=1-3, to=2-3]
	\arrow["{\alpha[1]}"', from=1-4, to=2-4]
	\arrow[from=2-1, to=2-2]
	\arrow[from=2-2, to=2-3]
	\arrow[from=2-3, to=2-4]
\end{tikzcd}
\]
One can compose morphisms of triangles in the obvious way and the notion of an isomorphism of triangles is defined as usual.

\begin{definition}
    A \emph{triangulated category} is an additive or $k$-linear category $\D$ with translation $T$ equipped with a class of \emph{distinguished triangles}, which is subject to the following axioms.
    \begin{enumerate}
        \item[TR1] \begin{enumerate}[(i)]
            \item Any triangle isomorphic to a distinguished triangle is also itself destinguished.
            \item For any $X$ the triangle
            \[
                X \xrightarrow{\id_X} X \longrightarrow 0 \longrightarrow X[1]
            \] 
            is distinguished.
            \item For any morphism $f \colon X \to Y$ there is a distinguished triangle of the form
            \[
                X \xrightarrow{\ f \ } Y \longrightarrow Z \longrightarrow X[1]
            \]
            \end{enumerate}
        \item[TR2]
        \item[TR3]
        \item[TR4]    
    \end{enumerate} 
\end{definition}

% \begin{definition}
%     A \emph{triangulated category} is an additive or $k$-linear category $\D$ together with two additional pieces of structure. 
%     \begin{enumerate}    
%         \item An additive or $k$-linear auto-equivalence
%         \[
%         T\colon \D \to \D,
%         \]
%         called a \emph{translation functor}. Usually we denote its action on an object $X$ of $\D$ with $X[1]$ and its action on a morphism $f$ with $f[1]$ instead of $T(X)$ and $T(f)$.
        
%         \item A collection of \emph{distinguished triangles}, which are triplets of composable morphisms $(u, v, w)$ of category $\D$ having the form 
%         \[
%             X\xrightarrow{u} Y \xrightarrow{v} Z \xrightarrow{w} X[1],
%             \]
%     \end{enumerate}
%     The above is constrained by the following axioms.

%     \begin{enumerate}
%         \item[\text{TR1}] 
%         \item[TR2] 
%     \end{enumerate}
% \end{definition}

\begin{remark}
    For a distinguished triangle
    \[
        X\xrightarrow{\ u \ } Y \xrightarrow{\ v \ } Z \xrightarrow{\ w \ } X[1]
    \]
    morphisms $u$ and $v$ are said to be of degree $0$ and morphism $w$ is said to be of degree $+1$. They are also sometimes diagramatically depicted as triangles, with the markings on morphisms describing their respective degrees.
    \[\begin{tikzcd}[column sep = small, row sep = 1.5em]
        % https://q.uiver.app/#q=WzAsMyxbMSwwLCJaIl0sWzAsMiwiWCJdLFsyLDIsIlkiXSxbMCwxLCIrMSIsMl0sWzEsMiwiMCJdLFsyLDAsIjAiLDJdXQ==
        & Z \\
        \\
        X && Y
        \arrow["{+1}"', from=1-2, to=3-1]
        \arrow["0", from=3-1, to=3-3]
        \arrow["0"', from=3-3, to=1-2]
    \end{tikzcd}\]
\end{remark}

\begin{definition}
    Let $\D$ and $\D'$ be triangulated categories with translation functors $T$ and $T'$ respectively. An additive functor $F \colon \D \to \D'$ is defined to be \emph{triangulated} or \emph{exact}, if the following two conditions are satisfied.
    \begin{enumerate}[(i)]
        \item There exists a natural isomorphism of functors 
        \[
            \eta \colon F \circ T \iso T' \circ F.
        \]
        \item For every distinguished triangle
        \[
            X\xrightarrow{u} Y \xrightarrow{v} Z \xrightarrow{w} X[1]
        \]
        in $\D$, the triangle
        \[
            F(X) \to F(Y) \to F(Z) \to F(X)[1] 
        \]
        is distinguished in $\D'$, where the last morphism (of degree $1$) is obtained as the composition $F(Z) \xrightarrow{Fw} F(X[1]) \xrightarrow{\eta_X} F(X)[1]$.
    \end{enumerate}
\end{definition}

\begin{definition}
    Let $H\colon \D \to \A$ be an additive functor between a triangulated category $\D$ and an abelian category $\A$. We say $H$ is a \emph{cohomological functor} if for every distinguished triangle $X \to Y \to Z \to X[1]$ in $\D$, the induced long sequence in $\A$
    \[
        \cdots \to H(X) \to H(Y) \to H(Z) \to H(X[1]) \to H(Y[1]) \to H(Z[1]) \to \cdots
    \]
    is exact.
\end{definition}

\begin{example}
    For any object $W$ in a triangulated category $\D$ the functors 
    \[
    \Hom_\D(W, -) \colon \D \to \Ab \text{ and } \Hom_\D( - , W) \colon \D^\op \to \Ab
    \]
    are cohomological.
\end{example}
