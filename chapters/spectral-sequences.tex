\section{Spectral sequences and how to use them}

In this chapter we will first define cohomological spectral sequences and discuss their meaning through applications related with derived categories.

At first, when encountering spectral sequences, one might think of them as just  bookkeeping devices encoding a tremendous amount of data, but we will soon see how elegantly one can infer certain properties related to derived categorical claims exploiting the fact that they can naturally be encoded with spectral sequences.
% from a spectral sequence associated to a given problem  that we can encode a problem in terms of such a sequence. 
% just by considering their form for example.
One of the most important spectral sequences, which is also very general within our scope of inspection, will be the Grothendieck spectral sequence relating the higher derived functors of two composable functors with the higher derived functors of their composition. Later on we will see that many useful and well-known spectral sequences occur as special cases of the Grothendieck spectral sequence. What follows was gathered mostly from \cite[Chapter 2, 2.1]{huybrechts2006fouriermukai} and \cite[Chapter III.7]{gelfand2002methods}.

For the time being we fix an abelian category $\mathcal A$ and start off with a definition.

\begin{definition}
    A \emph{(cohomological) spectral sequence} in an abelian category $\mathcal A$ consists of the following data on which we further impose two convergence conditions.
\begin{itemize}
    \item \emph{Sequence of pages.} 
    % A finite page at step $r \in \N$ is a bi-graded object $E^{\bullet, \bullet}_r$ together with differentials of bi-degree $(r, 1-r)$ \ie morphisms
    % \[
    %     d^{p,q}_r : E^{p,q}_r \to E^{p + r, q - r + 1}_r
    % \] 
    % satisfying
    % \[
    %     d^{p + r, q - r + 1}_r \circ d^{p,q}_r = 0,
    % \]
    % for each $p, q \in \Z$.
    A sequence of bi-graded objects $(E^{\bullet, \bullet}_r)_{r \in \N}$ equipped with differentials of bi-degree $(r, 1-r)$. The $r$-th term of this sequence is called the $r$-th \emph{page} and it consists of a lattice of objects $E^{p,q}_r$ of $\mathcal A$, for $p, q \in \Z$, and differentials
    \[
        d^{p,q}_r : E^{p,q}_r \to E^{p + r, q - r + 1}_r,
    \] 
    satisfying
    \[
        d^{p + r, q - r + 1}_r \circ d^{p,q}_r = 0,
    \]
    for each $p, q \in \Z$.
    \item \emph{Isomorphisms.} A collection of isomorphisms
    \[
        \alpha^{p,q}_r : H^{p,q}(E_r) \xrightarrow{\sim} E^{p,q}_{r + 1},
    \]
    for all $p, q \in \Z$ and $r \in \N$, where 
    \[
        H^{p,q}(E_r) := \ker(d^{p,q}_r)/\im(d^{p-r, q + r - 1}_r),
    \]
    which allow us to turn the pages.
    \item \emph{Transfinite page.} A bi-graded object $E^{\bullet, \bullet}_\infty$. 
    % Without any differentials.
    \item \emph{Goal of computation.} A sequence of objects $(E^n)_{n \in \Z}$ of the category $\mathcal A$.
\end{itemize}
The above collection of data also has to satisfy the following two convergence conditions.
\begin{enumerate}
    \item For each pair $(p,q)$, there exists $r_0 \geq 0$, such that for all $r \geq r_0$ we have
    \[
        d^{p,q}_r = 0 \quad \text{and} \quad d^{p + r, q - r + 1}_r = 0
    \]
    and the isomorphism $\alpha^{p,q}_r$ can be taken to be the identity. We then say that the $(p,q)$-term \emph{stabilizes} after page $r_0$ and we denote $E^{p,q}_{r_0}$ (along with all the subsequent $E^{p,q}_{r}$ for $r \geq r_0$) by $E^{p,q}_\infty$.
    \item For each $n \in \Z$ there is a decreasing \emph{regular}\footnote{In our case the filtration $(F^p E^n)_{p \in \Z}$ is \emph{regular}, whenever $\bigcap_p F^p E^n = \lim_p F^p E^n = 0$ and $\bigcup_p F^p E^n = \colim_p F^p E^n = E^n$.} filtration of $E^n$
    \[
        E^n \supseteq \dots \supseteq F^p E^n \supseteq F^{p + 1}E^n \supseteq \dots \supseteq 0
    \]
    and isomorphisms
    \[
        \beta^{p,q} : E^{p,q}_\infty \xrightarrow{\sim} F^p E^{p + q}/F^{p + 1}E^{p + q}
    \]
    for all $p, q \in \Z$.
\end{enumerate}

In this case we also denote the existence of such a spectral sequence by 
\[
    E^{p,q}_r \implies E^n.
\]
\end{definition}

\begin{remark}
    A few words are in order to justify us naming the sequence $(E^n)_{n \in \Z}$ our \emph{goal of computation}. Usually one is given a starting page or a small number of them and the first goal is to identify the transfinite page -- we are refering to convergence condition (a). Often one is able to infer the differentials degenerate after a number of turns of the pages from context or by observing the shape of the spectral sequence. For example \emph{first quadrant spectral sequences}, \ie the ones with non-trivial $E^{p,q}_r$ only for $(p,q)$ lying in the first quadrant, always satisfy condition (a).
    
    The second part of the computation is concerned with relating objects from the transfinte page $E^{\bullet, \bullet}_\infty$ with objects $E^n$. This is captured in the convergence condition (b), from which we can clearly observe that the intermediate quotients of the filtration $(F^p E^n)_{p \in \Z}$ for a fixed term $E^n$ lie on the anti-diagonal of the transfinite page passing through \eg $E^{n, 0}_\infty$.
    
    In condition (b) the existence of isomorphisms $\beta^{p,q}$ can also be restated by saying that $E^{p,q}_\infty$ fits into a short exact sequence
    \[
        0 \to F^{p + 1} E^n \to F^p E^n \to E^{p,q}_\infty \to 0.
    \]
    This observation becomes very useful when considering properties of objects of the category $\mathcal A$ which are closed under extensions, especially when the filtration of $E^n$ is finite.

    % In that case one can inductively reason about such a property of $E^n$ provided the filtration of $E^n$ is finite.
 \end{remark}

% Next we mention two ways of naturally obtaining a spectral sequence, the first one arises from a filtered chain complex. A filtration of a chain complex $K^\bullet \in \Ob(\com(\mathcal A))$ is a decreasing sequence of chain complexes 
% \[
%  K^\bullet \supseteq \dots \supseteq F^p K^\bullet \supseteq F^{p+1}K^\bullet \supseteq \dots \supseteq 0
% \]
% respecting the differential of $K^\bullet$ \ie $d(F^p K^n) \subseteq F^p K^{n+1}$. Using only this piece of information one can 

\info{mention Tohoku paper}

\begin{theorem}[Grothendieck]
    Let $F\colon \A \to \B$ and $G\colon \B \to \mathcal C$ be left exact functors between abelian categories. Let $\I_F$ be an $F$-adapted class in $\A$, $\I_G$ a $G$-adapted class of objects in $\B$ and suppose every object of $\I_F$ is sent to a $G$-acyclic object by the functor $F$. Then for every $A^\bullet$ in $K^+(\A)$ there exists a spectral sequence
    \[
        E^{p,q}_2 = \rderived{p}{G}(\rderived{q}{F}(A^\bullet)) \implies \rderived{p+q}{(G \circ F)}(A^\bullet).
    \] 
\end{theorem}