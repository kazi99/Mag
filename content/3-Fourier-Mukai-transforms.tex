\section{Fourier-Mukai transforms}

\begin{example}
    \label{Identifying fm transforms}
    % In this example we notice and identity some common functors from algebraic geometry appear as Fourier-Mukai transforms. 
\begin{enumerate}[label = (\roman*)]
    \item{
    $\id{\derb{X}} \iso \fm{\struct{\Delta_X}}$. Recall that $\struct{\Delta_X}$ is defined to be the push-forward $\Delta_*\struct{X}$, of the structure sheaf of $X$ along the diagonal embedding $\Delta \colon X \to X \times X$. Then for any object $E$ of $\derb{X}$ we compute 
    % \begin{align*}
    %     \fm{\struct{\Delta_X}} &= \rderived{}{p_*}(\derivedtensor{\struct{\Delta_X}}{\lderived{}{p^*}(E)}{\struct{X \times X}}) \\
    %     &\iso \rderived{}{p_*}(\derivedtensor{\Delta_*\struct{X}}{\lderived{}{p^*}(E)}{\struct{X \times X}}) \\
    %     &\iso \rderived{}{p_*}(\derivedtensor{\Delta_*\struct{X}}{\lderived{}{\Delta^*}(\lderived{}{p^*}(E))}{\struct{X \times X}}) \\
    % \end{align*}
    \begin{align*}
        \fm{\struct{\Delta_X}} &= p_*(\struct{\Delta_X} \otimes_{\struct{X \times X}} p^*(E)) \\
        &= p_*(\Delta_*\struct{X} \otimes_{\struct{X \times X}} p^*(E)) \\
        &\iso p_*(\Delta_*(\struct{X} \otimes_{\struct{X}} \Delta^*p^*(E))) \\
        % &\iso \struct{X} \otimes E \\
        &\iso \id{\derb{X}}(E).
    \end{align*}
    All isomorphisms are natural in $E$, thus proving $\id{\derb{X}} \iso \fm{\struct{\Delta_X}}$.
    }
    \item{$(-)[1] \iso \fm{\struct{\Delta_X}[1]}$. A similar computation as above identifies the translation functor as a Fourier-Mukai transform with kernel $\struct{\Delta_X}[1]$.}
    \item{$\rderived{}{f_*} \iso \fm{\struct{\Gamma_f}}$. For a morphism $f \colon X \to Y$, let $\struct{\Gamma_f}$ denote the push-forward of the sheaf $\struct{X}$ along the morphism $i\colon X \to X \times Y$ induced by the identity $\id{X}\colon X \to X$ and $f\colon X \to Y$.  Then for any object $E$ of $\derb{X}$ we have
    \begin{align*}
        \fm{\struct{\Gamma_f}} &= p_*(\struct{\Gamma_f} \otimes_{\struct{X \times Y}} q^*(E)) \\
        &= p_*(i_*\struct{X} \otimes_{\struct{X \times X}} q^*(E)) \\
        &\iso p_*(i_*(\struct{X} \otimes_{\struct{X}} i^*q^*(E))) \\
        % &\iso \struct{X} \otimes E \\
        &\iso f_*(E).
    \end{align*}
    % \begin{align*}
    %     \fm{\struct{\Gamma_f}} &= \rderived{}{p_*}(\struct{\Gamma_f} \otimes_{\struct{X \times Y}} \lderived{}{q^*}(E)) \\
    %     &= \rderived{}{p_*}(i_*\struct{X} \otimes_{\struct{X \times X}} \lderived{}{q^*}(E)) \\
    %     &\iso \rderived{}{p_*}(\rderived{}{i_*}(\struct{X} \otimes_{\struct{X}} \lderived{}{i^*}\lderived{}{q^*}(E))) \\
    %     % &\iso \struct{X} \otimes E \\
    %     &\iso \rderived{}{f_*}(E).
    % \end{align*}
    Note that we have used $p \circ i = f$ and $q \circ i = \id{X}$ passing to the last line. As all the isomorphisms are natural, we have $\rderived{}{f_*} \iso \fm{\struct{\Gamma_f}}$.
    }
\end{enumerate}
\end{example}

\begin{theorem}[\text{\cite[Theorem 2.2]{Orlov-K3}}]
    \label{Orlov's theorem}
    Let $X$ and $Y$ be smooth projective varieties over a field $k$. Suppose $F \colon \derb{X} \to \derb{Y}$ is a fully faithful functor admitting a right (and, consequently, left) adjoint functor. Then there exists up to an isomorphism a unique object $E$ of $\derb{X \times Y}$ such that $F$ and $\fm{E}$ are naturally isomorphic.
\end{theorem}

\begin{proposition}
    \label{Composition of fm is fm}
    Let $\fm{\E^\bullet} \colon \derb{X} \to \derb{Y}$ and $\fm{\F^\bullet} \colon \derb{Y} \to \derb{Z}$ be Fourier-Mukai transforms with kernels $\E^\bullet$ and $\F^\bullet$ belonging to $\derb{X \times Y}$ and $\derb{Y \times Z}$ respectively. Then the composition $\fm{\F^\bullet} \circ \fm{\E^\bullet} \colon \derb{X} \to \derb{Z}$ is also a Fourier-Mukai transform with kernel $\mathcal R^\bullet = ...$, which is sometimes also called the \emph{convolution} of $\E^\bullet$ and $\F^\bullet$.
\end{proposition}

The following is a beautiful application of Orlov's Theorem \ref{Orlov's theorem}. It is also an important statement by itself, providing evidence for why the bounded derived category of coherent sheaves is a nicely behaved invariant for smooth and projective varieties. In particular it says that $\derb{-}$ detects the dimension and triviality of the canonical bundle.  
% showing that the bounded derived categories of coherent sheaves on smooth and projective varieties over $k$ detect dimension and triviality of the canonical bundle. 
\begin{theorem}
    \label{Db detects dimension and triviality of canonical bundle}
    Suppose $X$ and $Y$ are smooth projective varieties over $k$. Assume
    \[
        \derb{X} \natiso \derb{Y}.
    \]
    Then $\dim X = \dim Y$ and if the canonical bundle $\omega_X$ of $X$ is trivial, the canonical bundle $\omega_Y$ of $Y$ is trivial as well.
\end{theorem}

\begin{proof}
    By Orlov's Theorem \ref{Orlov's theorem} the equivalence $\derb{X} \natiso \derb{Y}$ is witnessed by a Fourier-Mukai transform $\fm{E}$ with the kernel $E$ belonging to $\derb{X \times Y}$. Since $\fm{E}$ is an equivalence its left and right adjoints agree. As they are both given by Fourier-Mukai transforms their respective kernels $E_L$ and $E_R$, introduced in \ref{}, must be isomorphic in $\derb{X \times Y}$. Writing this out we see that
    \[
        E^\vee \iso E^\vee \otimes (q^*\omega_X \otimes p^*\omega_Y)[\dim Y - \dim X].
    \]
    As tensoring by a line bundle, namely $q^*\omega_X \otimes p^*\omega_Y$ is exact, both tensor products are underived, but more importantly, the cohomology of complexes $E^\vee$ and ${E^\vee \otimes (q^*\omega_X \otimes p^*\omega_Y)}$ agree in all degrees. Since $E$ is a bounded complex, its derived dual $E^\vee = \rderived{}{\localhom[X \times Y]^\bullet(E, \struct{X \times Y})}$ is bounded as well. Then comparing cohomology intermediately\footnote{As $\fm{E}$ is an equivalence, the kernel $E$ cannot be isomorphic to the zero object in $\derb{X \times Y}$, implying that $E^\vee$ has non-trivial cohomology at least in some degrees.} gives 
    \[
        \dim X = \dim Y.
    \]
    
    For the second part, the assumption $\omega_X \iso \struct{X}$, simplifies the Serre functor $S_X$ of $\derb{X}$ to 
    \[
        S_X \colon \derb{X} \to \derb{X} \qquad S_X(-) = (-)[n], 
    \]
    where $n$ denotes the dimension of $X$ and $Y$. Since equivalences of triangulated categories are known to commute with Serre functors by \ref{}, we deduce from $\fm{E} \circ S_X \natiso S_Y \circ \fm{E}$ that the Serre functor $S_Y$ also simplifies to $(-)[n]$. By Definition \ref{Serre functor on Db(X)} this may be expressed as
    \[
        (-)[n] \natiso S_Y(-) = (-) \otimes_{\struct{Y}} \omega_Y[n].
    \]
    Plugging in the structure sheaf $\struct{Y}$, we see that $\omega_Y \iso \struct{Y}$ holds in $\coherent{Y}$. The last assertion 
    % about $\coherent{Y}$ 
    follows from fully faithfulness of the inclusion $\coherent{Y} \to \derb{Y}$ according to Proposition \ref{A -> D(A) fully faithful}. 
\end{proof}

\subsection{on $K$-groups}


\subsection{on rational cohomology}
\label{Subsection: FM transform on cohomology}

\begin{proposition}
    \label{Composition of cohomological fm is fm}
    Let $\fmcoh{\E^\bullet} \colon H^\bullet(X, \Q) \to H^\bullet(Y, \Q)$ and $\fmcoh{\F^\bullet} \colon H^\bullet(Y, \Q) \to H^\bullet(Z, \Q)$ be cohomological Fourier-Mukai transforms for $\E^\bullet$ and $\F^\bullet$ belonging to $\derb{X \times Y}$ and $\derb{Y \times Z}$ respectively. Then the composition $\fmcoh{\F^\bullet} \circ \fmcoh{\E^\bullet}$ is also a cohomological Fourier-Mukai transform with kernel $\mathcal R^\bullet$ of Proposition \ref{}
\end{proposition}

\begin{theorem}[Grothendieck-Riemann-Roch]
    \label{Grothendieck-Riemann-Roch}
    Let
\end{theorem}

\begin{remark}
    What's the deal with Chow groups $A_\Q(X)$    
\end{remark}

\begin{corollary}[Hirzebruch-Riemann-Roch]
    \label{Hirzebruch-Riemann-Roch}
    \[
        \chi(X, \E) = \int_X \ch{\E}\tdclass{X}.
    \]
\end{corollary}

\begin{remark}
    Meaning of integral $\int_X \ch{\E}\tdclass{X} = \pairing{\ch{\E}\tdclass{X}}{[X]}$
\end{remark}

\begin{proof}
    from GRR.
\end{proof}

\begin{definition}
    Let $f \colon X \to Y$ be a continuous map between closed orientable manifolds.  define the \emph{Gysin} or \emph{umkehr map} on cohomology as 
    \[
        f_! \colon
    \]
\end{definition}

\begin{proposition}
    \label{cohomological projection formula}
    Let $f \colon X \to Y$ be a continuous map between closed orientable manifolds $X$ and $Y$. Then for all $\alpha \in \cohom{*}{X}{\Z}$ and $\beta \in \cohom{*}{Y}{\Z}$
    \[
        f_!(\alpha) \smallsmile \beta = f_!(\alpha \smallsmile f^*\beta).
    \]
\end{proposition}

\begin{remark}
    In the proof of the above proposition we use the following formulas, which are easily seen to follow from the definitions of $\smallsmile$ and $\smallfrown$ operations on singular cohomology and homology. They may also be found in \cite[\S VI, Theorem 5.2]{Bredon1993}.
    \begin{enumerate}[label = (\roman*)]
        \item{Let $\eta \in \cohom{k}{X}{\Z}$, $\theta \in \cohom{\ell}{X}{\Z}$ and $\sigma \in H_{k+\ell+m}(X, \Z)$, then 
        \[
            (\sigma \smallfrown \eta) \smallfrown \theta = 
            \sigma \smallfrown (\eta \smallsmile \theta)
        \]
        holds in $H_m(X, \Z)$.}
        \item{Let $\sigma \in H_{k + \ell}(X, \Z)$ and $\eta \in \cohom{k}{Y}{\Z}$ then 
        \[
            f_*(\sigma \smallfrown f^*\eta) = f_*\sigma \smallfrown \eta
        \]
        holds in $H_\ell(Y, \Z)$.
        }
    \end{enumerate}
\end{remark}

\begin{proof}
    Let $n$ be the dimension of $X$ and let $D_X \colon H_k(X) \to \cohom{n-k}{X}{\Z}$ denote the inverse of $[X] \smallfrown -$, which we now to be an isomorphism by Poincaré duality \cite[text]{keylist}. Similarly let $m$ be the dimension of $Y$ and
\end{proof}

\begin{proposition}
    \label{Hodge lattice, fm transform interaction}
    Then for every pair of integers $(p,q)$
    \begin{equation}
        \label{eq: Hodge lattice, fm transform}
        \fmcoh{E}_\C\left(H^{p,q}(X)\right) \subseteq \bigoplus_{s - t = p - q} H^{s,t}(Y).
    \end{equation}
\end{proposition}