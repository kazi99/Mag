\section{Fourier--Mukai transforms}
\label{Chapter: Fourier-Mukai transforms}

Let $X$ and $Y$ be smooth projective varieties over $k$ and denote the canonical projections 
\[
    p \colon X \times Y \to X \quad \text{and} \quad q \colon X \times Y \to Y.
\]

\begin{definition}
    The \emph{Fourier--Mukai transform} associated to an object $\E$ of $\derb{X \times Y}$ is defined to be the functor
    \[
        \fm{\E} \colon \derb{X} \to \derb{Y} \qquad \fm{\E} := \rderived{}{q_*}(\derivedtensor{\E}{\lderived{}{p^*}(-)}{\struct{X \times Y}}).
    \]
    The object $\E$ is called the \emph{kernel} of $\fm{\E}$.
\end{definition}

Note that as the projection $p \colon X \times Y \to X$ is flat, the pull-back functor $p^*$ need not be derived. Many examples of known functors are seen to be of Fourier--Mukai type.

\begin{example}
    \label{Identifying fm transforms}
    % In this example we notice and identity some common functors from algebraic geometry appear as Fourier-Mukai transforms. 
\begin{enumerate}[label = (\roman*)]
    \item{
    $\id{\derb{X}} \iso \fm{\struct{\Delta}}$. Recall that $\struct{\Delta}$ is defined to be the push-forward $\Delta_*\struct{X}$, of the structure sheaf of $X$ along the diagonal embedding $\Delta \colon X \to X \times X$. Then for any object $\F$ of $\derb{X}$ we compute 
    % \begin{align*}
    %     \fm{\struct{\Delta_X}} &= \rderived{}{p_*}(\derivedtensor{\struct{\Delta_X}}{\lderived{}{p^*}(E)}{\struct{X \times X}}) \\
    %     &\iso \rderived{}{p_*}(\derivedtensor{\Delta_*\struct{X}}{\lderived{}{p^*}(E)}{\struct{X \times X}}) \\
    %     &\iso \rderived{}{p_*}(\derivedtensor{\Delta_*\struct{X}}{\lderived{}{\Delta^*}(\lderived{}{p^*}(E))}{\struct{X \times X}}) \\
    % \end{align*}
    \begin{align*}
        \fm{\struct{\Delta}} &= p_*(\struct{\Delta} \otimes_{\struct{X \times X}} p^*(\F)) \\
        &= p_*(\Delta_*\struct{X} \otimes_{\struct{X \times X}} p^*(\F)) \\
        &\iso p_*(\Delta_*(\struct{X} \otimes_{\struct{X}} \Delta^*p^*(\F))) \\
        % &\iso \struct{X} \otimes E \\
        &\iso \id{\derb{X}}(\F).
    \end{align*}
    All isomorphisms are natural in $\F$, thus proving $\id{\derb{X}} \iso \fm{\struct{\Delta}}$.
    }
    \item{$(-)[1] \iso \fm{\struct{\Delta}[1]}$. A similar computation as above identifies the translation functor as a Fourier-Mukai transform with kernel $\struct{\Delta}[1]$.}
    \item{$\rderived{}{f_*} \iso \fm{\struct{\Gamma_f}}$. For a morphism $f \colon X \to Y$, let $\struct{\Gamma_f}$ denote the push-forward of the sheaf $\struct{X}$ along the morphism $i\colon X \to X \times Y$ induced by the identity $\id{X}\colon X \to X$ and $f\colon X \to Y$.  Then for any object $\F$ of $\derb{X}$ we have
    \begin{align*}
        \fm{\struct{\Gamma_f}} &= p_*(\struct{\Gamma_f} \otimes_{\struct{X \times Y}} q^*(\F)) \\
        &= p_*(i_*\struct{X} \otimes_{\struct{X \times X}} q^*(\F)) \\
        &\iso p_*(i_*(\struct{X} \otimes_{\struct{X}} i^*q^*(\F))) \\
        % &\iso \struct{X} \otimes E \\
        &\iso f_*\F.
    \end{align*}
    % \begin{align*}
    %     \fm{\struct{\Gamma_f}} &= \rderived{}{p_*}(\struct{\Gamma_f} \otimes_{\struct{X \times Y}} \lderived{}{q^*}(E)) \\
    %     &= \rderived{}{p_*}(i_*\struct{X} \otimes_{\struct{X \times X}} \lderived{}{q^*}(E)) \\
    %     &\iso \rderived{}{p_*}(\rderived{}{i_*}(\struct{X} \otimes_{\struct{X}} \lderived{}{i^*}\lderived{}{q^*}(E))) \\
    %     % &\iso \struct{X} \otimes E \\
    %     &\iso \rderived{}{f_*}(E).
    % \end{align*}
    Note that we have used $p \circ i = f$ and $q \circ i = \id{X}$ passing to the last line. As all the isomorphisms are natural, we have $\rderived{}{f_*} \iso \fm{\struct{\Gamma_f}}$.
    }
\end{enumerate}
\end{example}

In his paper \cite{Mukai1981} Mukai discovered that every Fourier-Mukai transform admits left and right adjoints. 

\begin{proposition}[Mukai]
    \label{Adjoints of FM transforms}
    Let $\fm{\E} \colon \derb{X} \to \derb{Y}$ be a Fourier-Mukai transform associated to an object $\E$ of $\derb{X \times Y}$. Then $\fm{\E}$ admits left and right adjoints
    \[
        \fm{\ladj{\E}} \colon \derb{Y} \to \derb{X} \quad \text{and} \quad \fm{\radj{\E}} \colon \derb{Y} \to \derb{X},
    \]
    which are evidently also Fourier-Mukai with their respective kernels
    \[
        \ladj{\E} = \E^\vee \otimes_{\struct{X \times Y}} q^*\omega_Y[\dim Y] \quad \text{and} \quad 
        \radj{\E} = \E^\vee \otimes_{\struct{X \times Y}} p^*\omega_X[\dim X].
    \]
\end{proposition}

\begin{proof}
    See \cite[\S 5, Proposition 5.9]{huybrechts2006fouriermukai}. This follows from a highly non-trivial fact that the derived push-forward $\rderived{}{f_*} \colon \derb{X} \to \derb{Y}$ along a proper map $f \colon X \to Y$ admits a \emph{right} adjoint $f^! = \lderived{}{f^*}(-) \otimes_\struct{X} \omega_f[\dim f]$, where $\omega_f = \omega_X \otimes_\struct{X} f^*\omega_Y^*$ and $\dim f = \dim Y - \dim X$. This is a consequence of Grothendieck--Verdier duality \cite[\S VII, Corollary 4.3]{Hartshorne1966}. Since $X$ and $Y$ are projective the projection   
\end{proof}

\begin{remark}
    \label{another description of adjoints of FM}
    Using Serre functors we are able to write down an alternative description of left and right adjoints of $\fm{\E}$, namely
    \begin{equation}
        \label{eq: fm adjoints with serre functors}
        \fm{\ladj{\E}} = \fm{\E^\vee} \circ S_Y \quad \text{and} \quad 
        \fm{\radj{\E}} = S_X \circ \fm{\E^\vee}.
    \end{equation}
    Indeed, a straight forward computation, using the fact that pull-backs commute with tensor products, shows
    \begin{align*}
        \fm{\ladj{E}}(\F) &= p_*(\ladj{\E} \otimes_{\struct{X \times Y}} q^*\F) \\
        &= p_*(\E^\vee \otimes_{\struct{X \times Y}} q^*\omega_Y[\dim Y] \otimes_{\struct{X \times Y}} q^*\F) \\
        &\iso p_*(\E^\vee \otimes_{\struct{X \times Y}} q^*(\F \otimes_{\struct{Y}} \omega_Y[\dim Y])) \\
        &= \fm{\E^\vee}(S_Y(\F))
    \end{align*}
    for any $\F$ of $\derb{Y}$. For the other, using the projection formula \eqref{eq: projection formula}, we have
    \begin{align*}
        \fm{\radj{E}}(\F) &= p_*(\radj{\E} \otimes_{\struct{X \times Y}} q^*\F) \\
        &= p_*(\E^\vee \otimes_{\struct{X \times Y}} p^*\omega_X[\dim X] \otimes_{\struct{Y}} q^*\F) \\
        & \iso p_*(\E^\vee \otimes_{\struct{X \times Y}} q^*\F) \otimes_\struct{X} \omega_X[\dim X] \\
        % &\iso p_*(\E^\vee \otimes_{\struct{X \times Y}} q^*(\F \otimes_{\struct{Y}} \omega_Y[\dim Y])) \\
        &= S_X(\fm{\E^\vee}(\F))
    \end{align*}
    for any $\F$ of $\derb{X}$. The equations \eqref{eq: fm adjoints with serre functors} also agree with the more general method of Proposition \ref{Serre functors and adjoints} for obtaining a right adjoint of a functor provided that a left adjoint is known, when Serre functors are present.
    %  in the presence of Serre functors 
\end{remark}

The following is a celebrated theorem by Orlov, which shed a new light on the topic of triangulated functors between bounded derived categories of coherent sheaves on smooth projective varieties. Its main advantage lies in the fact that instead of studying fully faithful triangulated functors on their own, one can study them in simpler terms using their Fourier--Mukai kernels. 
% in the fact that we can study 
% It can be interpreted to provide a way 

\begin{theorem}[\text{\cite[Theorem 2.2]{Orlov-K3}}]
    \label{Orlov's theorem}
    Let $X$ and $Y$ be smooth projective varieties over a field $k$. Suppose $F \colon \derb{X} \to \derb{Y}$ is a fully faithful functor admitting a right (and, consequently, left) adjoint functor. Then there exists up to an isomorphism a unique object $\E$ of $\derb{X \times Y}$ such that $F$ and $\fm{\E}$ are naturally isomorphic.
\end{theorem}

\begin{proposition}
    \label{Composition of fm is fm}
    % Let $\fm{\E^\bullet} \colon \derb{X} \to \derb{Y}$ and $\fm{\F^\bullet} \colon \derb{Y} \to \derb{Z}$ be Fourier-Mukai transforms with kernels $\E^\bullet$ and $\F^\bullet$ belonging to $\derb{X \times Y}$ and $\derb{Y \times Z}$ respectively. Then the composition $\fm{\F^\bullet} \circ \fm{\E^\bullet} \colon \derb{X} \to \derb{Z}$ is also a Fourier-Mukai transform with kernel $\mathcal R^\bullet = ...$, which is sometimes also called the \emph{convolution} of $\E^\bullet$ and $\F^\bullet$.
    Let $\fm{\E} \colon \derb{X} \to \derb{Y}$ and $\fm{\F} \colon \derb{Y} \to \derb{Z}$ be Fourier-Mukai transforms with kernels $\E$ and $\F$ belonging to $\derb{X \times Y}$ and $\derb{Y \times Z}$ respectively. Then the composition $\fm{\F} \circ \fm{\E} \colon \derb{X} \to \derb{Z}$ is also a Fourier-Mukai transform with kernel
    \[
        \G = (\pi_{XZ})_*(\pi_{XY}^*\E \otimes \pi_{YZ}^*\F),
    \]
    which is sometimes also called the \emph{convolution} of $\E^\bullet$ and $\F^\bullet$. Morphisms $\pi_{XY}$, $\pi_{YZ}$ and $\pi_{XZ}$ denote the canonical projections from $X \times Y \times Z$ to $X \times Y$, $Y \times Z$ and $X \times Z$, respectively.
\end{proposition}

\begin{proof}
    See \cite[\S 5, 5.10]{huybrechts2006fouriermukai}. The proof utilizes the projection formula \eqref{eq: projection formula} and the flat base change formula \eqref{eq:}. Essentially this means that it is far less sophisticated then the proof of the existence of adjoints.    
\end{proof}

The following is a beautiful application of Orlov's Theorem \ref{Orlov's theorem}. It is also an important statement by itself, providing evidence for why the bounded derived category of coherent sheaves is a nicely behaved invariant for smooth projective varieties. In particular it says that $\derb{-}$ detects the dimension and triviality of the canonical bundle.  
% showing that the bounded derived categories of coherent sheaves on smooth and projective varieties over $k$ detect dimension and triviality of the canonical bundle. 
\begin{theorem}
    \label{Db detects dimension and triviality of canonical bundle}
    Suppose $X$ and $Y$ are smooth projective varieties over $k$. Assume
    \[
        \derb{X} \natiso \derb{Y}.
    \]
    Then $\dim X = \dim Y$ and if the canonical bundle $\omega_X$ of $X$ is trivial, the canonical bundle $\omega_Y$ of $Y$ is trivial as well.
\end{theorem}

\begin{proof}
    By Orlov's Theorem \ref{Orlov's theorem} the equivalence $\derb{X} \natiso \derb{Y}$ is witnessed by a Fourier-Mukai transform $\fm{\E}$ with a kernel $\E$ belonging to $\derb{X \times Y}$. Since $\fm{\E}$ is an equivalence its left and right adjoints agree. As they are both given by Fourier-Mukai transforms their respective kernels $\ladj{\E}$ and $\radj{\E}$, introduced in Proposition \ref{Adjoints of FM transforms}, must be isomorphic in $\derb{X \times Y}$. Writing this out we see that
    \[
        \E^\vee \iso \E^\vee \otimes (q^*\omega_X \otimes p^*\omega_Y)[\dim Y - \dim X].
    \]
    As tensoring by a line bundle, namely $q^*\omega_X \otimes p^*\omega_Y$ is exact, both tensor products are underived, but more importantly, the cohomology of complexes $\E^\vee$ and ${\E^\vee \otimes (q^*\omega_X \otimes p^*\omega_Y)}$ agree in all degrees. Since $\E$ is a bounded complex of coherent sheaves, its derived dual $\E^\vee = \rderived{}{\localhom[X \times Y]^\bullet(\E, \struct{X \times Y})}$ is bounded as well. Then comparing cohomology intermediately\footnote{As $\fm{\E}$ is an equivalence, the kernel $\E$ cannot be isomorphic to the zero object in $\derb{X \times Y}$, implying that $\E^\vee$ has non-trivial cohomology at least in some degrees.} gives 
    \[
        \dim X = \dim Y.
    \]
    
    For the second part, the assumption $\omega_X \iso \struct{X}$, simplifies the Serre functor $S_X$ of $\derb{X}$ to 
    \[
        S_X \colon \derb{X} \to \derb{X} \qquad S_X \natiso (-)[n], 
    \]
    where $n$ denotes the dimensions of both $X$ and $Y$. Since equivalences of triangulated categories are known to commute with Serre functors by Proposition \ref{Serre functors commute with equivalences}, we deduce from $\fm{\E} \circ S_X \natiso S_Y \circ \fm{\E}$ that the Serre functor $S_Y$ of $\derb{Y}$ also simplifies to $(-)[n]$. By Definition \ref{Serre functor on Db(X)} this may be expressed as
    \[
        (-)[n] \natiso S_Y = (-) \otimes_{\struct{Y}} \omega_Y[n].
    \]
    Plugging in the structure sheaf $\struct{Y}$, we see that $\omega_Y \iso \struct{Y}$ holds in $\coherent{Y}$. The last assertion 
    % about $\coherent{Y}$ 
    follows from fully faithfulness of the inclusion $\coherent{Y} \to \derb{Y}$ according to Proposition \ref{A -> D(A) fully faithful}. 
\end{proof}

\subsection{...on \emph{K}-groups}
\label{Subsection: FM transform on K-theory}

We introduce the $K$-theoretic Fourier-Mukai transforms as a passageway between categorical and cohomological Fourier-Mukai transforms. Throughout this section we assume our schemes are smooth projective complex varieties and our morphisms are proper. We do this so that we have for example access to propositions like \ref{In Db(X) everything is a complex of vector bundles}, \ref{push-forward on Db(X)}, etc.

Given a smooth projective complex variety $X$ we define its \emph{$K$-group} as follows. Consider the free abelian group generated by isomorphism classes of coherent sheaves on $X$. The quotient of this group with its subgroup generated by expressions of the form $[\E] - [\F] + [\G]$, when there is a short exact sequence $0 \to \E \to \F \to \G \to 0$, is the $K$-group of $X$. We denote it by $\kgroup{X}$. Extending this notion of additivity on short exact sequences, we define for any bounded complex $\F^\bullet$ of $\derb{X}$ 
\[
    [\F^\bullet] := \sum_{i \in \Z}(-1)^i[\F^i] \in \kgroup{X}.
\]
\begin{remark}
    Observe that any complex $\F^\bullet$ may be split up into a series of short exact sequences of the form
    \[
        0 \to \ker d^{i}_\F \to \F^i \to \im d^i_\F \to 0.
    \]
    In $\kgroup{X}$ this means that $[\F^i] = [\ker d^{i}_\F] + [\im d^i_\F]$. From the short exact sequence defining the cohomology $\mathcal H^i(\F^\bullet)$ we also see that $[\mathcal H^i(\F^\bullet)] = [\im d^{i-1}_\F] + [\ker d^{i}_\F]$. Using these two formulas one reparenthesizes the definition of $[\F^\bullet]$ to obtain
    \[
        [\F^\bullet] = \sum_{i \in \Z}(-1)^i[\mathcal H^i(\F^\bullet)].
    \]
    In particular this shows that the operation $[-]$ of forming a class in $\kgroup{X}$ out of a complex $\F^\bullet$ of $\derb{X}$ is an invariant of its isomorphism class in $\derb{X}$.
\end{remark}


The group $\kgroup{X}$ also comes equipped with a multiplication operation. On isomorphism classes of locally free sheaves $\E$ and $\F$ this operation is given by
\[
    [\E]\cdot[\F] = [\E \otimes_{\struct{X}} \F].
\]

\begin{definition}
    Let $f \colon X \to Y$ be a proper morphism between smooth projective complex varieties. 
    % Let $\F$ be a coherent sheaf on $X$. 
    The \emph{push-forward map} defined on equivalence classes of coherent sheaves is
    \[
        f_! \colon \kgroup{X} \to \kgroup{Y}, \qquad f_!([\F]) = \sum_{i \in \Z}(-1)^i[\rderived{i}{f_*}\F].
    \]
    By the following remark this definition is sound.
\end{definition}

\begin{remark}
    Existence of a long exact sequence for right derived functors associated to a short exact sequence of coherent sheaves (Proposition \ref{LES for right derived functor}) implies that $f_!$ is well-defined.
\end{remark}

The following proposition captures compatibilities of categorical derived push-forward and derived pull-back functors with their respective $K$-theoretical counterparts.  

\begin{proposition}
    \label{push-forward and pull-back compatibilities}
    Let $f \colon X \to Y$ be a proper morphism between smooth projective complex varieties. For every bounded complex $\E^\bullet$ of $\derb{X}$ and $\F^\bullet$ of $\derb{Y}$, we have
    \[
        [\rderived{}{f_*}\E^\bullet] = f_!([\E^\bullet]) \qquad \text{and} \qquad [\lderived{}{f^*}\F^\bullet] = f^*([\F^\bullet]).
    \]
\end{proposition}

\begin{proof}
    
    We see the second equality holds true, because $\F^\bullet$ may be replaced by its locally free resolution $\E_0^\bullet$, guaranteed to exist by Proposition \ref{In Db(X) everything is a complex of vector bundles}. and noting that the pull-back functor $f^*$ . 
    % For the second equation it is enough to verify it for a single coherent sheaf $\F$ on $Y$, because this time both sides of the equation are clearly additive in $\F$.
\end{proof}

% In particular we will utilize the Grothendieck-Riemann-Roch theorem, which will later on help us relate the two in Section \ref{Subsection: FM transform on cohomology}.

\begin{definition}
    Let $X$ and $Y$ be smooth projective complex varieties. The \emph{Fourier--Mukai transform} with kernel $\xi \in \kgroup{X \times Y}$ on the level of $K$-groups is defined to be
    \[
        \fmkt{\xi} \colon \kgroup{X} \to \kgroup{Y}, \qquad \fmkt{\xi} := q_!(\xi \cdot p^*(-)).
    \]
\end{definition}

\begin{corollary}
    Let $X$ and $Y$ be smooth projective complex varieties and $\E^\bullet$ a bounded complex of $\derb{X \times Y}$. Then
    \[
        \fmkt{[\E^\bullet]}([-]) = [\fm{\E^\bullet}(-)].
    \]
\end{corollary}

\begin{proof}
    This is a direct consequence of Proposition \ref{push-forward and pull-back compatibilities}.
\end{proof}

\subsection{...on rational cohomology}
\label{Subsection: FM transform on cohomology}

A Fourier--Mukai type map exists also between the rational cohomology modules of $X$ and $Y$ associated to any cohomology class $\alpha \in \cohom{*}{X \times Y}{\Q}$. In order to introduce it we first need to define the correct analog of the ``push-forward map'' on cohomology. We are again assuming that $X$ and $Y$ are smooth and projective varieties over $\C$ and we identify them with their respective complex analytic spaces, which are in this case compact connected complex manifolds. As such they are orientable as real manifolds, admitting Poincaré duality to relate their homology and cohomology. Let $n$ denote the \emph{real} dimension of $X$ and let $[X] \in H_n(X, \Q)$ denote the (rational) fundamental class of $X$. By Poincaré duality \cite[\S VI, Theorem 8.3]{Bredon1993} the homomorphism
\[
    [X] \smallfrown - \colon \cohom{k}{X}{\Q} \to H_{n-k}(X, \Q)
\]
is an isomorphism and we denote its inverse with $D_X \colon H_{n-k}(X, \Q) \to \cohom{k}{X}{\Q}$.

\begin{definition}
    \label{Definition of umkehr map}
    Let $f \colon X \to Y$ be a continuous map between two compact connected complex manifolds. Let $n$ and $m$ denote the real dimensions of $X$ and $Y$, respectively.
    %  $m$ the dimension of $Y$.
    %  of real dimension $n$ and $m$ respectively.
    The \emph{Gysin map}\footnote{This map is sometimes also called the \emph{umkehr map}. The German word ``umkehr'' pointing out the fact that the map goes in the opposite direction of the standard pull-back.} $f_! \colon \cohom{k}{X}{\Q} \to \cohom{m-n+k}{Y}{\Q}$ is defined to be the composition
    \[
        f_! \colon \quad \cohom{k}{X}{\Q} \xrightarrow{[X] \smallfrown -} H_{n-k}(X, \Q) \xrightarrow{\ f_* \ } H_{n-k}(Y, \Q) \xrightarrow{D_Y} \cohom{m-n+k}{Y}{\Q}.
    \]
\end{definition}

\begin{remark}
    In other words $f_!$ is precisely the map, which fits into the commutative diagram below
    \[\begin{tikzcd}
        % https://q.uiver.app/#q=WzAsNCxbMCwwLCJcXGJ1bGxldCJdLFsyLDAsIlxcYnVsbGV0Il0sWzAsMiwiXFxidWxsZXQiXSxbMiwyLCJcXGJ1bGxldCJdLFswLDEsIltYXSBcXHNtYWxsZnJvd24gLSJdLFsyLDMsIltZXSBcXHNtYWxsZnJvd24gLSJdLFsxLDMsImZfKiJdLFswLDIsImZfISIsMl1d
        \cohom{k}{X}{\Q} && H_{n-k}(X, \Q) \\
        \\
        \cohom{m-n+k}{X}{\Q} && H_{n-k}(X, \Q).
        \arrow["{[X] \smallfrown -}", from=1-1, to=1-3]
        \arrow["{f_!}"', from=1-1, to=3-1]
        \arrow["{f_*}", from=1-3, to=3-3]
        \arrow["{[Y] \smallfrown -}", from=3-1, to=3-3]
    \end{tikzcd}\]
\end{remark}

\begin{remark}
    For a topological space $X$ we will distinguish between the collection of all cohomology groups, considered concisely as a graded object $\cohom{\bullet}{X}{\Z} = \bigoplus_{i\in \Z} \cohom{i}{X}{\Z}$, and its (graded) cohomology ring, which we will denote by $\cohom{*}{X}{\Z}$.
\end{remark}

\begin{proposition}[Cohomological projection formula]
    \label{cohomological projection formula}
    Let $f \colon X \to Y$ be a continuous map between closed orientable manifolds $X$ and $Y$. Then for all $\alpha \in \cohom{*}{X}{\Z}$ and $\beta \in \cohom{*}{Y}{\Z}$
    \begin{equation}
        \label{eq: cohomological projection formula}
        f_!(\alpha) \smallsmile \beta = f_!(\alpha \smallsmile f^*\beta).
    \end{equation}
\end{proposition}

\begin{remark}
    In the proof of the above proposition we use the following formulas, which are easily seen to follow by unwrapping the definitions of operations $\smallsmile$ and $\smallfrown$ on singular cohomology and homology. They may also be found in \cite[\S VI, Theorem 5.2]{Bredon1993}.
    \begin{enumerate}[label = (\roman*)]
        \item{Let $\eta \in \cohom{k}{X}{\Z}$, $\theta \in \cohom{\ell}{X}{\Z}$ and $\sigma \in H_{k+\ell+m}(X, \Z)$, then 
        \[
            (\sigma \smallfrown \eta) \smallfrown \theta = 
            \sigma \smallfrown (\eta \smallsmile \theta)
        \]
        holds in $H_m(X, \Z)$.}
        \item{Let $\sigma \in H_{k + \ell}(X, \Z)$ and $\eta \in \cohom{k}{Y}{\Z}$ then 
        \[
            f_*(\sigma \smallfrown f^*\eta) = f_*\sigma \smallfrown \eta
        \]
        holds in $H_\ell(Y, \Z)$.
        }
    \end{enumerate}
\end{remark}

\begin{proof}[Proof of Proposition \ref{cohomological projection formula}]
    % Let $n$ and $m$ denote the real dimensions of $X$ and $Y$ respectively.
    We compute that
    \begin{align*}
        f_!(\alpha \smallsmile f^*\beta) &=
        D_Y(f_*([X] \smallfrown (\alpha \smallsmile f^*\beta))) \\
        &= D_Y(f_*(([X] \smallfrown \alpha) \smallfrown f^*\beta)) \\
        &= D_Y(f_*([X] \smallfrown \alpha) \smallfrown \beta) \\
        &= D_Y(([Y] \smallfrown f_!(\alpha)) \smallfrown \beta) \\
        &= D_Y([Y] \smallfrown (f_!(\alpha) \smallsmile \beta)) \\
        &= f_!(\alpha) \smallsmile \beta. \qedhere
    \end{align*}
\end{proof}

\begin{proposition}[Cohomological base change formula]
    Consider the following pull-back square of closed orientable manifolds and assume $g$ and $v$ are fibrations.
    \[\begin{tikzcd}[row sep = 3em]
        % https://q.uiver.app/#q=WzAsNCxbMCwwLCJYJyJdLFswLDEsIlgiXSxbMSwxLCJZIl0sWzEsMCwiWSciXSxbMSwyLCJmIl0sWzMsMiwiZyJdLFswLDEsInYiLDJdLFswLDMsInUiXSxbMCwyLCIiLDEseyJzdHlsZSI6eyJuYW1lIjoiY29ybmVyIn19XV0=
        {X'} & {Y'} \\
        X & Y
        \arrow["u", from=1-1, to=1-2]
        \arrow["v"', from=1-1, to=2-1]
        \arrow["\lrcorner"{anchor=center, pos=0.125}, draw=none, from=1-1, to=2-2]
        \arrow["g", from=1-2, to=2-2]
        \arrow["f", from=2-1, to=2-2]
    \end{tikzcd}\]
    Then
    \begin{equation}
        \label{eq: cohomological base change}
        v_! \circ u^* = f^* \circ g_!
    \end{equation}
    as homomorphisms $\cohom{\bullet}{Y'}{\Q} \to \cohom{\bullet}{X}{\Q}$.
\end{proposition}

\begin{proof}
    \info{needs reference}    
\end{proof}

As before let $p \colon X \times Y \to X$ and $q \colon X \times Y \to Y$ denote the canonical projections, now in the category of complex manifolds. 

\begin{definition}
    The \emph{cohomological Fourier-Mukai transform} associated to a cohomology class $\alpha \in \cohom{\bullet}{X \times Y}{\Q}$ is defined to be the homomorphism
    \[
        f^\alpha \colon \cohom{\bullet}{X}{\Q} \longrightarrow \cohom{\bullet}{Y}{\Q} \qquad \beta \longmapsto q_!(\alpha \smallsmile p^*\beta).
    \]
\end{definition}

We would now like to relate this cohomological Fourier-Mukai transform back to the $K$-theoretic and the categorical one. This will be achieved by applying the famous Grothendieck--Riemann--Roch theorem \ref{Grothendieck-Riemann-Roch}, for which we first need to introduce the Chern character and the Todd class. 

\begin{proposition}
    \label{Composition of cohomological fm is fm}
    Let $\fmcoh{\E} \colon H^\bullet(X, \Q) \to H^\bullet(Y, \Q)$ and $\fmcoh{\F} \colon H^\bullet(Y, \Q) \to H^\bullet(Z, \Q)$ be cohomological Fourier--Mukai transforms for kernels $\E$ and $\F$ belonging to $\derb{X \times Y}$ and $\derb{Y \times Z}$ respectively. Then the composition $\fmcoh{\F} \circ \fmcoh{\E}$ is also a cohomological Fourier-Mukai transform given by the kernel $\G$ of Proposition \ref{Composition of fm is fm}.
\end{proposition}

\begin{proof}
    The proof of this result is analogous to the proof of Proposition \ref{Composition of fm is fm}, where instead we use the cohomological projection and base change formulas, \eqref{eq: cohomological projection formula} and \eqref{eq: cohomological base change}, respectively.
\end{proof}

\subsubsection*{Grothendieck--Riemann--Roch theorem}

In order to relate Fourier--Mukai functors on bounded derived categories of smooth projective varieties over $\C$ with their rational cohomology counterparts we will employ the famous Grothendieck--Riemann--Roch theorem. For us to be able to state it, we first introduce the Chern character and the Todd class of a smooth complex variety.

\begin{definition}
    \label{Definition of Chern character}
    Let $X$ be a smooth projective variety over $\C$. The \emph{Chern character} is a ring homomorphism
    \[
        \ch[] \colon \kgroup{X} \to \cohom{*}{X}{\Q}
    \] 
    enjoying many desirable properties, such as
    \begin{enumerate}[label = (\roman*)]
        \item{For a line bundle $\mathcal L$ on $X$: $\ch[](\mathcal L) = e^{\cclass[1]{\mathcal L}} = 1 + \cclass[1]{\mathcal L} + \frac{1}{2!}\cclass[1]{\mathcal L}^2 + \frac{1}{3!}\cclass[1]{\mathcal L}^3 + \cdots$.
        % \sum_{k=0}^{\infty}\frac{1}{k!}\cclass[1]{\mathcal{L}}^k$.
        } \label{chern character on line bundle}
        \item{\footnote{
            See \ref{Definition of v vee} for the definiton of $(-)^\vee$ and \cite[Lemma C.7]{vanBree2020} for a proof.
        }For any coherent sheaf $\E$ on $X$: $\ch[]{\E^\vee} = \ch[]{\E}^\vee$}
        \item{Naturality for pull-back: $f^*\ch[]{e} = \ch[]{f^*e}$ for any proper morphism $f \colon X \to Y$.}
        \item{The Chern character takes values in $\cohom{2\bullet}{X}{\Q}$, \ie cohomology groups of \emph{even} degree.}
    \end{enumerate} 
\end{definition}

\begin{remark}
    Such a homomorphism in fact exists and is usually constructed by first specifying it on line bundles as in \ref{chern character on line bundle}. Then utilizing the so-called \emph{Splitting principle} \cite[text]{Fulton1998} it is possible to extend the Chern character to arbitrary vector bundles on $X$. Further, one defines the Chern character of a coherent sheaf $\E$ on $X$ using a bounded locally free resolution, mentioned already in Proposition \ref{In Db(X) everything is a complex of vector bundles}. Lastly, one extends the Chern character to arbitrary bounded complexes of coherent sheaves by additivity \ie
    \[
        \ch[](\F^\bullet) = \sum_{i \in \Z}(-1)^i\ch[](\F^i).
    \] 
    % one uses the fact that every coherent sheaf on $X$ has a bounded resolution with vector bundles, mentioned already in the proof of Proposition \ref{In Db(X) everything is a complex of vector bundles}.
\end{remark}

% The Chern character exists 



The \emph{Todd class} of a smooth complex projective variety $X$ is a cohomology class, living in even degrees, denoted by
\[
    \tdclass{X} \in \cohom{2\bullet}{X}{\Q}.
\]
% will be denoted by $\tdclass{X}$. 
As with the Chern character we refrain from properly defining or constructing $\tdclass{X}$, but give the first few terms of its series expansion in terms of Chern classes of $X$. It goes as follows
\[
    \tdclass{X} = 1 + \frac{1}{2}\cclass[1](X) + \frac{1}{12}(\cclass[1]{X}^2 + \cclass[2]{X}) + \frac{1}{12}\cclass[1](X)\cclass[2](X) + \cdots.
\] 

% \begin{definition}
%     The \emph{Todd class} of a smooth complex variety $X$ is defined to be
%     \[
%         \tdclass{X} = 
%     \]
%     where $\tangent{X}$ is the tangent sheaf of $X$, \ie $\tangent{X} = \Omega_X^*$.  
%     We define this to also be the Todd class of the corresponding compact complex manifold $X$ considered as an analytic space. 
% \end{definition}



% \begin{remark}
%     From $\Omega_{X \times Y} \iso p^*\Omega_X \oplus q^*\Omega_Y$, we also see that $\tangent{X \times Y}$ so
%     \[
%         \tdclass{X \times Y} = p^*\tdclass{X} \smallsmile q^*\tdclass{Y}
%     \]
%     Since the degree $0$ part of the Todd class is $1$, we see that it is invertible in the ring $\cohom{*}{X}{\Q}$ and its inverse can be given by a power series.
% \end{remark}

\begin{theorem}[Grothendieck--Riemann--Roch]
    \label{Grothendieck-Riemann-Roch}
    Let $f \colon X \to Y$ be a proper morphism of smooth complex varieties. Then for all $e \in \kgroup{X}$, equation 
    \begin{equation}
        \label{eq: GRR}
        \ch[]{f_!(e)} \smallsmile \tdclass{Y} = f_!(\ch[]{e} \smallsmile \tdclass{X})
    \end{equation}
    holds in $\cohom{*}{Y}{\Q}$.
\end{theorem}

\begin{proof}
    A proof may be found in \cite[\S 15]{Fulton1998}.
    % The statement is \cite[\S 15, Theorem 15.2]{Fulton1998} and the proof may be found in Chapter 15 of this reference.
\end{proof}

\begin{remark}
    In the formulation of the Grothendieck--Riemann--Roch theorem in \cite[\S 15, Theorem 15.2]{Fulton1998} the equation \eqref{eq: GRR} holds inside the so-called (rational) \emph{Chow ring} $A(Y)_\Q = A(Y) \otimes_{\Z} \Q$ instead of $\cohom{*}{Y}{\Q}$. We will not define this ring, but mention that there exists a push-forward map also on these Chow rings $f_* \colon A(X) \to A(Y)$ and homomorphisms $\operatorname{cl} \colon A(X) \to \cohom{*}{X}{\Z}$, for each smooth complex variety $X$, called \emph{cycle maps}, which are covariant in $X$ and compatible with Chern classes. This enables us to state the theorem in the way that we did. More on this topic can be found in \cite[\S 19]{Fulton1998}.
    % such that the diagram below commutes.
    % \[\begin{tikzcd}
    %     % https://q.uiver.app/#q=WzAsNCxbMCwwLCJcXGJ1bGxldCJdLFsyLDAsIlxcYnVsbGV0Il0sWzAsMiwiXFxidWxsZXQiXSxbMiwyLCJcXGJ1bGxldCJdLFswLDEsIltYXSBcXHNtYWxsZnJvd24gLSJdLFsyLDMsIltZXSBcXHNtYWxsZnJvd24gLSJdLFsxLDMsImZfKiJdLFswLDIsImZfISIsMl1d
    %     A(X)_\Q && H^{*}(X, \Q) \\
    %     \\
    %     A(Y)_\Q && H^{*}(Y, \Q)
    %     \arrow["", from=1-1, to=1-3]
    %     \arrow["{f_*}"', from=1-1, to=3-1]
    %     \arrow["{f_!}", from=1-3, to=3-3]
    %     \arrow["", from=3-1, to=3-3]
    % \end{tikzcd}\]
    % This enables us to state the theorem in the way that we did. 
\end{remark}

For a coherent sheaf $\E$ on $X$ we already know that $\cohom{i}{X}{\E} = 0$ for $i > \dim X$ by a finiteness result of Serre (Theorem \ref{Serre finitness}). The concept of an \emph{Euler characteristic} of $\E$, set out to be
\[
    \chi(X, \E) = \sum_{i \in \Z} (-1)^i \dim_\C H^i(X, \E),
\] 
is then well-defined.

\begin{corollary}[Hirzebruch--Riemann--Roch]
    \label{Hirzebruch-Riemann-Roch}
    Let $X$ be a smooth complex projective variety and $\E$ a coherent sheaf on $X$. Then
    \begin{equation}
        \label{eq: Hirzebruch-Riemann-Roch formula}
        \chi(X, \E) = \int_X \ch{\E}\tdclass{X}.
    \end{equation}
\end{corollary}

% Evaluated against a fundamental class $[X]$ 

\begin{remark}
    The integral on the right hand side we will take to be $\int_X \ch{\E}\tdclass{X} = \pairing{\ch{\E}\tdclass{X}}{[X]}$. 
    % In other words to compute the right side we only calculate the top degree contributions of the product $\ch{\E}\td{X}$
    Letting $n$ denote the real dimension of $X$, computing the integral amounts to calculating the $n$-th degree contributions of the product $\ch{\E}\td{X}$ and evaluating them into a rational number through the isomorphism $\cohom{n}{X}{\Q} \iso \Q$, given by the choice of a fundamental class $[X]$.
\end{remark}

\begin{proof}
    We will use the Grothendieck--Riemann--Roch formula for the structure morphism $\varepsilon \colon X \to \spec{\C}$. The Chern character on a point is completely determined by its image of the trivial line bundle $\mathcal L = \struct{\spec{\C}}$. This is because $\coherent{\spec{\C}}$ is equivalent to the category of finite dimensional $\C$-vector spaces and $\ch[]$ is additive. In this case we have
    \[
        \ch[]{\mathcal L} = e^{\cclass[1]{\mathcal L}} = e^0 = 1.
    \]
    Therefore after identifying the ring $\cohom{*}{\spec{\C}}{\Q}$ with $\Q$, we see that $\ch[](-) = \dim_\C(-)$. We also note that $\tdclass{\spec{\C}} = 1$ in this cohomology ring. The left-hand side of \eqref{eq: GRR} is then
    \[
        \ch[]{\varepsilon_!(\E)}\smallsmile 1 = \ch[]{\sum_{i \in \Z}(-1)^i[\rderived{i}{\varepsilon_*}\E]} = \sum_{i\in \Z}(-1)^i\dim_\C\cohom{i}{X}{\E} = \chi(X, \E).
    \]  
    Evaluating the right-hand side only amounts to interpreting the meaning of $\varepsilon_!$ according to remark before this proof.
    % Whereas the right-hand side is 
    % \[
    %     \varepsilon_!(\ch[]{\E} \smallsmile \tdclass{X}) = \langle \ch[]{\E} \smallsmile \tdclass{X}, [X] \rangle = \int_X \ch[]{\E} \smallsmile \tdclass{X}.
    % \]
\end{proof}

\subsubsection*{Mukai vector}

Of special importance for relating the $K$-theoretic Fourier--Mukai transforms with their cohomological counterparts is the Mukai vector. 

\begin{definition}
    The \emph{Mukai vector} of a class $e \in \kgroup{X}$ is defined to be
    \[
        v(e) := \ch[](e)\sqrt{\tdclass{X}}.
    \]
    The Mukai vector of a complex $\E^\bullet$ of $\derb{X}$ is defined to be the Mukai vector of the corresponding class $[\E^\bullet] \in \kgroup{X}$, so $v(\E^\bullet) := v([\E^\bullet])$.
\end{definition}

\begin{remark}
    \label{sqrt of cohomology class}
    Recall that for compact manifolds $X$ the cohomology ring $\cohom{*}{X}{\Q}$ is finite dimensional. Thus we can get away with defining the square root of a cohomology class $\alpha \in \cohom{*}{X}{\Q}$ by a power series
    % As a consequence we may for example define the square root of a cohomology class $\alpha \in \cohom{*}{X}{\Q}$ by a power series
    \[
        \sqrt{\alpha} = 1 + \tfrac{1}{2}\alpha - \tfrac{1}{8}\alpha^2 + \tfrac{1}{16}\alpha^3 + \cdots \in \cohom{*}{X}{\Q}.
    \]
    Defining the exponential $e^\alpha \in \cohom{*}{X}{\Q}$ is also done in a similar manner. 
\end{remark}

\begin{proposition}
    \label{Mukai vector FM transform interaction}
    For each kernel $\xi \in \kgroup{X \times Y}$ and any $e \in \kgroup{X}$ we have
    \[
        \fmcoh{v(\xi)}(v(e)) = v(\fmkt{\xi}(e))
    \]  
\end{proposition}

\begin{proof}
    This equation follows from the following computation
    \begin{align*}
        \fmcoh{v(\xi)}(v(e)) &=
        q_!(v(\xi) \smallsmile p^*v(e)) \\
        &= q_!(\ch[](\xi)\sqrt{\tdclass{X \times Y}} \smallsmile p^*(\ch[]{e}\sqrt{\tdclass{X}})) \\
        &= q_!(\ch[](\xi)p^*\sqrt{\tdclass{X}} \smallsmile q^*\sqrt{\tdclass{Y}} \smallsmile \ch[]{p^*(e)}p^*\sqrt{\tdclass{X}}) & \text{(Def. \ref{Definition of Chern character})} \\
        &= q_!(\ch[](\xi \cdot p^*(e))\tdclass{X \times Y} \smallsmile q^*\sqrt{\tdclass{Y}}\inv) & \text{(Formula \eqref{eq: projection formula})}\\
        &= q_!(\ch[](\xi \cdot p^*(e))\tdclass{X \times Y}) \smallsmile \sqrt{\tdclass{Y}}\inv & \text{(Formula \eqref{eq: GRR})} \\
        &= \ch[]{q_!(\xi \cdot p^*(e))}\sqrt{\tdclass{Y}} \\
        &= v(\fmkt{\xi}(e)). & & \qedhere
    \end{align*}
\end{proof}

\begin{definition}
    \label{Definition of v vee}
    For a vector $v \in \cohom{2\bullet}{X}{\Q} = \bigoplus_{i \in \Z} \cohom{2i}{X}{\Q}$ confined in \emph{even} degrees, with components $v = (v_0, v_1, v_2, v_3, \dots)$, we define $v^\vee = (v_0, -v_1, v_2, -v_3, \dots)$.
    % define
    % $v^\vee \in \cohom{2\bullet}{X}{\Q}$ with components $(v^\vee)_{2i} = (-1)^i v_{2i}$. In other words for $v = (v_0, v_1, v_2,\dots)$ 
\end{definition}

This operation is clearly natural for pull-backs and fixes all cohomology classes living in degrees, which are multiples of $4$.  

\begin{definition}
    \label{Definition of Mukai pairing}
    For any $v$, $w \in \cohom{2\bullet}{X}{\Q}$ coming from the even part, we define the \emph{Mukai pairing} on $\cohom{2\bullet}{X}{\Q}$ to be
    \[
        \pairing{v}{w} = - \int_X v^\vee \smallsmile w.
    \]
\end{definition}

Since the Mukai vector of any complex $\E^\bullet$ is spread out only in the even cohomology groups, we can pair their Mukai vectors with each other in this way. Another way of pairing up (bounded complexes of) coherent sheaves is given by the \emph{Euler pairing}, defined for $\E^\bullet$ and $\F^\bullet$ belonging to $\derb{X}$ to be the alternating sum
\begin{equation}
    \label{Euler pairing}
    \chi(\E^\bullet, \F^\bullet) = \sum_{i \in \Z} (-1)^i \dim_\C \Ext^i_\struct{X}(\E^\bullet, \F^\bullet).
\end{equation}
The following proposition says that these two pairings are basically the same, differing only by a sign. 

\begin{proposition}
    Let $\E^\bullet$ and $\F^\bullet$ be complexes of coherent sheaves belonging to $\derb{X}$. Then
    \[
        \pairing{v(\E^\bullet)}{v(\F^\bullet)} = -\chi(\E^\bullet, \F^\bullet)
    \]
\end{proposition}

\begin{proof}
    This is a straight forward application of the Hirzebruch--Riemann--Roch formula \eqref{eq: Hirzebruch-Riemann-Roch formula}, once we realize that we have the following isomorphism in $\derb{X}$
    \[
        {\E^\bullet}^\vee \derivedtensor{}{}{\struct{X}} \F^\bullet \iso \rderived{}{\localhom[X]^\bullet(\E^\bullet, \F^\bullet)}.
    \]
\end{proof}

\subsubsection*{Hodge structures}

This short section serves as a quick recollection on the very basics of Hodge theory and ends with a proposition describing a relationship between a cohomological Fourier--Mukai transform and some Hodge structures. 

\begin{definition}
    A \emph{Hodge structure of weight $n \in \Z$} on a finitely generated free abelian group or a finite dimensional $\Q$-vector space $V$ is a direct sum decomposition of $V_\C = V \otimes_\Z \C$, consisting of subspaces $V^{p,q} \subseteq V_\C$ for $p$, $q \in \Z$ of the form
    \[  
        V_\C = \bigoplus_{p+q = n} V^{p,q},
    \]
    satisfying $\ols{V^{p,q}} = V^{q,p}$ for all pairs $p$, $q \in \Z$. If $V$ is a $\Z$-module, we say it is equipped with an \emph{integral} Hodge structure and if $V$ is a $\Q$-vector space, we say the Hodge structure is \emph{rational}.
    
    A $\Z$-linear (\resp $\Q$-linear) homomorphism $f \colon V \to W$ between two integral (\resp rational) Hodge structures of weight $n$ is said to \emph{preserve the Hodge structures}, if
    \[
        f_\C(V^{p,q}) \subseteq W^{p,q}
    \]  
    for all $p$, $q \in \Z$, with $p + q = n$, where $f_\C$ denotes the complexification $f \otimes \id{\C}$.
\end{definition}

\begin{remark}
    Complex conjugation on $V_\C = V \otimes_\Z \C$ is defined on elementary tensors as $\ols{v \otimes \lambda} = v \otimes \ols{\lambda}$.
\end{remark}

A prototypical example of a Hodge structure of weight $n$ comes from complex geometry and is given by the following decomposition of the $n$-th cohomology group $\cohom{n}{X}{\C}$ of a compact complex manifold $X$
\[
    H^n(X, \Q)_\C = \bigoplus_{p+q = n} H^{p,q}(X).
\]
The groups $H^{p,q}(X)$ are defined to be the \emph{Dolbeault cohomology groups} or $\cohom{q}{X}{\Omega_X^p}$. Here $\Omega_X^p$ denotes the sheaf of holomorphic $p$-forms or more precisely the $p$-th exterior power $\bigwedge^p \Omega_X$ of the sheaf of Kähler differentials $\Omega_X$ on $X$. The Hodge decomposition of $\cohom{n}{X}{\Q}$ then arises from the well-known Hodge decomposition theorem through identifying the Dolbeault cohomology groups as certain subspaces of globally defined $(p,q)$-forms, called \emph{harmonic} forms. This viewpoint also allows one to see that the $\smallsmile$-product on the cohomology ring $\cohom{*}{X}{\C}$ defines  


% We will describe the groups $H^{p,q}(X)$ using our established language of derived functors but refer the reader to consult \cite[text]{keylist} for many of the details left out.

For example in the case of a $2$-dimensional complex manifold  

Cohomological Fourier--Mukai transforms unfortunately do not preserve the Hodge structures in general\footnote{In the same breath we have to mention that for K3 surfaces this is true, as we shall see in Lemma \ref{}, but that will }, but they interact with Hodge structures in a particular way, which is worth noting. 

\begin{proposition}
    \label{Hodge lattice, fm transform interaction}
    Let $X$ and $Y$ be complex smooth projective varieties and let $\fmcoh{\E} \colon \cohom{\bullet}{X}{\Q} \to \cohom{\bullet}{Y}{\Q}$ be a Cohomological Fourier--Mukai transform associated to a kernel $\E$ belonging to $\derb{X \times Y}$.
    Then for every pair of integers $(p,q)$
    \begin{equation}
        \label{eq: Hodge lattice, fm transform}
        \fmcoh{\E}_\C\left(H^{p,q}(X)\right) \subseteq \bigoplus_{s - t = p - q} H^{s,t}(Y).
    \end{equation}
\end{proposition}