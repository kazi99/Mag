\section{K3 surfaces}

\subsection{Algebraic and complex}

K3 surfaces come in two flavours, algebraic and complex, essentially equivalent, but formally allowing us to effortlessly pass between the realms of algebraic and complex geometry. 

A \emph{variety over a the field of complex numbers $\C$} is a separated integral scheme of finite type over $\C$. We say that a variety $X$ over $\C$ is \emph{smooth}, if the stalks $\struct{X, x}$ are regular\footnote{A noetherian local ring $(A, \mathfrak{m})$ is \emph{regular}, if $\dim A = \dim_k \mathfrak{m}/\mathfrak{m}^2$, where $\dim A$ denotes the \emph{Krull dimension of $A$} and $k$ denotes the residue field $A/\mathfrak{m}$.} local rings for every $x \in X$. A variety $X$ over $\C$ is \emph{projective}, if the structure morphism $X \to \spec{\C}$ is projective, meaning there exists a closed immersion $X \hookrightarrow \PP^n_\C$ for some $n \in \N$. A variety of dimension two will be called a \emph{surface}. 

\begin{definition}
    A smooth and projective variety $X$ over $\C$ of dimension two is a \emph{K3 surface}, if 
    \[
        \omega_X \iso \struct{X} \quad \text{and} \quad \cohom{1}{X}{\struct{X}} = 0.
    \]
\end{definition}



\begin{definition}
    A compact connected complex manifold $X$ of dimension two is a \emph{K3 surface}, if it is simply connected and $\omega_X \iso \struct{X}$.
\end{definition}

\begin{example}
    Smooth quartic in $\PP^3$.
\end{example}

\subsection{Main invariants}

\subsubsection{Cohomology}
\subsubsection{Intersection pairing}

The transcendental lattice of $X$ will be denoted by $\transc{X}$.

\begin{proposition}
    \label{intersection pairing on K3 is even}
    Let $X$ be a K3 surface. Then the intersection pairing on $H^2(X, \Z)$ given by the $\smallsmile$-product is an even lattice. 
\end{proposition}

\begin{proposition}
    \label{Todd class of K3}
    Todd class of a K3
\end{proposition}

\begin{example}
    
\end{example}

\subsubsection{Hodge structure}

\begin{theorem}
    Suppose $X$ and $Y$ are smooth projective varieties over $\C$. Assume $X$ is a K3 surface and
    \[
        \derb{X} \natiso \derb{Y}.
    \]
    Then $Y$ is a K3 surface as well. 
\end{theorem}

\begin{proof}
    By Theorem \ref{Db detects dimension and triviality of canonical bundle} we already know $Y$ is a surface with a trivial canonical bundle. It is left to show that $\cohom{1}{Y}{\struct{Y}} = 0$. By Proposition \ref{Hodge lattice, fm transform interaction} we see that 
    \[
        h^{0,1}(X) + h^{1,2}(X) = h^{0,1}(Y) + h^{1,2}(Y). 
    \]
    However $h^{0,1}(X) = 0$ and $h^{1,2}(X) = h^{2,1}(X) = \dim \cohom{1}{X}{\omega_X} = \dim \cohom{1}{X}{\struct{X}} = 0$. Thus $h^{0,1}(Y) = 0$, proving that $Y$ is a K3 surface.
\end{proof}

\subsection{Two important theorems}

\begin{theorem}[Torelli]
    \label{Classical Torelli theorem}
    Let $X$ and $Y$ be K3 surfaces over $\C$. Then $X$ and $Y$ are isomorphic if and only if there exists a Hodge isometry
    \[
        f\colon \cohom{2}{X}{\Z} \to \cohom{2}{Y}{\Z}.
    \]
\end{theorem}

\begin{remark}
    For a topological space $X$ we will distinguish between the collection of all cohomology groups, considered concisely as a graded object $\cohom{\bullet}{X}{\Z} = \bigoplus_{i\in \Z} \cohom{i}{X}{\Z}$, and its (graded) cohomology ring, which we will denote by $\cohom{*}{X}{\Z}$.
\end{remark}