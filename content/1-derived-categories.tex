\section{Derived categories}

The first goal of this section is to construct the derived category of an abelian category $\A$ and equip it with a triangulated structure. Our arguments, although specialized to the homotopy category $\\kcat{\A}$ and the class of quasi-isomorphisms, do not differ tremendously from the general theory of localization of categories. 
% Our arguments will try to be minimal and will not rely on the already established theory of localization of categories. Our approach might at times seem less elegant, but we include the exposition for completness nonetheless.
A more comprehensive and formal treatment of this topic is laid out in \cite[Ch.~7]{kashiwara2006categories} or \cite{milicic-dercat}.
The second part covers the construction of derived functors. We see how the established framework of derived categories nicely lends itself for the definition of derived functors and we also relate them back to the classical higher derived functors. We close the section by establishing a correspondence between the hom-sets of $\derplus{\A}$ and certain $\operatorname{Ext}$-modules.
% more general framework realtes back to the classical higher derived functors. 

\subsection{Derived category of an abelian category}

In algebraic geometry cohomology of a geometric object, like a scheme or a variety, $X$ with respect to some coherent sheaf $\F$ plays a very important role. One way of computing $H^i(X, \F)$, which we shall also feature in section \ref{Derived categories in geometry}, involves the following. Instead of intrinsically \info{intrinsically might not be the best word as resolutions in some sense do represent a sheaf intrinsically in the sense of generators and relations (Hilbert syzygy thm)...}
studying the sheaf $\F$, we represent it with a so called \emph{resolution}, which consists of a complex of sheaves $F^\bullet$, built up from sheaves $F^i$, for $i \in \Z$, belonging to some class of sheaves, which is well behaved under cohomology, and a quasi-isomorphism of the form $F^\bullet \to \F$ or $\F \to F^\bullet$. 
% After observing that using two different resolutions of $\F$ 
After noting that the sheaf $\F$ can be seen as a complex concentrated in degree $0$ and observing that replacing a resolution of $\F$ with another one
results in computing isomorphic cohomology groups, we are motivated  
not to distinguish the sheaf $\F$ from its resolutions in the ambient (homotopy) category of complexes any longer. 
% identify the sheaf $\F$ with the collection of all its resolutions. 
Taking a step back, we would like to modify the homotopy category $\kcat{\coherent{X}}$ in such a way that $\F$ is identified with all its resolutions, or in other words, we want all the quasi-isomorphisms of $\kcat{\coherent{X}}$ to turn into isomorphisms in a ``universal way''. We will take the latter to be our inspiration for the definition of a derived category $\dercat{\A}$ of a general abelian category $\A$. This is stated more formally in the form of the ensuing universal property.

% assigning the derived category $\dercat{\A}$ to a general abelian category $\A$. 

% is to consider a certain complex built from objects all originating from some special class, in a way representing $\F$, called its \emph{resolution}. In a way we would like to identify $\F$ with its resolution and the correct relation, which acheves this, is via quasi-isomorphisms. Inherently, computing the cohomology from said complex losses some information 

% One of the most important invariants of a geometric object $X$ in algebraic geometry is its cohomology $H^i(X, \F)$ with respect to some (coherent) sheaf $\F$. 


% The central idea in this part of algebraic geometry is 

% Our goal in general is then fairly simple, modify the homotopy category $\\kcat{\A}$ in such a way that all the quasi-isomorphisms become isomorphisms in a ``universal way''. This is stated more formally in the form of the ensuing universal property.

\begin{definition}
    Let $\A$ be an abelian category and $\\kcat{\A}$ its homotopy category. A triangulated category $\dercat{\A}$ together with a triangulated functor $Q\colon \\kcat{\A} \to \dercat{\A}$ is the \emph{derived category of $\A$}, if it satisfies:
    \begin{enumerate}[(i)]
        \item For every quasi-isomorphism $s$ in $\\kcat{\A}$, $Q(s)$ is an isomorphism in $\dercat{\A}$.
        \item For any triangulated category $\D$ and any functor $F\colon \\kcat{\A} \to \D$, sending quasi-isomorphisms $s$ in $\\kcat{\A}$ to isomorphisms $F(s)$ in $\D$,
        % for which $F(s)$ is an isomorphism in $\D$ for all quasi-isomorphisms $s$ in $\kcat{\A}$, 
        there exists a triangulated functor $F_0 \colon \dercat{\A} \to \D$, which is unique up to a unique natural isomorphism, such that $F \natiso F_0 \circ Q$.
        % , and moreover $F_0$ is required to be unique up to a unique natural isomorphism.
        In other words, the diagram below commutes up to natural isomorphism. 
        \[\begin{tikzcd}
            % https://q.uiver.app/#q=WzAsMyxbMCwwLCJLKFxcQSkiXSxbMCwxLCJEKFxcQSkiXSxbMSwwLCJcXEQiXSxbMCwyLCJGIl0sWzAsMSwiUSIsMl0sWzEsMiwiRl9cXEEiLDIseyJzdHlsZSI6eyJib2R5Ijp7Im5hbWUiOiJkYXNoZWQifX19XV0=
            {\\kcat{\A}} & \D \\
            {\dercat{\A}}
            \arrow["F", from=1-1, to=1-2]
            \arrow["Q"', from=1-1, to=2-1]
            \arrow["{F_0}"', dashed, from=2-1, to=1-2]
        \end{tikzcd}\]
    \end{enumerate}
\end{definition}

\begin{remark}
    We recognise this definition as a special case of localization of categories \cite[Def.~7.1.1.]{kashiwara2006categories}. In particular it defines $\dercat{\A}$ to be the \emph{localization of a triangulated category} $\\kcat{\A}$ by the family of all quasi-isomorphisms in $\\kcat{\A}$.
\end{remark}

A naive way of constructing $\dercat{\A}$ out of $\\kcat{\A}$ would be to artificially add the inverses to all the quasi-isomorphisms in $\\kcat{\A}$ and then impose the correct collection of relations on the newly constructed class of morphisms. As this can quickly lead us to some set theoretic problems, we will construct a specific model, which acheves this, instead. Our construction is a priori not going to result in a locally small\footnote{A category $\mathcal C$ is called \emph{locally small} if for all objects $X$ and $Y$ of $\mathcal C$ the hom-sets $\Hom_\mathcal{C}(X, Y)$ are actual sets.} category, but as we shall soon see in practice all the categories we will be concerned with will be locally small. 
% still going to reach outside the scope of sets, but as we will se in practise 
% this can quickly become very tricky, we will construct a specific model instead.
% This will be achieved by constructing a specific model. 
% Instead we will construct a specific model, which achieves this. 

To start, we first need a technical lemma resembling the Ore condition from non-commutative algebra.

\begin{lemma}
    \label{Ore condition}
    Let $f \colon A^\bullet \to B^\bullet$ and $s\colon C^\bullet \to B^\bullet$ belong to the homotopy category $\\kcat{\A}$, with $s$ being a quasi-isomorphism. Then there exists a quasi-isomorphism $u\colon C_0^\bullet \to A^\bullet$ and a morphism $g\colon C_0^\bullet \to C^\bullet$, such that the diagram below commutes in $\kcat{\A}$.
    \[\begin{tikzcd}
        % https://q.uiver.app/#q=WzAsNCxbMCwyLCJBXlxcYnVsbGV0Il0sWzIsMiwiQl5cXGJ1bGxldCJdLFsyLDAsIkNeXFxidWxsZXQiXSxbMCwwLCJDXlxcYnVsbGV0XzAiXSxbMCwxLCJmIl0sWzIsMSwicyJdLFszLDIsImciLDAseyJzdHlsZSI6eyJib2R5Ijp7Im5hbWUiOiJkYXNoZWQifX19XSxbMywwLCJ1IiwyLHsic3R5bGUiOnsiYm9keSI6eyJuYW1lIjoiZGFzaGVkIn19fV0sWzMsMCwiXFxzaW0iLDAseyJzdHlsZSI6eyJib2R5Ijp7Im5hbWUiOiJub25lIn0sImhlYWQiOnsibmFtZSI6Im5vbmUifX19XSxbMiwxLCJcXHNpbSIsMix7InN0eWxlIjp7ImJvZHkiOnsibmFtZSI6Im5vbmUifSwiaGVhZCI6eyJuYW1lIjoibm9uZSJ9fX1dXQ==
        {C^\bullet_0} && {C^\bullet} \\
        \\
        {A^\bullet} && {B^\bullet}
        \arrow["g", dashed, from=1-1, to=1-3]
        \arrow["u", dashed, from=1-1, to=3-1]
        \arrow["\sim"', draw=none, from=1-1, to=3-1]
        \arrow["s", from=1-3, to=3-3]
        \arrow["\sim"', draw=none, from=1-3, to=3-3]
        \arrow["f", from=3-1, to=3-3]
    \end{tikzcd}\]
\end{lemma}

\begin{proof}
    \info{add proof}
\end{proof}

Equipped with the preceding lemma, we are now in a position to construct the derived category $\dercat{\A}$ of an abelian category $\A$. The derived category $\dercat{\A}$ will consist of
\[
    \textsl{Objects of $\dercat{\A}$:} \quad \text{chain complexes in $\A$},
\]
\ie the class of objects of $\\kcat{\A}$ or $\com(\A)$, and a class of morphisms, which is a bit more intricate to define. 
% Its class of objects is formed by chain complexes in $\A$ \ie equals the class of objects of $\kcat{\A}$ and $\com(\A)$, but the definition of morphisms in this case is a bit more intricate. 
For fixed complexes $A^\bullet$ and $B^\bullet$ we define the hom-set\footnote{What will be defined here is a priori not necessarily a set, but a class, so $\dercat{\A}$, defined in this section, is not a category in the usual sense. We will however prove that in specific cases some variants of the derived category, especially concrete ones used later on, will from categories in the usual sense.} $\Hom_{\dercat{\A}}(A^\bullet, B^\bullet)$ in the following way.

\vspace{0.3 cm}

\noindent
\textsc{Hom-sets.}
A \emph{left roof spanned on $A^\bullet$ and $B^\bullet$} is a pair of morphisms $s \colon A^\bullet \to C^\bullet$ and $f\colon C^\bullet \to B^\bullet$ in the homotopy category $\kcat{\A}$, where $s$ is a quasi-isomorphism. This roof is depicted in the following diagram
\begin{equation}
    \label{eq: left roof}
    \begin{tikzcd}[row sep = small, column sep = small]
    % https://q.uiver.app/#q=WzAsMyxbMCwxLCJBIl0sWzIsMCwiQyJdLFs0LDEsIkIiXSxbMSwyLCJmIiwyXSxbMSwwLCJzIl0sWzEsMCwiXFxzaW0iLDJdXQ==
	&& C^\bullet \\
	A^\bullet &&&& B^\bullet
	\arrow["s", from=1-3, to=2-1]
	\arrow["\sim"', from=1-3, to=2-1]
	\arrow["f"', from=1-3, to=2-5]
    \end{tikzcd}
\end{equation}
or denoted by $(s,f)$.
Dually, one also obtains the notion of a \emph{right roof spanned on $A^\bullet$ and $B^\bullet$}, which is a pair of morphisms $g\colon A^\bullet \to C^\bullet$ and $u\colon B^\bullet \to C^\bullet$, where $u$ is a quasi-isomorphism, and is depicted below.
\[\begin{tikzcd}[row sep = small, column sep = small]
    % https://q.uiver.app/#q=WzAsMyxbMCwxLCJBIl0sWzIsMCwiQyJdLFs0LDEsIkIiXSxbMSwyLCJmIiwyXSxbMSwwLCJzIl0sWzEsMCwiXFxzaW0iLDJdXQ==
	&& C^\bullet \\
	A^\bullet &&&& B^\bullet
	\arrow["g"', to=1-3, from=2-1]
	\arrow["\sim"', to=1-3, from=2-5]
	\arrow["u", to=1-3, from=2-5]
\end{tikzcd}\] 
Our construction of $\Hom_{\dercat{\A}}(A^\bullet, B^\bullet)$ will be based on left roofs, for nothing is gained or lost by picking either one of the two. Both work just as well and are in fact equvalent. Despite our arbitrary choice, it is still benefitial to consider both, as we will sometimes switch between the two whenever convenient. 

\begin{definition}
    Two left roofs $A^\bullet \xleftarrow{\ s_0 \ } C_0^\bullet \xrightarrow{\ f_0 \ } B^\bullet$ and $A^\bullet \xleftarrow{\ s_1 \ } C_1^\bullet \xrightarrow{\ f_1 \ } B^\bullet$ are defined to be \emph{equivalent}, if there exists a quasi-isomorphism $u\colon C^\bullet \to C_0^\bullet$ and a morphism ${g\colon C^\bullet \to C_1^\bullet}$ in $\kcat{\A}$, for which the diagram below commutes (in $\kcat{\A}$). 
    \begin{equation}
        \label{eq: equivalence of left roofs}
    \begin{tikzcd}[row sep = 1.5em, column sep = 2em]
        % https://q.uiver.app/#q=WzAsNSxbMCwyLCJBIl0sWzEsMSwiQ18wIl0sWzQsMiwiQiJdLFszLDEsIkNfMSJdLFsyLDAsIkMiXSxbMSwyXSxbMywwLCJcXHNpbSIsMCx7ImxhYmVsX3Bvc2l0aW9uIjo2MH1dLFszLDJdLFs0LDEsIlxcc2ltIiwyXSxbNCwzLCJnIiwyXSxbMSwwLCJcXHNpbSIsMl0sWzQsMSwidSJdXQ==
        && C^\bullet \\
        & {C_0^\bullet} && {C_1^\bullet} \\
        A^\bullet &&&& B^\bullet
        \arrow["\sim"', from=1-3, to=2-2]
        \arrow["u", from=1-3, to=2-2]
        \arrow["g"', from=1-3, to=2-4]
        \arrow["\sim"', from=2-2, to=3-1]
        \arrow[from=2-2, to=3-5]
        \arrow["\sim"{pos=0.6}, from=2-4, to=3-1]
        \arrow[from=2-4, to=3-5]
    \end{tikzcd}
    \end{equation}
We denote this relation by $\equiv$.
\end{definition}

Note that since $C^\bullet \to C_0^\bullet \to A^\bullet$ is a quasi-isomorphism, the same is true for the composition ${C^\bullet \to C_1^\bullet \to A^\bullet}$, concluding that $g\colon C^\bullet \to C^\bullet_1$ is a quasi-isomorphism. Also observe that in the diagram \eqref{eq: equivalence of left roofs} we may find a new left roof, namely 
\begin{equation*}
    \begin{tikzcd}[row sep = small, column sep = small]
    % https://q.uiver.app/#q=WzAsMyxbMCwxLCJBIl0sWzIsMCwiQyJdLFs0LDEsIkIiXSxbMSwyLCJmIiwyXSxbMSwwLCJzIl0sWzEsMCwiXFxzaW0iLDJdXQ==
	&& C^\bullet \\
	A^\bullet &&&& B^\bullet,
	\arrow["s_0 \circ u"', from=1-3, to=2-1]
	\arrow["\sim", from=1-3, to=2-1]
	\arrow["f_1 \circ g", from=1-3, to=2-5]
    \end{tikzcd}
\end{equation*}
which is also equivalent to the two roofs we started with $(s_0, f_0)$ and $(s_1, f_1)$. 

\begin{lemma}
    The equivalence of left roofs on $A^\bullet$ and $B^\bullet$ is an equivalence relation.
\end{lemma}

\begin{proof}
    The relation is clearly reflexive. We take both $u$ and $g$ to be $\id{C^\bullet}$. By the note above it is also symmetric, for $g$ is a quasi-isomorphism. It remains to show transitivity. Suppose left roofs $A^\bullet \longleftarrow C_0^\bullet \longrightarrow B^\bullet$ and $A^\bullet \longleftarrow C_1^\bullet \longrightarrow B^\bullet$ are equivalent and left roofs $A^\bullet \longleftarrow C_1^\bullet \longrightarrow B^\bullet$ and $A^\bullet \longleftarrow C_2^\bullet \longrightarrow B^\bullet$ are equivalent. This is witnessed by the solid diagram below.
    \[\begin{tikzcd}
        % https://q.uiver.app/#q=WzAsOCxbMCwzLCJBIl0sWzEsMiwiQ18wIl0sWzYsMywiQiJdLFs1LDIsIkNfMiJdLFszLDIsIkNfMSJdLFsyLDEsIkRfMCJdLFs0LDEsIkRfMSJdLFszLDAsIkMiXSxbMSwyXSxbMywwLCJcXHNpbSIsMCx7ImxhYmVsX3Bvc2l0aW9uIjo2MH1dLFszLDJdLFsxLDAsIlxcc2ltIiwyXSxbNSwxLCJcXHNpbSIsMix7ImNvbG91ciI6WzI0MCw2MCw2MF19LFsyNDAsNjAsNjAsMV1dLFs1LDQsIiIsMCx7ImNvbG91ciI6WzI0MCw2MCw2MF19XSxbNCwwLCJcXHNpbSIsMix7ImxhYmVsX3Bvc2l0aW9uIjo3MH1dLFs0LDJdLFs2LDMsIiIsMCx7ImNvbG91ciI6WzMwMCw2MCw2MF19XSxbNiw0LCJcXHNpbSIsMix7ImNvbG91ciI6WzMwMCw2MCw2MF19LFszMDAsNjAsNjAsMV1dLFs3LDUsIlxcc2ltIiwyLHsic3R5bGUiOnsiYm9keSI6eyJuYW1lIjoiZGFzaGVkIn19fV0sWzcsNiwiIiwwLHsic3R5bGUiOnsiYm9keSI6eyJuYW1lIjoiZGFzaGVkIn19fV1d
        &&& C^\bullet \\
        && {D_0^\bullet} && {D_1^\bullet} \\
        & {C_0^\bullet} && {C_1^\bullet} && {C_2^\bullet} \\
        A^\bullet &&&&&& B^\bullet
        \arrow["\sim"', dashed, from=1-4, to=2-3]
        \arrow[dashed, from=1-4, to=2-5]
        \arrow["\sim"', color={rgb,255:red,92;green,92;blue,214}, from=2-3, to=3-2]
        \arrow[color={rgb,255:red,92;green,92;blue,214}, from=2-3, to=3-4]
        \arrow["\sim"', color={rgb,255:red,214;green,92;blue,214}, from=2-5, to=3-4]
        \arrow[color={rgb,255:red,214;green,92;blue,214}, from=2-5, to=3-6]
        \arrow["\sim"', from=3-2, to=4-1]
        \arrow[curve={height=12pt}, from=3-2, to=4-7]
        \arrow["\sim"'{pos=0.2}, from=3-4, to=4-1]
        \arrow[from=3-4, to=4-7]
        \arrow["\sim"{pos=0.6}, curve={height=-12pt}, from=3-6, to=4-1]
        \arrow[from=3-6, to=4-7]
    \end{tikzcd}\]
    \info{I don't like how this diagram looks...}
    By lemma \ref{Ore condition} this diagram may be completed with the dashed arrows at the top, proving that $A^\bullet \leftarrow C_0^\bullet \rightarrow B^\bullet$ and $A^\bullet \leftarrow C_2^\bullet \rightarrow B^\bullet$ are equivalent.
\end{proof}

For a left roof \eqref{eq: left roof} we let $\left[s \setminus f\right]$ or $\left[A^\bullet \xleftarrow{\ s\ } C^\bullet \xrightarrow{\ f \ } B^\bullet\right]$ denote its equivalence class under $\equiv$. 

We then define $\Hom_{\dercat{\A}}(A^\bullet, B^\bullet)$ to be the class of left roofs spanned by $A^\bullet$ and $B^\bullet$, quotiented by the relation $\equiv$. That is
% \begin{equation*}
%     \Hom_{\dercat{\A}}(A^\bullet, B^\bullet) := 
%     \left\{ (s, f) \in \coprod_{C^\bullet \in \Ob(\kcat{\A})} \Hom_{\kcat{\A}}(C^\bullet, A^\bullet) \times \Hom_{\kcat{\A}}(C^\bullet, B^\bullet) \ \middle\vert \ \text{$s$ quasi-isomorphism} \right\}
% \end{equation*}
\[
\Hom_{\dercat{\A}}(A^\bullet, B^\bullet) := \\ 
\left\{
\left[
\begin{array}{l c r}
& C^\bullet & \\
\quad \swarrow & & \searrow \quad\\
A^\bullet & & B^\bullet
\end{array}
\right]_\equiv
\, \middle| \,
\begin{array}{l}
C^\bullet \in \Ob K(A), \\
f \in \Hom_{K(A)}(C^\bullet, B^\bullet), \\
s \in \Hom_{K(A)}(C^\bullet, A^\bullet) \text{ quasi-iso.}
\end{array}
\right\}.
\]

\vspace{0,3 cm}

\noindent
\textsc{Composition.} Next we define the compositon operations
\[
    \circ \colon \Hom_{\dercat{\A}}(A_0^\bullet, A_1^\bullet) \times \Hom_{\dercat{\A}}(A_1^\bullet, A_2^\bullet) \to \Hom_{\dercat{\A}}(A_0^\bullet, A_2^\bullet)
\]
for all objects $A_0^\bullet$, $A_1^\bullet$ and $A_2^\bullet$ of $\dercat{\A}$. Let $\phi_0\colon A_0^\bullet \to A_1^\bullet$ and $\phi_1\colon A_1^\bullet \to A_2^\bullet$ be a pair of composable morphisms in $\dercat{\A}$. Next, pick their respective left roof representatives, $A_0^\bullet \xleftarrow{\ s_0\ } C_0^\bullet \xrightarrow{\ f_0 \ } A_1^\bullet$ and $A_1^\bullet \xleftarrow{\ s_1\ } C_1^\bullet \xrightarrow{\ f_1 \ } A_2^\bullet$, and concatenate them according to the solid zig-zag diagram below. 
\[\begin{tikzcd}[row sep = small, column sep = small]
    % https://q.uiver.app/#q=WzAsNixbMCwyLCJBXzAiXSxbMiwxLCJDXzAiXSxbNCwyLCJBXzEiXSxbOCwyLCJBXzIiXSxbNiwxLCJDXzEiXSxbNCwwLCJDIl0sWzEsMiwiZl8wIiwyXSxbMSwwLCJzXzAiXSxbMSwwLCJcXHNpbSIsMix7Im9mZnNldCI6LTEsInN0eWxlIjp7ImJvZHkiOnsibmFtZSI6Im5vbmUifSwiaGVhZCI6eyJuYW1lIjoibm9uZSJ9fX1dLFs0LDIsInNfMSJdLFs0LDMsImZfMSIsMl0sWzUsMSwidSIsMCx7InN0eWxlIjp7ImJvZHkiOnsibmFtZSI6ImRhc2hlZCJ9fX1dLFs1LDQsImciLDIseyJzdHlsZSI6eyJib2R5Ijp7Im5hbWUiOiJkYXNoZWQifX19XSxbNCwyLCJcXHNpbSIsMix7Im9mZnNldCI6LTEsInN0eWxlIjp7ImJvZHkiOnsibmFtZSI6Im5vbmUifSwiaGVhZCI6eyJuYW1lIjoibm9uZSJ9fX1dLFs1LDEsIlxcc2ltIiwyLHsib2Zmc2V0IjotMSwic3R5bGUiOnsiYm9keSI6eyJuYW1lIjoibm9uZSJ9LCJoZWFkIjp7Im5hbWUiOiJub25lIn19fV1d
	&&&& C^\bullet \\
	&& {C_0^\bullet} &&&& {C_1^\bullet} \\
	{A_0^\bullet} &&&& {A_1^\bullet} &&&& {A_2^\bullet}
	\arrow["u", dashed, from=1-5, to=2-3]
	\arrow["\sim"', shift left, draw=none, from=1-5, to=2-3]
	\arrow["g"', dashed, from=1-5, to=2-7]
	\arrow["{s_0}", from=2-3, to=3-1]
	\arrow["\sim"', shift left, draw=none, from=2-3, to=3-1]
	\arrow["{f_0}"', from=2-3, to=3-5]
	\arrow["{s_1}", from=2-7, to=3-5]
	\arrow["\sim"', shift left, draw=none, from=2-7, to=3-5]
	\arrow["{f_1}"', from=2-7, to=3-9]
\end{tikzcd}\]
By lemma \ref{Ore condition} there are morphisms $u\colon C^\bullet \to C_0^\bullet$ and $g\colon C^\bullet \to C_1^\bullet$, depicted with dashed arrows, completing the diagram in $\kcat{\A}$. In this way we obtain a left roof, formed by a quasi-isomorphism $s_0 \circ u$ and a morphism $f_1 \circ g$, the equivalence class of which we define to be the composition 
\[
    \phi_1 \circ \phi_0 :=
    \left[A_0^\bullet \xleftarrow{s_0 \circ u} C^\bullet \xrightarrow{f_1 \circ g} A_2^\bullet\right].
\]
It can be shown that this is a well defined composition that is also associative. We leave out the proof, because it is routine, but refer the reader to a very detailed account by Miličić \cite[Ch.~1.3]{milicic-dercat}. 

\vspace{0,3 cm}

\noindent
\textsc{Identities.}
The identity morphism on an object $A^\bullet$ of $\dercat{\A}$ is defined to be the equivalence class of $A^\bullet \xleftarrow{\id{A^\bullet}} A^\bullet \xrightarrow{\id{A^\bullet}} A^\bullet$. It is not difficult to check that these classes play the role of identity morphisms in $\dercat{\A}$. 

\vspace{0,3 cm}

\noindent
\textsc{Functor $Q\colon \\kcat{\A} \to \dercat{\A}$.} The \emph{localization} functor $Q\colon \\kcat{\A} \to \dercat{\A}$ is an identity on objects functor with the action on morphisms defined by the assignment 
\[\begin{tikzcd}[row sep = 0.5em, column sep = 0.75em]
    % https://q.uiver.app/#q=WzAsNCxbMSwwLCJcXEhvbV97XFxkZXJjYXR7XFxBfX0oQV5cXGJ1bGxldCwgSV5cXGJ1bGxldCkiXSxbMCwwLCJcXEhvbV97fShBXlxcYnVsbGV0LCBJXlxcYnVsbGV0KSJdLFswLDEsIlxcbGVmdCggZlxcY29sb24gQV5cXGJ1bGxldCBcXHRvIEleXFxidWxsZXRcXHJpZ2h0KSJdLFsxLDEsIlxcbGVmdFtBXlxcYnVsbGV0IFxceGxlZnRhcnJvd3sgXFwgXFxpZHtBXlxcYnVsbGV0fSB9IEFeXFxidWxsZXQgXFx4cmlnaHRhcnJvd3sgXFwgZiBcXCB9IEleXFxidWxsZXRcXHJpZ2h0XSJdLFsyLDMsIiIsMix7InN0eWxlIjp7InRhaWwiOnsibmFtZSI6Im1hcHMgdG8ifX19XSxbMSwwXV0=
	{\Hom_{\kcat{\A}}(A^\bullet, B^\bullet)} & {\Hom_{\dercat{\A}}(A^\bullet, B^\bullet)} \\
	{\left( f\colon A^\bullet \to B^\bullet\right)} & {\left[A^\bullet \xleftarrow{ \ \id{A^\bullet} } A^\bullet \xrightarrow{ \ f \ } B^\bullet\right].}
	\arrow["{Q}",from=1-1, to=1-2]
	\arrow[maps to, from=2-1, to=2-2]
\end{tikzcd}\]


\vspace{0,3 cm}

\noindent
\textsc{$k$-linear structure.}

% We first define the biproduct on $\dercat{\A}$.
\info{finish subsection}

\vspace{0,3 cm}

\noindent
\textsc{Triangulated structure.} The shift functor $T \colon \dercat{\A} \to \dercat{\A}$ is defined 



\begin{remark}
    All that has been defined and established in this section with left roofs can analogously also be done with right roofs.
\end{remark}

As with the homotopy category of complexes $\kcat{\A}$, we also have the following bounded versions of the derived category $\dercat{\A}$, appearing as full triangulated subcategories of the unbounded variant $\dercat{\A}$ by proposition \ref{triangulated subcategory proposition}.

\begin{center}
    % \begin{tabular}{c l}
    %     $\derplus{\A}$ & spanned on complexes in $\A$ bounded below. \\
    %     $\derminus{\A}$ & spanned on complexes in $\A$ bounded above. \\
    %     $\derb{\A}$ & spanned on bounded complexes in $\A$. \\    
    % \end{tabular}
    \begin{tabular}{c l}
        $\derplus{\A}$ & spanned on objects of $\kplus{\A}$. \\
        $\derminus{\A}$ & spanned on objects of $\kminus{\A}$. \\
        $\derb{\A}$ & spanned on objects of $\kb{\A}$. \\    
    \end{tabular}
\end{center}

% ============================================================== %

\subsubsection{Derived category of an abelian category with enough injectives}

After possibly working with proper classes when constructing the derived category, this section will once again place us back on familiar grounds of set theory. Aside from these concerns the establishing result of this section will have a very important practical application -- construction of derived functors. Under an assumption on the abelian category $\A$, we will establish an equivalence of $\derplus{\A}$ with the homotopy category of a full additive subcategory of $\A$, spanned on \emph{injective objects}, which we define presently. 

\begin{definition}
    An object $I$ of an abelian category $\A$ is called \emph{injective}\footnote{Dually, an object $P$ of $\A$ is \emph{projective}, if the functor $\Hom_\A(P, -) \colon \A \to \mod{k}$ is exact, and $\A$ is said to \emph{have enough projectives}, if every object $A$ of $\A$ is a quotient of some projective object \ie there is an epimorphism $P \twoheadrightarrow A$ for some projective object $P$.
}, if the functor $\Hom_\A(-, I)\colon \A^\op \to \mod{k}$ is exact. Equivalently whenever for any monomorphism ${A \hookrightarrow B}$ and morphism $A \to I$ there is a morphism $B \to I$, for which the following diagram commutes.
    \[\begin{tikzcd}[column sep = 2em]
        % https://q.uiver.app/#q=WzAsNCxbMSwxLCJBIl0sWzIsMSwiQiJdLFsxLDAsIkkiXSxbMCwxLCIwIl0sWzMsMF0sWzAsMSwiIiwwLHsic3R5bGUiOnsidGFpbCI6eyJuYW1lIjoiaG9vayIsInNpZGUiOiJ0b3AifX19XSxbMCwyXSxbMSwyLCIiLDEseyJzdHlsZSI6eyJib2R5Ijp7Im5hbWUiOiJkYXNoZWQifX19XV0=
        & I \\
        0 & A & B
        \arrow[from=2-1, to=2-2]
        \arrow[from=2-2, to=1-2]
        \arrow[hook, from=2-2, to=2-3]
        \arrow[dashed, from=2-3, to=1-2]
    \end{tikzcd}\]
\end{definition}
A category $\A$ is said to \emph{have enough injectives} if every object $A$ of $\A$ embeds into an injective object \ie there is a monomorphism $A \hookrightarrow I$ for some injective object $I$. 

\begin{remark}
    A simple verification shows, that the full subcategory of an abelian or $k$-linear category $\A$, spanend on all injective objects of $\A$ is a $k$-linear category, as the zero object $0$ is injective and the biproduct of two injective objects is again injective. We denote this category by $\J$.
\end{remark}

We will use injectives to 

We call a quasi-isomorphism $f \colon A^\bullet \to I^\bullet$ an \emph{injective resolution} of the complex $A^\bullet$. Later we will also see that this injective resolution is unique up to homotopy \ie any two injective resolutions of $A^\bullet$ are homotopically equivalent. Throughout this section differenitials of $I^\bullet$ will be denoted with $\delta^i\colon I^i \to I^{i+1}$.

\begin{proposition}
    \label{injective resolution}
    % Suppose $\A$ contains enough injectives. Then every $A^\bullet$ in $\kplus{\A}$ has an injective resolution.
    Suppose $\A$ contains enough injectives. Then for every $A^\bullet$ in $\kplus{\A}$ there is a quasi-isomorphism $f^\bullet \colon A^\bullet \to I^\bullet$, where $I^\bullet \in \Ob \kplus{\A}$ is a complex of injectives.
\end{proposition}

\begin{proof}
    We will inductively construct a complex $I^\bullet$ in $\kplus{\A}$, built up from injectives, and a quasi-isomorphism $f^\bullet \colon A^\bullet \to I^\bullet$. For simplicity assume $A^i = 0$ for $i < 0$. 
    
    The base case is trivial. Define $I^i = 0$ and $f^i = 0$ for $i < 0$ and take $f^0\colon A^0 \to I^0$ to be the morphism obtained from the assumption about $\A$ having enough injectives.
    \[\begin{tikzcd}[column sep = 2em]
        % https://q.uiver.app/#q=WzAsOCxbMSwwLCIwIl0sWzIsMCwiQV4wIl0sWzMsMCwiQV4xIl0sWzQsMCwiXFxjZG90cyJdLFswLDAsIlxcY2RvdHMiXSxbMiwxLCJJXjAiXSxbMSwxLCIwIl0sWzAsMSwiXFxjZG90cyJdLFs0LDBdLFswLDFdLFsxLDJdLFsyLDNdLFsxLDVdLFs3LDZdLFs2LDVdXQ==
        \cdots & 0 & {A^0} & {A^1} & \cdots \\
        \cdots & 0 & {I^0}
        \arrow[from=1-1, to=1-2]
        \arrow[from=1-2, to=1-3]
        \arrow[from=1-3, to=1-4]
        \arrow[from=1-3, to=2-3]
        \arrow[from=1-4, to=1-5]
        \arrow[from=2-1, to=2-2]
        \arrow[from=2-2, to=2-3]
    \end{tikzcd}\]

    For the induction step suppose we have constructed the complex $I^\bullet$ and the morphism $f^\bullet$ up to index $n$. 
    \[\begin{tikzcd}[column sep = 2em]
        % https://q.uiver.app/#q=WzAsOCxbMSwwLCJBXntuLTF9Il0sWzIsMCwiQV5uIl0sWzMsMCwiQV57bisxfSJdLFs0LDAsIlxcY2RvdHMiXSxbMCwwLCJcXGNkb3RzIl0sWzIsMSwiSV5uIl0sWzEsMSwiSV57bi0xfSJdLFswLDEsIlxcY2RvdHMiXSxbNCwwXSxbMCwxLCJkXntuLTF9Il0sWzEsMiwiZF5uIl0sWzIsM10sWzEsNSwiZl57bn0iLDJdLFs3LDZdLFs2LDUsIlxcZGVsdGFee24tMX0iXSxbMCw2LCJmXntuLTF9IiwyXV0=
        \cdots & {A^{n-1}} & {A^n} & {A^{n+1}} & \cdots \\
        \cdots & {I^{n-1}} & {I^n}
        \arrow[from=1-1, to=1-2]
        \arrow["{d^{n-1}}", from=1-2, to=1-3]
        \arrow["{f^{n-1}}"', from=1-2, to=2-2]
        \arrow["{d^n}", from=1-3, to=1-4]
        \arrow["{f^{n}}"', from=1-3, to=2-3]
        \arrow[from=1-4, to=1-5]
        \arrow[from=2-1, to=2-2]
        \arrow["{\delta^{n-1}}", from=2-2, to=2-3]
    \end{tikzcd}\]
    Consider the cokernel of $d^{n-1}_{C(f)}$
    \[
        A^n \oplus I^{n-1} \xrightarrow{\ d^{n-1}_{C(f)} \ } A^{n+1} \oplus I^n \xrightarrow{\ e \ } \coker d^{n-1}_{C(f)} 
    \]
    and denote with $j \colon \coker d^{n-1}_{C(f)} \hookrightarrow I^{n+1}$ an embedding to an injective object $I^{n+1}$.
    %  we get from $\A$ having enough injectives. 
    After precomposing the morphism
    \[
        A^{n+1} \oplus I^n \xrightarrow{\ e \ } \coker d^{n-1}_{C(f)} \xrightarrow{ \ j \ } I^{n+1}
    \]
    with canonical embeddings of $A^{n + 1}$ and $I^{n}$ into their biproduct $A^{n+1} \oplus I^n$, we obtain morphisms $\delta^{n}\colon I^n \to I^{n+1}$ and $f^{n+1}\colon A^{n+1} \to I^{n+1}$. As $j \circ e$ is a morphism fitting into the following coproduct diagram,
    \[\begin{tikzcd}[row sep = 2em]
        % https://q.uiver.app/#q=WzAsNCxbMSwxLCJBXntuKzF9IFxcb3BsdXMgSV5uIl0sWzAsMCwiQV57bisxfSJdLFsyLDAsIklee259Il0sWzEsMywiSV57bisxfSJdLFswLDMsImogXFxjaXJjIGUiLDJdLFsxLDMsImZee24rMX0iLDIseyJjdXJ2ZSI6Mn1dLFsyLDMsIlxcZGVsdGFebiIsMCx7ImN1cnZlIjotMn1dLFsxLDBdLFsyLDBdXQ==
        {A^{n+1}} && {I^{n}} \\
        & {A^{n+1} \oplus I^n} \\
        \\
        & {I^{n+1}}
        \arrow[from=1-1, to=2-2]
        \arrow["{f^{n+1}}"', curve={height=12pt}, from=1-1, to=4-2]
        \arrow[from=1-3, to=2-2]
        \arrow["{\delta^n}", curve={height=-12pt}, from=1-3, to=4-2]
        \arrow["{j \circ e}"', from=2-2, to=4-2]
    \end{tikzcd}\]
    we see that $j \circ e = \langle f^{n+1} , \delta^n \rangle$. Next, we see that $\delta^n\delta^{n-1} = 0$ and $\delta^n f^n = f^{n+1}d^n$ from the computation
    \[
        0 = j \circ e \circ d^{n-1}_{C(f)} = \langle f^{n+1},\ \delta^n \rangle \begin{pmatrix}
            -d^n & 0 \\ f^n & \delta^{n-1}
        \end{pmatrix} = \langle \delta^n f^n - f^{n+1}d^n , \ \delta^n\delta^{n-1}\rangle\\
    \]
    Thus far we have constructed a chain complex $I^\bullet$ of injective objects and a chain map ${f \colon A^\bullet \to I^\bullet}$. It remanis to be shown that $f$ is a quasi-isomorphism. Recall, that $f$ is a quasi-isomorphism if and only if its cone $C(f)^\bullet$ is acyclic. By remark \ref{versions of cohomology}, we known
    \[
        H^n(C(f)^\bullet) = \ker(\coker d^{n-1}_{C(f)} \xrightarrow{\ d \ } A^{n+2} \oplus I^{n+1}),
    \]
    where $d$ is obtained from the universal property of $\coker d^{n-1}_{C(f)}$, induced by $d^n_{C(f)}$. Thus it suffices to show, that $d$ is a monomorphism. Consider the following diagram, which we will show to commute. 
    \[\begin{tikzcd}[column sep = 2em]
        % https://q.uiver.app/#q=WzAsNSxbMCwwLCJBXm5cXG9wbHVzIElee24tMX0iXSxbMiwwLCJBXntuKzF9XFxvcGx1cyBJXntufSJdLFs0LDAsIkFee24rMn1cXG9wbHVzIElee24rMX0iXSxbMywxLCJjb2tlciBkXntuLTF9X3tDKGYpfSJdLFs1LDEsIklee24rMX0iXSxbMCwxLCJkXntuLTF9X3tDKGYpfSJdLFsxLDIsImRee259X3tDKGYpfSJdLFsxLDMsImUiXSxbMywyLCJkIiwwLHsic3R5bGUiOnsiYm9keSI6eyJuYW1lIjoiZGFzaGVkIn19fV0sWzIsNCwicCJdLFszLDQsImoiXV0=
        {A^n\oplus I^{n-1}} && {A^{n+1}\oplus I^{n}} && {A^{n+2}\oplus I^{n+1}} \\
        &&& {\coker d^{n-1}_{C(f)}} && {I^{n+1}}
        \arrow["{d^{n-1}_{C(f)}}", from=1-1, to=1-3]
        \arrow["{d^{n}_{C(f)}}", from=1-3, to=1-5]
        \arrow["e", two heads, from=1-3, to=2-4]
        \arrow["p", from=1-5, to=2-6]
        \arrow["d", dashed, from=2-4, to=1-5]
        \arrow["j", hook, from=2-4, to=2-6]
    \end{tikzcd}\]
    Once we show, that its right most triangle is commutative, we see that $d$ is monomorphic, because $j$ is monomorphic. To see that the triangle in question commutes, we compute
    \[
        p \circ d \circ e = p \circ d^n_{C(f)} = \langle f^{n+1}, \delta^n \rangle = j \circ e.
    \] 
    As $e$ is epic, we may cancel it on the right, to arrive at $p \circ d = j$, which ends the proof.
\end{proof}

\begin{remark}
    As is evident from the proof by close inspection we have only used two facts about the class of all injective objects of $\A$ -- that every object of $\A$ embeds into some injective object and that the class of injectives is closed under finite direct sums. This will later on be used in subsection \ref{Subsection on F-adapted classes} when constructing a right derived functor of a given functor $F$ in the presence of an $F$-adapted class.
\end{remark}

\begin{theorem}
    \label{D+ = K+ and injectives}
    Assume $\A$ contains enough injectives and let $\J \subseteq \A$ denote the full subcategory on injecitve objects of $\A$. Then the inclusion $\kplus{\J} \hookrightarrow \kplus{\A}$ induces an equivalence of triangulated categories
    \[
        \kplus{\J} \iso \derplus{\A}.
    \]
\end{theorem}

\begin{remark}
    Theorem \ref{D+ = K+ and injectives} in particular shows that $\derplus{\A}$ is a category in the usual sense, \ie all its hom-sets are in fact sets.  
    \info{maybe I replace "usual" with locally small everywhere this comes up.}
\end{remark}

\begin{lemma}
    \label{acyclic to injective is nulhomotopic}
    Let $A^\bullet$ be any acyclic complex in $\kplus{\A}$ and $I^\bullet$ a complex of injectives from $K^+(J)$. Then
    \[
        \hom_{\kplus{\A}}(A^\bullet, I^\bullet) = 0.
    \]
    In other words every morphism from an acyclic complex to an injective one is nulhomotopic.
\end{lemma}

\begin{proof}
    We will show that $f \htpy 0$ for any chain map $f \colon A^\bullet \to I^\bullet$ by inductively constructing an appropriate homotopy $h$. For simplicity assume $A^i = 0$ and $I^i = 0$ for $i < 0$ and define $h^i \colon A^i \to I^{i-1}$ to be $0$ for $i \leq 0$. 
    
    As $A^\bullet$ is acyclic, the first non-trivial differential $d^0\colon A^0 \to A^1$ is mono. From $I^0$ being injective, we obtain a morphism $h^1\colon A^1 \to I^0$, satisfying
    \[
        f^0 = h^1 \circ d^0.
    \]
    Note that the above can actually be rewritten to $f^0 = h^1d^0 + \delta^{-1}h^0$, as $h^0 = 0$.

    For the induction step assume that we have already constructed $h^i \colon A^i \to I^{i-1}$, with $f^{i-1} = h^id^{i-1} + \delta^{i-2}h^{i-1}$ for all $i \leq n$. We are aiming to construct $h^{n+1} \colon A^{n+1} \to I^n$, for which $f^n = h^{n+1} d^n + \delta^{n-1}h^n$ holds. First, expand the differential $d^n$ into a composition of the canonical epimorphism $e \colon A^n \to \coker d^{n-1}$, followed by $j \colon \coker d^{n-1} \to A^{n+1}$ induced by the universal property of $\coker d^{n-1}$ by the map $d^n \colon A^n \to A^{n+1}$. Morphism $j$ is actually a monomorphism by acyclicity of $A^\bullet$, since we know that
    \[
        0 = H^n(A^\bullet) \iso \ker(j \colon \coker d^{n-1} \to A^{n+1}).
    \]
    Next, we see that by the universal property of $\coker d^{n-1}$, morphism $f^n - \delta^{n-1}h^n \colon A^n \to I^n$ induces a morphism ${g \colon \coker d^{n-1} \to I^n}$, because
    \begin{align*}
        (f^n - \delta^{n-1}h^n)d^{n-1} &= f^nd^{n-1} - \delta^{n-1}h^nd^{n-1} \\
        &= f^nd^{n-1} + \delta^{n-1}\delta^{n-2}h^{n-1} - \delta^{n-1}f^{n-1} \\
        &= 0.
    \end{align*}
    The second equality follows from the inductive hypothesis and the third one from $f$ being a chain map and $\delta^{n-1}$, $\delta^{n-2}$ being a differentials. 

    Lastly, for $I^n$ is injecive and $j$ mono, there is a morphism $h^{n+1} \colon A^{n+1} \to I^n$, satisfying $h^{n+1} \circ j = g$, thus after precomposing both sides with $e$, we arrive at $h^{n+1}d^n = f^n - \delta^{n-1}h^n$, which can be rewritten as
    \[
        f^n = h^{n+1}d^n + \delta^{n-1}h^n.
        \qedhere  
    \]
\end{proof}

\begin{lemma}
    \label{D+ K+ aux lemma 1}
    % Let $A^\bullet$ and $B^\bullet$ be complexes belonging to $\kplus{\A}$ and $I^\bullet$ a complex of injectives from $\kplus{\J}$.
    Let $A^\bullet$ and $B^\bullet$ belong to $\kplus{\A}$ and $I^\bullet \in \kplus{\J}$. Let $f \colon B^\bullet \to A^\bullet$ be a quasi-isomorphism, then 
    \[
        \Hom_{\kplus{\A}}(A^\bullet, I^\bullet) \xrightarrow{ \ f^* \ } \Hom_{\kplus{\A}}(B^\bullet, I^\bullet) 
    \]
    is an isomorphism of $k$-modules.
\end{lemma}

\begin{proof}
    In $\kplus{\A}$ we have a distinguished triangle $B^\bullet \to A^\bullet \to C(f)^\bullet \to B[1]^\bullet$, which induces a long exact sequence (of $k$-modules)
    \begin{multline*}
        \cdots \longrightarrow \Hom_{\kplus{\A}}(C(f)[-1]^\bullet, I^\bullet) \longrightarrow \\
        \longrightarrow \Hom_{\kplus{\A}}(A^\bullet, I^\bullet) \xrightarrow{ \ f^* \ }
        \Hom_{\kplus{\A}}(B^\bullet, I^\bullet) \longrightarrow \\
        \longrightarrow \Hom_{\kplus{\A}}(C(f)^\bullet, I^\bullet) \longrightarrow \cdots.
    \end{multline*}
    The reason being $\Hom_{\kplus{\A}}(-, I^\bullet)$ is a cohomological functor by example \ref{partial homs are cohomological functors}. As $f$ is a quasi-isomorphism, the cone $C(f)^\bullet$ is acyclic along with all its shifts. Since $I^\bullet$ is a complex of injectives, $f^*$ is clearly an isomorphism by lemma \ref{acyclic to injective is nulhomotopic}. 
\end{proof}

\begin{lemma}
    \label{D+ K+ aux lemma 2}
    Let $A^\bullet$ belong to $\kplus{\A}$ and $I^\bullet$ to $\kplus{\J}$. Then the morphism action 
    \begin{equation}
        \label{eq: morphism action of Q}
        \Hom_{\kplus{\A}}(A^\bullet, I^\bullet) \to \Hom_{\derplus{\A}}(A^\bullet, I^\bullet)
    \end{equation}
    of the localization functor $Q$ is an isomorphism of $k$-modules. 
\end{lemma}

\begin{proof}
    We already know this is a $k$-module homomorphism, thus it is enough to show that it is bijective. This is accomplished by constructing its inverse. Let $\phi \colon A^\bullet \to I^\bullet$ be a morphism in $\derplus{\A}$ and let $A^\bullet \xleftarrow{ \ s \ } B^\bullet \xrightarrow{ \ f \ } I^\bullet$ be its left roof representative. The morphism $s$ being a quasi-isomorphism implies that $s^* \colon \Hom_{\kplus{\A}}(A^\bullet, I^\bullet) \to \Hom_{\kplus{\A}}(B^\bullet, I^\bullet)$ is bijective by lemma \ref{D+ K+ aux lemma 1}, so there is a unique morphism $g\colon A^\bullet \to I^\bullet$, such that $f = g \circ s$ in $\kplus{\A}$. The inverse to \eqref{eq: morphism action of Q} is then defined by sending $\phi$ to $g$. 
    
    First let's argue why this map is well-defined \ie independent of the choice of left roof representative for $\phi$. We pick two left roof representatives $A^\bullet \xleftarrow{ \ s_0 \ } B_0^\bullet \xrightarrow{ \ f_0 \ } I^\bullet$ and $A^\bullet \xleftarrow{ \ s_1 \ } B_1^\bullet \xrightarrow{ \ f_1 \ } I^\bullet$ for $\phi$ and let $g_0$ and $g_1$ be such that $f_i = g_i \circ s_i$ for both $i$. As the roofs are equivalent there are quasi-isomorphisms $t_0 \colon C^\bullet \to B^\bullet_0$ and $t_1 \colon C^\bullet \to B^\bullet_1$, which fit into the following commutative diagram.
    \[\begin{tikzcd}
        % https://q.uiver.app/#q=WzAsNSxbMCwyLCJBIl0sWzEsMSwiQl8wIl0sWzQsMiwiSSJdLFszLDEsIkJfMSJdLFsyLDAsIkMiXSxbMSwyLCJmXzAiLDIseyJsYWJlbF9wb3NpdGlvbiI6NjB9XSxbMywwLCJzXzEiLDAseyJsYWJlbF9wb3NpdGlvbiI6NjB9XSxbMywyLCJmXzEiXSxbNCwzLCJ0XzEiXSxbMSwwLCJzXzAiLDJdLFs0LDEsInRfMCIsMl1d
        && C^\bullet \\
        & {B^\bullet_0} && {B^\bullet_1} \\
        A^\bullet &&&& I^\bullet
        \arrow["{t_0}"', from=1-3, to=2-2]
        \arrow["{t_1}", from=1-3, to=2-4]
        \arrow["{s_0}"', from=2-2, to=3-1]
        \arrow["{f_0}"'{pos=0.6}, from=2-2, to=3-5]
        \arrow["{s_1}"{pos=0.6}, from=2-4, to=3-1]
        \arrow["{f_1}", from=2-4, to=3-5]
    \end{tikzcd}\]
    Therefore $g_0s_0t_0 = f_0t_0 = f_1t_1 = g_1s_1t_1 = g_1s_0t_0$. As $s_0t_0$ is a quasi-isomorphism, we see that $g_0 = g_1$ by applying lemma \ref{D+ K+ aux lemma 1}.
    %  follows by applying lemma \ref{D+ K+ aux lemma 1} as the composition $s_0t_0$ is a quasi-isomorphism.
    
    % The following diagram shows why the constructed map is a left inverse to \eqref{eq: morphism action of Q}
    % \[
    %     \left[ A^\bullet \xleftarrow{ \ s \ } B^\bullet \xrightarrow{ \ f \ } I^\bullet \right] \longmapsto 
    %     \left( g\colon A^\bullet \to I^\bullet\right) \longmapsto 
    %     \left[A^\bullet \xleftarrow{ \ \id{A^\bullet} } A^\bullet \xrightarrow{ \ g \ } I^\bullet\right] = 
    %     \left[ A^\bullet \xleftarrow{ \ s \ } B^\bullet \xrightarrow{ \ f \ } I^\bullet \right]
    % \]

    % \begin{align*}
    %     \Hom_{\derplus{\A}}(A^\bullet, I^\bullet) \to \Hom_{\kplus{\J}}(A^\bullet, I^\bullet) \to \Hom_{\derplus{\A}}(A^\bullet, I^\bullet) \\
    %     \left[ A^\bullet \xleftarrow{ \ s \ } B^\bullet \xrightarrow{ \ f \ } I^\bullet \right] \longmapsto 
    %     \left( g\colon A^\bullet \to I^\bullet\right) \longmapsto 
    %     \left[A^\bullet \xleftarrow{ \ \id{A^\bullet} } A^\bullet \xrightarrow{ \ g \ } I^\bullet\right] = 
    %     \left[ A^\bullet \xleftarrow{ \ s \ } B^\bullet \xrightarrow{ \ f \ } I^\bullet \right]
    % \end{align*}

%     \[\begin{tikzcd}[column sep = small , row sep = small]
%         % https://q.uiver.app/#q=WzAsNixbMCwwLCJcXEhvbV97RF4rKFxcQSl9KEFeXFxidWxsZXQsIEleXFxidWxsZXQpIl0sWzQsMCwiXFxIb21fe0ReKyhcXEEpfShBXlxcYnVsbGV0LCBJXlxcYnVsbGV0KSJdLFsyLDAsIlxcSG9tX3tLXisoXFxKKX0oQV5cXGJ1bGxldCwgSV5cXGJ1bGxldCkiXSxbMCwxLCJcXGxlZnRbIEFeXFxidWxsZXQgXFx4bGVmdGFycm93eyBcXCBzIFxcIH0gQl5cXGJ1bGxldCBcXHhyaWdodGFycm93eyBcXCBmIFxcIH0gSV5cXGJ1bGxldCBcXHJpZ2h0XSJdLFsyLDEsIlxcbGVmdCggZ1xcY29sb24gQV5cXGJ1bGxldCBcXHRvIEleXFxidWxsZXRcXHJpZ2h0KSJdLFs0LDEsIlxcbGVmdFtBXlxcYnVsbGV0IFxceGxlZnRhcnJvd3sgXFwgXFxpZHtBXlxcYnVsbGV0fSB9IEFeXFxidWxsZXQgXFx4cmlnaHRhcnJvd3sgXFwgZyBcXCB9IEleXFxidWxsZXRcXHJpZ2h0XSA9ICAgICAgICAgIFxcbGVmdFsgQV5cXGJ1bGxldCBcXHhsZWZ0YXJyb3d7IFxcIHMgXFwgfSBCXlxcYnVsbGV0IFxceHJpZ2h0YXJyb3d7IFxcIGYgXFwgfSBJXlxcYnVsbGV0IFxccmlnaHRdIl0sWzAsMl0sWzIsMV0sWzMsNCwiIiwyLHsic3R5bGUiOnsidGFpbCI6eyJuYW1lIjoibWFwcyB0byJ9fX1dLFs0LDUsIiIsMix7InN0eWxlIjp7InRhaWwiOnsibmFtZSI6Im1hcHMgdG8ifX19XV0=
% 	{\Hom_{\derplus{\A}}(A^\bullet, I^\bullet)} && {\Hom_{\kplus{\J}}(A^\bullet, I^\bullet)} && {\Hom_{\derplus{\A}}(A^\bullet, I^\bullet)} \\
% 	{\left[ A^\bullet \xleftarrow{ \ s \ } B^\bullet \xrightarrow{ \ f \ } I^\bullet \right]} && {\left( g\colon A^\bullet \to I^\bullet\right)} && {\left[A^\bullet \xleftarrow{ \ \id{A^\bullet} } A^\bullet \xrightarrow{ \ g \ } I^\bullet\right] =          \left[ A^\bullet \xleftarrow{ \ s \ } B^\bullet \xrightarrow{ \ f \ } I^\bullet \right]}
% 	\arrow[from=1-1, to=1-3]
% 	\arrow[from=1-3, to=1-5]
% 	\arrow[maps to, from=2-1, to=2-3]
% 	\arrow[maps to, from=2-3, to=2-5]
% \end{tikzcd}\]

    % and why it is also a right inverse to \eqref{eq: morphism action of Q}
    % \[
    %     \left(  g\colon A^\bullet \to I^\bullet \right) \longmapsto
    %     \left[A^\bullet \xleftarrow{ \ \id{A^\bullet} } A^\bullet \xrightarrow{ \ g \ } I^\bullet\right] \longmapsto
    %     \left(  g\colon A^\bullet \to I^\bullet \right).
    %     \qedhere
    % \]


    % https://q.uiver.app/#q=WzAsNyxbMCwwLCJcXEhvbV97RF4rKFxcQSl9KEFeXFxidWxsZXQsIEleXFxidWxsZXQpIl0sWzIsMCwiXFxIb21fe0ReKyhcXEEpfShBXlxcYnVsbGV0LCBJXlxcYnVsbGV0KSJdLFsxLDAsIlxcSG9tX3tLXisoXFxKKX0oQV5cXGJ1bGxldCwgSV5cXGJ1bGxldCkiXSxbMCwxLCJcXGxlZnRbIEFeXFxidWxsZXQgXFx4bGVmdGFycm93eyBcXCBzIFxcIH0gQl5cXGJ1bGxldCBcXHhyaWdodGFycm93eyBcXCBmIFxcIH0gSV5cXGJ1bGxldCBcXHJpZ2h0XSJdLFsxLDEsIlxcbGVmdCggZ1xcY29sb24gQV5cXGJ1bGxldCBcXHRvIEleXFxidWxsZXRcXHJpZ2h0KSJdLFsyLDEsIlxcbGVmdFtBXlxcYnVsbGV0IFxceGxlZnRhcnJvd3sgXFwgXFxpZHtBXlxcYnVsbGV0fSB9IEFeXFxidWxsZXQgXFx4cmlnaHRhcnJvd3sgXFwgZyBcXCB9IEleXFxidWxsZXRcXHJpZ2h0XSJdLFszLDEsIlxcbGVmdFsgQV5cXGJ1bGxldCBcXHhsZWZ0YXJyb3d7IFxcIHMgXFwgfSBCXlxcYnVsbGV0IFxceHJpZ2h0YXJyb3d7IFxcIGYgXFwgfSBJXlxcYnVsbGV0IFxccmlnaHRdIl0sWzAsMl0sWzIsMV0sWzMsNCwiIiwyLHsic3R5bGUiOnsidGFpbCI6eyJuYW1lIjoibWFwcyB0byJ9fX1dLFs0LDUsIiIsMix7InN0eWxlIjp7InRhaWwiOnsibmFtZSI6Im1hcHMgdG8ifX19XSxbNSw2LCIiLDIseyJsZXZlbCI6Miwic3R5bGUiOnsiaGVhZCI6eyJuYW1lIjoibm9uZSJ9fX1dXQ==
% \[\begin{tikzcd}[row sep = 0.5em, column sep = 0.75em]
% 	{\Hom_{\derplus{\A}}(A^\bullet, I^\bullet)} & {\Hom_{\kplus{\J}}(A^\bullet, I^\bullet)} & {\Hom_{\derplus{\A}}(A^\bullet, I^\bullet)} \\
% 	{\left[ A^\bullet \xleftarrow{ \ s \ } B^\bullet \xrightarrow{ \ f \ } I^\bullet \right]} & {\left( g\colon A^\bullet \to I^\bullet\right)} & {\left[A^\bullet \xleftarrow{ \ \id{A^\bullet} } A^\bullet \xrightarrow{ \ g \ } I^\bullet\right]} & {\left[ A^\bullet \xleftarrow{ \ s \ } B^\bullet \xrightarrow{ \ f \ } I^\bullet \right]}
% 	\arrow[from=1-1, to=1-2]
% 	\arrow[from=1-2, to=1-3]
% 	\arrow[maps to, from=2-1, to=2-2]
% 	\arrow[maps to, from=2-2, to=2-3]
% 	\arrow[equals, from=2-3, to=2-4]
% \end{tikzcd}\]

% \[\begin{tikzcd}[row sep = 0.5em, column sep = 0.75em]
%     % https://q.uiver.app/#q=WzAsNixbMCwwLCJcXEhvbV97RF4rKFxcQSl9KEFeXFxidWxsZXQsIEleXFxidWxsZXQpIl0sWzIsMCwiXFxIb21fe0ReKyhcXEEpfShBXlxcYnVsbGV0LCBJXlxcYnVsbGV0KSJdLFsxLDAsIlxcSG9tX3tLXisoXFxKKX0oQV5cXGJ1bGxldCwgSV5cXGJ1bGxldCkiXSxbMCwxLCJcXGxlZnRbIEFeXFxidWxsZXQgXFx4bGVmdGFycm93eyBcXCBzIFxcIH0gQl5cXGJ1bGxldCBcXHhyaWdodGFycm93eyBcXCBmIFxcIH0gSV5cXGJ1bGxldCBcXHJpZ2h0XSJdLFsxLDEsIlxcbGVmdCggZ1xcY29sb24gQV5cXGJ1bGxldCBcXHRvIEleXFxidWxsZXRcXHJpZ2h0KSJdLFsyLDEsIlxcbGVmdFtBXlxcYnVsbGV0IFxceGxlZnRhcnJvd3sgXFwgXFxpZHtBXlxcYnVsbGV0fSB9IEFeXFxidWxsZXQgXFx4cmlnaHRhcnJvd3sgXFwgZyBcXCB9IEleXFxidWxsZXRcXHJpZ2h0XSJdLFswLDJdLFsyLDFdLFszLDQsIiIsMix7InN0eWxlIjp7InRhaWwiOnsibmFtZSI6Im1hcHMgdG8ifX19XSxbNCw1LCIiLDIseyJzdHlsZSI6eyJ0YWlsIjp7Im5hbWUiOiJtYXBzIHRvIn19fV1d
% 	{\Hom_{\derplus{\A}}(A^\bullet, I^\bullet)} & {\Hom_{\kplus{\J}}(A^\bullet, I^\bullet)} & {\Hom_{\derplus{\A}}(A^\bullet, I^\bullet)} \\
% 	{\left[ A^\bullet \xleftarrow{ \ s \ } B^\bullet \xrightarrow{ \ f \ } I^\bullet \right]} & {\left( g\colon A^\bullet \to I^\bullet\right)} & {\left[A^\bullet \xleftarrow{ \ \id{A^\bullet} } A^\bullet \xrightarrow{ \ g \ } I^\bullet\right]}
% 	\arrow[from=1-1, to=1-2]
% 	\arrow[from=1-2, to=1-3]
% 	\arrow[maps to, from=2-1, to=2-2]
% 	\arrow[maps to, from=2-2, to=2-3]
% \end{tikzcd}\]


Following the diagram below, it is clear why the constructed map is a right inverse to \eqref{eq: morphism action of Q} 
\[\begin{tikzcd}[row sep = 0.5em, column sep = 0.75em]
    % https://q.uiver.app/#q=WzAsNixbMSwwLCJcXEhvbV97RF4rKFxcQSl9KEFeXFxidWxsZXQsIEleXFxidWxsZXQpIl0sWzIsMCwiXFxIb21fe0teKyhcXEopfShBXlxcYnVsbGV0LCBJXlxcYnVsbGV0KSJdLFsyLDEsIlxcbGVmdCggZ1xcY29sb24gQV5cXGJ1bGxldCBcXHRvIEleXFxidWxsZXRcXHJpZ2h0KSJdLFswLDAsIlxcSG9tX3tLXisoXFxKKX0oQV5cXGJ1bGxldCwgSV5cXGJ1bGxldCkiXSxbMCwxLCJcXGxlZnQoIGdcXGNvbG9uIEFeXFxidWxsZXQgXFx0byBJXlxcYnVsbGV0XFxyaWdodCkiXSxbMSwxLCJcXGxlZnRbQV5cXGJ1bGxldCBcXHhsZWZ0YXJyb3d7IFxcIFxcaWR7QV5cXGJ1bGxldH0gfSBBXlxcYnVsbGV0IFxceHJpZ2h0YXJyb3d7IFxcIGcgXFwgfSBJXlxcYnVsbGV0XFxyaWdodF0iXSxbMCwxXSxbNCw1LCIiLDIseyJzdHlsZSI6eyJ0YWlsIjp7Im5hbWUiOiJtYXBzIHRvIn19fV0sWzUsMiwiIiwyLHsic3R5bGUiOnsidGFpbCI6eyJuYW1lIjoibWFwcyB0byJ9fX1dLFszLDBdXQ==
	{\Hom_{\kplus{\A}}(A^\bullet, I^\bullet)} & {\Hom_{\derplus{\A}}(A^\bullet, I^\bullet)} & {\Hom_{\kplus{\A}}(A^\bullet, I^\bullet)} \\
	{\left( g\colon A^\bullet \to I^\bullet\right)} & {\left[A^\bullet \xleftarrow{ \ \id{A^\bullet} } A^\bullet \xrightarrow{ \ g \ } I^\bullet\right]} & {\left( g\colon A^\bullet \to I^\bullet\right).}
	\arrow["Q", from=1-1, to=1-2]
	\arrow[from=1-2, to=1-3]
	\arrow[maps to, from=2-1, to=2-2]
	\arrow[maps to, from=2-2, to=2-3]
\end{tikzcd}\]
Lastly we see that it is also a left inverse by the following diagram \[\begin{tikzcd}[row sep = 0.5em, column sep = 0.75em]
    % https://q.uiver.app/#q=WzAsNixbMCwwLCJcXEhvbV97RF4rKFxcQSl9KEFeXFxidWxsZXQsIEleXFxidWxsZXQpIl0sWzIsMCwiXFxIb21fe0ReKyhcXEEpfShBXlxcYnVsbGV0LCBJXlxcYnVsbGV0KSJdLFsxLDAsIlxcSG9tX3tLXisoXFxKKX0oQV5cXGJ1bGxldCwgSV5cXGJ1bGxldCkiXSxbMCwxLCJcXGxlZnRbIEFeXFxidWxsZXQgXFx4bGVmdGFycm93eyBcXCBzIFxcIH0gQl5cXGJ1bGxldCBcXHhyaWdodGFycm93eyBcXCBmIFxcIH0gSV5cXGJ1bGxldCBcXHJpZ2h0XSJdLFsxLDEsIlxcbGVmdCggZ1xcY29sb24gQV5cXGJ1bGxldCBcXHRvIEleXFxidWxsZXRcXHJpZ2h0KSJdLFsyLDEsIlxcbGVmdFtBXlxcYnVsbGV0IFxceGxlZnRhcnJvd3sgXFwgXFxpZHtBXlxcYnVsbGV0fSB9IEFeXFxidWxsZXQgXFx4cmlnaHRhcnJvd3sgXFwgZyBcXCB9IEleXFxidWxsZXRcXHJpZ2h0XSJdLFswLDJdLFsyLDFdLFszLDQsIiIsMix7InN0eWxlIjp7InRhaWwiOnsibmFtZSI6Im1hcHMgdG8ifX19XSxbNCw1LCIiLDIseyJzdHlsZSI6eyJ0YWlsIjp7Im5hbWUiOiJtYXBzIHRvIn19fV1d
	{\Hom_{\derplus{\A}}(A^\bullet, I^\bullet)} & {\Hom_{\kplus{\A}}(A^\bullet, I^\bullet)} & {\Hom_{\derplus{\A}}(A^\bullet, I^\bullet)} \\
	{\left[ A^\bullet \xleftarrow{ \ s \ } B^\bullet \xrightarrow{ \ f \ } I^\bullet \right]} & {\left( g\colon A^\bullet \to I^\bullet\right)} & {\left[A^\bullet \xleftarrow{ \ \id{A^\bullet} } A^\bullet \xrightarrow{ \ g \ } I^\bullet\right]}
	\arrow[from=1-1, to=1-2]
	\arrow["Q", from=1-2, to=1-3]
	\arrow[maps to, from=2-1, to=2-2]
	\arrow[maps to, from=2-2, to=2-3]
\end{tikzcd}\]
along with observing that $A^\bullet \xleftarrow{ \ \id{A^\bullet} } A^\bullet \xrightarrow{ \ g \ } I^\bullet$ and $A^\bullet \xleftarrow{ \ s \ } B^\bullet \xrightarrow{ \ f \ } I^\bullet$ are equivalent roofs.
\end{proof}

\begin{proof}[Proof of theorem \ref{D+ = K+ and injectives}]
    % We start of with a

    % \noindent
    % % \textbf{Preliminary claim.} Let $A^\bullet$ and $B^\bullet$ be complexes belonging to $\kplus{\A}$ and $I^\bullet$ a complex of injectives from $\kplus{\J}$.
    % \textbf{Preliminary claim.} Let $A^\bullet$ and $B^\bullet$ belong to $\kplus{\A}$ and $I^\bullet \in \kplus{\J}$. Let $f \colon B^\bullet \to A^\bullet$ be a quasi-isomorphism, then 
    % \[
    %     \Hom_{\kplus{\A}}(A^\bullet, I^\bullet) \xrightarrow{ \ - \circ f \ } \Hom_{\kplus{\A}}(B^\bullet, I^\bullet) 
    % \]
    % is a bijection.
    As $\kplus{\J} \to \derplus{\A}$ is a triangulated functor, we only need to show that it is fully faithful and essentially surjective. Let $I^\bullet$ and $J^\bullet$ be objects of $\kplus{\J}$. Then
    \[
        \Hom_{\kplus{\J}}(I^\bullet, J^\bullet) \xrightarrow{\ = \ } \Hom_{\kplus{\A}}(I^\bullet, J^\bullet) \xrightarrow{\ \eqref{D+ K+ aux lemma 2} \ } \Hom_{\derplus{\A}}(I^\bullet, J^\bullet)
    \]
    is a bijection, showing fully faithfulness (the last map is a bijection by lemma \ref{D+ K+ aux lemma 2}). Essential surjectivity is clear from the existence of injective resolutions (\cf proposition \ref{injective resolution}) because quasi-isomorphisms now play the role of isomorphisms in $\derplus{\A}$.
\end{proof}

\info{add a little argument for why injective res. are unique up to htpy.}

% We leave out the proof for why this operation is associative, but refer the intersted reader to \cite{milicic-dercat} or \cite[\S 7]{kashiwara2006categories} for a more detailed and general approach astablishing the theory of localization of categories.
% \begin{remark}
%     This lemma reminds us of the Ore condition in the context of localization in non-commutative algebra as so analogously there is also a dual version of the lemma. 
% \end{remark}

% ============================================================== %

\subsubsection{Subcategories of derived categories}

We have already seen $\derplus{\A}$, $\derminus{\A}$ and $\derb{\A}$ be defined as full triangulated subcategories of $\dercat{\A}$ on the class of certain bounded complexes taking values in $\A$.  



% ============================================================== %

\subsection{Derived functors}

% ============================================================== %

\subsubsection*{$F$-adapted classes}