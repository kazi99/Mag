\section{Derived categories}

The first goal of this section is to construct the derived category of an abelian category $\A$ and equip it with a triangulated structure. Our arguments, although specialized to the homotopy category $\kcat{\A}$ and the class of quasi-isomorphisms, do not differ tremendously from the general theory of localization of categories. 
% Our arguments will try to be minimal and will not rely on the already established theory of localization of categories. Our approach might at times seem less elegant, but we include the exposition for completness nonetheless.
A more comprehensive and formal treatment of this topic is laid out in \cite[Chapters 7, 10, 13]{kashiwara2006categories} or \cite{milicic-dercat}.
The second part covers the construction of derived functors. We see how the established framework of derived categories nicely lends itself for the definition of derived functors and we also relate them back to the classical higher derived functors. To close the chapter, we prove a correspondence between the hom-sets of $\derplus{\A}$ and certain $\Ext$-modules.
% more general framework realtes back to the classical higher derived functors. 

\subsection{Derived categories of abelian categories}

In algebraic geometry cohomology of a geometric object, like a scheme or a variety, $X$ with respect to some coherent sheaf $\F$ plays a very important role. One way of computing $H^i(X, \F)$, which we shall also feature in section \ref{Derived categories in geometry}, involves the following. Instead of intrinsically \info{intrinsically might not be the best word as resolutions in certain cases do represent a sheaf intrinsically in the sense of generators and relations... (Hilbert syzygy thm)}
studying the sheaf $\F$, we represent it with a so called \emph{resolution}, which consists of a complex of sheaves $F^\bullet$, built up from sheaves $F^i$, for $i \in \Z$, belonging to some class of sheaves, which is well behaved under cohomology, and a quasi-isomorphism of the form $F^\bullet \to \F$ or $\F \to F^\bullet$. 
% After observing that using two different resolutions of $\F$ 
After noting that the sheaf $\F$ can be seen as a complex concentrated in degree $0$ and observing that replacing a resolution of $\F$ with another one
results in computing isomorphic cohomology groups, we are motivated  
not to distinguish the sheaf $\F$ from its resolutions in the ambient (homotopy) category of complexes any longer. 
% identify the sheaf $\F$ with the collection of all its resolutions. 
Taking a step back, we would like to modify the homotopy category $\kcat{\coherent{X}}$ in such a way that $\F$ is identified with all its resolutions, or in other words, we want all the quasi-isomorphisms of $\kcat{\coherent{X}}$ to turn into isomorphisms in a ``universal way''. We will take the latter to be our inspiration for the definition of the derived category $\dercat{\A}$ of a general abelian category $\A$. This is stated more formally in the form of the ensuing universal property.

% assigning the derived category $\dercat{\A}$ to a general abelian category $\A$. 

% is to consider a certain complex built from objects all originating from some special class, in a way representing $\F$, called its \emph{resolution}. In a way we would like to identify $\F$ with its resolution and the correct relation, which acheves this, is via quasi-isomorphisms. Inherently, computing the cohomology from said complex losses some information 

% One of the most important invariants of a geometric object $X$ in algebraic geometry is its cohomology $H^i(X, \F)$ with respect to some (coherent) sheaf $\F$. 


% The central idea in this part of algebraic geometry is 

% Our goal in general is then fairly simple, modify the homotopy category $\kcat{\A}$ in such a way that all the quasi-isomorphisms become isomorphisms in a ``universal way''. This is stated more formally in the form of the ensuing universal property.

% \begin{definition}
%     \label{universal property of D(A)}
%     Let $\A$ be an abelian category and $\kcat{\A}$ its homotopy category. A triangulated category $\dercat{\A}$ together with a triangulated functor $Q\colon \kcat{\A} \to \dercat{\A}$ is the \emph{derived category of $\A$}, if it satisfies:
%     \info{maybe I replace triangulated in this def. w/ $k$-linear and then add to proposion 2.8, that when $Q$ and $F$ are assumed to be triangulated $F_0$ is as well. This is so that I can use it directly to construct homology functors on $\dercat{\A}$}
%     \begin{enumerate}[label = (\roman*)]
%         \item For every quasi-isomorphism $s$ in $\kcat{\A}$, $Q(s)$ is an isomorphism in $\dercat{\A}$.
%         \item For any triangulated category $\D$ and any functor $F\colon \kcat{\A} \to \D$, sending quasi-isomorphisms $s$ in $\kcat{\A}$ to isomorphisms $F(s)$ in $\D$,
%         % for which $F(s)$ is an isomorphism in $\D$ for all quasi-isomorphisms $s$ in $\kcat{\A}$, 
%         there exists a triangulated functor $F_0 \colon \dercat{\A} \to \D$, which is unique up to a unique natural isomorphism, such that $F \natiso F_0 \circ Q$.
%         % , and moreover $F_0$ is required to be unique up to a unique natural isomorphism.
%         In other words, the diagram below commutes up to natural isomorphism. 
%         \[\begin{tikzcd}
%             % https://q.uiver.app/#q=WzAsMyxbMCwwLCJLKFxcQSkiXSxbMCwxLCJEKFxcQSkiXSxbMSwwLCJcXEQiXSxbMCwyLCJGIl0sWzAsMSwiUSIsMl0sWzEsMiwiRl9cXEEiLDIseyJzdHlsZSI6eyJib2R5Ijp7Im5hbWUiOiJkYXNoZWQifX19XV0=
%             {\kcat{\A}} & \D \\
%             {\dercat{\A}}
%             \arrow["F", from=1-1, to=1-2]
%             \arrow["Q"', from=1-1, to=2-1]
%             \arrow["{F_0}"', dashed, from=2-1, to=1-2]
%         \end{tikzcd}\]
%     \end{enumerate}
% \end{definition}

\begin{definition}
    \label{universal property of D(A)}
    Let $\A$ be an abelian category and $\kcat{\A}$ its homotopy category. A category $\dercat{\A}$ together with a functor $Q\colon \kcat{\A} \to \dercat{\A}$ is the \emph{derived category of $\A$}, if it satisfies:
    \begin{enumerate}[label = (\roman*)]
        \item For every quasi-isomorphism $s$ in $\kcat{\A}$, $Q(s)$ is an isomorphism in $\dercat{\A}$.
        \item For any category $\D$ and any functor $F\colon \kcat{\A} \to \D$, sending quasi-isomorphisms $s$ in $\kcat{\A}$ to isomorphisms $F(s)$ in $\D$,
        % for which $F(s)$ is an isomorphism in $\D$ for all quasi-isomorphisms $s$ in $\kcat{\A}$, 
        there exists a functor $F_0 \colon \dercat{\A} \to \D$, which is unique up to a unique natural isomorphism, such that $F \natiso F_0 \circ Q$.
        % , and moreover $F_0$ is required to be unique up to a unique natural isomorphism.
        In other words, the diagram below commutes up to natural isomorphism. 
        \[\begin{tikzcd}
            % https://q.uiver.app/#q=WzAsMyxbMCwwLCJLKFxcQSkiXSxbMCwxLCJEKFxcQSkiXSxbMSwwLCJcXEQiXSxbMCwyLCJGIl0sWzAsMSwiUSIsMl0sWzEsMiwiRl9cXEEiLDIseyJzdHlsZSI6eyJib2R5Ijp7Im5hbWUiOiJkYXNoZWQifX19XV0=
            {\kcat{\A}} & \D \\
            {\dercat{\A}}
            \arrow["F", from=1-1, to=1-2]
            \arrow["Q"', from=1-1, to=2-1]
            \arrow["{F_0}"', dashed, from=2-1, to=1-2]
        \end{tikzcd}\]
    \end{enumerate}
\end{definition}

\begin{remark}
    We recognize this definition as a special case of localization of categories \cite[\S 7, Definition 7.1.1.]{kashiwara2006categories} or \cite[\S III.2, Definition 1]{gelfand2002methods}. In particular it defines $\dercat{\A}$ to be the \emph{localization of a triangulated category} $\kcat{\A}$ by the family of all quasi-isomorphisms in $\kcat{\A}$.
\end{remark}

A naive way of constructing $\dercat{\A}$ out of $\kcat{\A}$ would be to artificially add the inverses to all the quasi-isomorphisms in $\kcat{\A}$ and then impose the correct collection of relations on the newly constructed class of morphisms. As this can quickly lead us to some set theoretic problems, we will construct a specific model, which achieves this, instead. Our construction is a priori not going to result in a locally small\footnote{A category $\mathcal C$ is called \emph{locally small} if for all objects $X$ and $Y$ of $\mathcal C$ the hom-sets $\Hom_\mathcal{C}(X, Y)$ are actual sets.} category, but as we shall soon see in practice all the categories we will be concerned with will be locally small. 
% still going to reach outside the scope of sets, but as we will se in practise 
% this can quickly become very tricky, we will construct a specific model instead.
% This will be achieved by constructing a specific model. 
% Instead we will construct a specific model, which achieves this. 

\subsubsection{Construction}

To start, we first need a technical lemma resembling the Ore condition from non-commuta\-tive algebra.

\begin{lemma}
    \label{Ore condition}
    Let $f \colon A^\bullet \to B^\bullet$ and $s\colon C^\bullet \to B^\bullet$ belong to the homotopy category $\kcat{\A}$, with $s$ being a quasi-isomorphism. Then there exists a quasi-isomorphism $u\colon C_0^\bullet \to A^\bullet$ and a morphism $g\colon C_0^\bullet \to C^\bullet$, such that the diagram below commutes in $\kcat{\A}$.
    \[\begin{tikzcd}[column sep = 2em]
        % https://q.uiver.app/#q=WzAsNCxbMCwyLCJBXlxcYnVsbGV0Il0sWzIsMiwiQl5cXGJ1bGxldCJdLFsyLDAsIkNeXFxidWxsZXQiXSxbMCwwLCJDXlxcYnVsbGV0XzAiXSxbMCwxLCJmIl0sWzIsMSwicyJdLFszLDIsImciLDAseyJzdHlsZSI6eyJib2R5Ijp7Im5hbWUiOiJkYXNoZWQifX19XSxbMywwLCJ1IiwyLHsic3R5bGUiOnsiYm9keSI6eyJuYW1lIjoiZGFzaGVkIn19fV0sWzMsMCwiXFxzaW0iLDAseyJzdHlsZSI6eyJib2R5Ijp7Im5hbWUiOiJub25lIn0sImhlYWQiOnsibmFtZSI6Im5vbmUifX19XSxbMiwxLCJcXHNpbSIsMix7InN0eWxlIjp7ImJvZHkiOnsibmFtZSI6Im5vbmUifSwiaGVhZCI6eyJuYW1lIjoibm9uZSJ9fX1dXQ==
        {C^\bullet_0} && {C^\bullet} \\
        \\
        {A^\bullet} && {B^\bullet}
        \arrow["g", dashed, from=1-1, to=1-3]
        \arrow["u", dashed, from=1-1, to=3-1]
        \arrow["\sim"', draw=none, from=1-1, to=3-1]
        \arrow["s", from=1-3, to=3-3]
        \arrow["\sim"', draw=none, from=1-3, to=3-3]
        \arrow["f", from=3-1, to=3-3]
    \end{tikzcd}\]
\end{lemma}

\begin{proof}
    \info{add proof}
\end{proof}

Equipped with the preceding lemma, we are now in a position to construct the derived category $\dercat{\A}$ of an abelian category $\A$. The derived category $\dercat{\A}$ will consist of
\[
    \textsl{Objects of $\dercat{\A}$:} \quad \text{chain complexes in $\A$},
\]
\ie the class of objects of $\kcat{\A}$ or $\com(\A)$, and a class of morphisms, which is quite intricate to define. 
% Its class of objects is formed by chain complexes in $\A$ \ie equals the class of objects of $\kcat{\A}$ and $\com(\A)$, but the definition of morphisms in this case is a bit more intricate. 
For fixed complexes $A^\bullet$ and $B^\bullet$ we define the hom-set\footnote{What will be defined here is a priori not necessarily a set, but a class, so $\dercat{\A}$, defined in this section, is not a category in the usual sense. We will however prove that in specific cases some variants of the derived category, especially concrete ones used later on, will from categories in the usual sense.} $\Hom_{\dercat{\A}}(A^\bullet, B^\bullet)$ in the following way.

\vspace{0.3 cm}

\noindent
\textsc{Hom-sets.}
A \emph{left roof spanned on $A^\bullet$ and $B^\bullet$} is a pair of morphisms $s \colon A^\bullet \to C^\bullet$ and $f\colon C^\bullet \to B^\bullet$ in the homotopy category $\kcat{\A}$, where $s$ is a quasi-isomorphism. This roof is depicted in the following diagram
\begin{equation}
    \label{eq: left roof}
    \begin{tikzcd}[row sep = small, column sep = small]
    % https://q.uiver.app/#q=WzAsMyxbMCwxLCJBIl0sWzIsMCwiQyJdLFs0LDEsIkIiXSxbMSwyLCJmIiwyXSxbMSwwLCJzIl0sWzEsMCwiXFxzaW0iLDJdXQ==
	&& C^\bullet \\
	A^\bullet &&&& B^\bullet
	\arrow["s", from=1-3, to=2-1]
	\arrow["\sim"', from=1-3, to=2-1]
	\arrow["f"', from=1-3, to=2-5]
    \end{tikzcd}
\end{equation}
and denoted by $(s,f)$.
Dually, one also obtains the notion of a \emph{right roof spanned on $A^\bullet$ and $B^\bullet$}, which is a pair of morphisms $g\colon A^\bullet \to C^\bullet$ and $u\colon B^\bullet \to C^\bullet$, where $u$ is a quasi-isomorphism, and is depicted below.
\[\begin{tikzcd}[row sep = small, column sep = small]
    % https://q.uiver.app/#q=WzAsMyxbMCwxLCJBIl0sWzIsMCwiQyJdLFs0LDEsIkIiXSxbMSwyLCJmIiwyXSxbMSwwLCJzIl0sWzEsMCwiXFxzaW0iLDJdXQ==
	&& C^\bullet \\
	A^\bullet &&&& B^\bullet
	\arrow["g"', to=1-3, from=2-1]
	\arrow["\sim"', to=1-3, from=2-5]
	\arrow["u", to=1-3, from=2-5]
\end{tikzcd}\] 
Our construction of $\Hom_{\dercat{\A}}(A^\bullet, B^\bullet)$ will be based on left roofs, for nothing is gained or lost by picking either one of the two. Both work just as well and are in fact equivalent (see \cite[Remark 7.1.18]{kashiwara2006categories}). Despite our arbitrary choice, it is still beneficial to consider both, as we will sometimes switch between the two whenever convenient. 

\begin{definition}
    \label{Definition of equivalence of roofs}
    Two left roofs $A^\bullet \xleftarrow{\ s_0 \ } C_0^\bullet \xrightarrow{\ f_0 \ } B^\bullet$ and $A^\bullet \xleftarrow{\ s_1 \ } C_1^\bullet \xrightarrow{\ f_1 \ } B^\bullet$ are defined to be \emph{equivalent}, if there exists a quasi-isomorphism $u\colon C^\bullet \to C_0^\bullet$ and a morphism ${g\colon C^\bullet \to C_1^\bullet}$ in $\kcat{\A}$, for which the diagram below commutes (in $\kcat{\A}$). 
    \begin{equation}
        \label{eq: equivalence of left roofs}
    \begin{tikzcd}[row sep = small, column sep = small]
        % [row sep = 1.5em, column sep = 2em]
        % https://q.uiver.app/#q=WzAsNSxbMCwyLCJBIl0sWzEsMSwiQ18wIl0sWzQsMiwiQiJdLFszLDEsIkNfMSJdLFsyLDAsIkMiXSxbMSwyXSxbMywwLCJcXHNpbSIsMCx7ImxhYmVsX3Bvc2l0aW9uIjo2MH1dLFszLDJdLFs0LDEsIlxcc2ltIiwyXSxbNCwzLCJnIiwyXSxbMSwwLCJcXHNpbSIsMl0sWzQsMSwidSJdXQ==
        && C^\bullet \\
        & {C_0^\bullet} && {C_1^\bullet} \\
        A^\bullet &&&& B^\bullet
        \arrow["\sim"', from=1-3, to=2-2]
        \arrow["u", from=1-3, to=2-2]
        \arrow["g"', from=1-3, to=2-4]
        \arrow["\sim"', from=2-2, to=3-1]
        \arrow["\sim"{pos=0.6}, from=2-4, to=3-1]
        \arrow[from=2-4, to=3-5]
        \arrow[from=2-2, to=3-5, crossing over]
    \end{tikzcd}
    \end{equation}
We denote this relation by $\equiv$.
\end{definition}

Note that since $C^\bullet \to C_0^\bullet \to A^\bullet$ is a quasi-isomorphism, the same is true for the composition ${C^\bullet \to C_1^\bullet \to A^\bullet}$, concluding that $g\colon C^\bullet \to C^\bullet_1$ is a quasi-isomorphism. Also observe that in the diagram \eqref{eq: equivalence of left roofs} we may find a new left roof, namely 
\begin{equation*}
    \begin{tikzcd}[row sep = small, column sep = small]
    % https://q.uiver.app/#q=WzAsMyxbMCwxLCJBIl0sWzIsMCwiQyJdLFs0LDEsIkIiXSxbMSwyLCJmIiwyXSxbMSwwLCJzIl0sWzEsMCwiXFxzaW0iLDJdXQ==
	&& C^\bullet \\
	A^\bullet &&&& B^\bullet,
	\arrow["s_0 \circ u"', from=1-3, to=2-1]
	\arrow["\sim", from=1-3, to=2-1]
	\arrow["f_1 \circ g", from=1-3, to=2-5]
    \end{tikzcd}
\end{equation*}
which is also equivalent to the two roofs we started with $(s_0, f_0)$ and $(s_1, f_1)$. 

\begin{lemma}
    The equivalence of left roofs on $A^\bullet$ and $B^\bullet$ is an equivalence relation.
\end{lemma}

\begin{proof}
    The relation is clearly reflexive. We take both $u$ and $g$ to be $\id{C^\bullet}$. By the note above it is also symmetric, for $g$ is a quasi-isomorphism. It remains to show transitivity. Suppose left roofs $A^\bullet \longleftarrow C_0^\bullet \longrightarrow B^\bullet$ and $A^\bullet \longleftarrow C_1^\bullet \longrightarrow B^\bullet$ are equivalent and left roofs $A^\bullet \longleftarrow C_1^\bullet \longrightarrow B^\bullet$ and $A^\bullet \longleftarrow C_2^\bullet \longrightarrow B^\bullet$ are equivalent. This is witnessed by the diagrams
    \begin{center}
        \begin{tabular}{l c r}
            \begin{tikzcd}[column sep = small, row sep = small]
                % https://q.uiver.app/#q=WzAsNSxbMCwyLCJBXlxcYnVsbGV0Il0sWzEsMSwiQ18wXlxcYnVsbGV0Il0sWzQsMiwiQl5cXGJ1bGxldCJdLFszLDEsIkNfMV5cXGJ1bGxldCJdLFsyLDAsIkReXFxidWxsZXQiXSxbMSwyXSxbMywwLCJcXHNpbSIsMCx7ImxhYmVsX3Bvc2l0aW9uIjo2MH1dLFszLDJdLFsxLDAsIlxcc2ltIiwyXSxbNCwxLCJcXHNpbSIsMl0sWzQsMywiZyIsMl0sWzQsMSwidSJdXQ==
                && {D_0^\bullet} \\
                & {C_0^\bullet} && {C_1^\bullet} \\
                {A^\bullet} &&&& {B^\bullet}
                \arrow["\sim"', from=1-3, to=2-2]
                \arrow["", from=1-3, to=2-2]
                \arrow[""', from=1-3, to=2-4]
                \arrow["\sim"', from=2-2, to=3-1]
                \arrow["\sim"{pos=0.7}, from=2-4, to=3-1]
                \arrow[from=2-4, to=3-5]
                \arrow[from=2-2, to=3-5, crossing over]
            \end{tikzcd} & 
            and &
            \begin{tikzcd}[column sep = small, row sep = small]
                % https://q.uiver.app/#q=WzAsNSxbMCwyLCJBXlxcYnVsbGV0Il0sWzEsMSwiQ18wXlxcYnVsbGV0Il0sWzQsMiwiQl5cXGJ1bGxldCJdLFszLDEsIkNfMV5cXGJ1bGxldCJdLFsyLDAsIkReXFxidWxsZXQiXSxbMSwyXSxbMywwLCJcXHNpbSIsMCx7ImxhYmVsX3Bvc2l0aW9uIjo2MH1dLFszLDJdLFsxLDAsIlxcc2ltIiwyXSxbNCwxLCJcXHNpbSIsMl0sWzQsMywiZyIsMl0sWzQsMSwidSJdXQ==
                && {D_1^\bullet} \\
                & {C_1^\bullet} && {C_2^\bullet} \\
                {A^\bullet} &&&& {B^\bullet.}
                \arrow["\sim"', from=1-3, to=2-2]
                \arrow["", from=1-3, to=2-2]
                \arrow[""', from=1-3, to=2-4]
                \arrow["\sim"', from=2-2, to=3-1]
                \arrow["\sim"{pos=0.7}, from=2-4, to=3-1]
                \arrow[from=2-4, to=3-5]
                \arrow[from=2-2, to=3-5, crossing over]
            \end{tikzcd}
        \end{tabular}
    \end{center}
    % \[\begin{tikzcd}
    %     % https://q.uiver.app/#q=WzAsOCxbMCwzLCJBIl0sWzEsMiwiQ18wIl0sWzYsMywiQiJdLFs1LDIsIkNfMiJdLFszLDIsIkNfMSJdLFsyLDEsIkRfMCJdLFs0LDEsIkRfMSJdLFszLDAsIkMiXSxbMSwyXSxbMywwLCJcXHNpbSIsMCx7ImxhYmVsX3Bvc2l0aW9uIjo2MH1dLFszLDJdLFsxLDAsIlxcc2ltIiwyXSxbNSwxLCJcXHNpbSIsMix7ImNvbG91ciI6WzI0MCw2MCw2MF19LFsyNDAsNjAsNjAsMV1dLFs1LDQsIiIsMCx7ImNvbG91ciI6WzI0MCw2MCw2MF19XSxbNCwwLCJcXHNpbSIsMix7ImxhYmVsX3Bvc2l0aW9uIjo3MH1dLFs0LDJdLFs2LDMsIiIsMCx7ImNvbG91ciI6WzMwMCw2MCw2MF19XSxbNiw0LCJcXHNpbSIsMix7ImNvbG91ciI6WzMwMCw2MCw2MF19LFszMDAsNjAsNjAsMV1dLFs3LDUsIlxcc2ltIiwyLHsic3R5bGUiOnsiYm9keSI6eyJuYW1lIjoiZGFzaGVkIn19fV0sWzcsNiwiIiwwLHsic3R5bGUiOnsiYm9keSI6eyJuYW1lIjoiZGFzaGVkIn19fV1d
    %     &&& C^\bullet \\
    %     && {D_0^\bullet} && {D_1^\bullet} \\
    %     & {C_0^\bullet} && {C_1^\bullet} && {C_2^\bullet} \\
    %     A^\bullet &&&&&& B^\bullet
    %     \arrow["\sim"', dashed, from=1-4, to=2-3]
    %     \arrow[dashed, from=1-4, to=2-5]
    %     \arrow["\sim"', color={rgb,255:red,92;green,92;blue,214}, from=2-3, to=3-2]
    %     \arrow[color={rgb,255:red,92;green,92;blue,214}, from=2-3, to=3-4]
    %     \arrow["\sim"', color={rgb,255:red,214;green,92;blue,214}, from=2-5, to=3-4]
    %     \arrow[color={rgb,255:red,214;green,92;blue,214}, from=2-5, to=3-6]
    %     \arrow["\sim"', from=3-2, to=4-1]
    %     \arrow[curve={height=12pt}, from=3-2, to=4-7]
    %     \arrow["\sim"'{pos=0.2}, from=3-4, to=4-1]
    %     \arrow[from=3-4, to=4-7]
    %     \arrow["\sim"{pos=0.6}, curve={height=-12pt}, from=3-6, to=4-1]
    %     \arrow[from=3-6, to=4-7]
    % \end{tikzcd}\]
    By lemma \ref{Ore condition} the right roof $D_0^\bullet \to C_1^\bullet \leftarrow D_1^\bullet$ may be completed to form a commutative square
    \[\begin{tikzcd}[column sep = 2em]
        % https://q.uiver.app/#q=WzAsNCxbMCwyLCJEXzBeXFxidWxsZXQiXSxbMiwyLCJDXzFcXGJ1bGxldCJdLFsyLDAsIkReXFxidWxsZXRfMSJdLFswLDAsIkNeXFxidWxsZXQiXSxbMCwxXSxbMiwxLCJcXHNpbSJdLFszLDAsIlxcc2ltIiwyLHsic3R5bGUiOnsiYm9keSI6eyJuYW1lIjoiZGFzaGVkIn19fV0sWzMsMiwiIiwwLHsic3R5bGUiOnsiYm9keSI6eyJuYW1lIjoiZGFzaGVkIn19fV1d
        {C^\bullet} && {D^\bullet_1} \\
        \\
        {D_0^\bullet} && {C_1^\bullet,}
        \arrow[dashed, from=1-1, to=1-3]
        \arrow["\sim"', dashed, from=1-1, to=3-1]
        \arrow["\sim", from=1-3, to=3-3]
        \arrow[from=3-1, to=3-3]
    \end{tikzcd}\]
    proving that $A^\bullet \leftarrow C_0^\bullet \rightarrow B^\bullet$ and $A^\bullet \leftarrow C_2^\bullet \rightarrow B^\bullet$ are equivalent.
\end{proof}

For a left roof \eqref{eq: left roof} we let $\left[s \setminus f\right]$ or $\left[A^\bullet \xleftarrow{\ s\ } C^\bullet \xrightarrow{\ f \ } B^\bullet\right]$ denote its equivalence class under $\equiv$. 

We then define $\Hom_{\dercat{\A}}(A^\bullet, B^\bullet)$ to be the class of left roofs spanned by $A^\bullet$ and $B^\bullet$, quotiented by the relation $\equiv$. That is
% \begin{equation*}
%     \Hom_{\dercat{\A}}(A^\bullet, B^\bullet) := 
%     \left\{ (s, f) \in \coprod_{C^\bullet \in \Ob(\kcat{\A})} \Hom_{\kcat{\A}}(C^\bullet, A^\bullet) \times \Hom_{\kcat{\A}}(C^\bullet, B^\bullet) \ \middle\vert \ \text{$s$ quasi-isomorphism} \right\}
% \end{equation*}
% \[
% \Hom_{\dercat{\A}}(A^\bullet, B^\bullet) := \\ 
% \left\{
% \left[
% \begin{array}{l c r}
% & C^\bullet & \\
% \quad \swarrow & & \searrow \quad \ \\
% A^\bullet & & B^\bullet
% \end{array}
% \right]_\equiv
% \, \middle| \,
% \begin{array}{l}
% C^\bullet \in \Ob \kcat{\A}, \\
% f \in \Hom_{\kcat{\A}}(C^\bullet, B^\bullet), \\
% s \in \Hom_{\kcat{\A}}(C^\bullet, A^\bullet) \text{ quasi-iso.}
% \end{array}
% \right\}.
% \]
\[
\Hom_{\dercat{\A}}(A^\bullet, B^\bullet) := 
\left\{
\left[
\begin{tikzcd}[row sep = small, column sep = 0.8em]
    % https://q.uiver.app/#q=WzAsMyxbMCwxLCJBIl0sWzIsMCwiQyJdLFs0LDEsIkIiXSxbMSwyLCJmIiwyXSxbMSwwLCJzIl0sWzEsMCwiXFxzaW0iLDJdXQ==
	&& C^\bullet \\
	A^\bullet &&&& B^\bullet
	\arrow["s", from=1-3, to=2-1]
	\arrow["\sim"', from=1-3, to=2-1]
	\arrow["f"', from=1-3, to=2-5]
\end{tikzcd}
\right]_\equiv
\, \middle| \,
\begin{array}{l}
C^\bullet \in \Ob \kcat{\A}, \\
f \in \Hom_{\kcat{\A}}(C^\bullet, B^\bullet), \\
s \in \Hom_{\kcat{\A}}(C^\bullet, A^\bullet) \text{ quasi-iso.}
\end{array}
\right\}.
\]

\vspace{0,3 cm}

\noindent
\textsc{Composition.} Next we define the composition operations
\[
    \circ \colon \Hom_{\dercat{\A}}(A_0^\bullet, A_1^\bullet) \times \Hom_{\dercat{\A}}(A_1^\bullet, A_2^\bullet) \to \Hom_{\dercat{\A}}(A_0^\bullet, A_2^\bullet),
\]
for all objects $A_0^\bullet$, $A_1^\bullet$ and $A_2^\bullet$ of $\dercat{\A}$. Let $\phi_0\colon A_0^\bullet \to A_1^\bullet$ and $\phi_1\colon A_1^\bullet \to A_2^\bullet$ be a pair of composable morphisms in $\dercat{\A}$. Next, pick their respective left roof representatives, $A_0^\bullet \xleftarrow{\ s_0\ } C_0^\bullet \xrightarrow{\ f_0 \ } A_1^\bullet$ and $A_1^\bullet \xleftarrow{\ s_1\ } C_1^\bullet \xrightarrow{\ f_1 \ } A_2^\bullet$, and concatenate them according to the solid zig-zag diagram below. 
\[\begin{tikzcd}[row sep = small, column sep = small]
    % https://q.uiver.app/#q=WzAsNixbMCwyLCJBXzAiXSxbMiwxLCJDXzAiXSxbNCwyLCJBXzEiXSxbOCwyLCJBXzIiXSxbNiwxLCJDXzEiXSxbNCwwLCJDIl0sWzEsMiwiZl8wIiwyXSxbMSwwLCJzXzAiXSxbMSwwLCJcXHNpbSIsMix7Im9mZnNldCI6LTEsInN0eWxlIjp7ImJvZHkiOnsibmFtZSI6Im5vbmUifSwiaGVhZCI6eyJuYW1lIjoibm9uZSJ9fX1dLFs0LDIsInNfMSJdLFs0LDMsImZfMSIsMl0sWzUsMSwidSIsMCx7InN0eWxlIjp7ImJvZHkiOnsibmFtZSI6ImRhc2hlZCJ9fX1dLFs1LDQsImciLDIseyJzdHlsZSI6eyJib2R5Ijp7Im5hbWUiOiJkYXNoZWQifX19XSxbNCwyLCJcXHNpbSIsMix7Im9mZnNldCI6LTEsInN0eWxlIjp7ImJvZHkiOnsibmFtZSI6Im5vbmUifSwiaGVhZCI6eyJuYW1lIjoibm9uZSJ9fX1dLFs1LDEsIlxcc2ltIiwyLHsib2Zmc2V0IjotMSwic3R5bGUiOnsiYm9keSI6eyJuYW1lIjoibm9uZSJ9LCJoZWFkIjp7Im5hbWUiOiJub25lIn19fV1d
	&&&& C^\bullet \\
	&& {C_0^\bullet} &&&& {C_1^\bullet} \\
	{A_0^\bullet} &&&& {A_1^\bullet} &&&& {A_2^\bullet}
	\arrow["u", dashed, from=1-5, to=2-3]
	\arrow["\sim"', shift left, draw=none, from=1-5, to=2-3]
	\arrow["g"', dashed, from=1-5, to=2-7]
	\arrow["{s_0}", from=2-3, to=3-1]
	\arrow["\sim"', shift left, draw=none, from=2-3, to=3-1]
	\arrow["{f_0}"', from=2-3, to=3-5]
	\arrow["{s_1}", from=2-7, to=3-5]
	\arrow["\sim"', shift left, draw=none, from=2-7, to=3-5]
	\arrow["{f_1}"', from=2-7, to=3-9]
\end{tikzcd}\]
By Lemma \ref{Ore condition} there are morphisms $u\colon C^\bullet \to C_0^\bullet$ and $g\colon C^\bullet \to C_1^\bullet$, depicted with dashed arrows, completing the diagram in $\kcat{\A}$. In this way we obtain a left roof, formed by a quasi-isomorphism $s_0 \circ u$ and a morphism $f_1 \circ g$, the equivalence class of which we define to be the composition 
\[
    \phi_1 \circ \phi_0 :=
    \left[A_0^\bullet \xleftarrow{s_0 \circ u} C^\bullet \xrightarrow{f_1 \circ g} A_2^\bullet\right].
\]
It can be shown that this is a well defined composition, independent of the choice of left roof representatives of $\phi_0$ and $\phi_1$ and independent of the choice of the peak $C^\bullet$ and morphisms $u\colon C^\bullet \to C_0^\bullet$ and $g\colon C^\bullet \to C_1^\bullet$, for which the square at the top of the diagram commutes. It is also true that the operation is associative. We leave out the proof, because it is routine, but refer the reader to a very detailed account by Miličić \cite[Ch.~1.3]{milicic-dercat}. 

\vspace{0,3 cm}

\noindent
\textsc{Identities.}
The identity morphism on an object $A^\bullet$ of $\dercat{\A}$ is defined to be the equivalence class of $A^\bullet \xleftarrow{\id{A^\bullet}} A^\bullet \xrightarrow{\id{A^\bullet}} A^\bullet$. It is not difficult to check that these classes play the role of identity morphisms in $\dercat{\A}$. 

\vspace{0,3 cm}

\noindent
\textsc{Functor $Q\colon \kcat{\A} \to \dercat{\A}$.} The \emph{localization} functor $Q\colon \kcat{\A} \to \dercat{\A}$ is an identity on objects functor with the action on morphisms defined by the assignment 
\[\begin{tikzcd}[row sep = 0.3em, column sep = 0.75em]
    % https://q.uiver.app/#q=WzAsNCxbMSwwLCJcXEhvbV97XFxkZXJjYXR7XFxBfX0oQV5cXGJ1bGxldCwgSV5cXGJ1bGxldCkiXSxbMCwwLCJcXEhvbV97fShBXlxcYnVsbGV0LCBJXlxcYnVsbGV0KSJdLFswLDEsIlxcbGVmdCggZlxcY29sb24gQV5cXGJ1bGxldCBcXHRvIEleXFxidWxsZXRcXHJpZ2h0KSJdLFsxLDEsIlxcbGVmdFtBXlxcYnVsbGV0IFxceGxlZnRhcnJvd3sgXFwgXFxpZHtBXlxcYnVsbGV0fSB9IEFeXFxidWxsZXQgXFx4cmlnaHRhcnJvd3sgXFwgZiBcXCB9IEleXFxidWxsZXRcXHJpZ2h0XSJdLFsyLDMsIiIsMix7InN0eWxlIjp7InRhaWwiOnsibmFtZSI6Im1hcHMgdG8ifX19XSxbMSwwXV0=
	{\Hom_{\kcat{\A}}(A^\bullet, B^\bullet)} & {\Hom_{\dercat{\A}}(A^\bullet, B^\bullet)} \\
	{\left( f\colon A^\bullet \to B^\bullet\right)} & {\left[A^\bullet \xleftarrow{ \ \id{A^\bullet} } A^\bullet \xrightarrow{ \ f \ } B^\bullet\right].}
	\arrow["{Q}",from=1-1, to=1-2]
	\arrow[maps to, from=2-1, to=2-2]
\end{tikzcd}\]

% \vspace{0,3 cm}

\noindent
\textsc{$k$-linear structure.} To equip the hom-sets of $\dercat{\A}$ with a $k$-module structure, we consult the following lemma resembling finding a common denominator in the context of left roofs.

\begin{lemma}
    \label{common denominator for left roofs}
    Let $\phi_0 \colon A^\bullet \to B^\bullet$ and $\phi_1 \colon A^\bullet \to B^\bullet$ be two morphisms in $\dercat{\A}$ represented by left roofs
    \begin{center}
    \begin{tabular}{l c r}
        \begin{tikzcd}[row sep = small, column sep = small]
            % https://q.uiver.app/#q=WzAsMyxbMCwxLCJBIl0sWzIsMCwiQ18wIl0sWzQsMSwiQiJdLFsxLDIsImZfMCJdLFsxLDAsInNfMCIsMl0sWzEsMCwiXFxzaW0iLDAseyJzdHlsZSI6eyJib2R5Ijp7Im5hbWUiOiJub25lIn0sImhlYWQiOnsibmFtZSI6Im5vbmUifX19XV0=
            && {C_0^\bullet} \\
            A^\bullet &&&& B^\bullet
            \arrow["{s_0}"', from=1-3, to=2-1]
            \arrow["\sim", draw=none, from=1-3, to=2-1]
            \arrow["{f_0}", from=1-3, to=2-5]
        \end{tikzcd}
        & \quad and \quad &
        \begin{tikzcd}[row sep = small, column sep = small]
            % https://q.uiver.app/#q=WzAsMyxbMCwxLCJBIl0sWzIsMCwiQ18wIl0sWzQsMSwiQiJdLFsxLDIsImZfMCJdLFsxLDAsInNfMCIsMl0sWzEsMCwiXFxzaW0iLDAseyJzdHlsZSI6eyJib2R5Ijp7Im5hbWUiOiJub25lIn0sImhlYWQiOnsibmFtZSI6Im5vbmUifX19XV0=
            && {C_1^\bullet} \\
            A^\bullet &&&& B^\bullet
            \arrow["{s_1}"', from=1-3, to=2-1]
            \arrow["\sim", draw=none, from=1-3, to=2-1]
            \arrow["{f_1}", from=1-3, to=2-5]
        \end{tikzcd}
    \end{tabular}
    \end{center}
    Then there is a quasi-isomorphism $s \colon C^\bullet \to A^\bullet$ and morphisms $g_0$, $g_1 \colon C^\bullet \to B^\bullet$, such that 
    \begin{center}
    \begin{tabular}{l c r}
        \begin{tikzcd}[row sep = small, column sep = small]
            % https://q.uiver.app/#q=WzAsMyxbMCwxLCJBIl0sWzIsMCwiQ18wIl0sWzQsMSwiQiJdLFsxLDIsImZfMCJdLFsxLDAsInNfMCIsMl0sWzEsMCwiXFxzaW0iLDAseyJzdHlsZSI6eyJib2R5Ijp7Im5hbWUiOiJub25lIn0sImhlYWQiOnsibmFtZSI6Im5vbmUifX19XV0=
            && {C^\bullet} \\
            A^\bullet &&&& B^\bullet
            \arrow["{s}"', from=1-3, to=2-1]
            \arrow["\sim", draw=none, from=1-3, to=2-1]
            \arrow["{g_0}", from=1-3, to=2-5]
        \end{tikzcd}
        & \quad and \quad &
        \begin{tikzcd}[row sep = small, column sep = small]
            % https://q.uiver.app/#q=WzAsMyxbMCwxLCJBIl0sWzIsMCwiQ18wIl0sWzQsMSwiQiJdLFsxLDIsImZfMCJdLFsxLDAsInNfMCIsMl0sWzEsMCwiXFxzaW0iLDAseyJzdHlsZSI6eyJib2R5Ijp7Im5hbWUiOiJub25lIn0sImhlYWQiOnsibmFtZSI6Im5vbmUifX19XV0=
            && {C^\bullet} \\
            A^\bullet &&&& B^\bullet
            \arrow["{s}"', from=1-3, to=2-1]
            \arrow["\sim", draw=none, from=1-3, to=2-1]
            \arrow["{g_1}", from=1-3, to=2-5]
        \end{tikzcd}
    \end{tabular}
    \end{center}
    represent $\phi_0$ and $\phi_1$ respectively.
\end{lemma}

\begin{proof}
    \info{add proof}    
\end{proof}

Using notation from lemma \ref{common denominator for left roofs}, we define the addition operation on $\Hom_{\dercat{\A}}(A^\bullet, B^\bullet)$ by the rule
\[
    \phi_0 + \phi_1 := \left[ A^\bullet \xleftarrow{\ s\ } C^\bullet \xrightarrow{ g_0 + g_1 } B^\bullet \right].
\]
It is well defined, associative, commutative and has a neutral element $0 = Q(0)$. The action of scalars of the ring $k$ is defined as  
\[
    \lambda \phi_0 := \left[ A^\bullet \xleftarrow{\ s_0\ } C^\bullet \xrightarrow{ \lambda f_0 } B^\bullet \right].
\]
Moreover, the composition law $\circ$ is $k$-bilinear and the localization functor $Q \colon \kcat{\A} \to \dercat{\A}$ is an additive functor. 

The zero complex $0^\bullet$ plays the role of the zero object of $\dercat{\A}$ and the direct sum of two complexes is seen to exist and is induced from the direct sum in $\kcat{\A}$ by applying the localization functor $Q$.


\vspace{0,3 cm}

\noindent
\textsc{Triangulated structure.} The translation functor $T \colon \dercat{\A} \to \dercat{\A}$ is defined to be the usual translation functor on objects and a morphism represented by some roof is sent to the equivalence class of that roof on which we have acted with the translation functor of $\kcat{\A}$. This action respects the equivalence relations, which define the hom-sets of $\dercat{\A}$, and thus induces a well defined functor, which is also an additive auto-equivalence, the quasi-inverse being given by translation in the other direction.  

The class of distinguished triangles in $\dercat{\A}$ is defined to consist of all triangles $A^\bullet \to B^\bullet \to C^\bullet \to A[1]^\bullet$, for which there exists a distinguished triangle $A_0^\bullet \to B_0^\bullet \to C_0^\bullet \to A_0[1]^\bullet$ in $\kcat{\A}$, such that $Q(A_0^\bullet) \to Q(B_0^\bullet) \to Q(C_0^\bullet) \to Q(A_0[1]^\bullet)$ is isomorphic to $A^\bullet \to B^\bullet \to C^\bullet \to A[1]^\bullet$ in $\dercat{\A}$. Spelling this out in the case of left roofs, we see that this condition implies the existence of another distinguished triangle $A_1^\bullet \to B_1^\bullet \to C_1^\bullet \to A_1[1]^\bullet$ in $\kcat{\A}$, such that the following commutative diagram, where vertical arrows are all quasi-isomorphisms exists in $\kcat{\A}$.
\[\begin{tikzcd}[]
    % https://q.uiver.app/#q=WzAsMTIsWzAsMSwiQSJdLFsxLDEsIkIiXSxbMiwxLCJDIl0sWzMsMSwiQVsxXSJdLFswLDIsIlgnIl0sWzEsMiwiWSciXSxbMiwyLCJaJyJdLFszLDIsIlgnWzFdIl0sWzAsMCwiWCciXSxbMSwwLCJZJyJdLFsyLDAsIlonIl0sWzMsMCwiWCdbMV0iXSxbMCwxXSxbMSwyXSxbMiwzXSxbNCw1XSxbNSw2XSxbNiw3XSxbMCw0XSxbMSw1XSxbMyw3XSxbOCw5XSxbOSwxMF0sWzEwLDExXSxbMCw4XSxbMSw5XSxbMiwxMF0sWzMsMTFdLFsyLDZdXQ==
	{A^\bullet} & {B^\bullet} & {C^\bullet} & {A[1]^\bullet} \\
	{A_1^\bullet} & {B_1^\bullet} & {C_1^\bullet} & {A_1[1]^\bullet} \\
    {A_0^\bullet} & {B_0^\bullet} & {C_0^\bullet} & {A_0[1]^\bullet}
	\arrow[from=1-1, to=1-2]
	\arrow[from=1-2, to=1-3]
	\arrow[from=1-3, to=1-4]
	\arrow[from=2-1, to=1-1]
	\arrow[from=2-1, to=2-2]
	\arrow[from=2-1, to=3-1]
	\arrow[from=2-2, to=1-2]
	\arrow[from=2-2, to=2-3]
	\arrow[from=2-2, to=3-2]
	\arrow[from=2-3, to=1-3]
	\arrow[from=2-3, to=2-4]
	\arrow[from=2-3, to=3-3]
	\arrow[from=2-4, to=1-4]
	\arrow[from=2-4, to=3-4]
	\arrow[from=3-1, to=3-2]
	\arrow[from=3-2, to=3-3]
	\arrow[from=3-3, to=3-4]
\end{tikzcd}\]

Verification, that the above defines the structure of a triangulated category on $\dercat{\A}$, is left out, but we remark, that it follows naturally from the triangulated structure on $\kcat{\A}$ and refer the reader to \cite[Chapter 2, Theorem 1.6.1]{milicic-dercat} or \cite[Chapter 10, Theorem 10.2.3]{kashiwara2006categories}.

\begin{remark}
    All that has been defined and established in this section with left roofs can analogously also be done with right roofs.
\end{remark}

\begin{proposition}
    \label{universal property for D(A) proposition}
    The constructed derived category $\dercat{\A}$ of an abelian category $\A$ together with the localization functor $Q \colon \kcat{\A} \to \dercat{\A}$ satisfies the universal property of Definition \ref{universal property of D(A)}. Additionally, in notation of Definition \ref{universal property of D(A)}, if category $\D$ and the functor $F \colon \kcat{\A} \to \D$ are $k$-linear or triangulated, the induced functor $F_0$ is as well.
\end{proposition}

\begin{proof}
    For (i) suppose $s\colon A^\bullet \to B^\bullet$ is a quasi-isomorphism. Then the inverse of $Q(s)$ is a morphism $\phi\colon B^\bullet \to A^\bullet$ represented by the left roof $B^\bullet \xleftarrow{\ s \ } A^\bullet \xrightarrow{\id{A^\bullet}} A^\bullet$. Clearly $\phi \circ Q(s) = \id{A^\bullet}$, whereas $Q(s) \circ \phi$, represented by $B^\bullet \xleftarrow{s} A^\bullet \xrightarrow{s} B^\bullet$, can be seen to equal $\id{B^\bullet}$ by the following diagram.
    \[\begin{tikzcd}[column sep = small, row sep = small]
        % https://q.uiver.app/#q=WzAsNSxbMCwyLCJCXlxcYnVsbGV0Il0sWzEsMSwiQV5cXGJ1bGxldCJdLFs0LDIsIkJeXFxidWxsZXQiXSxbMywxLCJCXlxcYnVsbGV0Il0sWzIsMCwiQV5cXGJ1bGxldCJdLFsxLDIsInMiLDJdLFszLDAsIiIsMCx7ImxhYmVsX3Bvc2l0aW9uIjo2MCwibGV2ZWwiOjIsInN0eWxlIjp7ImhlYWQiOnsibmFtZSI6Im5vbmUifX19XSxbMywyLCIiLDAseyJsZXZlbCI6Miwic3R5bGUiOnsiaGVhZCI6eyJuYW1lIjoibm9uZSJ9fX1dLFsxLDAsInMiLDJdLFs0LDEsIiIsMix7ImxldmVsIjoyLCJzdHlsZSI6eyJoZWFkIjp7Im5hbWUiOiJub25lIn19fV0sWzQsMywicyIsMl1d
        && {A^\bullet} \\
        & {A^\bullet} && {B^\bullet} \\
        {B^\bullet} &&&& {B^\bullet}
        \arrow[equals, from=1-3, to=2-2]
        \arrow["s"', from=1-3, to=2-4]
        \arrow["s"', from=2-2, to=3-1]
        \arrow[equals, from=2-4, to=3-1]
        \arrow[equals, from=2-4, to=3-5]
        \arrow["s"'{pos=0.7}, from=2-2, to=3-5, crossing over]
    \end{tikzcd}\]% , where all the unmarked arrows are identities.
    For (ii) the functor $F_0 \colon \dercat{\A} \to \kcat{\A}$ is defined by the assignment on
    \begin{center}
    \begin{tabular}{r r c l}
        \textsl{Objects:} & $A^\bullet$ & $\longmapsto$ & $F(A^\bullet)$. \\
        \textsl{Morphisms:} & $[A^\bullet \xleftarrow{s} C^\bullet \xrightarrow{f} B^\bullet]$ & $\longmapsto$ & $F(f) \circ F(s)\inv$.
    \end{tabular}
    \end{center}
    This is well defined and functorial for the functor $F\colon \kcat{\A} \to \D$ sends quasi-isomorphisms of $\kcat{\A}$ to isomorphisms of $\D$. Moreover, $F_0$ is a $k$-linear functor, as can quickly be computed by an application of Lemma \ref{common denominator for left roofs}. Using the same notation as in the lemma, we have
    \begin{gather*}
        F_0(\phi_0 + \phi_1) = F(g_0 + g_1) \circ F(s)\inv = F(g_0) \circ F(s)\inv + F(g_1) \circ F(s)\inv = F_0(\phi_0) + F_1(\phi_1) \\
        \text{and} \quad F_0(\lambda \phi_0) = F(\lambda f_0) \circ F(s_0)\inv = \lambda (F(f_0) \circ F(s_0)\inv) = \lambda F_0(\phi_0).
    \end{gather*}
    It is clear from the definition, that $F_0 \circ Q = F$ and that $F_0$ is unique up to a natural isomorphism for which the diagram \eqref{universal property of D(A)} commutes. 

    Lastly, when $\D$ and $F$ are triangulated, the functor $F_0$ also commutes with the translation functors of $\dercat{\A}$ and $\D$, because $Q$ as the identity on objects and $F$ commutes with translation functors of $\kcat{\A}$ and $\D$. From the fact that distinguished triangles in $\dercat{\A}$ are by definition exactly those triangles, which are isomorphic to $Q$-images of distinguished triangles of $\kcat{\A}$, and the fact that $F$ maps distinguished triangles of $\dercat{\A}$ to distinguished triangles of $\D$, we can conclude that $F_0$ is triangulated as well. 
\end{proof}


% ============================================================== %


\subsubsection{Cohomology}

Since cohomology already appeared as an indispensable tool on the level of the homotopy category $\kcat{\A}$, it would be nice to also have it be accessible at the level of derived categories. 
% also have access to it at the level of derived categories. 
We can in fact define cohomology functors on $\dercat{\A}$ by inducing them from the functors $H^i \colon \kcat{\A} \to \A$, which by definition send quasi-isomorphisms of $\kcat{\A}$ to isomorphisms of $\A$, using the universal property of Definition \ref{universal property of D(A)}. In this way we obtain $k$-linear functors 
\[
    H^i \colon \dercat{\A} \rightarrow \A, \qquad i \in \Z
\]
for $i \in \Z$, which as expected return the $i$-th cohomology of a complex $A^\bullet$ and return the morphism $H^i(f) \circ H^i(s)\inv$ when applied to a morphism in $\dercat{\A}$, represented by a left roof \eqref{eq: left roof}.


% ============================================================== %


\subsubsection{Subcategories of derived categories}

% We have already seen $\derplus{\A}$, $\derminus{\A}$ and $\derb{\A}$ be defined as full triangulated subcategories of $\dercat{\A}$ on the class of certain bounded complexes taking values in $\A$.  

As with the homotopy category of complexes $\kcat{\A}$, we also have the bounded versions of the derived category $\dercat{\A}$. They are constructed in the exact same way as $\dercat{\A}$ above, with the adjustment, that we replace every instance of the category $\kcat{\A}$ with either $\kplus{\A}$, $\kminus{\A}$ or $\kb{\A}$, to obtain $\derplus{\A}$, $\derminus{\A}$ or $\derb{\A}$, respectively. They are clearly all triangulated as well. All three bounded variants naturally come with the inclusion functors
\[
    \derplus{\A} \to \dercat{\A}, \qquad \derminus{\A} \to \dercat{\A}, \qquad \derb{\A} \to \dercat{\A},
\]
% \[\begin{tikzcd}[row sep = small]
%     % https://q.uiver.app/#q=WzAsNCxbMSwwLCJCIl0sWzAsMSwiQSJdLFsyLDEsIkMiXSxbMSwyLCJEIl0sWzEsMF0sWzAsMl0sWzEsM10sWzMsMl1d
% 	& \derplus{\A} \\
% 	\derb{\A} && \dercat{\A}, \\
% 	& \derminus{\A}
% 	\arrow[from=1-2, to=2-3]
% 	\arrow[from=2-1, to=1-2]
% 	\arrow[from=2-1, to=3-2]
% 	\arrow[from=3-2, to=2-3]
% \end{tikzcd}\]
defined to be the identity on objects and send a morphism, \ie equivalence class of some roof representative with respect to the ``bounded variant'' of the equivalence relation \ref{Definition of equivalence of roofs}, where, again, every instance of $\kcat{\A}$ is replaced by either $\kplus{\A}$, $\kminus{\A}$ or $\kb{\A}$, to the equivalence class of that same roof representative, but now under the original equivalence relation. All three functors are clearly also triangulated.  

% appearing as full triangulated subcategories of the unbounded variant $\dercat{\A}$ by proposition \ref{triangulated subcategory proposition}.
% As with the homotopy category of complexes $\kcat{\A}$, we also have the following bounded versions of the derived category $\dercat{\A}$, appearing as full triangulated subcategories of the unbounded variant $\dercat{\A}$ by proposition \ref{triangulated subcategory proposition}.

% \begin{center}
%     % \begin{tabular}{c l}
%     %     $\derplus{\A}$ & spanned on complexes in $\A$ bounded below. \\
%     %     $\derminus{\A}$ & spanned on complexes in $\A$ bounded above. \\
%     %     $\derb{\A}$ & spanned on bounded complexes in $\A$. \\    
%     % \end{tabular}
%     \begin{tabular}{c l}
%         $\derplus{\A}$ & spanned on objects of $\kplus{\A}$. \\
%         $\derminus{\A}$ & spanned on objects of $\kminus{\A}$. \\
%         $\derb{\A}$ & spanned on objects of $\kb{\A}$. \\    
%     \end{tabular}
% \end{center}

There is also another (equivalent) way of obtaining the categories mentioned above, which will sometimes be more convenient to work with. We introduce them in the form of the following proposition.

\begin{proposition}
    \label{Subcategories of D(A) but bigger}
    Let $\A$ be an abelian category. 
    \begin{enumerate}[label = (\roman*)]
        % \item{The functor $\derplus{\A} \to \dercat{\A}$ (\resp $\derminus{\A} \hookrightarrow \dercat{\A}$) is fully faithful and induces an equivalence between $\derplus{\A}$ and the full triangulated subcategory of $\dercat{\A}$ spanned on (possibly unbounded) complexes $A^\bullet$, for which there is an integer $n \in \Z$, such that $H^i(A^\bullet) \iso 0$ for all $i < n$ (\resp $i > n$).}
        \item{The functor $\derplus{\A} \to \dercat{\A}$ is fully faithful and induces an equivalence between $\derplus{\A}$ and the full triangulated subcategory of $\dercat{\A}$ spanned on (possibly unbounded) complexes $A^\bullet$, for which there is an integer $n \in \Z$, such that $H^i(A^\bullet) \iso 0$ for all $i < n$.}
        \item{The functor $\derminus{\A} \to \dercat{\A}$ is fully faithful and induces an equivalence between $\derminus{\A}$ and the full triangulated subcategory of $\dercat{\A}$ spanned on (possibly unbounded) complexes $A^\bullet$, for which there is an integer $n \in \Z$, such that $H^i(A^\bullet) \iso 0$ for all $i > n$.} 
        \item{The functor $\derb{\A} \to \dercat{\A}$ is fully faithful and induces an equivalence between $\derb{\A}$ and the full triangulated subcategory of $\dercat{\A}$ spanned on (possibly unbounded) complexes $A^\bullet$, for which there is an integer $n \in \Z$, such that $H^i(A^\bullet) \iso 0$ for all $\abs{i} > n$.} 
    \end{enumerate}
\end{proposition}

To make the proof of this proposition conceptually clearer, we introduce two complexes associated to a complex $A^\bullet$, called its \emph{right and left truncations}
\begin{align*}
    \rtruncation{n}(A^\bullet) &= (\cdots \to A^{n-2} \to A^{n-1} \to \ker d^{n} \to 0 \to \cdots) \text{ \ and } \\
    \ltruncation{n}(A^\bullet) &= (\cdots \to 0 \to \coker d^{n-1} \to A^{n+1} \to A^{n+2} \to \cdots).
\end{align*}
They come equipped with the natural inclusion map $j \colon \rtruncation{n}(A^\bullet) \to A^\bullet$ and the natural quotient map $q \colon A^\bullet \to \ltruncation{n}(A^\bullet)$, satisfying the following claims
\begin{align}
    H^i(j) \colon H^i(\rtruncation{n}(A^\bullet)) \to H^i(A^\bullet) \text{ is an isomorphism for all } i \leq n. \label{right truncation isos on cohomology} \\
    H^i(q) \colon H^i(A^\bullet) \to H^i(\ltruncation{n}(A^\bullet)) \text{ is an isomorphism for all } i \geq n. \label{left trunctation isos on cohomology}
\end{align}
Indeed, claim \eqref{right truncation isos on cohomology} is clearly true for $i < n$, with $i = n$ being the only interesting case. Here the right commutative square of the diagram below
% \[\begin{tikzcd}[column sep = small, row sep = 1.5em]
%     % https://q.uiver.app/#q=WzAsNCxbMCwwLCJcXGtlciBkXm4iXSxbMiwwLCIwIl0sWzAsMiwiQV57bn0iXSxbMiwyLCJBXntuKzF9Il0sWzIsMywiZF5uIl0sWzAsMSwiMCJdLFsxLDMsImpee24rMX0iXSxbMCwyLCJqXm4iLDJdXQ==
% 	{\ker d^n} && 0 \\
% 	\\
% 	{A^{n}} && {A^{n+1}}
% 	\arrow["0", from=1-1, to=1-3]
% 	\arrow["{j^n}"', from=1-1, to=3-1]
% 	\arrow["{j^{n+1}}", from=1-3, to=3-3]
% 	\arrow["{d^n}", from=3-1, to=3-3]
% \end{tikzcd}\]
\[\begin{tikzcd}[column sep = small, row sep = 1.5em]
    % https://q.uiver.app/#q=WzAsNixbMiwwLCJcXGtlciBkXm4iXSxbNCwwLCIwIl0sWzIsMiwiQV57bn0iXSxbNCwyLCJBXntuKzF9Il0sWzAsMCwiQV57bi0xfSJdLFswLDIsIkFee24tMX0iXSxbMiwzLCJkXm4iXSxbMCwxLCIwIl0sWzEsMywial57bisxfSJdLFswLDIsImpebiIsMl0sWzQsMCwiXFxkZWx0YSJdLFs0LDUsImpee24tMX0iLDJdLFs1LDIsImRee24tMX0iXV0=
	{A^{n-1}} && {\ker d^n} && 0 \\
	\\
	{A^{n-1}} && {A^{n}} && {A^{n+1}}
	\arrow["\delta", from=1-1, to=1-3]
	\arrow["{j^{n-1}}"', from=1-1, to=3-1]
	\arrow["0", from=1-3, to=1-5]
	\arrow["{j^n}"', from=1-3, to=3-3]
	\arrow["{j^{n+1}}", from=1-5, to=3-5]
	\arrow["{d^{n-1}}", from=3-1, to=3-3]
	\arrow["{d^n}", from=3-3, to=3-5]
\end{tikzcd}\]
induces the identity morphism between the kernels of the horizontal differentials and the left commutative square induces the identity morphism between the images of the horizontal differential. Consequently $j$ induces the identity morphism on the $n$-th cohomology $H^n(j) \colon H^n(\rtruncation{n}(A^\bullet)) \to H^n(A^\bullet)$.
Dually, the claim \eqref{left trunctation isos on cohomology} also holds. 
% In order to make the proof of this proposition conceptually clearer, we introduce truncation functors $\tau_{\geq n} \colon \kcat{\A} \to \kplus{\A}$, defined by
% \begin{center}
%     \begin{tabular}{r r c l}
%         \textsl{Objects:} & $A^\bullet$ & $\longmapsto$ & $(\cdots \to 0 \to \coker d^{n-1} \to A^{n+1} \to A^{n+2} \to \cdots)$. \\
%         \textsl{Morphisms:} & $(f\colon A^\bullet \to B^\bullet)$ & $\longmapsto$ & $ 
%         (\tau_{\geq n}(f)^i)_{i \in \Z}$, 
%         % \text{, where } \tau_{\geq n}(f)^i = 
%         % \begin{cases}
%         %     0, \ i < n \\
%         %     \coker d^{n-1}_A \to \coker d^{n-1}_B, \ i = n \\ 
%         %     f^i, \ i > n.
%         % \end{cases}$.
%     \end{tabular}
% \end{center}
% where $\tau_{\geq n}(f)^i$ is the zero morphism for $i < n$, the induced morphism $\coker d^{n-1}_A \to \coker d^{n-1}_B$ for $i = n$ and morphisms $f^i\colon A^i \to B^i$ for $i > n$. 
% % Naturally for every $A^\bullet$ there is a quotient map $q \colon A^\bullet \to \ltruncation{n}(A^\bullet)$.

% We also have $\tau_{\leq n} \colon \kcat{\A} \to \kminus{\A}$, defined by
% \begin{center}
%     \begin{tabular}{r r c l}
%         \textsl{Objects:} & $A^\bullet$ & $\longmapsto$ & $(\cdots \to A^{n-2} \to A^{n-1} \to \ker d^{n} \to 0 \to \cdots)$. \\
%         \textsl{Morphisms:} & $(f\colon A^\bullet \to B^\bullet)$ & $\longmapsto$ & $ 
%         (\tau_{\leq n}(f)^i)_{i \in \Z}$, 
%     \end{tabular}
% \end{center}
% where $\tau_{\leq n}(f)^i$ is $f^i \colon A^i \to B^i$ for $i < n$, the induced morphism $\ker d^n_A \to \ker d^n_B$ for $i = n$ and the zero morphism for $i > 0$.
% % In this case there are natural inclusions $j \colon \rtruncation{n}(A^\bullet) \to A^\bullet$ for every $A^\bullet$.

% \begin{lemma}
%     For every complex $A^\bullet$ the natural quotient map $q \colon A^\bullet \to \ltruncation{n}(A^\bullet)$ induces isomorphisms $H^i(q)$ for $i \geq n$ and the natural inclusion map $j \colon \rtruncation{n}(A^\bullet) \to A^\bullet$ induces isomorphisms $H^i(j)$ for $i \leq n$.
% \end{lemma}

\begin{proof}[Proof of Proposition \ref{Subcategories of D(A) but bigger}]
    For (i) the functor $\derplus{\A} \rightarrow \dercat{\A}$ is readily seen to be fully faithful by considering morphisms to be represented by \emph{right} roofs. For example in the right roof $A^\bullet \to C^\bullet \leftarrow B^\bullet$ spanned on complexes $A^\bullet$ and $B^\bullet$ of $\derplus{\A}$, we first observe that $C^\bullet$ has bounded cohomology from below, thus $q \colon C^\bullet \to \ltruncation{n}(C^\bullet)$ is a quasi-isomorphism for some $n \in \Z$. Extending the aforementioned right roof with quasi-isomorphism $q$, to obtain $A^\bullet \to \ltruncation{n}(C^\bullet) \leftarrow B^\bullet$, then yields an equivalent right roof, whose peak now belongs to $\derplus{\A}$ and the equivalence class of which is therefore sent to the morphism represented by $A^\bullet \to C^\bullet \leftarrow B^\bullet$.
    % taking the viewpoint of morphisms being represented by \emph{right} roofs and extending their peaks using the quotient map into their left truncations. For example in right roof $A^\bullet \to C^\bullet \leftarrow B^\bullet$ spanned on complexes $A^\bullet$ and $B^\bullet$ of $\derplus{\A}$, we first observe that $C^\bullet$ has bounded   
    % truncating their peaks by extening them with the quotient map $q$. 
    % so we verify only that it is essentially surjective onto the subcategory of complexes with cohomology bounded from below. 

    To verify this functor is essentially surjective onto the subcategory of complexes with cohomology bounded from below,
    suppose $A^\bullet$ satisfies $H^i(A^\bullet) \iso 0$, for all $i < n$. Then by \eqref{left trunctation isos on cohomology} the quotient map $q \colon A^\bullet \to \ltruncation{n}(A^\bullet)$ is a quasi-isomorphism, inducing an isomorphism $A^\bullet \iso \ltruncation{n}(A^\bullet)$ in $\dercat{\A}$. 
    
    The case (ii) is proven analogously, now considering left roofs and right truncations. For (iii), the functor $\derb{\A} \to \dercat{\A}$ is viewed as the composition $\derb{\A} \to \derplus{\A} \to \dercat{\A}$. Here $\derb{\A} \to \derplus{\A}$ is seen to be fully faithful via a minor modification of the argument for (ii) and $\derplus{\A} \to \dercat{\A}$ is fully faithful by (i). Essential surjectivity onto the required subcategory then follows as the roof $A^\bullet \to \ltruncation{n}(A^\bullet) \leftarrow \rtruncation{n}(\ltruncation{n}(A^\bullet))$, for some $n \in \Z$, witnesses an isomorphism in $\dercat{\A}$ between a complex $A^\bullet$ with bounded cohomology and a bounded complex $\rtruncation{n}(\ltruncation{n}(A^\bullet))$. \qedhere
    % we view the inclusion $\derb{\A} \hookrightarrow \dercat{\A}$ as the composition $\derb{\A} \hookrightarrow \derplus{\A} \hookrightarrow \dercat{\A}$
    
    % claim that the quotient map $q \colon A^\bullet \to \ltruncation{n}(A^\bullet)$ is actually a quasi-isomorphism.
    
    % This shows that there is an isomorphism in $\dercat{\A}$ between $A^\bullet$ and a complex $\ltruncation{n}(A^\bullet)$, which belongs to $\derplus{\A}$. 
\end{proof}

\info{Move footnote about $\A \to \kcat{\A}$ being fully faithful over here.}

\begin{proposition}
    The natural functor $\A \to \dercat{\A}$ is fully faithful and identifies $\A$ with the full subcategory of $\dercat{\A}$, spanned by complexes $A^\bullet$, satisfying $H^i(A^\bullet) \iso 0$, for all $i \neq 0$.
\end{proposition}

\begin{proof}
    For fully faithfulness see \cite[Chapter 13, Proposition 13.1.10]{kashiwara2006categories}. Essential surjectivity onto the required subcategory is \cite[Chapter 13, Proposition 13.1.12]{kashiwara2006categories} and follows from a similar argument as in the proof of Proposition \ref{Subcategories of D(A) but bigger} (iii) by taking $n$ to be $0$, thus showing that in $\dercat{\A}$ the complex $A^\bullet$ with trivial cohomology in non-zero degrees is isomorphic to $H^0(A^\bullet)$ concentrated in degree $0$.
\end{proof}
% \begin{proof}
%     We first show the morphism action
%     \begin{equation}
%         \label{eq: morphism action of A -> D(A)}
%         \Hom_\A(A, B) \to \Hom_{\dercat{\A}}(A, B)
%     \end{equation}
%     is injective. Suppose morphisms $f_0$ and $f_1 \colon A \to B$ are sent to equivalent roofs, described by the next diagram.
%     \[\begin{tikzcd}[row sep = small, column sep = small]
%         % [row sep = 1.5em, column sep = 2em]
%         % https://q.uiver.app/#q=WzAsNSxbMCwyLCJBIl0sWzEsMSwiQ18wIl0sWzQsMiwiQiJdLFszLDEsIkNfMSJdLFsyLDAsIkMiXSxbMSwyXSxbMywwLCJcXHNpbSIsMCx7ImxhYmVsX3Bvc2l0aW9uIjo2MH1dLFszLDJdLFs0LDEsIlxcc2ltIiwyXSxbNCwzLCJnIiwyXSxbMSwwLCJcXHNpbSIsMl0sWzQsMSwidSJdXQ==
%         && C^\bullet \\
%         & {A} && {A} \\
%         A &&&& B
%         \arrow["\sim", from=1-3, to=2-2]
%         \arrow["u"', from=1-3, to=2-2]
%         \arrow["g", from=1-3, to=2-4]
%         \arrow[equals, from=2-2, to=3-1]
%         \arrow[equals, from=2-4, to=3-1]
%         \arrow["f_0", from=2-4, to=3-5]
%         \arrow["f_1"'{pos=0.7}, from=2-2, to=3-5, crossing over]
%     \end{tikzcd}\]
%     Then firstly we see that $u = g$ and taking the $0$-th cohomology, shows $f_0 = f_1$. For surjectivity we assign to a morphism $\phi \in \Hom_{\dercat{\A}}(A, B)$, represented by a roof $A \to C^\bullet \leftarrow B$, 


%     To see $\A \to \dercat{\A}$ is fully faithful one uses the $0$-th cohomology functor $H^0$ on roof representatives
%     % The functor $\A \to \dercat{\A}$ can be written as the composition $\A \to \derplus{\A} \to \dercat{\A}$, so we only need to verify fully faithfulness of $\A \to \derplus{\A}$. 
%     % This is seen through the fact that any quasi-isomorphism between two complexes concentrated in degree $0$ is an isomorphism. 
%     % That the functor is fully faithful is a consequence of the fact that it can be written as the composition $\A \to \derplus{\A} \to \dercat{\A}$ and any quasi-isomorphism between two complexes concentrated in degree $0$ is an isomorphism. 
%     Essential surjectivity follows from a similar argument as in the proof of Proposition \ref{Subcategories of D(A) but bigger} (iii) by taking $n$ to be $0$, thus showing that in $\dercat{\A}$ the complex $A^\bullet$ with trivial cohomology in non-zero degrees is isomorphic to $H^0(A^\bullet)$ concentrated in degree $0$. \info{this proof is wack.} 
% \end{proof}

% ============================================================== %

\subsubsection{Derived categories of abelian categories with enough injectives}

After possibly working with proper classes, when constructing the derived category, this section will once again place us back on familiar grounds of set theory. Aside from these concerns the establishing result of this section will have a very important practical application -- construction of derived functors. Under an assumption on the abelian category $\A$, we will establish an equivalence of $\derplus{\A}$ with the homotopy category of a full additive subcategory of $\A$, spanned on \emph{injective objects}, which we define presently. 

\begin{definition}
    An object $I$ of an abelian category $\A$ is \emph{injective}\footnote{Dually, an object $P$ of $\A$ is \emph{projective}, if $P$ is injective in $\A^\op$, or more explicitly, the functor $\Hom_\A(P, -) \colon \A \to \mod{k}$ is exact. Category $\A$ is said to \emph{have enough projectives}, if every object $A$ of $\A$ is a quotient of some projective object \ie there is an epimorphism $P \twoheadrightarrow A$ for some projective object $P$.
}, if the functor 
\[
    \Hom_\A(-, I)\colon \A^\op \to \mod{k}
\]
is exact. Equivalently, $I$ is injective, whenever for any monomorphism ${A \hookrightarrow B}$ and morphism $A \to I$ there is a morphism $B \to I$, for which the following diagram commutes.
    \[\begin{tikzcd}[column sep = 2em]
        % https://q.uiver.app/#q=WzAsNCxbMSwxLCJBIl0sWzIsMSwiQiJdLFsxLDAsIkkiXSxbMCwxLCIwIl0sWzMsMF0sWzAsMSwiIiwwLHsic3R5bGUiOnsidGFpbCI6eyJuYW1lIjoiaG9vayIsInNpZGUiOiJ0b3AifX19XSxbMCwyXSxbMSwyLCIiLDEseyJzdHlsZSI6eyJib2R5Ijp7Im5hbWUiOiJkYXNoZWQifX19XV0=
        & I \\
        0 & A & B
        \arrow[from=2-1, to=2-2]
        \arrow[from=2-2, to=1-2]
        \arrow[hook, from=2-2, to=2-3]
        \arrow[dashed, from=2-3, to=1-2]
    \end{tikzcd}\]
\end{definition}
A category $\A$ is said to \emph{have enough injectives}, if every object $A$ of $\A$ embeds into an injective object \ie there is a monomorphism $A \hookrightarrow I$ for some injective object $I$. 

\begin{remark}
    A simple verification shows, that the full subcategory of an abelian or $k$-linear category $\A$, spanned on all injective objects of $\A$ is a $k$-linear category, as the zero object $0$ is injective and the biproduct of two injective objects is again injective. We denote this category by $\J$.
\end{remark}

Injective objects will be utilized as building blocks for representing complexes. They are the preferred class of objects to work with in this context because they enable a favourable interplay between cohomology and homotopy. 
For example, as a consequence of lemma \ref{acyclic to injective is nulhomotopic}, every acyclic bounded below complex of injectives $I^\bullet$ is actually nullhomotopic \ie isomorphic to the object $0$ in $\kplus{\A}$. 
In light of this, consider a complex $A^\bullet$ of $\kplus{\A}$. A complex of injectives $I^\bullet$ of $\kplus{\J}$ together with a quasi-isomorphism $f \colon A^\bullet \to I^\bullet$ is called an \emph{injective resolution} of $A^\bullet$.
Later we will also see that this injective resolution is unique up to homotopy equivalence.
% of their nice behaviour with respect to cohomology and homotopy

% We call a quasi-isomorphism $f \colon A^\bullet \to I^\bullet$ an \emph{injective resolution} of the complex $A^\bullet$. Later we will also see that this injective resolution is unique up to homotopy \ie any two injective resolutions of $A^\bullet$ are homotopically equivalent. 
% Throughout this section differentials of $I^\bullet$ will be denoted with $\delta^i\colon I^i \to I^{i+1}$.

\begin{proposition}
    \label{injective resolution}
    % Suppose $\A$ contains enough injectives. Then every $A^\bullet$ in $\kplus{\A}$ has an injective resolution.
    Suppose $\A$ contains enough injectives. Then for every $A^\bullet$ in $\kplus{\A}$ there is a quasi-isomorphism $f \colon A^\bullet \to I^\bullet$, where $I^\bullet \in \Ob \kplus{\A}$ is a complex of injectives.
\end{proposition}

\begin{proof}
    We will inductively construct a complex $I^\bullet$ in $\kplus{\A}$, built up from injectives, and a quasi-isomorphism $f \colon A^\bullet \to I^\bullet$. For simplicity assume $A^i = 0$ for $i < 0$. 
    
    The base case is trivial. Define $I^i = 0$ and $f^i = 0$ for $i < 0$ and take $f^0\colon A^0 \to I^0$ to be the morphism obtained from the assumption about $\A$ having enough injectives.
    \[\begin{tikzcd}[column sep = 3em, row sep = 2.7em]
        % https://q.uiver.app/#q=WzAsOCxbMSwwLCIwIl0sWzIsMCwiQV4wIl0sWzMsMCwiQV4xIl0sWzQsMCwiXFxjZG90cyJdLFswLDAsIlxcY2RvdHMiXSxbMiwxLCJJXjAiXSxbMSwxLCIwIl0sWzAsMSwiXFxjZG90cyJdLFs0LDBdLFswLDFdLFsxLDJdLFsyLDNdLFsxLDVdLFs3LDZdLFs2LDVdXQ==
        \cdots & 0 & {A^0} & {A^1} & \cdots \\
        \cdots & 0 & {I^0}
        \arrow[from=1-1, to=1-2]
        \arrow[from=1-2, to=1-3]
        \arrow[from=1-3, to=1-4]
        \arrow[from=1-3, to=2-3]
        \arrow[from=1-4, to=1-5]
        \arrow[from=2-1, to=2-2]
        \arrow[from=2-2, to=2-3]
        \arrow[from=1-2, to=2-2]
    \end{tikzcd}\]

    For the induction step suppose we have constructed injective objects $I^i$, with differentials ${\delta^i \colon I^i \to I^{i+1}}$, and morphisms $f^i$ for all $i$ up to $n$, such that the diagram below commutes. 
    \[\begin{tikzcd}[column sep = normal, row sep = 3em]
        % https://q.uiver.app/#q=WzAsOCxbMSwwLCJBXntuLTF9Il0sWzIsMCwiQV5uIl0sWzMsMCwiQV57bisxfSJdLFs0LDAsIlxcY2RvdHMiXSxbMCwwLCJcXGNkb3RzIl0sWzIsMSwiSV5uIl0sWzEsMSwiSV57bi0xfSJdLFswLDEsIlxcY2RvdHMiXSxbNCwwXSxbMCwxLCJkXntuLTF9Il0sWzEsMiwiZF5uIl0sWzIsM10sWzEsNSwiZl57bn0iLDJdLFs3LDZdLFs2LDUsIlxcZGVsdGFee24tMX0iXSxbMCw2LCJmXntuLTF9IiwyXV0=
        \cdots & {A^{n-1}} & {A^n} & {A^{n+1}} & \cdots \\
        \cdots & {I^{n-1}} & {I^n}
        \arrow[from=1-1, to=1-2]
        \arrow["{d^{n-1}}", from=1-2, to=1-3]
        \arrow["{f^{n-1}}"', from=1-2, to=2-2]
        \arrow["{d^n}", from=1-3, to=1-4]
        \arrow["{f^{n}}"', from=1-3, to=2-3]
        \arrow[from=1-4, to=1-5]
        \arrow[from=2-1, to=2-2]
        \arrow["{\delta^{n-1}}", from=2-2, to=2-3]
    \end{tikzcd}\]
    Consider the cokernel of the differential\footnote{Strictly speaking this is not a differential of a complex yet, because we have not specified the chain map $f$ entirely. But this is easily fixed by setting $I^i = 0$ and $f^i$ to be the zero morphism $A^i \to 0$, for $i > n$.} $d^{n-1}_{C(f)}$
    \[
        A^n \oplus I^{n-1} \xrightarrow{\ d^{n-1}_{C(f)} \ } A^{n+1} \oplus I^n \xrightarrow{\ e \ } \coker d^{n-1}_{C(f)} 
    \]
    and denote with $j \colon \coker d^{n-1}_{C(f)} \hookrightarrow I^{n+1}$ an embedding to an injective object $I^{n+1}$.
    %  we get from $\A$ having enough injectives. 
    After precomposing the morphism
    \[
        A^{n+1} \oplus I^n \xrightarrow{\ e \ } \coker d^{n-1}_{C(f)} \xrightarrow{ \ j \ } I^{n+1}
    \]
    with canonical embeddings of $A^{n + 1}$ and $I^{n}$ into their biproduct $A^{n+1} \oplus I^n$, we obtain morphisms $\delta^{n}\colon I^n \to I^{n+1}$ and $f^{n+1}\colon A^{n+1} \to I^{n+1}$. As $j \circ e$ is a morphism fitting into the following coproduct diagram,
    % \[\begin{tikzcd}[row sep = 2em]
    %     % https://q.uiver.app/#q=WzAsNCxbMSwxLCJBXntuKzF9IFxcb3BsdXMgSV5uIl0sWzAsMCwiQV57bisxfSJdLFsyLDAsIklee259Il0sWzEsMywiSV57bisxfSJdLFswLDMsImogXFxjaXJjIGUiLDJdLFsxLDMsImZee24rMX0iLDIseyJjdXJ2ZSI6Mn1dLFsyLDMsIlxcZGVsdGFebiIsMCx7ImN1cnZlIjotMn1dLFsxLDBdLFsyLDBdXQ==
    %     {A^{n+1}} && {I^{n}} \\
    %     & {A^{n+1} \oplus I^n} \\
    %     \\
    %     & {I^{n+1}}
    %     \arrow[from=1-1, to=2-2]
    %     \arrow["{f^{n+1}}"', curve={height=12pt}, from=1-1, to=4-2]
    %     \arrow[from=1-3, to=2-2]
    %     \arrow["{\delta^n}", curve={height=-12pt}, from=1-3, to=4-2]
    %     \arrow["{j \circ e}"', from=2-2, to=4-2]
    % \end{tikzcd}\]
    \[\begin{tikzcd}[column sep = small, row sep = 2em]
        % https://q.uiver.app/#q=WzAsNCxbMSwxLCJBXntuKzF9IFxcb3BsdXMgSV5uIl0sWzAsMSwiQV57bisxfSJdLFsxLDAsIklee259Il0sWzIsMiwiSV57bisxfSJdLFswLDMsImogXFxjaXJjIGUiLDIseyJsYWJlbF9wb3NpdGlvbiI6NDAsInN0eWxlIjp7ImJvZHkiOnsibmFtZSI6ImRhc2hlZCJ9fX1dLFsxLDMsImZee24rMX0iLDIseyJjdXJ2ZSI6Mn1dLFsyLDMsIlxcZGVsdGFebiIsMCx7ImN1cnZlIjotMn1dLFsxLDBdLFsyLDBdXQ==
        & {I^{n}} \\
        {A^{n+1}} & {A^{n+1} \oplus I^n} \\
        && {I^{n+1}}
        \arrow[from=1-2, to=2-2]
        \arrow["{\delta^n}", curve={height=-18pt}, from=1-2, to=3-3]
        \arrow[from=2-1, to=2-2]
        \arrow["{f^{n+1}}"', curve={height=18pt}, from=2-1, to=3-3]
        \arrow["{j \circ e}"'{pos=0.4}, dashed, from=2-2, to=3-3]
    \end{tikzcd}\]
    we see that $j \circ e = \langle f^{n+1} , \delta^n \rangle$. Next, we see that $\delta^n\delta^{n-1} = 0$ and $\delta^n f^n = f^{n+1}d^n$ from the computation
    \[
        0 = j \circ e \circ d^{n-1}_{C(f)} = \langle f^{n+1},\ \delta^n \rangle \begin{pmatrix}
            -d^n & 0 \\ f^n & \delta^{n-1}
        \end{pmatrix} = \langle \delta^n f^n - f^{n+1}d^n , \ \delta^n\delta^{n-1}\rangle\\
    \]
    Thus far we have constructed a chain complex $I^\bullet$ of injective objects and a chain map ${f \colon A^\bullet \to I^\bullet}$. It remanis to be shown that $f$ is a quasi-isomorphism. Recall, that $f$ is a quasi-isomorphism if and only if its cone $C(f)^\bullet$ is acyclic. By Remark \ref{versions of cohomology}, we known
    \[
        H^n(C(f)^\bullet) = \ker(\coker d^{n-1}_{C(f)} \xrightarrow{\ d \ } A^{n+2} \oplus I^{n+1}),
    \]
    where $d$ is obtained from the universal property of $\coker d^{n-1}_{C(f)}$, induced by $d^n_{C(f)}$. Thus it suffices to show, that $d$ is a monomorphism. Consider the following diagram, which we will show to commute. 
    \[\begin{tikzcd}[column sep = 2em]
        % https://q.uiver.app/#q=WzAsNSxbMCwwLCJBXm5cXG9wbHVzIElee24tMX0iXSxbMiwwLCJBXntuKzF9XFxvcGx1cyBJXntufSJdLFs0LDAsIkFee24rMn1cXG9wbHVzIElee24rMX0iXSxbMywxLCJjb2tlciBkXntuLTF9X3tDKGYpfSJdLFs1LDEsIklee24rMX0iXSxbMCwxLCJkXntuLTF9X3tDKGYpfSJdLFsxLDIsImRee259X3tDKGYpfSJdLFsxLDMsImUiXSxbMywyLCJkIiwwLHsic3R5bGUiOnsiYm9keSI6eyJuYW1lIjoiZGFzaGVkIn19fV0sWzIsNCwicCJdLFszLDQsImoiXV0=
        {A^n\oplus I^{n-1}} && {A^{n+1}\oplus I^{n}} && {A^{n+2}\oplus I^{n+1}} \\
        &&& {\coker d^{n-1}_{C(f)}} && {I^{n+1}}
        \arrow["{d^{n-1}_{C(f)}}", from=1-1, to=1-3]
        \arrow["{d^{n}_{C(f)}}", from=1-3, to=1-5]
        \arrow["e", two heads, from=1-3, to=2-4]
        \arrow["p", from=1-5, to=2-6]
        \arrow["d", dashed, from=2-4, to=1-5]
        \arrow["j", hook, from=2-4, to=2-6]
    \end{tikzcd}\]
    Once we show, that its right most triangle is commutative, we see that $d$ is monomorphic, because $j$ is monomorphic. To see that the triangle in question commutes, we compute
    \[
        p \circ d \circ e = p \circ d^n_{C(f)} = \langle f^{n+1}, \delta^n \rangle = j \circ e.
    \] 
    As $e$ is epic, we may cancel it on the right, to arrive at $p \circ d = j$, which ends the proof.
\end{proof}

\begin{remark}
    As is evident from the proof by close inspection we have only used two facts about the class of all injective objects of $\A$ -- that every object of $\A$ embeds into some injective object and that the class of injectives is closed under finite direct sums. This will later on be used in subsection \ref{Subsection on F-adapted classes} when constructing a right derived functor of a given functor $F$ in the presence of an $F$-adapted class.
\end{remark}

\begin{theorem}
    \label{D+ = K+ and injectives}
    Assume $\A$ contains enough injectives and let $\J \subseteq \A$ denote the full subcategory on injecitve objects of $\A$. Then the inclusion $\kplus{\J} \hookrightarrow \kplus{\A}$ induces an equivalence of triangulated categories
    \[
        \kplus{\J} \iso \derplus{\A}.
    \]
\end{theorem}

\begin{remark}
    Theorem \ref{D+ = K+ and injectives} in particular shows that $\derplus{\A}$ is a category in the usual sense, \ie all its hom-sets are in fact sets.  
    \info{maybe I replace "usual" with locally small everywhere this comes up.}
\end{remark}

\begin{lemma}
    \label{acyclic to injective is nulhomotopic}
    Let $A^\bullet$ be any acyclic complex in $\kplus{\A}$ and $I^\bullet$ a complex of injectives from $\kplus{\J}$. Then
    \[
        \hom_{\kplus{\A}}(A^\bullet, I^\bullet) = 0.
    \]
    In other words every morphism from an acyclic complex to an injective one is nulhomotopic.
\end{lemma}

\begin{proof}
    We will show that $f \htpy 0$ for any chain map $f \colon A^\bullet \to I^\bullet$ by inductively constructing an appropriate homotopy $h$. For simplicity assume $A^i = 0$ and $I^i = 0$ for $i < 0$ and define $h^i \colon A^i \to I^{i-1}$ to be $0$ for $i \leq 0$. 
    
    As $A^\bullet$ is acyclic, the first non-trivial differential $d^0\colon A^0 \to A^1$ is mono. From $I^0$ being injective, we obtain a morphism $h^1\colon A^1 \to I^0$, satisfying
    \[
        f^0 = h^1 \circ d^0.
    \]
    Note that the above can actually be rewritten to $f^0 = h^1d^0 + \delta^{-1}h^0$, as $h^0 = 0$.

    For the induction step assume that we have already constructed $h^i \colon A^i \to I^{i-1}$, with $f^{i-1} = h^id^{i-1} + \delta^{i-2}h^{i-1}$ for all $i \leq n$. We are aiming to construct $h^{n+1} \colon A^{n+1} \to I^n$, for which $f^n = h^{n+1} d^n + \delta^{n-1}h^n$ holds. First, expand the differential $d^n$ into a composition of the canonical epimorphism $e \colon A^n \to \coker d^{n-1}$, followed by $j \colon \coker d^{n-1} \to A^{n+1}$ induced by the universal property of $\coker d^{n-1}$ by the map $d^n \colon A^n \to A^{n+1}$. Morphism $j$ is actually a monomorphism by acyclicity of $A^\bullet$, since we know that
    \[
        0 = H^n(A^\bullet) \iso \ker(j \colon \coker d^{n-1} \to A^{n+1}).
    \]
    Next, we see that by the universal property of $\coker d^{n-1}$, morphism $f^n - \delta^{n-1}h^n \colon A^n \to I^n$ induces a morphism ${g \colon \coker d^{n-1} \to I^n}$, because
    \begin{align*}
        (f^n - \delta^{n-1}h^n)d^{n-1} &= f^nd^{n-1} - \delta^{n-1}h^nd^{n-1} \\
        &= f^nd^{n-1} + \delta^{n-1}\delta^{n-2}h^{n-1} - \delta^{n-1}f^{n-1} \\
        &= 0.
    \end{align*}
    The second equality follows from the inductive hypothesis and the third one from $f$ being a chain map and $\delta^{n-1}$, $\delta^{n-2}$ being a differentials. 

    Lastly, for $I^n$ is injecive and $j$ mono, there is a morphism $h^{n+1} \colon A^{n+1} \to I^n$, satisfying $h^{n+1} \circ j = g$, thus after precomposing both sides with $e$, we arrive at $h^{n+1}d^n = f^n - \delta^{n-1}h^n$, which can be rewritten as
    \[
        f^n = h^{n+1}d^n + \delta^{n-1}h^n.
        \qedhere  
    \]
\end{proof}

\begin{lemma}
    \label{D+ K+ aux lemma 1}
    % Let $A^\bullet$ and $B^\bullet$ be complexes belonging to $\kplus{\A}$ and $I^\bullet$ a complex of injectives from $\kplus{\J}$.
    Let $A^\bullet$ and $B^\bullet$ belong to $\kplus{\A}$ and $I^\bullet \in \kplus{\J}$. Let $f \colon B^\bullet \to A^\bullet$ be a quasi-isomorphism, then 
    \[
        \Hom_{\kplus{\A}}(A^\bullet, I^\bullet) \xrightarrow{ \ f^* \ } \Hom_{\kplus{\A}}(B^\bullet, I^\bullet) 
    \]
    is an isomorphism of $k$-modules.
\end{lemma}

\begin{proof}
    In $\kplus{\A}$ we have a distinguished triangle $B^\bullet \to A^\bullet \to C(f)^\bullet \to B[1]^\bullet$, which induces a long exact sequence (of $k$-modules)
    \begin{multline*}
        \cdots \longrightarrow \Hom_{\kplus{\A}}(C(f)^\bullet, I^\bullet) \longrightarrow \\
        \longrightarrow \Hom_{\kplus{\A}}(A^\bullet, I^\bullet) \xrightarrow{ \ f^* \ }
        \Hom_{\kplus{\A}}(B^\bullet, I^\bullet) \longrightarrow \\
        \longrightarrow \Hom_{\kplus{\A}}(C(f)[-1]^\bullet, I^\bullet) \longrightarrow \cdots,
    \end{multline*}
    since $\Hom_{\kplus{\A}}(-, I^\bullet)$ is a cohomological functor by example \ref{partial homs are cohomological functors}. As $f$ is a quasi-isomorphism, the cone $C(f)^\bullet$ is acyclic along with all its shifts, thus showing that
    \[
        \Hom_{\kplus{\A}}(C(f)[i]^\bullet, I^\bullet) = 0,
    \]
    for all $i \in \Z$ by Lemma \ref{acyclic to injective is nulhomotopic}, since $I^\bullet$ is a complex of injectives. From the long exact sequence it then clearly follows that $f^*$ is an isomorphism. 
\end{proof}

\begin{lemma}
    \label{D+ K+ aux lemma 2}
    Let $A^\bullet$ belong to $\kplus{\A}$ and $I^\bullet$ to $\kplus{\J}$. Then the morphism action 
    \begin{equation}
        \label{eq: morphism action of Q}
        \Hom_{\kplus{\A}}(A^\bullet, I^\bullet) \to \Hom_{\derplus{\A}}(A^\bullet, I^\bullet)
    \end{equation}
    of the localization functor $Q$ is an isomorphism of $k$-modules. 
\end{lemma}

\begin{proof}
    We already know this is a $k$-module homomorphism, thus it is enough to show that it is bijective. This is accomplished by constructing its inverse. Let $\phi \colon A^\bullet \to I^\bullet$ be a morphism in $\derplus{\A}$ and let $A^\bullet \xleftarrow{ \ s \ } B^\bullet \xrightarrow{ \ f \ } I^\bullet$ be its left roof representative. The morphism $s$ being a quasi-isomorphism implies that $s^* \colon \Hom_{\kplus{\A}}(A^\bullet, I^\bullet) \to \Hom_{\kplus{\A}}(B^\bullet, I^\bullet)$ is bijective by lemma \ref{D+ K+ aux lemma 1}, so there is a unique morphism $g\colon A^\bullet \to I^\bullet$, such that $f = g \circ s$ in $\kplus{\A}$. The inverse to \eqref{eq: morphism action of Q} is then defined by sending $\phi$ to $g$. 
    
    First let's argue why this map is well defined \ie independent of the choice of left roof representative for $\phi$. We pick two left roof representatives $A^\bullet \xleftarrow{ \ s_0 \ } B_0^\bullet \xrightarrow{ \ f_0 \ } I^\bullet$ and $A^\bullet \xleftarrow{ \ s_1 \ } B_1^\bullet \xrightarrow{ \ f_1 \ } I^\bullet$ for $\phi$ and let $g_0$ and $g_1$ be such that $f_i = g_i \circ s_i$ for both $i$. As the roofs are equivalent there are quasi-isomorphisms $t_0 \colon C^\bullet \to B^\bullet_0$ and $t_1 \colon C^\bullet \to B^\bullet_1$, which fit into the following commutative diagram.
    \[\begin{tikzcd}[row sep = small, column sep = small]
        % https://q.uiver.app/#q=WzAsNSxbMCwyLCJBIl0sWzEsMSwiQl8wIl0sWzQsMiwiSSJdLFszLDEsIkJfMSJdLFsyLDAsIkMiXSxbMSwyLCJmXzAiLDIseyJsYWJlbF9wb3NpdGlvbiI6NjB9XSxbMywwLCJzXzEiLDAseyJsYWJlbF9wb3NpdGlvbiI6NjB9XSxbMywyLCJmXzEiXSxbNCwzLCJ0XzEiXSxbMSwwLCJzXzAiLDJdLFs0LDEsInRfMCIsMl1d
        && C^\bullet \\
        & {B^\bullet_0} && {B^\bullet_1} \\
        A^\bullet &&&& I^\bullet
        \arrow["{t_0}"', from=1-3, to=2-2]
        \arrow["{t_1}", from=1-3, to=2-4]
        \arrow["{s_0}"', from=2-2, to=3-1]
        \arrow["{s_1}"{pos=0.6}, from=2-4, to=3-1]
        \arrow["{f_1}", from=2-4, to=3-5]
        \arrow["{f_0}"'{pos=0.6}, from=2-2, to=3-5, crossing over]
    \end{tikzcd}\]
    Therefore $g_0s_0t_0 = f_0t_0 = f_1t_1 = g_1s_1t_1 = g_1s_0t_0$. As $s_0t_0$ is a quasi-isomorphism, we see that $g_0 = g_1$ by applying lemma \ref{D+ K+ aux lemma 1}.
    %  follows by applying lemma \ref{D+ K+ aux lemma 1} as the composition $s_0t_0$ is a quasi-isomorphism.
    
    % The following diagram shows why the constructed map is a left inverse to \eqref{eq: morphism action of Q}
    % \[
    %     \left[ A^\bullet \xleftarrow{ \ s \ } B^\bullet \xrightarrow{ \ f \ } I^\bullet \right] \longmapsto 
    %     \left( g\colon A^\bullet \to I^\bullet\right) \longmapsto 
    %     \left[A^\bullet \xleftarrow{ \ \id{A^\bullet} } A^\bullet \xrightarrow{ \ g \ } I^\bullet\right] = 
    %     \left[ A^\bullet \xleftarrow{ \ s \ } B^\bullet \xrightarrow{ \ f \ } I^\bullet \right]
    % \]

    % \begin{align*}
    %     \Hom_{\derplus{\A}}(A^\bullet, I^\bullet) \to \Hom_{\kplus{\J}}(A^\bullet, I^\bullet) \to \Hom_{\derplus{\A}}(A^\bullet, I^\bullet) \\
    %     \left[ A^\bullet \xleftarrow{ \ s \ } B^\bullet \xrightarrow{ \ f \ } I^\bullet \right] \longmapsto 
    %     \left( g\colon A^\bullet \to I^\bullet\right) \longmapsto 
    %     \left[A^\bullet \xleftarrow{ \ \id{A^\bullet} } A^\bullet \xrightarrow{ \ g \ } I^\bullet\right] = 
    %     \left[ A^\bullet \xleftarrow{ \ s \ } B^\bullet \xrightarrow{ \ f \ } I^\bullet \right]
    % \end{align*}

%     \[\begin{tikzcd}[column sep = small , row sep = small]
%         % https://q.uiver.app/#q=WzAsNixbMCwwLCJcXEhvbV97RF4rKFxcQSl9KEFeXFxidWxsZXQsIEleXFxidWxsZXQpIl0sWzQsMCwiXFxIb21fe0ReKyhcXEEpfShBXlxcYnVsbGV0LCBJXlxcYnVsbGV0KSJdLFsyLDAsIlxcSG9tX3tLXisoXFxKKX0oQV5cXGJ1bGxldCwgSV5cXGJ1bGxldCkiXSxbMCwxLCJcXGxlZnRbIEFeXFxidWxsZXQgXFx4bGVmdGFycm93eyBcXCBzIFxcIH0gQl5cXGJ1bGxldCBcXHhyaWdodGFycm93eyBcXCBmIFxcIH0gSV5cXGJ1bGxldCBcXHJpZ2h0XSJdLFsyLDEsIlxcbGVmdCggZ1xcY29sb24gQV5cXGJ1bGxldCBcXHRvIEleXFxidWxsZXRcXHJpZ2h0KSJdLFs0LDEsIlxcbGVmdFtBXlxcYnVsbGV0IFxceGxlZnRhcnJvd3sgXFwgXFxpZHtBXlxcYnVsbGV0fSB9IEFeXFxidWxsZXQgXFx4cmlnaHRhcnJvd3sgXFwgZyBcXCB9IEleXFxidWxsZXRcXHJpZ2h0XSA9ICAgICAgICAgIFxcbGVmdFsgQV5cXGJ1bGxldCBcXHhsZWZ0YXJyb3d7IFxcIHMgXFwgfSBCXlxcYnVsbGV0IFxceHJpZ2h0YXJyb3d7IFxcIGYgXFwgfSBJXlxcYnVsbGV0IFxccmlnaHRdIl0sWzAsMl0sWzIsMV0sWzMsNCwiIiwyLHsic3R5bGUiOnsidGFpbCI6eyJuYW1lIjoibWFwcyB0byJ9fX1dLFs0LDUsIiIsMix7InN0eWxlIjp7InRhaWwiOnsibmFtZSI6Im1hcHMgdG8ifX19XV0=
% 	{\Hom_{\derplus{\A}}(A^\bullet, I^\bullet)} && {\Hom_{\kplus{\J}}(A^\bullet, I^\bullet)} && {\Hom_{\derplus{\A}}(A^\bullet, I^\bullet)} \\
% 	{\left[ A^\bullet \xleftarrow{ \ s \ } B^\bullet \xrightarrow{ \ f \ } I^\bullet \right]} && {\left( g\colon A^\bullet \to I^\bullet\right)} && {\left[A^\bullet \xleftarrow{ \ \id{A^\bullet} } A^\bullet \xrightarrow{ \ g \ } I^\bullet\right] =          \left[ A^\bullet \xleftarrow{ \ s \ } B^\bullet \xrightarrow{ \ f \ } I^\bullet \right]}
% 	\arrow[from=1-1, to=1-3]
% 	\arrow[from=1-3, to=1-5]
% 	\arrow[maps to, from=2-1, to=2-3]
% 	\arrow[maps to, from=2-3, to=2-5]
% \end{tikzcd}\]

    % and why it is also a right inverse to \eqref{eq: morphism action of Q}
    % \[
    %     \left(  g\colon A^\bullet \to I^\bullet \right) \longmapsto
    %     \left[A^\bullet \xleftarrow{ \ \id{A^\bullet} } A^\bullet \xrightarrow{ \ g \ } I^\bullet\right] \longmapsto
    %     \left(  g\colon A^\bullet \to I^\bullet \right).
    %     \qedhere
    % \]


    % https://q.uiver.app/#q=WzAsNyxbMCwwLCJcXEhvbV97RF4rKFxcQSl9KEFeXFxidWxsZXQsIEleXFxidWxsZXQpIl0sWzIsMCwiXFxIb21fe0ReKyhcXEEpfShBXlxcYnVsbGV0LCBJXlxcYnVsbGV0KSJdLFsxLDAsIlxcSG9tX3tLXisoXFxKKX0oQV5cXGJ1bGxldCwgSV5cXGJ1bGxldCkiXSxbMCwxLCJcXGxlZnRbIEFeXFxidWxsZXQgXFx4bGVmdGFycm93eyBcXCBzIFxcIH0gQl5cXGJ1bGxldCBcXHhyaWdodGFycm93eyBcXCBmIFxcIH0gSV5cXGJ1bGxldCBcXHJpZ2h0XSJdLFsxLDEsIlxcbGVmdCggZ1xcY29sb24gQV5cXGJ1bGxldCBcXHRvIEleXFxidWxsZXRcXHJpZ2h0KSJdLFsyLDEsIlxcbGVmdFtBXlxcYnVsbGV0IFxceGxlZnRhcnJvd3sgXFwgXFxpZHtBXlxcYnVsbGV0fSB9IEFeXFxidWxsZXQgXFx4cmlnaHRhcnJvd3sgXFwgZyBcXCB9IEleXFxidWxsZXRcXHJpZ2h0XSJdLFszLDEsIlxcbGVmdFsgQV5cXGJ1bGxldCBcXHhsZWZ0YXJyb3d7IFxcIHMgXFwgfSBCXlxcYnVsbGV0IFxceHJpZ2h0YXJyb3d7IFxcIGYgXFwgfSBJXlxcYnVsbGV0IFxccmlnaHRdIl0sWzAsMl0sWzIsMV0sWzMsNCwiIiwyLHsic3R5bGUiOnsidGFpbCI6eyJuYW1lIjoibWFwcyB0byJ9fX1dLFs0LDUsIiIsMix7InN0eWxlIjp7InRhaWwiOnsibmFtZSI6Im1hcHMgdG8ifX19XSxbNSw2LCIiLDIseyJsZXZlbCI6Miwic3R5bGUiOnsiaGVhZCI6eyJuYW1lIjoibm9uZSJ9fX1dXQ==
% \[\begin{tikzcd}[row sep = 0.5em, column sep = 0.75em]
% 	{\Hom_{\derplus{\A}}(A^\bullet, I^\bullet)} & {\Hom_{\kplus{\J}}(A^\bullet, I^\bullet)} & {\Hom_{\derplus{\A}}(A^\bullet, I^\bullet)} \\
% 	{\left[ A^\bullet \xleftarrow{ \ s \ } B^\bullet \xrightarrow{ \ f \ } I^\bullet \right]} & {\left( g\colon A^\bullet \to I^\bullet\right)} & {\left[A^\bullet \xleftarrow{ \ \id{A^\bullet} } A^\bullet \xrightarrow{ \ g \ } I^\bullet\right]} & {\left[ A^\bullet \xleftarrow{ \ s \ } B^\bullet \xrightarrow{ \ f \ } I^\bullet \right]}
% 	\arrow[from=1-1, to=1-2]
% 	\arrow[from=1-2, to=1-3]
% 	\arrow[maps to, from=2-1, to=2-2]
% 	\arrow[maps to, from=2-2, to=2-3]
% 	\arrow[equals, from=2-3, to=2-4]
% \end{tikzcd}\]

% \[\begin{tikzcd}[row sep = 0.5em, column sep = 0.75em]
%     % https://q.uiver.app/#q=WzAsNixbMCwwLCJcXEhvbV97RF4rKFxcQSl9KEFeXFxidWxsZXQsIEleXFxidWxsZXQpIl0sWzIsMCwiXFxIb21fe0ReKyhcXEEpfShBXlxcYnVsbGV0LCBJXlxcYnVsbGV0KSJdLFsxLDAsIlxcSG9tX3tLXisoXFxKKX0oQV5cXGJ1bGxldCwgSV5cXGJ1bGxldCkiXSxbMCwxLCJcXGxlZnRbIEFeXFxidWxsZXQgXFx4bGVmdGFycm93eyBcXCBzIFxcIH0gQl5cXGJ1bGxldCBcXHhyaWdodGFycm93eyBcXCBmIFxcIH0gSV5cXGJ1bGxldCBcXHJpZ2h0XSJdLFsxLDEsIlxcbGVmdCggZ1xcY29sb24gQV5cXGJ1bGxldCBcXHRvIEleXFxidWxsZXRcXHJpZ2h0KSJdLFsyLDEsIlxcbGVmdFtBXlxcYnVsbGV0IFxceGxlZnRhcnJvd3sgXFwgXFxpZHtBXlxcYnVsbGV0fSB9IEFeXFxidWxsZXQgXFx4cmlnaHRhcnJvd3sgXFwgZyBcXCB9IEleXFxidWxsZXRcXHJpZ2h0XSJdLFswLDJdLFsyLDFdLFszLDQsIiIsMix7InN0eWxlIjp7InRhaWwiOnsibmFtZSI6Im1hcHMgdG8ifX19XSxbNCw1LCIiLDIseyJzdHlsZSI6eyJ0YWlsIjp7Im5hbWUiOiJtYXBzIHRvIn19fV1d
% 	{\Hom_{\derplus{\A}}(A^\bullet, I^\bullet)} & {\Hom_{\kplus{\J}}(A^\bullet, I^\bullet)} & {\Hom_{\derplus{\A}}(A^\bullet, I^\bullet)} \\
% 	{\left[ A^\bullet \xleftarrow{ \ s \ } B^\bullet \xrightarrow{ \ f \ } I^\bullet \right]} & {\left( g\colon A^\bullet \to I^\bullet\right)} & {\left[A^\bullet \xleftarrow{ \ \id{A^\bullet} } A^\bullet \xrightarrow{ \ g \ } I^\bullet\right]}
% 	\arrow[from=1-1, to=1-2]
% 	\arrow[from=1-2, to=1-3]
% 	\arrow[maps to, from=2-1, to=2-2]
% 	\arrow[maps to, from=2-2, to=2-3]
% \end{tikzcd}\]


Following the diagram below, it is clear why the constructed map is a right inverse to \eqref{eq: morphism action of Q} 
\[\begin{tikzcd}[row sep = 0.5em, column sep = 0.75em]
    % https://q.uiver.app/#q=WzAsNixbMSwwLCJcXEhvbV97RF4rKFxcQSl9KEFeXFxidWxsZXQsIEleXFxidWxsZXQpIl0sWzIsMCwiXFxIb21fe0teKyhcXEopfShBXlxcYnVsbGV0LCBJXlxcYnVsbGV0KSJdLFsyLDEsIlxcbGVmdCggZ1xcY29sb24gQV5cXGJ1bGxldCBcXHRvIEleXFxidWxsZXRcXHJpZ2h0KSJdLFswLDAsIlxcSG9tX3tLXisoXFxKKX0oQV5cXGJ1bGxldCwgSV5cXGJ1bGxldCkiXSxbMCwxLCJcXGxlZnQoIGdcXGNvbG9uIEFeXFxidWxsZXQgXFx0byBJXlxcYnVsbGV0XFxyaWdodCkiXSxbMSwxLCJcXGxlZnRbQV5cXGJ1bGxldCBcXHhsZWZ0YXJyb3d7IFxcIFxcaWR7QV5cXGJ1bGxldH0gfSBBXlxcYnVsbGV0IFxceHJpZ2h0YXJyb3d7IFxcIGcgXFwgfSBJXlxcYnVsbGV0XFxyaWdodF0iXSxbMCwxXSxbNCw1LCIiLDIseyJzdHlsZSI6eyJ0YWlsIjp7Im5hbWUiOiJtYXBzIHRvIn19fV0sWzUsMiwiIiwyLHsic3R5bGUiOnsidGFpbCI6eyJuYW1lIjoibWFwcyB0byJ9fX1dLFszLDBdXQ==
	{\Hom_{\kplus{\A}}(A^\bullet, I^\bullet)} & {\Hom_{\derplus{\A}}(A^\bullet, I^\bullet)} & {\Hom_{\kplus{\A}}(A^\bullet, I^\bullet)} \\
	{\left( g\colon A^\bullet \to I^\bullet\right)} & {\left[A^\bullet \xleftarrow{ \ \id{A^\bullet} } A^\bullet \xrightarrow{ \ g \ } I^\bullet\right]} & {\left( g\colon A^\bullet \to I^\bullet\right).}
	\arrow["Q", from=1-1, to=1-2]
	\arrow[from=1-2, to=1-3]
	\arrow[maps to, from=2-1, to=2-2]
	\arrow[maps to, from=2-2, to=2-3]
\end{tikzcd}\]
Lastly we see that it is also a left inverse by the following diagram \[\begin{tikzcd}[row sep = 0.5em, column sep = 0.75em]
    % https://q.uiver.app/#q=WzAsNixbMCwwLCJcXEhvbV97RF4rKFxcQSl9KEFeXFxidWxsZXQsIEleXFxidWxsZXQpIl0sWzIsMCwiXFxIb21fe0ReKyhcXEEpfShBXlxcYnVsbGV0LCBJXlxcYnVsbGV0KSJdLFsxLDAsIlxcSG9tX3tLXisoXFxKKX0oQV5cXGJ1bGxldCwgSV5cXGJ1bGxldCkiXSxbMCwxLCJcXGxlZnRbIEFeXFxidWxsZXQgXFx4bGVmdGFycm93eyBcXCBzIFxcIH0gQl5cXGJ1bGxldCBcXHhyaWdodGFycm93eyBcXCBmIFxcIH0gSV5cXGJ1bGxldCBcXHJpZ2h0XSJdLFsxLDEsIlxcbGVmdCggZ1xcY29sb24gQV5cXGJ1bGxldCBcXHRvIEleXFxidWxsZXRcXHJpZ2h0KSJdLFsyLDEsIlxcbGVmdFtBXlxcYnVsbGV0IFxceGxlZnRhcnJvd3sgXFwgXFxpZHtBXlxcYnVsbGV0fSB9IEFeXFxidWxsZXQgXFx4cmlnaHRhcnJvd3sgXFwgZyBcXCB9IEleXFxidWxsZXRcXHJpZ2h0XSJdLFswLDJdLFsyLDFdLFszLDQsIiIsMix7InN0eWxlIjp7InRhaWwiOnsibmFtZSI6Im1hcHMgdG8ifX19XSxbNCw1LCIiLDIseyJzdHlsZSI6eyJ0YWlsIjp7Im5hbWUiOiJtYXBzIHRvIn19fV1d
	{\Hom_{\derplus{\A}}(A^\bullet, I^\bullet)} & {\Hom_{\kplus{\A}}(A^\bullet, I^\bullet)} & {\Hom_{\derplus{\A}}(A^\bullet, I^\bullet)} \\
	{\left[ A^\bullet \xleftarrow{ \ s \ } B^\bullet \xrightarrow{ \ f \ } I^\bullet \right]} & {\left( g\colon A^\bullet \to I^\bullet\right)} & {\left[A^\bullet \xleftarrow{ \ \id{A^\bullet} } A^\bullet \xrightarrow{ \ g \ } I^\bullet\right]}
	\arrow[from=1-1, to=1-2]
	\arrow["Q", from=1-2, to=1-3]
	\arrow[maps to, from=2-1, to=2-2]
	\arrow[maps to, from=2-2, to=2-3]
\end{tikzcd}\]
along with observing that $A^\bullet \xleftarrow{ \ \id{A^\bullet} } A^\bullet \xrightarrow{ \ g \ } I^\bullet$ and $A^\bullet \xleftarrow{ \ s \ } B^\bullet \xrightarrow{ \ f \ } I^\bullet$ are equivalent roofs.
\end{proof}

\begin{proof}[Proof of theorem \ref{D+ = K+ and injectives}]
    % We start of with a

    % \noindent
    % % \textbf{Preliminary claim.} Let $A^\bullet$ and $B^\bullet$ be complexes belonging to $\kplus{\A}$ and $I^\bullet$ a complex of injectives from $\kplus{\J}$.
    % \textbf{Preliminary claim.} Let $A^\bullet$ and $B^\bullet$ belong to $\kplus{\A}$ and $I^\bullet \in \kplus{\J}$. Let $f \colon B^\bullet \to A^\bullet$ be a quasi-isomorphism, then 
    % \[
    %     \Hom_{\kplus{\A}}(A^\bullet, I^\bullet) \xrightarrow{ \ - \circ f \ } \Hom_{\kplus{\A}}(B^\bullet, I^\bullet) 
    % \]
    % is a bijection.
    As $\kplus{\J} \to \derplus{\A}$ is a triangulated functor, we only need to show that it is fully faithful and essentially surjective. Let $I^\bullet$ and $J^\bullet$ be objects of $\kplus{\J}$. Then
    \[
        \Hom_{\kplus{\J}}(I^\bullet, J^\bullet) \xrightarrow{\ = \ } \Hom_{\kplus{\A}}(I^\bullet, J^\bullet) \xrightarrow{\ \eqref{D+ K+ aux lemma 2} \ } \Hom_{\derplus{\A}}(I^\bullet, J^\bullet)
    \]
    is a bijection, showing fully faithfulness (the last map is a bijection by Lemma \ref{D+ K+ aux lemma 2}). Essential surjectivity is clear from the existence of injective resolutions (\cf Proposition \ref{injective resolution}) because quasi-isomorphisms now play the role of isomorphisms in $\derplus{\A}$.
\end{proof}

Within the scope of an abelian category $\A$ with enough injectives, theorem \ref{D+ = K+ and injectives} now offers some well known and classical results of homological algebra. 

\begin{corollary}
    Let $\A$ be an abelian category with enough injectives.
    \begin{enumerate}[label = (\roman*)]
        \item{Any two injective resolutions of a complex $A^\bullet$ of $\kplus{\A}$ are homotopically equivalent \ie isomorphic in $\kplus{\A}$}
        \item{For any morphism of complexes $f \colon A^\bullet \to B^\bullet$ in $\kplus{\A}$ and any injective resolutions $A^\bullet \to I^\bullet_A$ and $B^\bullet \to I^\bullet_B$ of $A^\bullet$ and $B^\bullet$ respectively, there is a unique up to homotopy chain map $I^\bullet_A \to I^\bullet_B$ for which the square below commutes up to homotopy.
        \[\begin{tikzcd}[row sep = normal]
            % https://q.uiver.app/#q=WzAsNCxbMiwwLCJBXlxcYnVsbGV0Il0sWzIsMiwiQl5cXGJ1bGxldCJdLFswLDAsIkleXFxidWxsZXRfQSJdLFswLDIsIkleXFxidWxsZXRfQiJdLFswLDEsImYiXSxbMCwyXSxbMSwzXSxbMiwzLCIiLDAseyJzdHlsZSI6eyJib2R5Ijp7Im5hbWUiOiJkYXNoZWQifX19XV0=
            {I^\bullet_A} && {A^\bullet} \\
            \\
            {I^\bullet_B} && {B^\bullet}
            \arrow[dashed, from=1-1, to=3-1]
            \arrow[from=1-3, to=1-1]
            \arrow["f", from=1-3, to=3-3]
            \arrow[from=3-3, to=3-1]
        \end{tikzcd}\]
        }
    \end{enumerate}
\end{corollary}

% \info{add a little argument for why injective res. are unique up to htpy.}

% We leave out the proof for why this operation is associative, but refer the intersted reader to \cite{milicic-dercat} or \cite[\S 7]{kashiwara2006categories} for a more detailed and general approach astablishing the theory of localization of categories.
% \begin{remark}
%     This lemma reminds us of the Ore condition in the context of localization in non-commutative algebra as so analogously there is also a dual version of the lemma. 
% \end{remark}



% ============================================================== %

\subsection{Derived functors}

The main goal of this section is to assign to a sensible additive functor $F \colon \A \to \B$ between two abelian categories an appropriate \emph{triangulated} functor on the level of derived categories, called a \emph{derived functor}. We will first deal with the easiest case, when $F$ is exact and then move on to the case, where $F$ will only be assumed to be left or right exact and instead require the domain category $\A$ to posses some nice properties (\eg contain enough injectives or contain an $F$-adapted class of objects). In practise the additional conditions $\A$ is required to satisfy are not very restrictive. 

\subsubsection{Derived functors of exact functors}
\label{subsection: Derived functors of exact functors}

We start with some establishing lemmas.

\begin{lemma}
    \label{qis iff cone acyclic}
    Let $f\colon A^\bullet \to B^\bullet$ be a chain map. Then $f$ is a quasi-isomorphism if and only if its associated cone $C(f)^\bullet$ is acyclic. 
\end{lemma}

\info{proof can reference a proposition from section on triangulated cats...}

\begin{proof}
    One only needs to consider the distinguished triangle $A^\bullet \xrightarrow{f} B^\bullet \to C(f)^\bullet \to A[1]^\bullet$ in $\kcat{\A}$ and its associated long exact sequence in cohomology.
\end{proof}

\begin{lemma}
    \label{F preserves qis iff preserves acyclic}
    Let $F\colon \kcat{\A} \to \kcat{\B}$ be a triangulated functor, then the following two conditions are equivalent. 
    % then $F$ preserves quasi-isomorphisms if and only if it preserves acyclic complexes.
    % \begin{enumerate}
    %     \item[\emph{(i)}] $F$ preserves quasi-isomorphisms
    %     \item[\emph{(ii)}] $F$ preserves acyclic complexes.
    % \end{enumerate}
    \begin{enumerate}[label = (\roman*)]
        \item For every quasi-isomorphism $s$ in $\kcat{\A}$, $F(s)$ is a quasi-isomorphism in $\kcat{\B}$.
        \item For every acyclic complex $A^\bullet$ of $\kcat{\A}$, $F(A^\bullet)$ is acyclic.
    \end{enumerate}
\end{lemma}

\begin{proof}
    $(\Rightarrow)$ Let $A^\bullet$ be an acyclic complex. Then the zero morphism $A^\bullet \to 0$ is a quasi-isomorphism and is by assumption mapped to a quasi-isomorphism $F(A^\bullet) \to 0$ by $F$. This means $F(A^\bullet)$ is acyclic.
    
    $(\Leftarrow)$ Let $f\colon A^\bullet \to B^\bullet$ be a quasi-isomorphism. The image of a distinguished triangle $A^\bullet \to B^\bullet \to C(f)^\bullet \to A^\bullet[1]$ in $\kcat{\A}$ by the triangulated functor $F$ is a distinguished triangle $FA^\bullet \to FB^\bullet \to F(C(f)^\bullet) \to F(A^\bullet)[1]$ in $\kcat{\B}$. By Lemma \ref{qis iff cone acyclic} $C(f)^\bullet$ is acyclic, implying $F(C(f)^\bullet)$ is acyclic by the assumption (ii). Hence again by Lemma \ref{qis iff cone acyclic} $F(f)$ is a quasi-isomorphism.
    % The cone $F(C(f)^\bullet)$ is then acyclic, because $C(f)^\bullet$ was acyclic by lemma \ref{qis iff cone acyclic}, thus showing $Ff$ is a quasi-isomorphism.  
\end{proof}

\begin{proposition}
    \label{Exact functor induces a derived functor}
    Let $F\colon \kcat{\A} \to \kcat{\B}$ be a triangulated functor and assume it satisfies one of the equivalent conditions of Lemma \ref{F preserves qis iff preserves acyclic}. Then there exists a triangulated functor $\dercat{\A} \to \dercat{\B}$ on the level of derived categories, for which the following diagram of triangulated functors commutes.
    \begin{equation}
        \label{eq: exact functor derives}
        \begin{tikzcd}[column sep = 1.5em]
        % https://q.uiver.app/#q=WzAsNCxbMCwwLCJLKFxcQSkiXSxbMiwwLCJLKFxcQikiXSxbMCwyLCJEKFxcQSkiXSxbMiwyLCJEKFxcQikiXSxbMCwxLCJGIl0sWzAsMl0sWzIsM10sWzEsM11d
        {\kcat{\A}} && {\kcat{\B}} \\
        \\
        {\dercat{\A}} && {\dercat{\B}}
        \arrow["F", from=1-1, to=1-3]
        \arrow["{Q_\A}"', from=1-1, to=3-1]
        \arrow["{Q_\B}", from=1-3, to=3-3]
        \arrow[from=3-1, to=3-3]
        \end{tikzcd}
    \end{equation}
\end{proposition}

\begin{proof}
    The claim follows from the universal property of $\dercat{\A}$ (\cf proposition \ref{universal property for D(A) proposition}), as $Q_\B \circ F$ sends quasi-isomorphisms of $\kcat{\A}$ to isomorphisms of $\dercat{\B}$.
\end{proof}

% The induced functor $\dercat{\A} \to \dercat{\B}$ will be called the \emph{derived functor of $F$}, often being denoted with $F$ as well.

Every $k$-linear functor $F\colon \A \to \B$ induces a triangulated functor on the level of homotopy categories $\kcat{\A} \to \kcat{\B}$, defined by the assignment.
\begin{center}
    \begin{tabular}{r l}
        \textsl{Objects:} & $A^\bullet \longmapsto \left(\cdots \to F(A^i) \xrightarrow{F(d^i)} F(A^{i+1}) \to \cdots \right)$. \\
        \textsl{Morphisms:} & $\left(f\colon A^\bullet \to B^\bullet\right) \longmapsto \left(F(f^i) \colon F(A^i) \to F(B^i)\right)_{i \in \Z}$.
    \end{tabular}
    \end{center}
 If moreover $F$ is assumed to be exact, then $\kcat{\A} \to \kcat{\B}$ satisfies the assumptions of lemma \ref{Exact functor induces a derived functor} and thus induces a triangulated functor $\dercat{\A} \to \dercat{\B}$ on the level of derived categories. We call this the \emph{derived functor of $F$} and often denote it with $F$ as well. We emphasise again that this convention will be used only in the case when $F\colon \A \to \B$ is an \emph{exact} functor between abelian categories.


% ============================================================== %

\subsubsection{Derived functors of left (right) exact functors}
% \subsubsection*{$F$-adapted classes}

Assume $F \colon \A \to \B$ is a l\emph{eft} exact functor between abelian categories $\A$ and $\B$. The theory for right exact functors is obtained in parallel by dualizing everything. A crucial downside of functors, which are no longer exact, is that their induced functors on the category of complexes no longer satisfy the two equivalent conditions of Lemma \ref{F preserves qis iff preserves acyclic}, so the same idea with the universal property of Definition \ref{universal property of D(A)}, which worked in the case of exact functors, cannot be applied in this case. Instead we will present two other more restrictive methods of constructing right derived functors, which are the following:
\begin{enumerate}[label = $\triangleright$]
    \item Assuming $\A$ has enough injectives, construct a \emph{right derived functor} 
    \[
        {\rderived{}{F} \colon \derplus{\A} \to \derplus{\B}} 
    \]
    for any left exact functor $F \colon \A \to \B$.
    \item Fix a left exact functor $F \colon \A \to \B$ and assume the category $\A$ contains a so-called \emph{$F$-adapted class} then construct a \emph{right derived functor} $\rderived{}{F} \colon \derplus{\A} \to \derplus{\B}$.
\end{enumerate}
These methods are however no less restrictive than the case of exact functors with our applications in mind.
These methods are however no less restrictive than the case of exact functors, when viewed through the perspective of our applications later on. 
Lastly we note, that the first method is in some sense a special instance of the second one, which will also be explained in this subsection.
% considered in the context of our applications later on.

\vspace{0.3 cm}

\noindent
\textsc{Method 1.} 
\label{method 1}
Let $F\colon \A \to \B$ be a left exact functor between abelian categories with $\A$ having enough injectives. By Theorem \ref{D+ = K+ and injectives} we know that the composition of $\kplus{\J} \to \kplus{\A}$ followed by the localization functor $Q_\A \colon \kplus{\A} \to \derplus{\A}$ is an equivalence of triangulated categories. The \emph{right derived functor of $F$} is then defined to be the composition
\begin{equation}
    \label{eq: construction of RF with injectives}
    \rderived{}{F}\colon \quad \derplus{\A} \xrightarrow{ \ \natiso \ } \kplus{\J} \hookrightarrow \kplus{\A} \xrightarrow{ \ F \ } \kplus{\B} \xrightarrow{\ Q_\B \ } \derplus{\B}
\end{equation}
The functor $\rderived{}{F}$ is clearly triangulated for it is a composition of triangulated functors. Note that in this case the functor $\rderived{}{F}$ does not necessarily make the square \eqref{eq: exact functor derives} commutative.

We will use this construction many times through the course of the thesis, so we reiterate the above definition by providing a simple recipe for computing the image of $\rderived{}{F}$ on complexes $A^\bullet$ of $\derplus{\A}$.   
% a complex $A^\bullet$ via the functor $\rderived{}{F}$.
% and conceptually computing the image of a right derived functor $\rderived{}{F}$ on a complex $A^\bullet$ from $\kplus{\A}$ will be achieved using the next two steps. 
\begin{enumerate}
    \item Pick an injective resolution $A^\bullet \to I_A^\bullet$ of $A^\bullet$.
    \item Apply the functor $F$ to the complex $I^\bullet_A$ term-wise to obtain $F(I^\bullet_A)$. As a complex
    \[
        \rderived{}{F}(A^\bullet) = F(I^\bullet_A).
    \]
\end{enumerate}

\begin{remark}
    \label{Deriving homotopy functors}
    In the slightly more general situation, if one is already given a triangulated functor $F \colon \kplus{\A} \to \kplus{\B}$ at the level of homotopy categories and assumes that $\A$ contains enough injectives, procedure \eqref{eq: construction of RF with injectives} again yields a right derived functor $\rderived{}{F} \colon \derplus{\A} \to \derplus{\B}$. This remark will be esspecially useful when considering the \emph{differentialy graded inner} $\Hom^\bullet$-functor of subsection \ref{subsection: Ext functors}, which does not appear as a functor induced by some left exact functor $\A \to \B$ on the level of abelian categories.
\end{remark}

\begin{remark}
    When considering a right exact functor $F \colon \A \to \B$, we instead assume the category $\A$ contains enough projectives. In this case we use the dualized version of Theorem \ref{D+ = K+ and injectives}, stating that $\kminus{\mathcal P} \natiso \derminus{\A}$, where $\mathcal P$ denotes the full subcategory of $\A$, spanned on projective objects of $\A$. The left derived functor is then defined as the composition
    % \[
    %     \lderived{}{F} \colon \derminus{\A} \to \derminus{\B}.
    % \]
    \[
        \lderived{}{F}\colon \quad \derminus{\A} \xrightarrow{ \ \natiso \ } \kminus{\mathcal P} \hookrightarrow \kminus{\A} \xrightarrow{ \ F \ } \kminus{\B} \xrightarrow{\ Q_\B \ } \derminus{\B}.
    \]
    Again, as a composition of triangulated functors, we see that $\lderived{}{F}$ is triangulated as well.
\end{remark}

\noindent
\textsc{Method 2.}
Unfortunately some of our categories will \emph{not} contain enough injectives, as can already be seen with the category of coherent sheaves $\coherent{X}$ of a scheme $X$ 
% differing from a point.
, which is not a point. 
In this case \textsc{Method 1} of constructing right derived functors will not work. Luckily however, there exists another way of obtaining a right derived functor $\rderived{}{F} \colon \derplus{\A} \to \derplus{\B}$, assigned to a left exact functor $F: \A \to \B$, when $\A$ does not contain enough injectives, but instead possesses a broader and less restrictive class of objects. To this end we first introduce \emph{$F$-adapted classes}.

\begin{definition}
    \label{F-adapted}
    A class of objects $\I_F \subseteq \A$ is \emph{adapted} to a \textit{left exact}\footnote{Dually, a class of objects $\mathcal P_F$ of $\A$ is \emph{adapted} to a \textit{right exact} functor $F \colon \A \to \B$ if it is closed under finite direct sums, every object of $\A$ is a quotient of some object of $\mathcal P_F$ and for every acyclic complex $P^\bullet$ in $\kminus{\A}$, with $P^i \in \mathcal P_F$, its image $F(P^\bullet)$ under $F$ is also acyclic.} functor $F \colon \A \to \B$, if the following three conditions are satisfied.
    \begin{enumerate}[label = (\roman*)]
        \item{$\I_F$ is stable under finite direct sums. \info{I started calling biproducts direct sums now...}} \label{F-adapted i} 
        \item{Every object $A$ of $\A$ \emph{embeds} into some object of $\I_F$, \ie there exists an object $I \in \I_F$ and a monomorphism $A \hookrightarrow I$.} \label{F-adapted ii}
        \item{For every acyclic complex $I^\bullet$ in $\kplus{\A}$, with $I^i \in \I_F$ for all $i \in \Z$, its image $F(I^\bullet)$ under $F$ is also acyclic.} \label{F-adapted iii}
    \end{enumerate}
\end{definition}

\begin{example}
    We observe, that whenever $\A$ contains enough injectives, the class of all injective objects forms an $F$-adapted class for \emph{every} left exact functor ${F\colon \A \to \B}$. Points (i) and (ii) are clearly true for the class of injective objects and point (iii) follows from Lemma \ref{acyclic to injective is nulhomotopic}. Indeed, it tells us that an acyclic complex of injectives $I^\bullet$ is nullhomotopic, which implies $F(I^\bullet)$ is nullhomotopic as well, therefore, in particular, also acyclic.
\end{example}

In the presence of an $F$-adapted class $\I$ we will now construct the right derived functor of $F$. Firstly we upgrade the class of objects $\I$ to a full additive subcategory of $\A$, also denoted by $\I$, having its class of objects be precisely the $F$-adapted class $\I$. 
% This is done by declaring $\hom_\J(X, Y) := \hom_\A(X, Y)$ for all objects $X$, $Y$ of the class $\I$. 
Then extending $F$ term-wise to a functor on the level of homotopy categories, also denoted by $F \colon \kplus{\I} \to \kplus{\A}$, utilizing the procedure outlined at the end of subsection \ref{subsection: Derived functors of exact functors}, we obtain a functor, which maps acyclic complexes to acyclic complexes because by definition the $F$-adapted class $\I$ satisfies condition \ref{F-adapted iii} of Definition \ref{F-adapted}. By Proposition \ref{Exact functor induces a derived functor} the functor $F$ then descends to a well defined functor on the level of derived categories
\begin{equation}
    \label{eq: D+(I) -> D+(B)}
    \derplus{\I} \to \derplus{\B}.
\end{equation}
Since we want to define the derived functor $\rderived{}{F}$, whose domain is $\derplus{\A}$, it remains to construct a functor $\derplus{\A} \to \derplus{\I}$, which we can then post-compose with $\derplus{\I} \to \derplus{\B}$ to obtain $\rderived{}{F}$. What we will show instead is the following.

\begin{proposition}
    \label{D+(I) = D+(A) F-adapted}
    The inclusion of categories $\I \hookrightarrow \A$ induces an equivalence of triangulated categories 
    \[
        \derplus{\I} \iso \derplus{\A}.
    \]
\end{proposition}

The proof of Proposition \ref{D+(I) = D+(A) F-adapted} hinges on the following lemma, closely resembling Lemma \ref{injective resolution}. In the same sprit, we call the complex $I^\bullet$ belonging to $\kplus{\I}$ together with a quasi-isomorphism $A^\bullet \to I^\bullet$ an $\I$-resolution of the complex $A^\bullet$.

\begin{lemma}
    \label{F-acyclic resolutions}
    Every complex $A^\bullet$ of $\kplus{\A}$ has an $\I$-resolution.
    % For every complex $A^\bullet$ in $\kplus{\A}$ there is a quasi-isomorphism $A^\bullet \to I^\bullet$, where $I^\bullet$ is a complex in $\kplus{\I}$.
\end{lemma}

\begin{proof}
    A close inspection of the proof of Proposition \ref{injective resolution} reveals, that formally only properties \ref{F-adapted i} and \ref{F-adapted ii} of Definition \ref{F-adapted}, which the class of all injective objects clearly satisfies, were used. Thus the same proof can be copied for this lemma to be true.
    % By inspecting the proof of Proposition \ref{injective resolution}, we see, that only conditions \ref{F-adapted i} and \ref{F-adapted ii} of Definition \ref{F-adapted} were actually used, so the proof of this lemma is just a copy 
\end{proof}

\begin{proof}[Proof of Proposition \ref{D+(I) = D+(A) F-adapted}]
    First, there exists a triangulated functor $\derplus{\I} \to \derplus{\A}$ by Proposition \ref{Exact functor induces a derived functor}, since the inclusion $\I \hookrightarrow \A$ induces an inclusion $\kplus{\I} \hookrightarrow \kplus{\A}$ sending acyclic complexes to acyclic ones. We claim, that functor $\derplus{\I} \to \derplus{\A}$ is an equivalence, so we will show that it is full, faithful and essentially surjective. 
    
    % Next we show that it is faithful \ie that the morphism action
    % \begin{equation}
    %     \label{eq: morphism action of D(J) - D(A)}
    %     \Hom_{\derplus{\I}}(I^\bullet, J^\bullet) \to \Hom_{\derplus{\A}}(I^\bullet, J^\bullet)
    % \end{equation}

    First, we consider the morphism action 
    \begin{equation}
        \label{eq: morphism action of D(J) - D(A)}
        \Hom_{\derplus{\I}}(I^\bullet, J^\bullet) \to \Hom_{\derplus{\A}}(I^\bullet, J^\bullet)
    \end{equation}
    and show it to be injective for all objects $I^\bullet$, $J^\bullet$ of $\derplus{\I}$.
    % is injective for all objects $I^\bullet$, $J^\bullet$ of $\derplus{\I}$. 
    Suppose two morphisms $\phi_0$, $\phi_1 \colon I^\bullet \to J^\bullet$, represented by \emph{right} roofs $I^\bullet \to I_0^\bullet \xleftarrow{\sim} J^\bullet$ and $I^\bullet \to I_1^\bullet \xleftarrow{\sim} J^\bullet$, are sent to the same morphism in $\Hom_{\derplus{\A}}(I^\bullet, J^\bullet)$, meaning that there is the following commutative diagram in $\kplus{\A}$.
    \[\begin{tikzcd}[row sep = small, column sep = small]
        % https://q.uiver.app/#q=WzAsNSxbMSwxLCJJXzAiXSxbMywxLCJBIl0sWzAsMiwiSSJdLFs0LDIsIkoiXSxbMiwwLCJCIl0sWzIsMF0sWzMsMCwiXFxzaW0iXSxbMiwxXSxbMywxLCJcXHNpbSIsMl0sWzAsNF0sWzEsNCwiXFxzaW0iLDJdXQ==
        && A^\bullet \\
        & {I_0^\bullet} && I_1^\bullet \\
        I^\bullet &&&& J^\bullet
        \arrow[from=2-2, to=1-3]
        \arrow["\sim"', from=2-4, to=1-3]
        \arrow[from=3-1, to=2-2]
        \arrow[from=3-1, to=2-4]
        \arrow["\sim"{pos=0.3}, from=3-5, to=2-2, crossing over]
        \arrow["\sim"', from=3-5, to=2-4]
    \end{tikzcd}\]
    Then simply extending the diagram by a quasi-isomorphism $A^\bullet \to I_2^\bullet$, whose existence is ensured by Lemma \ref{F-acyclic resolutions}, proves that $\phi_0$ and $\phi_1$ were in fact equal and that the action \eqref{eq: morphism action of D(J) - D(A)} is faithful.

    Next, we show that the action on morphisms is also surjective, \ie the homomorphism \eqref{eq: morphism action of D(J) - D(A)} is surjective. Pick any $\psi\colon I^\bullet \to J^\bullet$ in $\derplus{\A}$, which is represented by a right roof $I^\bullet \to A^\bullet \xleftarrow{\sim} J^\bullet$. As before, extending the roof by a quasi-isomorphism $A^\bullet \to I^\bullet_0$ from Lemma \ref{F-acyclic resolutions}, leaves us with a representative $I^\bullet \to I_0^\bullet \xleftarrow{\sim} J^\bullet$ of some morphism $\phi\colon I^\bullet \to J^\bullet$ in $\derplus{\I}$.
    \[\begin{tikzcd}[column sep = small, row sep = small]
        % https://q.uiver.app/#q=WzAsNSxbMSwxLCJJXzAiXSxbMywxLCJBIl0sWzAsMiwiSSJdLFs0LDIsIkoiXSxbMiwwLCJJXzAiXSxbMiwwXSxbMywwLCJcXHNpbSIsMCx7ImxhYmVsX3Bvc2l0aW9uIjo0MH1dLFsyLDFdLFszLDEsIlxcc2ltIiwyXSxbMCw0LCIiLDIseyJsZXZlbCI6Miwic3R5bGUiOnsiaGVhZCI6eyJuYW1lIjoibm9uZSJ9fX1dLFsxLDQsIlxcc2ltIiwyXV0=
        && {I_0^\bullet} \\
        & {I_0^\bullet} && A^\bullet \\
        I^\bullet &&&& J^\bullet
        \arrow[equals, from=2-2, to=1-3]
        \arrow["\sim"', from=2-4, to=1-3]
        \arrow[from=3-1, to=2-2]
        \arrow[from=3-1, to=2-4]
        \arrow["\sim"{pos=0.3}, from=3-5, to=2-2, crossing over]
        \arrow["\sim"', from=3-5, to=2-4]
    \end{tikzcd}\]
    % \[\begin{tikzcd}[column sep = small, row sep = 1em]
    %     % https://q.uiver.app/#q=WzAsNSxbMSwyLCJJXzAiXSxbMywyLCJBIl0sWzAsNCwiSSJdLFs0LDQsIkoiXSxbMiwwLCJJXzAiXSxbMiwwXSxbMywwLCJcXHNpbSIsMCx7ImxhYmVsX3Bvc2l0aW9uIjo0MH1dLFsyLDFdLFszLDEsIlxcc2ltIiwyXSxbMCw0LCIiLDIseyJsZXZlbCI6Miwic3R5bGUiOnsiaGVhZCI6eyJuYW1lIjoibm9uZSJ9fX1dLFsxLDQsIlxcc2ltIiwyXV0=
    %     && {I_0^\bullet} \\
    %     \\
    %     & {I_0^\bullet} && A^\bullet \\
    %     \\
    %     I^\bullet &&&& J^\bullet
    %     \arrow[equals, from=3-2, to=1-3]
    %     \arrow["\sim"', from=3-4, to=1-3]
    %     \arrow[from=5-1, to=3-2]
    %     \arrow[from=5-1, to=3-4]
    %     \arrow["\sim"{pos=0.4}, from=5-5, to=3-2]
    %     \arrow["\sim"', from=5-5, to=3-4]
    % \end{tikzcd}\]
    The above diagram then shows, that $\phi$ is sent to $\psi$ and thus the action on morphisms is surjective.
    
    Lastly, Lemma \ref{F-acyclic resolutions} asserts that our functor is essentially surjective.
\end{proof}

\begin{remark}
    Upon close inspection, we recognise Theorem \ref{D+ = K+ and injectives} as a special case of Proposition \ref{D+(I) = D+(A) F-adapted}, namely take the $F$-adapted class $\I$ to be the class of all injectives of $\A$, provided there is enough of them. Indeed, it can be shown that any quasi-isomorphism $f\colon I^\bullet \to J^\bullet$ between two bounded below complexes of injectives is an isomorphism in $\kplus{\I}$. This is done by showing both $f^*\colon \Hom_{\kplus{\I}}(J^\bullet, I^\bullet) \to \Hom_{\kplus{\I}}(I^\bullet, I^\bullet)$ and $f_* \colon \Hom_{\kplus{\I}}(J^\bullet, I^\bullet) \to \Hom_{\kplus{\I}}(J^\bullet, J^\bullet)$ are isomorphisms. Lemma \ref{D+ K+ aux lemma 1} already shows us that $f^*$ is an isomorphism and by a very similar argument as in the proof of the same lemma $f_*$ is as well.
\end{remark}

We can now define the \emph{right derived functor} $\rderived{}{F}$ as the composition
\[
    \rderived{}{F} \colon \quad \derplus{\A} \xrightarrow{\ \iso \ } \derplus{\I} \longrightarrow D^+(B),
\] 
where $\derplus{\A} \xrightarrow{\ \iso \ } \derplus{\I}$ denotes a quasi-inverse to the equivalence $\derplus{\I} \iso \derplus{\A}$ established in Proposition \ref{D+(I) = D+(A) F-adapted} and $\derplus{\I} \to \derplus{\B}$ is the functor \eqref{eq: D+(I) -> D+(B)}.
The procedure for computing the right derived functor $\rderived{}{F}(A^\bullet)$ of a complex $A^\bullet$ in $\derplus{\A}$ is then very similar to \textsc{Method 1}. One takes an $\I$-resolution $A^\bullet \to I^\bullet_A$ of $A^\bullet$ and then applies the functor $F$ to $I^\bullet_A$, \ie
\[
    \rderived{}{F}(A^\bullet) = F(I^\bullet_A).
\]

We mention that both methods give rise to right derived functors satisfying the following defining universal property. We only state this here, but refer the reader to \cite[\S III.6]{gelfand2002methods} for the claim and its proof or \cite[\S 13.3, \S 10.3, \S 7.3]{kashiwara2006categories} for a more comprehensive and high level treatment using the formalism of localization. 

\begin{definition}\cite[\S III.6, Definition 6]{gelfand2002methods}
    Let $F \colon \A \to \B$ be a left exact functor between abelian categories. The \emph{right derived functor of $F$} is a triangulated functor $\rderived{}{F} \colon \derplus{\A} \to \derplus{\B}$ together with a natural transformation $\varepsilon_F \colon Q_\B \circ F \Longrightarrow \rderived{}{F} \circ Q_\A$, such that for any triangulated functor $G \colon \derplus{\A} \to \derplus{\B}$ and any natural transformation $\varepsilon \colon Q_\B \circ F \Longrightarrow G \circ Q_\A$, there exists a unique natural transformation $\eta \colon \rderived{}{F} \Longrightarrow G$, for which
    \[\begin{tikzcd}[column sep = small]
        % https://q.uiver.app/#q=WzAsMyxbMCwxLCJhIl0sWzEsMCwiYiJdLFsyLDEsImMiXSxbMSwwLCJ4IiwyLHsibGV2ZWwiOjJ9XSxbMSwyLCJ5IiwwLHsibGV2ZWwiOjJ9XSxbMCwyLCJ6IiwwLHsibGV2ZWwiOjJ9XV0=
        & Q_\B \circ F \\
        \rderived{}{F} \circ Q_\A && G \circ Q_\A
        \arrow["\varepsilon_F"', Rightarrow, from=1-2, to=2-1]
        \arrow["\varepsilon", Rightarrow, from=1-2, to=2-3]
        \arrow["\eta Q_\A", Rightarrow, from=2-1, to=2-3]
    \end{tikzcd}\]
    is commutative in the functor category $\operatorname{Fct}(\derplus{\A}, \derplus{\B})$.  
\end{definition}

% \noindent
% \textsc{Method 2$^*$.} \info{I think I will delete this method 2$^*$.} There is one instance in this thesis in which the two described methods will not suffice, namely in deriving the \emph{differentialy graded} $\Hom^\bullet$-functor. In this case our starting point will be a triangulated functor already on the level of homotopy categories $F \colon \kplus{\A} \to \kplus{\B}$, which does not originate from any functor at the level of abelian categories $\A \to \B$. Analogous to Definition \ref{F-adapted}, we introduce the following.
% \begin{definition}
%     \label{F-adapted for homotopy category}
%     Let $F \colon \kplus{\A} \to \kplus{\B}$ be a triangulated functor. A full triangulated subcategory $\mathcal K_F \subseteq \kplus{\A}$ is siad to be \emph{$F$-adapted} if the following two conditions hold.
%     \begin{enumerate}[label = (\roman*)]
%         \item{For every complex $A^\bullet$ in $\kplus{\A}$, there is a } \label{F-adapted for homotopy category i}
%         \item{For every acyclic complex $A^\bullet$ of $\mathcal K_F$, the complex $F(A^\bullet)$ is acyclic as well.} \label{F-adapted for homotopy category ii}
%     \end{enumerate}
% \end{definition}

% In order to construct the right derived functor $\rderived{}{F} \colon \derplus{\A} \to \derplus{\B}$, we assume the existence of an $F$-adapted category $\mathcal K_F \subseteq \kplus{\A}$ and introduce the full additive subcategory $\D_F \subseteq \derplus{\A}$ spanned on objects of $\mathcal K_F$. Proposition \ref{triangulated subcategory proposition} guarantees that $\D_F$ is a full triangulated subcategory of $\derplus{\A}$ and is moreover equivalent to the latter by \ref{F-adapted for homotopy category i} of Definition \ref{F-adapted for homotopy category}. Then we can extend the funtor $F$ to a functor $\D_F \to \derplus{\B}$ given by the assignment
% \begin{center}
%     \begin{tabular}{r r c l}
%         \textsl{Objects:} & $A^\bullet$ & $\longmapsto$ & $F(A^\bullet)$. \\
%         \textsl{Morphisms:} & $[A^\bullet \xleftarrow{s} C^\bullet \xrightarrow{f} B^\bullet]$ & $\longmapsto$ & $Q_\B F(f) \circ Q_\B F(s)\inv$.
%     \end{tabular}
% \end{center}
% This is a well-defined functor because condition \ref{F-adapted for homotopy category ii} of Definition \ref{F-adapted for homotopy category} ensures $F$ sends quasi-isomorphisms to quasi-isomorphisms by Proposition \ref{F preserves qis iff preserves acyclic}.
% Lastly the right derived functor is then given as the composition
% \[
%     \rderived{}{F} \colon \quad \derplus{\A} \xrightarrow{\ \iso \ } \D_F \longrightarrow \derplus{\B}.
% \]
% \begin{example}
%     When $\A$ contains enough injectives, the category $\kplus{\J} \subseteq \kplus{\A}$ is $F$-adapted for any triangulated functor $F \colon \kplus{\A} \to \kplus{\B}$.
% \end{example}

% \vspace{3cm}

For two composable the following proposition tells us that first deriving and then composing or first composing and then deriving results in the same functor.

\begin{proposition}
    
\end{proposition}

\begin{proof}
    
\end{proof}

\subsubsection*{Higher derived functors}

Classically derived functors of a left exact functor $F \colon \A \to \B$ first appeared as a sequence of additive functors $\A \to \B$. We introduce them in the following definition, using our established (modern) viewpoint. 

\begin{definition}
    For a right derived functor $\rderived{}{F} \colon \derplus{\A} \to \derplus{\B}$ of a left exact functor $F \colon \A \to \B$, we define the \emph{higher derived functors of} $F$ as compositions
    \[
        \rderived{i}{F} := H^i \circ \rderived{}{F} \colon \quad \derplus{\A} \to \B
    \]
    for all $i \in \Z$. 
\end{definition}

Frequently we will be interested in the image of a higher derived functor $\rderived{i}{F}$ of just a single object $A$ belonging to category $\A$. In that case we will always interpret\footnote{More precisely, there is always a fully faithful functor $\A \to \kcat{\A}$ sending an object $A$ to a complex $\cdots \to 0 \to A \to 0 \to \cdots$ concentrated in degree $0$ and a morphism $f\colon A \to B$ to the homotopy class of the chain map defined by $f^0 = f$ and $f^i = 0$, for $i \neq 0$ \cite[Chapter 3, Lemma 1.3.5]{milicic-dercat}.} it as a bounded complex concentrated in degree $0$. 

\begin{example}
    For an object $A$ of $\A$, equipped with an $F$-adapted class $\I$, we can compute that $\rderived{i}{F}(A) \iso 0$, for $i < 0$, and $\rderived{0}{F}(A) \iso F(A)$. 
    % Indeed, considering $A$ as a complex concentrated in degree $0$, let $I^\bullet$ be its $\I$-resolution. 
    Indeed, consider an $\I$-resolution $I^\bullet$ of $A$.
    Then, as $I^i = 0$ for $i < 0$, we have that $\rderived{i}{F}(A) = H^i(F(I^\bullet)) \iso 0$ for $i < 0$. For the second claim, because $I^\bullet$ is quasi-isomorphic to $A$, we have $\ker(I^0 \to I^1) \iso A$, and since $F$ is left exact, we conclude that $F(A) \iso \ker(F(I^0) \to F(I^1)) \iso H^0(F(I^\bullet)) = \rderived{0}{F}(A)$.
\end{example}

\begin{definition}
    \label{F-acyclic}
    Suppose the right derived functor $\rderived{}{F}$ exists. We say an object $A$ of $\A$ is \emph{$F$-acyclic}, if $\rderived{i}{F}(A) \iso 0$ for all $i \neq 0$.
\end{definition}

A useful tool for gathering information about the action of the higher derived functors of a left exact functor $F$ on objects will be the long exact sequence associated to $F$ introduced in the following proposition. In view of this proposition we also interpret higher derived functors of $F$ as measuring the extent to which $F$ fails to be exact. 

\begin{proposition}
    Let $F \colon \A \to \B$ be a left exact functor and let $0 \to A \to B \to C \to 0$ be a short exact sequence in $\A$. Then there exists a long exact sequence
    \begin{multline}
        \label{eq: LES for left exact functor F}
        0 \to F(A) \to F(B) \to F(C) \to \rderived{1}{F}(A) \to \rderived{1}{F}(B) \to \rderived{1}{F}(C) \to \cdots \\
        \cdots \to \rderived{i}{F}(A) \to \rderived{i}{F}(B) \to \rderived{i}{F}(C) \to \rderived{i+1}{F}(A) \to \cdots.
    \end{multline}
\end{proposition}

\begin{proof}
    First, we describe how the short exact sequence gives rise to a distinguished triangle $A \to B \to C \to A[1]$ in $\derplus{\A}$, where $A$, $B$ and $C$ are considered as complexes concentrated in degree $0$.
    \info{finish this} 

    After first applying the triangulated functor $\rderived{}{F}$ and then the cohomological functor $H^0$ to the triangle $A \to B \to C \to A[1]$, we obtain a long exact sequence 
    \begin{multline*}
        0 \to H^0(\rderived{}{F}(A)) \to H^0(\rderived{}{F}(B)) \to H^0(\rderived{}{F}(C)) \to H^0(\rderived{}{F}(A)[1]) \to \cdots \\
        \cdots H^0(\rderived{}{F}(A)[i]) \to H^0(\rderived{}{F}(B)[i]) \to H^0(\rderived{}{F}(C)[i]) \to H^0(\rderived{}{F}(A)[i+1]) \to \cdots
    \end{multline*}
    by Proposition \ref{LES in cohomology}, whose terms are readily seen to be isomorphic to the terms of \eqref{eq: LES for left exact functor F}. \qedhere
\end{proof}

The rest of this section is devoted to proving some useful results relating $F$-acyclic and $F$-adapted objects. 

\begin{lemma}
    \label{F-adapted is F-acyclic}
    Every object of an $F$-adapted class is also $F$-acyclic.
\end{lemma}

\begin{proof}
    We only have to recall definition \ref{F-acyclic}. An object $I$ belonging to an $F$-adapted class $\I$ trivially forms its own $\I$-resolution $I^\bullet = (\cdots \to 0 \to I \to 0 \to \cdots)$, which allows us to compute $\rderived{i}{F}(I) = H^i(\rderived{}{F}(I^\bullet)) = H^i(F(I^\bullet)) \iso 0$, whenever $i \neq 0$ (and also $F(I)$ for $i = 0$).
\end{proof}

\begin{proposition}
    \label{F-acyclic is F-adapted}
    Let $F : \A \to \B$ be a left exact functor between two abelian categories. Let $\I \subset \A$ be an $F$-adapted class of objects in $\A$. Then the class of all $F$-acyclic objects of $\A$, denoted by $\I_F$, is also $F$-adapted. 
\end{proposition}

\begin{proof}
    We verify conditions (i)--(iii) of definition \ref{F-adapted}. (i) Since the higher derived functors $\rderived{i}{F}$ are additive, $\I_F$ is stable under finite sums. (ii) Holds by lemma \ref{F-adapted is F-acyclic}. (iii) Suppose $A^\bullet$ is an acyclic complex in $K^+(\A)$ with $A^i \in \I_F$ for all $i$. Then using acyclicity of $A^\bullet$, we may break this complex up into a series of short exact sequences 
    % as follows.
    % \begin{align*}
    %     0 \to A^0 \to & A^1 \to \ker d^2 \to 0 \\
    %     0 \to \ker d^2 \to & A^2 \to \ker d^3 \to 0 \\
    %     & \vdots
    % \end{align*} 
    \begin{equation}
        \label{eq: SES from acyclic cx}
        0 \to \ker d^i \to A^i \to \ker d^{i+1} \to 0 \quad \text{ for $i \in \Z$.}
    \end{equation}
    As our complex $A^\bullet$ is supported on $\N$, the first interesting short exact sequence happens at index $i = 1$, where $\ker d^1 = \im d^0 \iso A^0$
    It yields the following long exact sequence associated to $F$
    \begin{multline*}
        0 \to F(A^0) \to F(A^1) \to F(\ker d^2) \to \rderived{1}{F}(A^0) \to \cdots \\ \cdots \to \rderived{i}{F}(A^0) \to \rderived{i}{F}(A^1) \to \rderived{i}{F}(\ker d^2) \to \rderived{i+1}{F}(A^0) \to \cdots.
    \end{multline*}
    Exactness of this sequence together with $F$-acyclicity of $A^0$ and $A^1$ shows that $\ker d^2$ is also $F$-acyclic and by only considering the beginning few terms of the sequence, we obtain the short exact sequence 
    % From the fact that $A^0$ and $A^1$ are $F$-acyclic we see that $\ker d^2$ is $F$-acyclic and we obtain a short exact sequence 
    \[
        0 \to F(A^0) \to F(A^1) \to F(\ker d^2) \to 0.
    \]
    After inductively applying the same reasoning for each of the original short exact sequences of \eqref{eq: SES from acyclic cx} for $i > 1$, we are left with short exact sequences\footnote{To be more precise, we have used that $F$ is left exact in order to swap the order of $F$ and $\ker$ to reach \[\ker Fd^i \iso F(\ker d^i)\].}
    \[
        0 \to \ker Fd^i \to F(A^i) \to \ker Fd^{i+1} \to 0 \quad \text{ for $i \geq 1$.}
    \]
    Collecting them all back together we conclude that the complex $F(A^\bullet)$ is acyclic.
    % \info{There are inconsistencies with $F(A^\bullet) = K^+(F)(A^\bullet)$ and so on. $\to$ I am going to abuse notation and use $F(A^\bullet)$, bc this is also done in practise \eg $f_*\F^\bullet$, not $K^+(f_*)(\F^\bullet)$} \qedhere
\end{proof}

% \[\begin{tikzcd}[column sep = small, row sep = 1.8em]
%     % https://q.uiver.app/#q=WzAsNSxbMCwwLCJcXGtwbHVze1xcSn0iXSxbMiwwLCJcXGtwbHVze1xcQX0iXSxbMiwyLCJcXGRlcnBsdXN7XFxBfSJdLFs0LDAsIlxca3BsdXN7XFxCfSJdLFs0LDIsIlxcZGVycGx1c3tcXEJ9Il0sWzEsMiwiUV9cXEEiXSxbMiwwLCJcXG5hdGlzbyIsMl0sWzAsMSwiXFxpb3RhIiwwLHsic3R5bGUiOnsidGFpbCI6eyJuYW1lIjoiaG9vayIsInNpZGUiOiJ0b3AifX19XSxbMSwzLCJGIl0sWzMsNCwiUV9cXEIiXV0=
% 	{\kplus{\J}} && {\kplus{\A}} && {\kplus{\B}} \\
% 	\\
% 	&& {\derplus{\A}} && {\derplus{\B}}
% 	\arrow["\iota", hook, from=1-1, to=1-3]
% 	\arrow["F", from=1-3, to=1-5]
% 	\arrow["{Q_\A}", from=1-3, to=3-3]
% 	\arrow["{Q_\B}", from=1-5, to=3-5]
% 	\arrow["\natiso"', from=3-3, to=1-1]
% \end{tikzcd}\]

% In th subsection we will deal with the case, when $F$ is only left exact. The theory dually runs in parallel, when $F$ is right exact. 

\subsubsection{Ext functors}
\label{subsection: Ext functors}

As a first application of the established theory of derived functors we consider the $\Hom$-functor
\[
    \Hom_\A(A, -) \colon \A \to \mod{k},
\]
where $A$ is an object of an abelian category $\A$. Assume that category $\A$ contains enough injectives. By Example \ref{Homs are left exact functors}, we know that $\Hom_\A(A, -)$ is a left exact functor, so we can right derive it, to obtain the functor
% so there exists its right derived functor
\[
    \mathbf{R}\Hom_\A(A, -) \colon \derplus{\A} \to \derplus{\mod{k}}.
\] 
Taking the $i$-th cohomology, we obtain higher derived functors of $\Hom_\A(A, -)$, called the \emph{Ext-functors}
\[
    \Ext^i_\A(A, -) := \rderived{i}{\Hom_\A(A, -)} = H^i(\rderived{}{\Hom_\A(A, -)}).
\]
There exists a very beautiful connection relating the Ext-functors of category $\A$ and the hom-functors of the bounded below derived category $\derplus{\A}$ captured in the next proposion, given without proof, for in a moment we will present its generalized verison.

\begin{proposition}
    Let $\A$ be an abelian category with enough injectives and let $A$ and $B$ belong to $\A$. Then for all $i \in \Z$ there exist isomorphisms
    \[
        \Ext^i_\A(A, B) \iso \Hom_{\derplus{\A}}(A, B[i]).
    \]
\end{proposition}

For any two complexes $A^\bullet$ and $B^\bullet$ we can define a complex $\Hom^\bullet(A^\bullet, B^\bullet)$ of $k$-modules given by terms
\[
    \Hom^i(A^\bullet, B^\bullet) = \prod_{j \in \Z}\Hom_\A(A^j, B^{j + i}),
\]
\ie the set of all degree $i$ maps from $A^\bullet$ to $B^\bullet$, considered as graded objects of $\A^\Z$, and differentials 
\[
    d^i \colon \Hom^i(A^\bullet, B^\bullet) \to \Hom^{i+1}(A^\bullet, B^\bullet), \qquad 
    d^i(f) = \left(d^{i+j}_B \circ f^j + (-1)^{i+1} f^{j+1} \circ d^j_A\right)_{j \in \Z}.
\]
This differential will also be denoted by $\delta^i_B$.

By setting $\Hom^\bullet(A^\bullet, -)(B^\bullet) = \Hom^\bullet(A^\bullet, B^\bullet)$ and for a chain map $u\colon B^\bullet \to C^\bullet$ in $\com(\A)$ defining $u_* = \Hom^\bullet(A^\bullet, -)(u) \colon \Hom^\bullet(A^\bullet, B^\bullet) \to \Hom^\bullet(A^\bullet, C^\bullet)$ to be the chain map given by the collection $(u_*^i)_{i \in \Z}$, where $u_*^i$ sends $f = (f^j)_{j \in \Z} \in \Hom^i(A^\bullet, B^\bullet)$ to $u \circ f = (u^{i+j} \circ f^j)_{j \in \Z}$, we obtain a functor
\[
    \Hom^\bullet(A^\bullet, -) \colon \com(\A) \to \com(\mod{k}).
\]
Indeed, clearly $u_*$ is a graded morphism of degree $0$, which moreover commutes with the differentials, according to the following computation
\begin{align*}
    u^{i+1}_*\left(d^i_{\Hom^\bullet(A, B)}(f)\right) &= u^{i+1}_*\left(\left(d^{i+j}_B f^j + (-1)^{i+1} f^{j+1} d^j_A\right)_j\right) \\
    &= \left(u^{i+j+1} d^{i+j}_B f^j + (-1)^{i+1} u^{i+j+1} f^{j+1} d^j_A\right)_j \\
    &= \left(d^{i+j}_C u^{i+j} f^j + (-1)^{i+1} u^{i+j+1} f^{j+1} d^j_A\right)_j \\
    &= d^i_{\Hom^\bullet(A, C)}\left(\left(u^{i + j}f^j\right)_j\right) \\
    &= d^i_{\Hom^\bullet(A, C)}\left(u^i_*(f)\right),
\end{align*}
for all $f \in \Hom^i(A^\bullet, B^\bullet)$. While not difficult to check, the verification of functoriality is omitted. 

Next, we would like $\Hom^\bullet(A^\bullet, -)$ to descend to a homotopy functor $\kcat{\A} \to \kcat{\mod{k}}$. This is in fact so, as a null-homotopic chain map $u\colon B^\bullet \to C^\bullet$ is sent to a null-homotopic chain map $u_* \htpy 0$. Indeed, suppose homotopy $h \in \Hom^{-1}(B^\bullet, C^\bullet)$ witnesses $u \htpy 0$, then for all $i \in \Z$, we have
\[
    u^i = d^{i-1}_C \circ h^i + h^{i+1} \circ d^i_B.
\]
We let $h^i_*$ denote the morphism $\Hom^i(A^\bullet, B^\bullet) \to \Hom^{i-1}(A^\bullet, C^\bullet)$, given by sending $f$ to $(h^{i+j} \circ f^j)_{j \in \Z}$, and then compute
\begin{align*}
    (\delta^{i-1}_C \circ h^{i}_* + h^{i+1}_* \circ \delta^i_B)(f) &= 
    \delta^{i-1}_C\left((h^{i+j}f^j)_j\right) + h^{i+1}_*\left((d_B^{i+j} f^j + (-1)^{i+1}f^{j+1}d^j_A)_j\right) \\
    &= \left(d^{i+j-1}_C h^{i+j} f^j + (-1)^{i} h^{i+j+1} f^{j+1} d^j_A \right)_j  + \\ 
    & \qquad \qquad + \left(h^{i+j+1}d_B^{i+j}f^j + (-1)^{i+1} h^{i+j+1} f^{j+1} d^j_A \right)_j \\
    &= \left(d^{i+j-1}_C h^{i+j} f^j + h^{i+j+1}d_B^{i+j}f^j\right)_j \\
    &= \left((d^{i+j-1}_C h^{i+j} + h^{i+j+1}d_B^{i+j})\circ f^j\right)_j \\
    &= \left(u^{i+j} \circ f^j\right)_j \\
    &= u_*^{i}(f).
\end{align*}
Thus the homotopy $h_* = (h^i_*)_{i \in \Z}$ whitnesses $u_* \htpy 0$, allowing us to conclude that 
\[
    \Hom^\bullet(A^\bullet, -) \colon \kcat{\A} \to \kcat{\mod{k}}
\] 
is well defined. The preceding functor is also triangulated, because it commutes with the translation functors and it sends distinguished triangles of $\kcat{\A}$ to distinguished triangles of $\kcat{\mod{k}}$ This is because of the following isomorphism of chain complexes of $k$-modules,
\[
    C(u_*)^\bullet = C(\Hom^\bullet(A^\bullet, -)(u))^\bullet \cong \Hom^\bullet(A^\bullet, -)(C(u)^\bullet),
\]
which holds for any chain map $u \colon B^\bullet \to C^\bullet$. Indeed, we have
\begin{align*}
    C(u_*)^i 
    &= \Hom^{i+1}(A^\bullet, B^\bullet) \oplus \Hom^i(A^\bullet, C^\bullet) \\
    &= \prod_{j \in \Z}\Hom_\A(A^j, B^{i+j+1}) \oplus \prod_{j \in \Z}\Hom_\A(A^j, C^{i+j}) \\
    &\cong \prod_{j \in \Z} \left(\Hom_\A(A^j, B^{i+j+1}) \oplus \Hom_\A(A^j, C^{i+j})\right) \\
    &\cong \prod_{j \in \Z}\Hom_\A(A^j, B^{i+j+1} \oplus C^{i+j}) \\
    &= \prod_{j \in \Z}\Hom_\A(A^j, C(u)^{i+j}) \\
    &= \Hom^i(A^\bullet, C(u)^\bullet),
\end{align*}
where these isomorphisms also commute with the differentials. Pick $f \in \Hom^i(A^\bullet, C(u)^\bullet)$ which is given by a pair $f$


\vspace{3cm}



At last, for $A^\bullet$ belonging to $\kminus{\A}$ \ie bounded from \emph{above}\footnote{This assumption cannot be skipped, for otherwise we cannot claim, that the functor $\Hom^\bullet(A^\bullet, -)$ maps into the bounded \emph{below} homotopy category $\kplus{\A}$.}, we have a triangulated functor $\Hom^\bullet(A^\bullet, -) \colon \kplus{\A} \to \kplus{\mod{k}}$, which we may derive, in accordance with Remark \ref{Deriving homotopy functors}, of course assuming $\A$ contains enough injectives, to obtain
\[
    \rderived{}{\Hom^\bullet(A^\bullet, -)} \colon \derplus{\A} \longrightarrow \derplus{\mod{k}}.
\]
Its higher derived counterpart will be denoted by $\Ext^i_\A(A^\bullet, - ) := \rderived{i}{\Hom^\bullet(A^\bullet, -)}$.

% \begin{remark}
%     There is also a generalization of this result providing a way to identify the hom-functors in $\derplus{\A}$ applied not only to objects of $\A$, but to bounded below complexes. In that case, on the Ext-side, one considers the \emph{differentialy graded hom-functors} of $\kplus{\A}$, namely $\Hom^\bullet(-,-) \colon \kminus{\A}^\op \times \kplus{\A} \to \kcat{\mod{k}}$, for which $\Hom^\bullet(A^\bullet, -)$ turns out to be right derivable inducing its higher derived functors $\Ext^i(A^\bullet, -) = \rderived{i}{\Hom^\bullet(A^\bullet, -)}$. Then for all $i \in \Z$ and all $A^\bullet$ and $B^\bullet$ of $\derplus{\A}$ there are isomorphisms
%     \[
%         \Ext^i(A^\bullet, B^\bullet) \iso \Hom_{\derplus{\A}}(A^\bullet, B[i]^\bullet).
%     \]
%     \info{this has to be expaned, because I need it for Serre functors, naturality in $A^\bullet$ and $B^\bullet$ is important.}
% \end{remark}

\begin{remark}
    Behind the scenes of this whole process one actually consideres an additive bifunctor $F \colon \A \times \A' \to \A''$, with $\A$, $\A'$ and $\A''$ abelian, where the latter is assumed to contain countable products. From this bifunctor we induce another additive bifunctor on the level of chain complexes $F_{\pi} \colon \com(\A) \times \com(\A') \to \com(\A'')$ via the total complex of a double complex as described in \cite[\S 11.6]{kashiwara2006categories}.
    %  with terms $F^\bullet_\pi(X^\bullet, Y^\bullet) = \operatorname{tot_\pi}(F(X^\bullet, Y^\bullet))$, where $\operatorname{tot_\pi}(Z^{\bullet, \bullet})^n = \prod_{p+q = n} Z^{p,q}$. 
    It then turns out that $F_\pi$ induces a well defined triangulated bifunctor on the level of the homotopy category 
    \[
        \kcat{\A} \times \kcat{\A'} \to \kcat{\A''}.
    \]
    % $\kplus{\A} \times \kplus{\A'} \to \kplus{\A''}$, $\kminus{\A} \times \kminus{\A'} \to \kminus{\A''}$, $\kb{\A} \times \kcat{\A'} \to \kcat{\A''}$ and $\kcat{\A} \times \kb{\A'} \to \kcat{\A''}$.
\end{remark}

\begin{proposition}
    Let $\A$ be an abelian category with enough injectives. Then for every $A^\bullet$ of $\kminus{\A}$ and $B^\bullet$ belonging to $\kplus{\A}$ there natural isomorphisms
    \[
        \alpha_{A^\bullet, B^\bullet} : \Ext^i_\A(A^\bullet, B^\bullet) \iso \Hom_{\dercat{\A}}(A^\bullet, B[i]^\bullet).
    \]
\end{proposition}
\info{I think this is true, but I won't be able to show it as I need to use $\derplus{\A} \natiso \kplus{\J}$, since $A^\bullet$ is not bounded from below.}

\begin{proposition}
    \label{Homs in derived category}
    Let $\A$ be an abelian category with enough injectives. Then for every $A^\bullet$ and $B^\bullet$ belonging to $\derb{\A}$ there natural isomorphisms
    \[
        \alpha_{A^\bullet, B^\bullet} : \Ext^i_\A(A^\bullet, B^\bullet) \iso \Hom_{\derb{\A}}(A^\bullet, B[i]^\bullet).
    \]
\end{proposition}

\begin{lemma}
    \label{Homotopy hom = cohomology of inner hom}
    Let $A^\bullet$ and $B^\bullet$ be chain complexes in $\kcat{\A}$ and let $i \in \Z$. Then
    \[
        H^i(\Hom^\bullet(A^\bullet, B^\bullet)) = \Hom_{\kcat{\A}}(A^\bullet, B[i]^\bullet).
    \] 
\end{lemma}

\begin{proof}
    This is practically the definition of $\Hom_{\kcat{\A}}(A^\bullet, B[i]^\bullet)$. The cohomology $k$-module $H^i(\Hom^\bullet(A^\bullet, B^\bullet))$ is defined to be the quotient $\ker \delta^i / \im \delta^{i-1}$, thus let us identify $\ker \delta^i$ and $\im \delta^{i-1}$. Notice, that for $f \in \Hom^i(A^\bullet, B^\bullet)$ we have
    \[
        \delta^i(f) = \left(d^{i+j}_B \circ f^j + (-1)^{i+1} f^{j+1} \circ d^j_A\right)_{j \in \Z} = (-1)^i \left(d^j_{B[i]} \circ f^j - f^{j+1} \circ d^j_A\right)_{j \in \Z},
    \]
    thus $\ker \delta^i$ is the set of all chain maps $A^\bullet \to B[i]^\bullet$, and $\im \delta^{i-1}$ the set of all null-homotopic chain maps.
\end{proof}

\begin{proof}[Proof of Proposition \ref{Homs in derived category}]
    Let $B^\bullet \to I^\bullet$ be an injective resolution for $B^\bullet$. We follow the chain of isomorphisms
    \begin{align*}
        \Hom_{\derb{\A}}(A^\bullet, B[i]^\bullet) &= \Hom_{\derplus{\A}}(A^\bullet, B[i]^\bullet)  \\
        &\iso \Hom_{\derplus{\A}}(A^\bullet, I[i]^\bullet)  \\
        &\iso \Hom_{\kplus{\J}}(A^\bullet, I[i]^\bullet) &\text{(Theorem \ref{D+ = K+ and injectives})}\\
        &= H^i(\Hom^\bullet(A^\bullet, I^\bullet)) &\text{(Lemma \ref{Homotopy hom = cohomology of inner hom})} \\
        &= \Ext_\A^i(A^\bullet, B^\bullet). 
    \end{align*}\qedhere
\end{proof}