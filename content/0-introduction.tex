\section*{Introduction}
\phantomsection
\addcontentsline{toc}{section}{Introduction} % so that it is not numbered in table of contents

In algebraic geometry we study spaces like schemes and varieties in terms of different sheaves on them. Of special importance are the quasi-coherent and coherent sheaves, which bring us closer to algebra. In this sense we find important geometric data encoded in cohomology of various sheaves. For comparing or manipulating sheaves we often employ different functors, which are unfortunately not always perfectly in tune with cohomology. By this we mean that they are not necessarily exact as one already sees in the case of push-forward and pull-back functors, which are only left and right exact, respectively.
For instance let us take a look at the push forward functor along a morphism of schemes $f \colon X \to Y$. This functor is known to be left exact, but in general not always right exact. In more concrete terms this means that after applying the functor $f_*$ to a short exact sequence of sheaves $0 \to \F_0 \to \F_1 \to \F_2 \to 0$ on $X$ we obtain on $Y$ only an exact sequence of the form
\[
    0 \to f_*\F_0 \to f_*\F_1 \to f_*\F_2.
\]
One is then led to extend this sequence by additional terms to the right, which measure the extent to which $f_*$ fails to be exact, to obtain a long exact sequence
\begin{multline*}
    0 \to f_*\F_0 \to f_*\F_1 \to f_*\F_2 \to \rderived{1}{f_*}\F_0 \to \rderived{1}{f_*}\F_1 \to \cdots \\
    \cdots \to \rderived{i}{f_*}\F_0 \to \rderived{i}{f_*}\F_1 \to \rderived{i}{f_*}\F_2 \to \rderived{i+1}{f_*}\F_0 \to \cdots.
\end{multline*}
These additional terms show up as the images of the \emph{higher derived functors} of $f_*$. It turns out that the collection of all such higher derived functors actually emerges from a single derived functor whose natural domain is the \emph{derived category of coherent sheaves} $\derb{X}$. The derived category then becomes an interesting object of study by itself partly because of its complexity and also since it has proved itself to be an important invariant of varieties. In particular one is interested in the question of how much data is lost in the process of passing to the derived level? To this end we introduce the term of smooth projective varieties $X$ and $Y$ being \emph{derived equivalent} if they share equivalent bounded derived categories of coherent sheaves, \ie 
\[
    \derb{X} \iso \derb{Y}. 
\]
In certain cases like that of curves, the derived category turns out to capture essentially all the information about the curve. In that case, if a curve $C$ is derived equivalent to a smooth projective variety $X$, then necessarily $X$ must also be a curve and is isomorphic to $C$ \cite[\S 5]{huybrechts2006fouriermukai}. A similar phenomenon has been proven to occur by Bondal and Orlov in \cite{BondalOrlov2001}, where they showed that a pair of derived equivalent smooth projective varieties $X$ and $Y$ with $X$ admitting either an ample canonical or ample anti-canonical bundle implies that $X$ and $Y$ must necessarily be isomorphic. However, contrary to the above, there do exist pairs of derived equivalent varieties, which are not isomorphic. The first example of this was documented by Mukai in \cite{Mukai1981}, where he showed that an abelian variety and its dual are always derived equivalent however not necessarily isomorphic. This kind of behaviour is observed also for K3 surfaces which are of special importance to us. They are smooth projective surfaces $X$ with trivial canonical bundles, satisfying $\cohom{1}{X}{\struct{X}} = 0$. In this case the derived Torelli theorem, whose proof following \cite{Orlov2003} is the principal goal of this thesis, will in particular provide a practical criterion for determining when two K3 surfaces are derived equivalent. 

% is a curve sharing equivalent bounded derived categories of coherent sheaves with a  

K3 surfaces also show themselves to be intriguing from the point of view of \emph{Homological mirror symmetry}. Since they are exhibited as special instances of Calabi--Yau manifolds in dimension two, right after elliptic curves in dimension one, they turn out to be the next testing ground in which to tackle different conjectures. In this light the derived Torelli theorem becomes particularly prominent, because on the algebro-geometric side of homological mirror symmetry, where derived categories make an appearance, it allows one to replace them with a simpler and more convenient Mukai lattice. 

% though a very sophisticated lens of derived categories. K3 surface being special instances of Calabi--Yau manifolds 

% homological mirror symmetry.

% \begin{itemize}
%     \item{For us a \emph{variety} will mean an integral separated scheme of finite type over a field $k$. }
%     \item{Recall what a coherent and quasi-coherent sheaf is. } 
% \end{itemize}


% Bondal Orlov ample canonical bundle 


The following outline and structure of the thesis tries to highlight some of the most notable themes of each chapter.

The first chapter is devoted to establishing the necessary prerequisites to understand the definition of a derived category. We start with exploring the basics of abelian categories and touch upon exact, left exact and right exact functors between them. Next we introduce \emph{triangulated categories}, which in certain sense formalize the notion of short exact sequences from abelian categories. The chapter also contains some technical results, which we will apply in Chapter \ref{Chapter: Derived Torelli theorem}.
Building upon $k$-linear and abelian categories we introduce the categories of chain complexes and define the homotopy category of complexes. We will equip the latter with a triangulated structure, exhibiting the first example of a triangulated category we encounter in this thesis. We used \cite{kashiwara2006categories} as the main literature for abelian categories and categories of complexes and \cite{huybrechts2006fouriermukai} for triangulated categories.

The entirety of the second chapter is dedicated to first constructing the \emph{derived category of an abelian category} and second, to the construction of \emph{derived functors}. What is to be presented is a special case of a process called localization of (triangulated) categories. This method is quite similar to localization of rings and modules, with the central idea being the inversion of certain elements. In our case of derived categories we are led to invert a class of morphisms, which are called \emph{quasi-isomorphisms}, resulting in
% because this allows us to obtain 
a category which is well-behaved with respect to cohomology. In a way generalizing the idea of computing cohomology objects of a complex we introduce derived functors. Especially important will be the case of left (\resp right) exact functors between abelian categories, for which we will provide two methods of deriving them. The primary sources in writing this chapter were \cite{gelfand2002methods,milicic-dercat}.
%
%
% in the particular all the quasi-isomorphisms in the case of derived categories. We will see how the language of triangulated categories lends itself nicely to 
%
%
% In the third chapter we turn our attention to the principal goal of introducing derived categories 
%

In the third chapter we turn our attention to more concrete applications and study in detail the \emph{bounded derived category $\derb{X}$ of coherent sheaves} of a variety $X$, which is most often assumed to also be smooth and projective. Most notably we show that a connected variety has an indecomposable bounded derived category and that the derived category comes equipped with a spanning class. Both of these results will show up to be useful later on in Chapter \ref{Chapter: Derived Torelli theorem}. In the second part we deal with functors of geometric origin like the global sections functor, the pull-back and push-forward functors along a morphism and others and we will derive them following the methods outlined in Chapter \ref{Chapter: Derived categories}. We follow mainly \cite[\S 3]{huybrechts2006fouriermukai} for this part.

Chapter \ref{Chapter: Fourier-Mukai transforms} introduces \emph{Fourier--Mukai transforms}. For smooth projective varieties $X$ and $Y$ these functors map between their bounded derived categories in other words are functors of the form $\derb{X} \to \derb{Y}$. They form an important class of examples of triangulated functors between bounded derived categories and are essentially determined by a choice of a bounded complex of sheaves belonging to $\derb{X \times Y}$ called the \emph{kernel} of a Fourier--Mukai transform. We will see that many well-known triangulated functors between  derived categories 
% of smooth projective varieties 
are actually of Fourier--Mukai type, most surprisingly all fully faithful triangulated functors admitting adjoints are of this sort. This is due to an important theorem of Orlov \cite{Orlov2003}, which shed a new light on how one handles equivalences of derived categories, since it enables one to obtain a more practical description of these equivalences by utilizing their kernels. We will exploit precisely this feature in the proof of the derived Torelli theorem in Chapter \ref{Chapter: Derived Torelli theorem}. With the goal of eventually relating bounded derived categories of K3 surfaces with their cohomology, we will explain the process of first passing Fourier--Mukai transforms to the level of $K$-theory and afterwards to rational cohomology. This passage relies on the famous Grothendieck--Riemann--Roch theorem and leads to the inception of Mukai vectors. 
% also motivates the inception of Mukai vectors.
We also loosely skim over the topic of Hodge decompositions of complex projective manifolds and close with examining a nice interaction between Hodge decompositions and Fourier--Mukai transforms on cohomology. All the results on Fourier--Mukai transforms are sourced from \cite[\S 5]{huybrechts2006fouriermukai}, which are built on the pioneering work of Mukai \cite{Mukai1981} and later Orlov \cite{Orlov2003}.

% enables to and enables one to study them in terms of their kernels instead. 

% esspecially for studying the automorphism groups of bounded derived categories. 

% Triangulated equivalences between bounded derived categories of coherent sheaves on smooth projective varieties will be of of special interest to us due to a theorem of Orlov \cite{Orlov2003}. It makes it possible to exhibit all such equivalences as Fourier--Mukai transforms, thus enabling us to study them in terms of their kernels instead. 






In Chapter \ref{Chapter: K3} we meet \emph{K3 surfaces}. Their peculiar name was coined by André Weil in honor of Kodaira, Kummer and Kähler and the K2 mountain in the Karakoram mountain range.
We present two points of approach to K3 surfaces, first from the side of algebraic geometry and second through complex geometry and topology. Both viewpoints are fruitful to consider as their interplay lets us calculate the main invariants of these surfaces. We start by adapting Serre duality and computing the Hodge diamond, followed by identifying integral cohomology and the intersection form, which is known to be ubiquitous to the study of $4$-manifolds. Lastly, we present the classical Torelli theorem, which underscores how Hodge structures in combination with intersection forms completely determine the geometry of K3 surfaces. The theorem is attributed to the work of Pjatecki\u{\i}--Šapiro and Šafarevič \cite{PjateckiiShafarevich1971}. All of this and so much more on K3 surfaces is packaged concisely in \cite{Huybrechts2016}. 

% Lastly we introduce the classical Torelli theorem showcasing the importance of the combination of Hodge structures and intersection forms on K3 surfaces. 
% of the importance of the combination of the Hodge decomposition and the intersection form



Through the course of the last chapter, we will prove the derived Torelli theorem, which is due to many authors, most significantly Orlov \cite{Orlov2003} and Mukai \cite{Mukai1987}.
% but has its roots already in the work of Pjatecki\u{\i}--Šapiro and Šafarevič. 
% We first state it here and comment on it momentarily.
% Its statement is the following.
It states the following.

\begin{theorem*}
    % [Derived Torelli theorem]
    % \label{Derived torelli at the end}
    Let $X$ and $Y$ be K3 surfaces over the field of complex numbers $\C$. Then the following statements are equivalent.
    \begin{enumerate}[label = (\roman*)]
        \item{$X$ and $Y$ share equivalent bounded derived categories of coherent sheaves, 
        \[
            \derb{X} \iso \derb{Y}.
        \]
        }
        \item{There exists a Hodge isometry $f \colon \transc{X} \to \transc{Y}$ between the transcendental lattices of $X$ and $Y$.}
        \item{Mukai lattices $\mukai{X}$ and $\mukai{Y}$ of $X$ and $Y$, respectively, are Hodge isometric. 
        }
        \item{There exists 
        % a polarization on $X$ and 
        an isotropic Mukai vector $v \in \mukai{X}$ of divisibility $1$,
        such that $Y$ is isomorphic to a moduli space $M_v$ of stable sheaves on $X$ with Mukai vector $v$, representing \emph{some} moduli functor $\M_v$. Moreover $M_v$ is up to isomorphism the unique K3 surface, for which there exists a Hodge isometry
        %  of divisibility $1$, 
        % such that $Y$ is isomorphic to the unique K3 surface $M_v$, for which
        \[
            \cohom{2}{M_v}{\Z} \iso v^\perp / \Z v.
        \]
        }
        % moduli space of stable sheaves on $X$, with Mukai vector $v$.
    \end{enumerate}
\end{theorem*}

The first highlight of this theorem is the \emph{lattice of transcendental cycles}. It is exhibited as a Hodge sublattice of $\cohom{2}{X}{\Z}$ and contrasting it, the \emph{Mukai lattice} is displayed as an extension of the ordinary Hodge lattice $\cohom{2}{X}{\Z}$, encompassing the full integral cohomology of $X$. Interestingly these two Hodge lattices capture the same data of a complex K3 surface $X$ as is witnessed by the equivalence of points (ii) and (iii). In proving the implication (iii)$\implies$(i) we will invoke Orlov's theorem and follow the approach described in \cite{Orlov2003}. The other pinnacle of this proof is the \emph{moduli space} of stable sheaves on $X$. We will bypass its construction, which comes as a result of many people, most notably by \cite{GottscheHuybrechts1996,OGrady1997,huybrechts2006fouriermukai}, but will explain precisely how one can use such a moduli space together with a so-called \emph{universal bundle} to construct a triangulated equivalence between the bounded derived categories $\derb{X}$ and $\derb{Y}$. The other ties between the equivalent statements will be proven along the way. 

%  can be seen from the equivalence of points (ii) and (iii).

% The highlights of this chapter are the Mukai lattice and the moduli space of stable sheaves $M_v$. The Mukai lattice serves as an extension of the ordinary intersection form on $\cohom{2}{X}{\Z}$ encompassing the full integral cohomology of $X$. 

In Appendix \ref{Appendix A} we briefly present the concept of a (cohomological) \emph{spectral sequence}, which we will be using sprinkled in throughout the thesis. We mention the Grothendieck spectral sequence associated to a pair of composable left exact functors between abelian categories from which a multitude of other useful spectral sequences are acquired. Using this machinery we also prove some practical results, which are applied when deriving functors in Chapter \ref{Chapter: Derived categories in geometry}. 

Appendix \ref{appendix B} on lattice theory is there as support to first establish terminology and second to cover an important result of Nikulin \cite{Nikulin1980} regarding extensions of isometries of certain embedded lattices. We apply the latter in the proof of the derived Torelli theorem in Chapter \ref{Chapter: Derived Torelli theorem}. 
%    of lattice theory. Its purpose is to establish the required terminology but also 

\vspace{0.3cm}
\noindent
\textsl{Technicalities.}
A \emph{variety over a field $k$} is a separated integral scheme of finite type over $k$. We say that a variety $X$ over $k$ is \emph{smooth}, if the stalks $\struct{X, x}$ are regular\footnote{A noetherian local ring $(A, \mathfrak{m})$ is \emph{regular} if $\dim A = \dim_k \mathfrak{m}/\mathfrak{m}^2$, where $\dim A$ denotes the \emph{Krull dimension of $A$} and $k$ denotes the residue field $A/\mathfrak{m}$.} local rings for every $x \in X$. A variety $X$ over $k$ is \emph{projective}, if the structure morphism $X \to \spec{k}$ is projective, meaning there exists a closed immersion $X \hookrightarrow \PP^n_k$ for some $n \in \N$. A variety of dimension two will be called a \emph{surface}. For a scheme $X$ an $\struct{X}$-module $\F$ is called \emph{quasi-coherent}, if it is locally isomorphic to a cokernel of a morphism of free $\struct{X}$-modules in other words meaning that any point $x \in X$ has an open neighbourhood $U$ such that 
\[
    \OO_{X|U}^{\oplus I} \to \OO_{X|U}^{\oplus J} \to \F|_U \to 0
\]
is an exact sequence on $U$ for some index sets $I$ and $J$. The $\struct{X}$-module $\F$ is \emph{coherent} if it is locally isomorphic to a quotient of free \emph{finite rank} $\struct{X}$-modules. The index sets $I$ and $J$ in the above presentation would in this case be finite.
%  corresponding to the index sets $I$ and $J$ above being finite.