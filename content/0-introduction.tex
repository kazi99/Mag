\section*{Introduction}
\addcontentsline{toc}{section}{Introduction} % so that it is not numbered in table of contents


\vspace{3cm}

\begin{itemize}
    \item{For us a \emph{variety} will mean an integral separated scheme of finite type over a field $k$. }
    \item{Recall what a coherent and quasi-coherent sheaf is. } 
\end{itemize}

Through the course of the last chapter, we will prove the following theorem. 

\begin{theorem*}[Derived Torelli theorem]
    Let $X$ and $Y$ be K3 surfaces over the field of complex numbers $\C$. Then the following are equivalent.
    \begin{enumerate}[label = (\roman*)]
        \item{$X$ and $Y$ share equivalent bounded derived categories of coherent sheaves, 
        \[
            \derb{X} \iso \derb{Y}.
        \]
        }
        \item{There exists a Hodge isometry $f \colon \transc{X} \to \transc{Y}$ between the transcendental lattices of $X$ and $Y$.}
        \item{There exists 
        % a polarization on $X$ and 
        an isotropic Mukai vector $v \in \mukai{X}$ of divisibility $1$,
        such that $Y$ is isomorphic to a moduli space $M_v$ of stable sheaves on $X$ with Mukai vector $v$. Moreover $M_v$ is the unique K3 surface, for which there exists a Hodge isometry
        %  of divisibility $1$, 
        % such that $Y$ is isomorphic to the unique K3 surface $M_v$, for which
        \[
            \cohom{2}{M_v}{\Z} \iso v^\perp / \Z v.
        \]
        }
        % moduli space of stable sheaves on $X$, with Mukai vector $v$.
    \end{enumerate}
\end{theorem*}
