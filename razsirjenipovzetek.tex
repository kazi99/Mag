Zelo pomembno vlogo v algebraični geometriji igra kohomologija geometrijskega objekta, kot je shema ali raznoterost, $X$ glede na, denimo, koherentni snop $\F$ na $X$. Eden izmed načinov računanja kohomoloških grup $H^i(X, \F)$, ki ga bomo tudi spoznali v poglavju \ref{}, vključuje sledeče. 
% Namesto, da obravnavamo snop $\F$ samega po sebi,
Namesto, da 
\info{morda intrinzično ni prava beseda, ker so klasično predstavitve zajele ravno informacijo o generatorjih, relacijah,... } 
intrinzično obravnavamo snop $\F$, ga predstavimo s \ti \emph{resolucijo} ali \emph{predstavitvijo}, ki jo sestavljata kompleks snopov $F^\bullet$, členi katerega pripadajo nekemu razredu snopov, ki ima glede na kohomologijo določene ugodne lastnosti, in kvazi-izomorfizem $F^\bullet \to \F$ ali $\F \to F^\bullet$. Ker sprememba resolucije na kohomologijo ne bo imela vpliva in ker lahko vsak snop zase vidimo tudi kot kompleks zgoščen v stopnji $0$, želimo snop $\F$ obravnavati enako kot vse njegove resolucije. 
% identificirati skupaj z vsemi njegovimi resolucijami.
Pogledano od daleč, želimo homotopsko kategorijo $\K(\coherent{X})$ spremeniti tako, da se snop $\F$ identificira z vsemi svojimi resolucijami, \oz z drugimi besedami, želimo vse kvazi-izomorfizme v $\K(\coherent{X})$ spremeniti v izomorfizme. Slednje bo naše vodilo, da za splošno abelovo kategorijo $\A$ vpeljemo njej prirejeno izpeljano kategorijo $\dercat{\A}$.  

% Kohomološke grupe $H^i(X, \F)$, kot bomo tudi spoznali v poglavju \ref{}, lahko računamo na sledeč način. Najprej snop $\F$ predstavimo s \ti \emph{resolucijo} ali \emph{predstavitvijo}, ki jo sestavljata kompleks snopov $F^\bullet$, členi katerega pripadajo nekemu razredu snopov, ki ima glede na kohomologijo določene ugodne lastnosti, in kvazi-izomorfizem $F^\bullet \to \F$ ali $\F \to F^\bullet$. Nato 


